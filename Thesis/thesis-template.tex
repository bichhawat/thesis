% % % % % % % % % % % % % % % % % % % % %
% Saarland Ph.D. thesis template		%
% created by Annemarie Friedrich		%
% % % % % % % % % % % % % % % % % % % % %

% % % % % % % % % % % % % % % % % % %
% Language stuff					%
% % % % % % % % % % % % % % % % % % %

% this was for pdflatex
\usepackage[utf8]{inputenc}
\usepackage[T1]{fontenc} % adding this makes sf font look weird??
\usepackage[scaled=0.9]{helvet}

%Xelatex
% \usepackage{fontspec}
\usepackage[english]{babel}

% \setmainfont[Ligatures=TeX]{Minion Pro}
% \setmonofont{Inconsolata}
% \setsansfont{calibri}



% % % % % % % % % % % % % % % % % % %
% Layout and fonts					%
% % % % % % % % % % % % % % % % % % %

\usepackage{fancyhdr}
\usepackage{graphicx}

% layout for floats
\usepackage{float}
\usepackage{placeins}


\setlength{\parindent}{0pt}
% spacing between lines
\linespread{1.1}
\setlength{\parskip}{0.4ex plus 0.3ex minus 0.1ex}
% have nice hanging captions for figures etc.
\usepackage[labelfont=bf,format=hang]{caption}
% this needs to be before fancy header
\usepackage[includeheadfoot,a4paper,total={6in,
  9.8in},footskip=0pt,top=1in,bottom=1in,left=1in,
right=1in, bindingoffset=0.25in, heightrounded]{geometry}

%%%%%%%%%%%%%%%%%%%%%%%%%%%%%%
%% fancy header			    %%
%%%%%%%%%%%%%%%%%%%%%%%%%%%%%%

\pagestyle{fancyplain}
\renewcommand{\chaptermark}[1]{%
	\markboth{\thechapter.\ #1}{}}
\lhead[\fancyplain{}{\thepage}]%
{\fancyplain{}{\bfseries \sffamily \nouppercase \leftmark}}
\rhead[\fancyplain{}{\bfseries \sffamily \nouppercase \leftmark}]%
{\fancyplain{}{\thepage}}
\cfoot{} % no page number at bottom


%%%%%%%%%%%%%%%%%%
%%  title page  %%
%%%%%%%%%%%%%%%%%%

% Adapted from a template by Robert Dahlke and Sigmund Stintzing (LMI München 2002)
%\usepackage{german}

\newcommand{\ThesisTitle}[3]{
  \thispagestyle{empty}
  \vspace*{\stretch{1}}
  {\parindent0cm
  \rule{\linewidth}{.7ex}}
  \begin{flushright}
    \vspace*{\stretch{1}}
    \sffamily\bfseries\Huge
    #1\\
    \vspace*{\stretch{2}}
    \sffamily\Large
    #2
    \vspace*{\stretch{1}}
  \end{flushright}
  \rule{\linewidth}{.7ex}
  
  \vspace*{1cm}
  
  \begin{center}
	    \includegraphics[width=1.5in]{owl}
  \end{center}

  \vspace*{\stretch{1}}
  \begin{center}
    \Large A dissertation submitted towards the degree\\
    \Large Doctor of Engineering\\
    \Large of the Faculty of Mathematics and Computer Science\\
    \Large of Saarland University\\
     \vspace*{0.5cm}
    \large by\\
    \Large #2\\
    \vspace*{0.5cm}
    \large  \sffamily Saarbr\"ucken \\
    \large  #3
  \end{center}


  \colloquium	
}

\newcommand{\colloquium}[7]{%
  \newpage
  \thispagestyle{empty}

  \vspace*{\stretch{1}}

  \begin{flushleft}
    \begin{tabular}{p{7cm} | l}%
	\large Day of the Colloquium & #1\\[1mm]
	\large Dean of the Faculty & #2 \\[1mm]
	& \\[1mm]
	\large Chair of the Committee & #3 \\[1mm]
	\large First Reviewer & #4 \\[1mm]
	\large Second Reviewer & #5 \\[1mm]
	\large Third Reviewer & #6 \\[1mm]
	\large Academic Assistant & #7\\
    \end{tabular}%
  \end{flushleft}	

  \cleardoublepage
}


% % % % Chapter headings % % % % %
\usepackage[raggedright]{titlesec}

% fixes a bug in current texlive installation
% http://tex.stackexchange.com/questions/299969/titlesec-loss-of-section-numbering-with-the-new-update-2016-03-15
\usepackage{etoolbox}
\makeatletter
\patchcmd{\ttlh@hang}{\parindent\z@}{\parindent\z@\leavevmode}{}{}
\patchcmd{\ttlh@hang}{\noindent}{}{}{}
\makeatother

\titleformat{\chapter}[display]
{\normalfont\sffamily\LARGE\bfseries}
{\chaptertitlename\ \thechapter}{0pt}{\rmfamily\Huge\rule{\linewidth}{.2ex}\raggedright\\}[\vspace*{-0.6cm}\rule{\linewidth}{.2ex}]

\titleformat{\part}[display]
{\normalfont\sffamily\Huge\bfseries\centering}
{\partname\ \thepart\\ \rule{\linewidth}{.3ex}}{10pt}{}
 




% % % % % % % % % % % % % % % % % % %
% citation							%
% % % % % % % % % % % % % % % % % % %
\usepackage[numbers,sort&compress]{natbib}
\usepackage[toc,page]{appendix}
\usepackage{bibentry}
\nobibliography*

% \bibliographystyle{plainnat}


\usepackage{amsmath}
\usepackage{listings}
\lstset{
  numbers=left,
  numbersep=10pt,
  numberstyle=\tiny\color{gray},
  basicstyle=\ttfamily\small,
  escapeinside={@}{@},
  keywordstyle=\ttfamily,
  firstnumber=1,
  showspaces=false,
  numberfirstline=true,
  columns=fullflexible,
  captionpos=b,
  % language=Pascal,
  mathescape=true,
  xleftmargin=0.7in
}


\captionsetup{justification=centering}


\usepackage{tikz}
\usepackage{amsthm}
\usepackage{lmodern}%[lighttt]
\usepackage[T1]{fontenc}
\usepackage{amssymb}

\usepackage{hyperref}


%\usepackage{semantic}
\usepackage{mathpartir}

\usepackage{xspace}
\usepackage{fixltx2e}
\usepackage{framed}
\usepackage{relsize}


% % % % % % % % % % % % % % % % % % %
% graphics etc						%
% % % % % % % % % % % % % % % % % % %
\usepackage{graphicx}
% \usepackage[usenames,dvipsnames]{xcolor}
% todonotes: useful while writing!
\usepackage[backgroundcolor=yellow!30]{todonotes}
% for showing todonotes properly
\setlength{\marginparwidth}{2.2cm}
%\reversemarginpar

% % % % % % % % % % % % % % % % % % %
% useful custom commands			%
% % % % % % % % % % % % % % % % % % %

% dense underline
\newcommand{\dul}[1]{\underline{\smash{#1}}}


% % % % % % % % % % % % % % % % % % %
% stuff to create nice tables		%
% % % % % % % % % % % % % % % % % % %
\usepackage{booktabs} % For \toprule, \midrule and \bottomrule
\usepackage{siunitx} % Formats the units and values
%\usepackage{pgfplotstable} % Generates table from .csv
\usepackage{multirow}
\usepackage{longtable} % for tables breaking pages
\usepackage{array}
\usepackage{dcolumn}
%here we're setting up a version of the math fonts with normal x-width
\DeclareMathVersion{nxbold}
\SetSymbolFont{operators}{nxbold}{OT1}{cmr} {b}{n}
\SetSymbolFont{letters}  {nxbold}{OML}{cmm} {b}{it}
\SetSymbolFont{symbols}  {nxbold}{OMS}{cmsy}{b}{n}

%\usepackage{arydshln} % dashed lines
\usepackage{ragged2e}
\newcolumntype{P}[1]{>{\RaggedRight\hspace{0pt}}p{#1}}
\usepackage{adjustbox}
\usepackage{tabularx}
\newcolumntype{R}[2]{%
	>{\adjustbox{angle=#1,lap=\width-(#2)}\bgroup}%
	l%
	<{\egroup}%
}
\newcommand*\rot{\multicolumn{1}{R{45}{1em}}}% no optional argument here, please!
\newcommand*\rotVert{\multicolumn{1}{R{90}{1em}|}}% no optional argument here, please!


% % % % % % % % % % % % % % % % % % %
% environments						%
% % % % % % % % % % % % % % % % % % %
\usepackage{url}
\usepackage{indent}
\usepackage{textcomp}
\usepackage{enumerate}
\usepackage{verbatim} % for comments
\usepackage{fancyvrb}
\usepackage{listings} % for comments
\usepackage{qtree} % for syntactic trees (images)
\usepackage{booktabs} % for nice tables
\usepackage{colortbl}


\newenvironment{packed_enum}{
	\begin{enumerate}
		\setlength{\itemsep}{1pt}
		\setlength{\parskip}{0pt}
		\setlength{\parsep}{0pt}
		\setlength{\leftmargin}{0pt}
	}{\end{enumerate}}

\newenvironment{packed_enum_more_indent}{
	\begin{enumerate}
		\setlength{\itemsep}{1pt}
		\setlength{\parskip}{0pt}
		\setlength{\parsep}{0pt}
		\setlength{\leftmargin}{4pt}
	}{\end{enumerate}}

\newenvironment{packed_item}{
	\begin{itemize}
		\setlength{\itemsep}{1pt}
		\setlength{\parskip}{0pt}
		\setlength{\parsep}{0pt}
		\setlength{\leftmargin}{0pt}
	}{\end{itemize}}

\newenvironment{packed_list}{
	\begin{list}
		\setlength{\itemsep}{1pt}
		\setlength{\parskip}{0pt}
		\setlength{\parsep}{0pt}
		\setlength{\leftmargin}{0pt}
	}{\end{list}}

\setlength{\fboxsep}{0.3cm}
\setlength{\fboxrule}{1pt}

% Example environment (for language examples)
% use with \begin{example}
%... 
%\end{example}

\newcounter{examplecounter}
\newlength\myLeftmargin
\setlength{\myLeftmargin}{0pt}
\newenvironment{example}
{\refstepcounter{examplecounter}
\begin{samepage}
\begin{indentation}{2.5em}{0em} % first: indent on left side, second: indent on right side
	\setlength{\parsep}{0pt}
	\setlength{\parskip}{0pt}
	\setlength{\itemsep}{0pt}
	\begin{list}{\textbf{(\arabic{examplecounter})}}%
		{\global\addtolength\myLeftmargin{\parindent}%
			\global\addtolength\myLeftmargin{\parindent}
			\setlength\leftmargin{\myLeftmargin}%
		}
		\item\relax}
	{\end{list}
\end{indentation}
\end{samepage}
}
	
\usepackage{amssymb}
\usepackage{amsmath}
\setcounter{tocdepth}{1}
\setcounter{secnumdepth}{3}
\usepackage{graphicx}
\usepackage{multicol}
\usepackage{color}
\usepackage{hyperref}
\usepackage{listings}
\usepackage{fancyvrb}
\usepackage{amsthm}
\usepackage{mathtools}
\usepackage{algpseudocode}
\usepackage{semantic}
\usepackage{chngcntr}
\usepackage{url}
\usepackage{paralist}
\usepackage{enumitem}
\usepackage{framed}
\usepackage{mdframed}
\usepackage{mathpartir}
\usepackage{tabularx}
\usepackage{hhline}
\usepackage{parcolumns}

\counterwithin*{paragraph}{subsection}


% reset example count in each chapter
\makeatletter
\@addtoreset{examplecounter}{chapter}
\makeatother
	
\newcommand{\labex}[1]{\label{ex:#1}}
	
% modular references
\newcommand{\eref}[2][]{(\ref{ex:#2}#1)} %examples
\newcommand{\aref}[3][]{(\ref{ex:#2}#1#3)} %examples with (a) / (b)
\newcommand{\cref}[1]{Chapter~\ref{chap:#1}} % chapters
\newcommand{\dref}[1]{Definition~\ref{def:#1}} % definitions
\newcommand{\tref}[1]{Table~\ref{tab:#1}} % tables
\newcommand{\fref}[1]{Figure~\ref{fig:#1}} % figures
\newcommand{\sref}[1]{Section~\ref{sec:#1}} %sections
\newcommand{\apref}[1]{Appendix~\ref{app:#1}} %appendix sections

\newtheorem{mydef}{Definition}
\newtheorem{myLemma}{Lemma}
\newtheorem{myThm}{Theorem}
\newtheorem{myaxiom}{Axiom}
\newtheorem{mycor}{Corollary}
\newtheorem{myprop}{Proposition}


\newcommand{\expr}{\mathit{e}} 
\newcommand{\comm}{\mathit{c}}
\newcommand{\sk}{\texttt{skip}} 
% \newcommand{\lbl}{\mathit{label}} 
\newcommand{\pc}{\mathit{pc}} 
\newcommand{\lab}{\ell}
\newcommand{\TODO}[1]{\textcolor{red}{\bf #1}}
\newcommand{\pl}{\ensuremath{{}^\star}\xspace}
\newcommand{\TT}{\texttt}

% \newcommand{\inference}[3][]{\text{\smaller #1}\inferrule{ #2}{#3}}

\newcommand{\bp}{\mathit{b}}
\newcommand{\node}{\mathit{n}}
\newcommand{\cfg}{\mathcal{G}}
\newcommand{\pathG}[2]{#1 \rightarrow_p #2}
\newcommand{\dom}[2]{#1~\texttt{pd}~#2}
\newcommand{\ndom}[2]{#1~\texttt{Npd}~#2}
\newcommand{\IPD}[1]{\mathit{IPD}(#1)}
\newcommand{\SEN}{\texttt{SEN}}

\setlength{\parindent}{1em}
\setlength{\parskip}{0.5em}
\linespread{1.3}

\newcommand {\mV} {\mathcal{V}}
\newcommand {\mL} {\mathcal{L}}
\newcommand {\mH} {\mathcal{H}}
\newcommand {\mE} {\mathcal{E}}

\newcommand {\cstack} {\kappa}
\newcommand {\inst} {\iota}
\newcommand {\Succ} {\mathit{Succ}}
\newcommand {\Left} {\mathit{Left}}
\newcommand {\Right} {\mathit{Right}}
\newcommand {\src} {\mathit{src}}
\newcommand {\dst} {\mathit{dst}}
\newcommand {\isIPD} {\mathit{isIPD}}
\newcommand {\ipd} {\mathit{ipd}}
\newcommand {\cf} {\mathit{CF}}
\newcommand {\cond} {\mathit{cond}}
\newcommand {\push} {\mathit{push}}
\newcommand {\false} {\mathit{false}}
\newcommand {\true} {\mathit{true}}
\newcommand {\val} {\mathit{value}}
\newcommand {\base} {\mathit{base}}
\newcommand {\Empty} {\mathit{empty}}
\newcommand {\eV} {\mathit{excValue}}
\newcommand {\SC} {\mathit{sc}}
\newcommand {\LO} {\Lambda}
\newcommand{\CFG}{\mathit{CFG}}
\newcommand{\map}{\mu}
\newcommand{\dep}{\delta}
\newcommand{\op}{\oplus}
\newcommand{\eop}{\mathit{eop}}
\newcommand{\cop}{\oplus}
\newcommand{\aop}{\odot}
\newcommand{\lbl}{\mathit{\mathcal{A}}}
\newcommand{\bool}{\mathit{\texttt{bool}}}
\newcommand{\Vars}{\mathit{\texttt{V}}}
\newcommand{\dVars}{\dep}
% \newcommand{\know}{\kappa}
\newcommand{\aks}{\mathcal{K}}
\newcommand{\fin}{\texttt{stop}}
\newcommand{\attacker}{\ell}
\newcommand{\up}{\mathit{upgradeVars}}
\newcommand{\resetBS}{\mathit{resetBS}}
\newcommand{\tb}{\mathit{tt}}
\newcommand{\ff}{\mathit{ff}}
\newcommand {\veq}{\approx^{\beta}_L}
\newcommand {\sigeq}{\simeq_\attacker}
% \newcommand {\sigeq}{\underset{\attacker}{\simeq}}
\newcommand {\teq}{\approx^{\beta}_L}
\newcommand {\req}{\cong^{\beta}_L}
\newcommand {\sceq}{\sim^{\beta}_L}
\newcommand {\lsd}{\Gamma(!\sigma(\dst))}
\newcommand{\eq}{\sim_\attacker}
% \newcommand{\eq}{\underset{\attacker}{\sim}}
\newcommand{\bits}{\mathit{b}}
\newcommand{\trace}{\tau}
\newcommand{\emptyTrace}{\epsilon}
%% \newcommand{\proj}{\vert_\attacker}
\newcommand{\proj}{{\uparrow_\attacker}}
\newcommand{\budget}{\beta}
\newcommand{\blabel}{\mathbb{B}}
\newcommand{\ilabel}{\mathbb{L}}
\newcommand{\dec}{\texttt{declassify}}
\newcommand{\domain}{\mathit{dom}}
\newcommand{\pred}{\mathit{P}}
\newcommand{\entropy}{\mathit{H}}
\newcommand{\avg}{\mathit{avg}}
\newcommand{\Reduce}{\mathbb{R}}

\newcommand{\bscmd}{\underset{\pc}{\Downarrow}}
\newcommand{\bscmdup}{\underset{\pc \sqcup \ell}{\Downarrow}}
\newcommand{\bsexp}{\Downarrow}

\newcommand{\lirsem}{{~\underset{\pc}{\Downarrow}~}}
\newcommand{\simsem}{{~\underset{\pc}{\downarrow}~}}
\newcommand{\lirsemup}{{~\underset{\pc \sqcup k}{\Downarrow}~}}
\newcommand{\simsemup}{{~\underset{\pc \sqcup k}{\downarrow}~}}

% \newcommand{\lirsem}{{~\underset{\pc}{\downdownarrows}~}}
% \newcommand{\simsem}{{~\underset{\pc}{\downarrow}~}}
% \newcommand{\lirsemup}{{~\underset{\pc \sqcup (k, \{\})}{\downdownarrows}~}}
% \newcommand{\simsemup}{{~\underset{\pc \sqcup (k, \{\})}{\downarrow}~}}

\newcommand{\refrule}[1]{\textsc{\ref{#1}}}

\newcommand{\sys}{\mbox{WebPol}}

\makeatletter % allow us to mention @-commands
\def\arcr{\@arraycr}
\makeatother

\makeatletter
\newcommand{\labelthis}[2]{%
  \def\@currentlabel{#2}\ltx@label{#1}#2%
}
\makeatother

\makeatletter
\def\blfootnote{\xdef\@thefnmark{}\@footnotetext}
\makeatother

\hyphenation{term-i-na-tion-in-sen-si-tive}

% Local Variables:
% TeX-engine: xetex
% End:

