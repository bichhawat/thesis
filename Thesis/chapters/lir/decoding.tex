\section{Soundness and Decoding Semantics}
\label{sec:lir:decoding}

The LIR policy defined earlier enforces that the information released 
by a program about every secret is bounded by its pre-determined budget. 
However, to prove that the actual information released by the approach is 
soundly accounted for, it needs to be shown that the the budget
reduction performed as part of LIR is well-founded in an information
theoretic sense. To this end, the result of Shannon's source coding
theorem~\cite{shannon} is leveraged to utilize the trace generated
by the semantics containing the declassified values. 

Recall that Shannon's source coding theorem states that if 
different outputs of a program can be associated with uniquely-decodable
codes, i.e., given a bit string there are no two different  
interpretations of the sequence of code-words, the information 
leaked by the program about the secret inputs through these outputs 
as computed using Shannon entropy is upper-bounded by the average 
length of these codes. The trace projected to an adversary contains all  
the bits the adversary can observe about the secret inputs 
to the program. The projected trace can, thus, be regarded as a coding 
of the information released about the secrets to the adversary. 
Hence, if it can be proven that the trace projected to an adversary
is uniquely-decodable then its average length over all executions
is an upper bound on the information leaked by the program as 
computed by the definition of Shannon entropy. 

In order to prove the soundness of the approach, this section presents a  
decoding semantics that simulates the adversary's approach to decode the 
information it obtains via the trace. The purpose of the decoding 
semantics is to show that the adversary-projected trace 
corresponds to a uniquely-decodable code, i.e., decoding the trace would 
result in a unique final memory store for the adversary, thereby showing 
that our enforcement is sound. The salient features of the semantics are:
\begin{enumerate}
\item Decoding semantics is specialized to a fixed adversary at level
   $\attacker$.
\item Decoding semantics operates on adversary-projected data structures,
  i.e., an $\attacker$-projected memory, where the secret inputs are 
  replaced by $\star$ that represents an unknown 
  value (as shown in Definition~\ref{def:lir:msp}) and
  $\attacker$-projected budget store, where the budgets having a 
  budget-label higher than the adversary's level  
  are replaced by $0$ (as in Definition~\ref{def:lir:bmp}). 
  % $$\iota\proj = \mathop{\Large \forall}_{x \in \sigma} \Big(\iota\proj(x) = -^l
  % \wedge l \sqsubseteq \attacker ~?~ \blabel(x) ~:~ 0 \Big)$$
\item At declassification points, the decoding semantics
  reads a value from the $\attacker$-projected trace. 
\item Decoding semantics completely skips the code that is executed 
  under a $\pc$ influenced by secrets (having the value $\star$).
\end{enumerate}

% In order to prove that the projected trace corresponds to a uniquely-decodable
% code, this section presents a decoding semantics that is almost
% similar to the LIR semantics from Section~\ref{sec:semantics}, except
% for the following differences:
% \begin{enumerate}
% \item Decoding semantics is specialized to a fixed adversary,
%   represented as $\attacker$.
% \item Decoding semantics operates on an $\attacker$-projected memory
%   (where the secret inputs are replaced by $\star$, representing an unknown
%   value as shown in Definition~\ref{def:lir:msp}) and
%   $\attacker$-projected budget store (where secret budget 
%   values are replaced by $0$ as in Definition~\ref{def:lir:bmp}). 
%   % $$\iota\proj = \mathop{\Large \forall}_{x \in \sigma} \Big(\iota\proj(x) = -^l
%   % \wedge l \sqsubseteq \attacker ~?~ \blabel(x) ~:~ 0 \Big)$$
% \item At declassification points, the decoding semantics
%   reads a value from the trace, which is also $\attacker$-projected. 
% \item Decoding semantics completely skips the code that is executed under a
%   $\pc$ influenced by $\star$.
% \end{enumerate}

\begin{mydef}[Memory store projection]
\label{def:lir:msp}
The projection of a memory store, $\sigma$, w.r.t. an adversary at
level $\attacker$, written $\sigma\proj$, is defined as:
\begin{align*}
\sigma\proj = \lambda \emph{\TT{x}}. 
\begin{cases}
\sigma(\emph{\TT{x}}), &\text{if} ~ \big(\Gamma(\sigma(\emph{\TT{x}})) \sqsubseteq \attacker\big) \\
\star^{(\bot,\{\emph{\TT{x}}\})}, &\text{otherwise}
\end{cases}
\end{align*}
\end{mydef}
% \begin{mydef}[Memory store projection]
% \label{def:lir:msp}
% The projection of a memory store, $\sigma$, w.r.t. an adversary at
% level $\attacker$, written $\sigma\proj$, is defined as:
% $$\sigma\proj = \Pi(\sigma, \attacker) = 
% \left \{\emph{\TT{x}}~\middle|~ \left(\emph{\TT{x}} =
% \left\{\begin{array}{ll}
%         \emph{\TT{y}}, & \text{if} ~ (\emph{\TT{y}} \in \sigma) \wedge \big(\Gamma(\sigma(\emph{\TT{y}})) \sqsubseteq \attacker\big) \\
%          \star^{(\bot,\{\emph{\texttt{x}}\})}, & \text{otherwise}
%         \end{array}\right\} 
% \right)\right\}
% $$
% $$\sigma\proj = \mathop{\Large \forall}_{x \in \sigma} \Big(\sigma\proj(x) =
%   \big(\Gamma(\sigma(x)) \sqsubseteq \attacker ~?~ \sigma(x) ~:~ \star^{(\bot,
%     \{\texttt{x}\})} \big)\Big)$$
% \end{mydef}

\begin{figure}[!htbp]
\begin{framed}
\begin{mathparpagebreakable}
%%%%%%%%%%%%%%%%%%CONST
\inferrule*[left=\mbox{\labelthis{lir:exp:sc}{s-const}}]
{ } 
{\langle\sigma, \iota, \TT{n}, \trace \rangle \simsem
  \TT{n}^{(\bot,\{\})}, \iota, \trace}
\and
%%%%%%%%%%%%%%%%%%VAR
\inferrule*[left=\mbox{\labelthis{lir:exp:sv}{s-var}}]
{\sigma(\TT{x})  = \TT{v}^{(k, \dep_o)} \\ 
  \dep' = \Reduce(\dep_o, \iota, k_o, \blabel) \\
  \dep = \dep_o \setminus \dep' \\ 
  k = k_o  \bigsqcup\limits_{\TT{x}  \in \dep'} \ilabel(\TT{x})  }
{\langle\sigma, \iota, \TT{x}, \trace \rangle \simsem \TT{v}^{(k, \dep)},
  \iota, \trace}
\and
%%%%%%%%%%%%%%%%%%AOP
\inferrule*[left=\mbox{\labelthis{lir:exp:saop}{s-aop}}]
{\langle \sigma, \iota, \expr_1, \trace \rangle \simsem
  \TT{v}_1^{(k_1, \dep_1)}, \iota_1, \trace_1 \\ \langle \sigma,
  \iota_1,   \expr_2, \trace_1 \rangle \simsem \TT{v}_2^{(k_2, \dep_2)},
  \iota', \trace' \\ \TT{v}= \TT{v}_1 \aop \TT{v}_2 \\ (k, \dep) =
  (k_1, \dep_1) \sqcup (k_2, \dep_2) 
}
{ \langle\sigma,  \iota,  (\expr_1 \aop \expr_2), \trace \rangle
  \simsem\TT{v}^{(k, \dep)}, \iota', \trace'}
\and
%%%%%%%%%%%%%%%%%%COP
\inferrule*[left=\mbox{\labelthis{lir:exp:scop}{s-cop}}]
{\langle \sigma, \iota, \expr_1, \trace \rangle \simsem
  \TT{v}_1^{(k_1, \dep_1)}, \iota_1, \trace_1 \\ \langle \sigma,
  \iota_1,   \expr_2, \trace_1 \rangle \simsem \TT{v}_2^{(k_2, \dep_2)},
  \iota', \trace' \\ \TT{v}= \TT{v}_1 \cop \TT{v}_2 \\ (k, \dep) =
  (k_1, \dep_1) \sqcup (k_2, \dep_2) 
}
{ \langle\sigma,  \iota,  (\expr_1 \cop \expr_2), \trace \rangle
  \simsem\TT{v}^{(k, \dep)}, \iota', \trace'}
\and
%%%%%%%%%%%%%%%%%%OPER-COP
\inferrule*[left=\mbox{\labelthis{lir:exp:sdn}{s-dcopn}}]
{
\langle\sigma,  \iota,  (\expr_1 \cop \expr_2), \trace \rangle
  \simsem\TT{v}^{(k, \dep)}, \iota', \trace' \\
  \pc \not\sqsubseteq k 
}
{\langle\sigma,  \iota, \dec(\expr_1 \op \expr_2), \trace
  \rangle \simsem\TT{v}^{(k, \dep)}, \iota', \trace'}
\and
%%%%%%%%%%%%%%%%%%OPER-COP
\inferrule*[left=\mbox{\labelthis{lir:exp:sdr}{s-dcopr}}]
{
  \langle\sigma,  \iota,  (\expr_1 \cop \expr_2), \trace \rangle
  \simsem\TT{v}_o^{(k_o, \dep_o)}, \iota_1, \trace_1 \\
  pc \sqsubseteq k_o   \\
  \dep' = \Delta(\dep_o, k_o, \ilabel) \\
  \dep'' = \Reduce(\dep',\iota_1, k_o, \blabel) \\ 
  \dep = \dep' \setminus \dep'' \\ 
  m =  k_o \bigsqcup\limits_{\TT{x} \in \dep''}
  \ilabel(\TT{x}) \bigsqcup\limits_{\TT{y} \in \dep} \blabel(\TT{y}) \\
%\iota' = \mathbb{I}(\iota_1, \dep)\\
 ( \TT{v}, k, \iota', \trace') =  \left\{\begin{array}{ll}
        (\TT{v}_t, k_t, \mathbb{I}(\iota_1, \dep_t), \trace''), & \textit{if} ~(\trace_1 = \TT{v}_t^{k_t} :: \trace'') \wedge (k_t = m) \wedge (\dep \neq \emptyset)\arcr
        (\TT{v}_{o}, m, \iota_1, \trace_1), & \textit{otherwise}
        \end{array}\right\} 
% pc \sqsubseteq k_o   \\
%  \dep' = \Delta(\dep_o, k_o, \ilabel) \\
% \dep'' = \Reduce(\dep',\iota_2, k_o, \blabel) \\ 
%   \dep = \dep' \setminus \dep'' \\ 
% m =  k_o \bigsqcup\limits_{\TT{x} \in \dep''}
%   \ilabel(\TT{x}) \bigsqcup\limits_{\TT{y} \in \dep} \blabel(\TT{y}) \\
% \iota' = \mathbb{I}(\iota_2, \dep)\\
%  (\trace', \TT{v}, k) =  \left\{\begin{array}{ll}
%         (\trace'', \TT{v}_t, k_t), & \textit{if} ~~ (\dep \neq \emptyset) \wedge (\trace_2 = \TT{v}_t^{k_t} :: \trace'') \arcr
%         (\trace_2, \TT{v}_{o}, m), & \textit{otherwise}
%         \end{array}\right\} 
% \trace' = \left\{\begin{array}{ll}
%         \trace'', & \textit{if} ~ (\dep \neq \emptyset) \wedge (\trace_2 =\TT{v}:: \trace'') \arcr
%         \trace_2, & \textit{otherwise}
%         \end{array}\right\} \\
% (\dep \neq \emptyset) ~?~ (\trace_2 =\TT{v}^l :: \trace') : 
% (\trace' = \trace_2 \wedge v=\TT{v}_o \wedge l = l_o \bigsqcup\limits_{x \in \dep}
%   (\ilabel(\TT{x}) \sqcup \blabel(\TT{x})))
}
{\langle\sigma,  \iota, \dec(\expr_1 \op \expr_2), \trace
  \rangle \simsem\TT{v}^{(k, \{\})}, \iota', \trace'}
% \and
% %%%%%%%%%%%%%%%%%%OPER-COP
% \inferrule*[left=\mbox{\labelthis{lir:exp:sdn}{s-dcopn}}]
% {\langle \sigma,  \iota, \expr_1, \trace \rangle \simsem
%   \TT{v}_1^{(k_1, \dep_1)}, \iota_1, \trace_1 \\
%   \langle \sigma, \iota_1, \expr_2,\trace_1  \rangle \simsem
%   \TT{v}_2^{(k_2, \dep_2)}, \iota_2, \trace_2  \\\\
%   \TT{v} = \TT{v}_1 \op \TT{v}_2
%   \\ \dep = \dep_1  \cup \dep_2 \\ k = k_1 \sqcup k_2
%   \\ pc \not\sqsubseteq k \\
% }
% {\langle\sigma,  \iota, \dec(\expr_1 \op \expr_2), \trace
%   \rangle \simsem\TT{v}^{(k, \dep)}, \iota_2, \trace_2}
\end{mathparpagebreakable}
\end{framed}
\caption{Decoding semantics of expressions}
\label{fig:lir:sem-sim-e}
\end{figure}

\begin{figure}[!tb]
\begin{framed}
\begin{mathpar}
%%%%%%%%%%%%%%%%%%SKIP
\inferrule*[left=\mbox{\labelthis{lir:cmd:ssk}{s-skip}}]
{ }
{\langle   \sigma, \iota, \sk, \trace \rangle \simsem
  \langle \sigma, \iota, \trace \rangle}
\and
%%%%%%%%%%%%%%%%%%ASSN
\inferrule*[left=\mbox{\labelthis{lir:cmd:sa}{s-assn}}]
{ \pc \sqsubseteq \Gamma(\sigma(\TT{x})) \\
\langle   \sigma, \iota, \expr, \trace \rangle
\simsem \TT{n}^{(m, {\dep'})}, \iota', \trace'
\\  k = \pc \sqcup m \\ \dep = \Delta(\dep', k, \ilabel)
}
{\langle   \sigma,  \iota, (\TT{x} := \expr), \trace \rangle
 \simsem \langle \sigma[\TT{x} \mapsto
  \TT{n}^{(k,{\dep})}], \iota', \trace' \rangle}
\and
%%%%%%%%%%%%%%%%%%SEQ
\inferrule*[left=\mbox{\labelthis{lir:cmd:ss}{s-seq}}]
  {\langle   \sigma, \iota, \comm_1, \trace \rangle \simsem
  \langle  \sigma', \iota', \trace_1 \rangle  \\ \langle   \sigma',
  \iota', \comm_2, \trace_1 \rangle \simsem \langle
  \sigma'',\iota'', \trace' \rangle
  }
  {\langle   \sigma, \iota, \comm_1;\comm_2, \trace \rangle
  \simsem \langle \sigma'', \iota'', \trace' \rangle}
\and
%%%%%%%%%%%%%%%%%%IFN
\inferrule*[left=\mbox{\labelthis{lir:cmd:sien}{s-if-else-n}}]
  {\langle   \sigma, \iota,  \expr, \trace \rangle \simsem
 \TT{v}^{(k_o,{\dep})}, \iota', \trace_1 \\\TT{v}\neq \star
  \\
i = \left\{\begin{array}{ll}
        1, & \textit{if} ~ (\TT{v} = \texttt{true}) \arcr
        2, & \textit{otherwise}
        \end{array}\right\} \\
k = k_o \bigsqcup\limits_{\TT{x} \in \dep} \ilabel(\TT{x})
\\ \langle   \sigma, \iota', \comm_i, \trace_1
  \rangle \simsemup \langle \sigma', \iota'',
  \trace' \rangle}
  {\langle   \sigma, \iota,
  (\texttt{if}~\expr~\texttt{then}~\comm_1~\texttt{else}~\comm_2),
  \trace \rangle \simsem  \langle \sigma', \iota'',
  \trace' \rangle}
\and
%%%%%%%%%%%%%%%%%%IFS
\inferrule*[left=\mbox{\labelthis{lir:cmd:sies}{s-if-else-s}}]
  {\langle \sigma, \iota,  \expr, \trace \rangle \simsem
  \star^{(\_, \_)}, \iota', \trace'
  }
  {\langle   \sigma, \iota,
  (\texttt{if}~\expr~\texttt{then}~\comm_1~\texttt{else}~\comm_2),
  \trace \rangle \simsem  \langle \sigma, \iota', \trace' \rangle}
\and
%%%%%%%%%%%%%%%%%%W_F_S
\inferrule*[left=\mbox{\labelthis{lir:cmd:swfs}{s-while-fs}}]
  {\langle   \sigma, \iota, \expr, \trace \rangle \simsem
 \TT{v}^{(\_, \_)}, \iota', \trace' \\ (\TT{v} = \star) \vee (\TT{v} = \texttt{false})
  }
  {\langle   \sigma,\iota,
  \texttt{while}~\expr~\texttt{do}~\comm, \trace
  \rangle \simsem  \langle \sigma, \iota', \trace' \rangle}
\and
%%%%%%%%%%%%%%%%%%W_T
\inferrule*[left=\mbox{\labelthis{lir:cmd:swt}{s-while-t}}]
  {\langle   \sigma, \iota, \expr, \trace \rangle \simsem
  \texttt{true}^{(k_o,{\dep})}, \iota_1, \trace_1 \\
  k = k_o \bigsqcup\limits_{x \in \dep} \ilabel(\TT{x})
  % \forall x \in \dep.(l := l \sqcup \ilabel(\TT{x}))
  \\ \langle   \sigma, \iota_1, \comm;
  \texttt{while}~\expr~\texttt{do}~\comm, \trace_1
  \rangle \simsemup \langle \sigma', \iota', \trace_2 \rangle
  }
  {\langle   \sigma,  \iota,
  \texttt{while}~\expr~\texttt{do}~\comm, \trace
  \rangle \simsem  \langle \sigma'',\iota'', \trace'' \rangle}

\end{mathpar}
\end{framed}
\caption{Decoding semantics of commands}
\label{fig:lir:sem-sim-c}
\end{figure}

Program configurations for expressions and commands are extended with
the projected trace and projections of the memory and budget store,
given by $\langle \sigma\proj, \iota\proj, \expr, \trace\proj \rangle$
and $\langle \sigma\proj, \iota\proj, \comm, \trace\proj \rangle$,
respectively. $\sigma\proj$, $\iota\proj$, and $\trace\proj$
represent the $\attacker$-projected memory, budget-map and trace,
respectively.  

The semantics of expression and command evaluation is shown in
Figures~\ref{fig:lir:sem-sim-e} and~\ref{fig:lir:sem-sim-c} and is
mostly similar to LIR semantics presented earlier
(Figures~\ref{fig:lir:sem-e} and~\ref{fig:lir:sem-c}) except that the
decoding semantics takes the trace  $\trace$ as input. The rules
generate a new trace $\trace'$, which is a suffix of the original
trace $\trace$. The  main difference in the semantics of expressions
shows up in the \refrule{lir:exp:sdr} rule --- when declassification
occurs, the  declassified value is read from the trace (as opposed to
the one computed by local evaluation in the sub-expressions, which is
generally $\star$). The main difference in the semantics of commands
is in the branching rules. Branching or looping on $\star$ values
skips the execution as shown in rules \refrule{lir:cmd:sies} and
\refrule{lir:cmd:swfs}.  

The proof for the projected-trace being uniquely-decodable is based on the
decoding semantics progressing on the trace generated by the LIR
semantics and resulting in a unique final memory with respect to the 
adversary. This requires defining equivalence between the memory 
stores in the two semantics. The memory equivalence
(Definition~\ref{def:lir:simseq}) is defined based on the 
observational equivalence of the values in the original memory store and 
the projected memory store (Definition~\ref{def:lir:eq}). 
The idea behind the definition of value equivalence is that if a value is 
visible to the adversary in the original memory store, then the value is 
also visible to the adversary in the projected store 
(Definition~\ref{def:lir:eq-1}). If the value is not visible to the 
adversary in the original memory store, then either the value shall never 
be declassified to the adversary in either semantics 
(Definition~\ref{def:lir:eq-2a} and~\ref{def:lir:eq-2b}) or the value can 
be declassified to the adversary under both the semantics 
(Definition~\ref{def:lir:eq-2c}). 

\begin{mydef}
\label{def:lir:eq}
Two values $\emph{\TT{v}}_o$ and $\emph{\TT{v}}_s$, where $\emph{\TT{v}}_o$ is a value in the original
memory store and $\emph{\TT{v}}_s$ is a value in the projected memory store for
simulation, are observationally equivalent at level $\attacker$,
written $\emph{\TT{v}}_o \sigeq \emph{\TT{v}}_s$ iff
\begin{enumerate}[label=\emph{\arabic*}., ref={\ref{def:lir:eq}.\arabic*}]
\item\label{def:lir:eq-1} $\emph{\TT{v}}_o = \emph{\TT{n}}_o^{(k, \dep)}$, $\emph{\TT{v}}_s =
  \emph{\TT{n}}_s^{(k',\dep')}$, $\emph{\TT{n}}_o = \emph{\TT{n}}_s$ and
  $(k = k') \sqsubseteq \attacker$ and $\dep = \dep'$ (or)
\item\label{def:lir:eq-2} $\emph{\TT{v}}_o = \emph{\TT{n}}_o^{(k, \dep)}$, $\Gamma(\emph{\TT{v}}_o) \not\sqsubseteq
  \attacker$,  $\emph{\TT{v}}_s = \star^{(k',\dep')}$ and either:
  \begin{enumerate}[label=\emph{(\alph*)}, ref={\ref{def:lir:eq-2}(\alph*)}]
  \item\label{def:lir:eq-2a} $k \not\sqsubseteq \attacker ~\wedge ~k'
    \not\sqsubseteq \attacker$
  \item \label{def:lir:eq-2b} $k \sqsubseteq \attacker~\wedge~\exists \emph{\TT{x}} \in
    \dep. (\blabel(\emph{\TT{x}}) \not\sqsubseteq \attacker) ~\wedge ~k' \not\sqsubseteq
    \attacker$
  \item\label{def:lir:eq-2c} $(k = k') \sqsubseteq \attacker~ \wedge
    ~\dep = \dep'$
  \end{enumerate}
\end{enumerate}
\end{mydef}

\begin{mydef}
\label{def:lir:simseq}
Given a memory store $\sigma$ and a projected memory store
$\sigma'\proj$, their equivalence $\sigeq$ at level
$\attacker$ is defined as: 
$\forall \emph{\TT{x}}.~ \sigma(\emph{\TT{x}}) \sigeq
\sigma'\proj(\emph{\TT{x}})$ 
\end{mydef}


Under a given memory $\sigma$ and budget store
$\iota$, if a program executes to completion under the LIR semantics
and generates a trace $\trace$, then for any chosen adversary
$\attacker$ the same program executing under $\attacker$-projected 
memory ($\sigma\proj$) and budget store ($\iota\proj$) along with the
projected trace $\trace\proj$ results in an observationally equivalent
final memory stores obtained at the end of both the executions with
respect to the adversary $\attacker$ (Theorem~\ref{thm:lir:csim}).

\begin{myThm}[Simulation Theorem]
\label{thm:lir:csim}
If 
$\langle \sigma, \iota, \comm \rangle \lirsem \langle \sigma', \iota',
\trace \rangle$,  then 
% $\exists \sigma'. \sigma \sigeq \sigma_1\proj \wedge \langle
$\langle \sigma\proj, \iota\proj, \comm, \trace\proj
\rangle  \simsem \langle \sigma'',\iota'',
\trace'' \rangle$ such that $\sigma' \sigeq \sigma''$,
$\iota' \eq \iota''$ and $\trace'' = \emptyTrace$
\end{myThm}

The proof of the theorem and the various auxiliary lemmas 
required to prove the theorem are described in
Appendix~\ref{sec:app:lir}.  As a sanity check, when the budgets of
all the secrets in the store are $0$, the LIR semantics satisfy
non-interference (Corollary~\ref{cor:lir:ni}). 

\begin{mycor}[Non-interference]
\label{cor:lir:ni}
If $\langle \sigma_1, \iota_1, \comm \rangle
\lirsem \langle \sigma_1', \iota_1', \trace_1 \rangle$, 
 $\langle \sigma_2, \iota_2, \comm \rangle
\lirsem \langle \sigma_2', \iota_2', \trace_2 \rangle$, 
 $\sigma_1 \eq \sigma_2$, and $\iota_1\proj = \iota_2\proj = 0$, 
then $\sigma_1' \eq \sigma_2'$.
\end{mycor}

