\section{LIR Enforcement}
\label{sec:formalization}

\subsection{Language and Syntax}
This section describes a runtime enforcement of LIR for the simple
imperative language shown in Figure~\ref{basic:syntax} extended with
the \dec~operator as shown in Figure~\ref{fig:syntaxLIR}. The
comparison and arithmetic operators are separated as $\op$ and $\aop$,
respectively, for the rules. The language
is sufficient for describing the key idea of LIR --- appropriate
deduction of budgets at declassification of Boolean expressions involving
secrets. 

\begin{figure}[!htbp]
\begin{align*}
  \expr	:=~&\TT{n}~\arrowvert~\TT{x}~\arrowvert~\expr_1 \aop \expr_2~\arrowvert~
             \expr_1 \op \expr_2~\arrowvert~\dec(\expr_1 \op \expr_2)\\
  \comm	:=~&\sk~\arrowvert~\TT{x} :=
             \expr~\arrowvert~\comm_1;\comm_2~\arrowvert~\texttt{if}~\expr~\texttt{then}~\comm_1~\texttt{else}~\comm_2~\arrowvert~ 
   \texttt{while}~\expr~\texttt{do}~\comm
\end{align*}
\caption{Syntax of the language}
\label{fig:syntaxLIR}
\end{figure}

\subsubsection{\textbf{Value-labels, budgets and budget-labels}}
Every input (coming in via store variables) to the program has an
initial immutable confidentiality label associated with it referred to
as the \emph{value-label}, which is an element of a security lattice
and represented using $k, l, m$. An immutable map $\ilabel$
represents the mapping from a variable to its value-label. Along with
the value-label, every input is also associated with a \emph{budget}
and an immutable \emph{budget-label}, which is also an element 
of the security lattice. The budget of an input
represents an upper bound on the amount of information that is allowed
to leak about that input. As budgets are publicly visible, changes to
budgets in secret contexts can result in additional leaks as we
demonstrate later. Such leaks are handled using the budget-label by  
disallowing budget reduction if the budget-label is not at least as
high as the current program context ($\pc$). Additionally, the
budget-label indicates the confidentiality level to which information
about the secret value can be released. An immutable budget-map $\blabel$
tracks the budget-label associated with each variable. 
% Since the budgets change as the program
% executes, they themselves can be a channel of information leak.
The budgets are represented as $\TT{n}^l$, where $\TT{n}$ is the
budget and $l$ is the budget-label. We require that the budget-label 
is lower than the value-label for a variable, i.e., 
$\forall \TT{x}.\blabel(\TT{x}) \sqsubset \ilabel(\TT{x})$.
%, i.e., a variable having a budget
%of $1$ and the budget-label $L$ is denoted as $1^L$.

\subsubsection{\textbf{Label format and dependencies}}
Instead of directly tracking the label on the values, variable
dependencies are tracked. As every variable has
an associated value-label, budget, and budget-label, tracking
dependencies enables access to this meta-data for every dependency
individually. This assists the enforcement by determining the right
variables to deduct budget from, which is required at the point of
declassification. These dependencies are split into two parts and the
label on a value is represented as a pair: $(l, \dep)$. The variable
is itself represented as $\TT{x} = \TT{n}^{(l, \dep)}$, where \TT{n}
is the value in the variable \TT{x} and $(l, \dep)$ is the label on
the value. $\dep$, referred to as dependency-set, is a set of
variables on whose initial values the current value
depends. Information can only be released about the variables that are
included in the dependency-set. The first part of the label $l$,
referred to as the secrecy-level of the value, is an element of the 
security lattice and the join of the 
value-labels (upper-bound) of all the dependencies for which no more
information is allowed to (or can) be released. 

Once the budget of a variable expires (becomes $0$), the value-label of that
variable is joined with the secrecy-level of the current value and
the variable is removed from the dependency-set of the current value. 
For instance, assume that initially $\TT{x} = \TT{n}^{(\bot, \{\TT{x}, \TT{y}\})}$ 
such that $\ilabel(\TT{x}) = k$ and $\ilabel(\TT{y}) = m$, once the budget of
$\TT{y}$ expires, $\TT{x} = \TT{n}^{(m, \{\TT{x}\})}$. The
secrecy-level of a value $l$ represents the minimal level at which the
value can be observed, i.e., the current value in the variable can never 
be released or declassified to a level below $l$. Initially, every
variable \TT{x} depends only on itself and is labeled 
$(\bot, \{\TT{x}\})$, where $\bot$ represents the least
element in the lattice. 

The join operation on the labels returns the join of the
secrecy-levels and the union of the dependency-sets as the final
label, i.e., 
$$(l, \dep) \sqcup (l', \dep') = (l \sqcup l', \dep \cup \dep')$$
Similarly, the ordering on the labels is defined as:
$$(l, \dep) \sqsubseteq (l', \dep') = (l \sqsubseteq l', \dep \subseteq \dep')$$
The function $\Lambda(\TT{x})$ returns the current label of the value
in \TT{x}, i.e., if $\TT{x} = \TT{n}^{(l, \dep)}$, then
$$\Lambda(\TT{x}) = \Lambda(\TT{n}^{(l, \dep)}) = {(l, \dep)}$$ 
The function $\Gamma(\TT{n}^{(l, \dep)})$ returns the current
confidentiality level of the value, i.e., 
$$\Gamma(\TT{n}^{(l, \dep)}) = l \bigsqcup\limits_{\TT{x} \in
	\dep} \ilabel(\TT{x})$$ 
Note that the value in \TT{x} is visible to an $\attacker$-level
adversary only if $\Gamma(\TT{x}) \sqsubseteq \attacker$ while
$\Lambda(\TT{x})$ allows us to track the required dependencies. 
% The function $\depend(\TT{x})$ returns all the dependencies of
% $\TT{x}$.
$\pc$ denotes the current program-context level, and 
contains \emph{only} the secrecy-level, represented as $(l, \{\})$,
i.e., none of the dependencies in the $\pc$ are allowed to release any
information through the variables being assigned within the branch and
are hence joined with the secrecy-level $l$ of the $\pc$ before
entering the branch or loop. This decision is justified below with an
example. The function $\Gamma(\pc)$ returns the secrecy-level $l$ of
the $\pc$. When the meaning is clear from the context, we use $\pc$ 
instead of $\Gamma(\pc)$.  

\subsection{Key Aspects of the Enforcement}
\label{sec:enf-design}

%% We begin by describing the subtleties involved in enforcement that motivated the
%% design of the semantics:
\subsubsection{\textbf{Dependency Tracking}}
\label{aspect:dt} 
A program execution can release information through variables that are
dependent on the initial value of some secret variables. Instead of
directly tracking abstract labels, which gives a bound on the secrecy
but loses the actual input variable dependency, variable dependencies
are tracked. These dependencies determine the  variables to deduct the
budget from. Dependency tracking is required for a correct analysis 
and to prevent an adversary from laundering secret information. To
illustrate why dependency tracking is imperative for a sound
analysis, consider the example in Listing~\ref{egDep} with the 
security lattice $L \sqsubset H$. Assume that the initial values in
\texttt{sec} and \texttt{h} are both secret having the value-label $H$ 
and a budget of $1^L$; the initial values in \texttt{pub} and
\texttt{i} are both public having the value-label $L$.

\begin{lstlisting}[float,caption=Leak due to dependent variables, label=egDep]
 pub $=$ 0;
 h $=$ sec $\%$ 2;  @\label{a:t}@
 sec $=$ sec $/$ 2;
 if (declassify(h $==$ 1))  @\label{if:h}@
    pub $=$ pub $|$ 1; @\label{if:a}@
 i $=$ sec $\%$ 2; @\label{a:2}@
 if (declassify(i $==$ 1)) @\label{if:i}@
    pub $=$ pub $|$ 2; @\label{if:a1}@
\end{lstlisting}

Suppose that at the assignment on line~\ref{a:t} only the
confidentiality level of \texttt{sec} is carried over to the label of
\texttt{h} but the actual variable dependencies are not tracked.
Without tracking the dependent variables, the declassification on
line~\ref{if:h} would release the value of \texttt{h} (the last bit of
\texttt{sec}) to \texttt{pub} deducting the budget of \texttt{h} but
not \texttt{sec}. As a result another bit of information about
\texttt{sec} can be leaked later (through the \TT{declassify} on
line~\ref{a:2}), which should not have been allowed as the budget of
\texttt{sec} is just $1$. A trivial solution to this problem would be
to reduce the budget of all the dependencies at the point of
declassification. 

With dependency tracking, initially \texttt{h} and \texttt{sec} are 
dependent on themselves, i.e., \texttt{h} and \texttt{sec},
respectively. The assignment on line~\ref{a:t} would update the
dependence of \texttt{h} on \texttt{sec}. As a result, the
declassification on line~\ref{if:h} would deduct the budget from
\texttt{sec} and not \texttt{h}. On line~\ref{a:2}, \texttt{i} is
updated with the next bit of \texttt{sec} and the dependency-set of
\texttt{i} is updated to contain only \texttt{sec}. As \texttt{sec}'s
budget is expired on line~\ref{if:i}, the monitor joins the
secrecy-level of \texttt{sec} in the new program context, making it
$H$, thus limiting the amount of information released about 
the initial value of \texttt{sec}.  

% \medskip
\begin{lstlisting}[float,caption=Example to illustrate budget-label
  constraints,label=egbr] 
 if (a $==$ 0) @\label{br0}@
   x $=$ b @\label{brxb}@
 z $=$ declassify(x $==$ 1) @\label{brz}@
 y $=$ declassify(b $==$ 0) @\label{bry}@
\end{lstlisting}
\subsubsection{\textbf{Budget-label Constraints}}
\label{aspect:blc} 
The reduction of budget for all variables in the dependency-set can 
result in additional unaccounted leaks. 
We illustrate the leak using the example in Listing~\ref{egbr} and 
the security lattice $L \sqsubset M \sqsubset H$. Assume that
\TT{a} and \TT{b} are two secrets having value-labels $\ilabel(\TT{a})
= M$ and $\ilabel(\TT{b}) = H$ and the current labels $\Lambda(\TT{a})
= (M, \{\})$ and $\Lambda(\TT{b}) = (L, \{\texttt{b}\})$, respectively.  
The initial budgets of \TT{a} and \TT{b} are $0$ and $1^L$,
respectively. Also assume that the label of \TT{x} is $\Lambda(\TT{x})
= (M, \{\})$. Considering two executions of the program depending on
whether the value of \TT{a} is $0$ or not, we show that not imposing
the condition $\forall \TT{x} \in \dep. l \sqsubseteq \blabel(\TT{x})$
when reducing the budget of a secret variable labeled $(l, \dep)$ can
result in additional leaks.  

When \texttt{a} is $0$, \texttt{x} is assigned \texttt{b} on
line~\ref{brxb} with a label $(M, \{\texttt{b}\})$ (the join of the
label of \TT{a} and \TT{b}). On line~\ref{brz}, \texttt{x} can be
declassified to level $M$ ($M \sqcup L$ as the budget of \texttt{b} is 
$1^L$). Thus on line~\ref{brz}, the label of the value in \texttt{z}
becomes $(M, \{\})$ and the budget of \texttt{b} becomes $0$. On  
line~\ref{bry}, as \texttt{b}'s budget has expired, \texttt{b}'s
current label becomes $(H, \{\})$; \texttt{y} is also labeled $(H,
\{\})$. In the other run when \texttt{a} is not $0$, \texttt{x}
retains its label of $(M, \{\})$ on line~\ref{brz}. As the
dependency-set is empty, no declassification occurs on
line~\ref{brz}. As the budget of \texttt{b} on line~\ref{bry} is still
$1^L$, the value on line~\ref{bry} is declassified to $L$. Thus, the
label of \texttt{y} after the assignment becomes $(L, \{\})$.  Thus,
in the first run an 
$L$-level adversary cannot observe any value while in the second run,
the adversary can see the value as it is labeled $L$. As the adversary
knows the budget of \texttt{b}, it can conclude that in the first case
the branch on line~\ref{br0} was taken and in the second case it was
not taken. This leaks an additional bit of information about
\texttt{a} although the budget of \texttt{a} doesn't allow any
information release. 

This happens because a purely dynamic information flow monitor
cannot account for dependencies in alternate branches. To prevent 
this leak, when declassifying a value labeled $(l, \dep)$, the 
budget is reduced for only those variables in the dependency-set 
$\dep$ that have a budget-label at least as high as $l$, i.e., 
$\forall \TT{x} \in \dep. l \sqsubseteq \blabel(\TT{x})$. 
By imposing this condition, in the first run of the above example, 
the value of \texttt{x} labeled $(M, \{\texttt{b}\})$ on line~\ref{brz} 
is not declassified as the budget-label of \texttt{b} 
($\blabel(\texttt{b}) = L$) is lower than the secrecy-level of the 
value ($M$). The execution in the first run then proceeds 
similar to the second run, thereby preventing the additional leak.  

% \medskip
\begin{lstlisting}[float,caption=Example to illustrate budget reduction,label=egbred]
 if (a $==$ 0) @\label{egbr0}@
   x $=$ declassify(b $==$ 0) @\label{egbrxb}@
 z $=$ declassify(b $==$ 1) @\label{egbrz}@
\end{lstlisting}
\subsubsection{\textbf{Budget Reduction}}
\label{aspect:br}
The budget of a secret variable is publicly visible, which can lead to
additional information leaks due to budget reduction in a secret context. 
We impose two additional conditions to prevent such leaks: 
\begin{enumerate}
\item A value cannot be declassified to a level lower than 
  the current program context $\pc$, as this could leak information
  about the context itself
\item The budget of a variable can only be reduced in a $\pc$ lower
  than or equal to the budget-label of the variable.
\end{enumerate} 
In the current setting, this would mean that if the 
label of the value being declassified is $(l, \dep)$ where $\dep$
contains a variable \TT{x}, then information about \TT{x} is only
declassified (the budget of \TT{x} is only reduced) if $\pc
\sqsubseteq l \sqcup \blabel(\TT{x})$. 

Without these checks in place, the monitor could leak additional
information as illustrated by the example in Listing~\ref{egbred}. 
As before, assume that \TT{a} and \TT{b} are two secrets having
value-labels $\ilabel(\TT{a}) = M$ and $\ilabel(\TT{b}) = H$ and the
current labels $\Lambda(\TT{a}) = (M, \{\})$ and $\Lambda(\TT{b}) = (L,
\{\texttt{b}\})$, respectively, such that $L \subseteq M \subseteq H$.   
The initial budgets of \TT{a} and \TT{b} are $0$ and $1^L$,
respectively. Also assume that the label of \TT{x} is $\Lambda(\TT{x})
= (M, \{\})$. Again, consider two executions of the program depending
on whether the value of \TT{a} is $0$ or not. 
Assume that the current context ($\pc$) 
is not checked for before declassifying a value. 

When \texttt{a} is $0$, the value of \texttt{b == 0} is declassified
on line~\ref{egbrxb} making \TT{b}'s budget $0$. \TT{x} is assigned
the label $(M, \{\})$ (the join of the label of \TT{a} and the
budget-label of \TT{b}). On line~\ref{egbrz}, as \texttt{b}'s budget
has expired, \texttt{b}'s current label becomes $(H, \{\})$ and
\texttt{z} is labeled $(H, \{\})$. In the other run when \texttt{a} is
not $0$, \texttt{x} retains its label of $(M, \{\})$ on
line~\ref{egbrz} and the budget of \TT{b} also remains unchanged. As
the budget of \texttt{b} on line~\ref{egbrz} is still $1^L$, the value is
declassified to the level $L$. Thus, in the second run the label of
\texttt{z} after the assignment becomes $(L, \{\})$.  While in the
first run an $L$-level adversary cannot observe any value, in the
second run  the adversary can see the value as it is labeled $L$. As
the adversary knows the budget of \texttt{b}, it can conclude that in
the first case  the branch on line~\ref{egbr0} was taken and in the
second case it was not taken. This leaks an additional bit of
information about \texttt{a} although the budget of \texttt{a} doesn't
allow any information release. 

By allowing the declassification operation only when $\pc \sqsubseteq l \sqcup
\blabel(\TT{x})$, the additional leak can be prevented. With this
condition in place, in the first run of the program, the value of 
\TT{b == 0} on line~\ref{egbrxb} is not declassified as the current
program context $M$ is not lesser than or equal to $L$, which is the
join of the secrecy-level and the budget-label of \TT{b}. Instead, the
declassification occurs on line~\ref{egbrz} thereby preventing any
additional information leaks. 

% \medskip
\subsubsection{\textbf{Handling implicit flows}}
\label{aspect:if}
Dynamic IFC approaches handle implicit flows via control constructs by
employing a program context ($\pc$) label. The NSU check presented in
Section~\ref{sec:bg-nsu} does not allow assignments 
to public variables in a sensitive context ($\pc$). Since variable
dependencies are used in addition to the secrecy-levels, those
dependencies also need to be checked for during assignment
operations. For an assignment to succeed, it is important that
no new dependencies are handed over to the variable as part of the
assignment operation. 

Directly using dependencies in the $\pc$ could lead to
additional information leaks. This happens because the assignments
that happen under a $\pc$ would carry the dependencies of
the $\pc$ too while in an alternate run those dependencies might not
be present in the label of the assigned variable. Such mismatch of
dependencies can leak information as illustrated using an example in
Listing~\ref{egimp}.

\begin{lstlisting}[float,caption=Illustration of handling of
  implicit flows,label=egimp]
 if (x $\leq$ 10) @\label{impbr}@
   x $=$ y @\label{impassn}@
 z $=$ declassify(x $\leq$ 5) @\label{impdec}@
\end{lstlisting}

For the program in Listing~\ref{egimp}, assume that \texttt{x},
\texttt{y} and \texttt{z} have value-labels of $H$, $H$ and $L$ ($L
\sqsubseteq H$), and budgets of $1^L$, $0$ and $0$ assigned to them,
respectively. Also assume that their current labels are of the
form $\Lambda(\TT{x}) = (L, \{\texttt{x}\})$, $\Lambda(\TT{y}) = (H,
\{\})$ and $\Lambda(\TT{z}) = (L, \{\})$. 
Consider two runs of the program with $\texttt{x} \leq 10$ and
$\texttt{x} > 10$. Suppose that the check for dependencies is included
in the $\pc$, i.e., the assignment succeeds only if $\pc = (l, \dep)
\sqsubseteq (k_o, \dep_o)$ where $(k_o, \dep_o)$ is the label of the
variable \texttt{x}.  When $\texttt{x} \leq 10$, the branch on
line~\ref{impbr} is taken and the $\pc$ on line~\ref{impassn} would be
$(L, \{\texttt{x}\})$, which would allow the assignment (as the label 
of \TT{x} is not less sensitive than the $\pc$). As a result the label
of \texttt{x} would be updated to $(H, \{\TT{x}\})$ (join of $\pc$ and
label of \texttt{y}). On line~\ref{impdec}, no information is
released as the budget-label of \texttt{x} is lower than its
secrecy-level ($H$) (as explained above in
Section~\ref{aspect:blc}). In the other run when 
$\texttt{x} > 10$, the branch is not taken and the value of
$\texttt{x} \leq 5$ on line~\ref{impdec} is released to the level $L$
as the label of \texttt{x} on line~\ref{impdec} is $(L,
\{\texttt{x}\})$ and its budget is $1^L$. This leads to different
sensitivity of \texttt{z} in the two runs which leads to leaking an
additional bit about \TT{x} in the first run without being accounted
for in its budget. 

To prevent this leak, we compute the combined label of all the
dependencies of the value in the predicate before using it in the
$\pc$, i.e., we carry only the secrecy-level in the $\pc$. Thus, in
the first run of the above example, the $\pc$ on line~\ref{impassn}
becomes $(H, \{\})$ (because $\ilabel(\texttt{x}) = H$) and the
assignment fails due to the no-sensitive-upgrade check. To ensure
soundness, it is required that $\pc \sqsubseteq k$, where $k$ is the
secrecy-level of the variable being assigned, i.e., the secrecy-level
of the variable being assigned is at least as high as the
secrecy-level of the $\pc$. 

% \medskip

\subsubsection{\textbf{Permissiveness}}
\label{aspect:perm}
The overall confidentiality of the underlying value with a
label $(l, \dep)$ is the join of $l$ with the value-label of all the
variables in the dependency-set. Thus, all those dependencies whose 
value-label is below the secrecy-level of the label can be removed as 
keeping those dependencies does not affect the confidentiality of the
value. This improves the permissiveness of the technique by not
marking the release of information for variables that have a label
lesser than or equal to the current secrecy-level or the $\pc$
because no information can be released below the current secrecy-level 
of the variable or the current $\pc$. For instance, consider a 
value $\TT{n}^{(k, \{\TT{x}\})}$. If $\ilabel(\TT{x}) \sqsubseteq k$, 
then \TT{x} is removed from the dependency-set and the value is
represented as $n^{(k, \{\})}$. As explained earlier, budgets are
deducted only when $\pc \sqsubseteq k$, thus, the budgets of any
variable whose value-label is below the $\pc$ are not deducted from.  
 
\begin{lstlisting}[float,caption=Example to illustrate permissiveness,label=egp]
 if (med $\leq$ 0) 
    x $=$ declassify(sec $==$ y); @\label{ifb}@
 if (declassify(y $\neq$ 100))   @\label{if1}@
    pub $=$ 1;  @\label{ifb1}@
\end{lstlisting}

To illustrate the gain in permissiveness by enforcing the condition
specified above, consider the program in Listing~\ref{egp}. Assume
that, \texttt{med}, \texttt{x}, \texttt{sec}, \texttt{y}, and
\texttt{pub}  are labeled $(M, \{\})$, $(M, \{\})$, $(M,
\{\texttt{sec}\}), (L, \{\texttt{y}\})$, and $(L, \{\texttt{pub}\})$,
respectively, with their respective value-labels being
$\ilabel(\texttt{sec}) = H, \ilabel(\texttt{y}) = M, 
\ilabel(\texttt{pub}) = L$ such that $L \sqsubset M \sqsubset
H$. Also, assume that the budgets of \texttt{sec} and \texttt{y} are
$1^M$ and $1^L$, respectively. The program context label ($\pc$) on
line~\ref{ifb} is $(M, \{\})$. The expression 
$\TT{sec} == \TT{y}$ on line~\ref{ifb} has the label $(M, \{\TT{sec},
\TT{y} \})$.

Without the check for the current secrecy-level, the
expression evaluation on line~\ref{ifb} releases the value of 
\texttt{sec} and \texttt{y} into \texttt{x}. However as the
secrecy-level is $M$, the final value of \texttt{x} would have the
secrecy-level of at least $M$. Thus, information about \texttt{y} is
released to the levels $M$ and above, who already had access to the
value of \texttt{y}. Subsequently, on line~\ref{if1} as the budget of
\texttt{y} has expired, no more information about \texttt{y} is
released and the assignment on line~\ref{ifb1} fails. Thus, no useful
information about \texttt{y} is released at any point in the program,
yet the program is terminated because of the NSU check. With the check 
for current secrecy-level and the context label, the budget of
\texttt{y} is not deducted on line~\ref{ifb}, which allows the check
on line~\ref{if1} to release information about \texttt{y} 
thereby allowing the assignment on line~\ref{ifb1} to succeed.


\subsection{Semantics}
\label{sec:semantics}
To formally define the information released by a program starting from some
initial memory containing secret values, big-step semantics is defined
for the language shown in Figure~\ref{fig:syntaxLIR} that releases
some information about initial secret values at declassification of 
Boolean expressions.
% For simplicity of
% exposition, we explain the enforcement for a simple WHILE
% language. The key idea of the policy, however, can be generalized to
% any language.
% We define a big-step semantics for the language in
% Figure~\ref{fig:syntaxLIR}.
% and instrument them with the kind
% of budget and label checks described above.

Program configurations are extended with $\iota$ representing the
budget store, which is a map from the variables to the budget of these
initial  values (inputs to the program). The configurations for
expressions ($\expr$) and commands ($\comm$) are, thus, represented
by $\langle \sigma, \iota, \expr \rangle$ and $\langle \sigma, \iota, 
\comm \rangle$, respectively where $\sigma$ represents the memory
store as before. 
$\langle\sigma, \iota, \expr \rangle \lirsem \TT{n}^{(k, \dep)}, \iota',
\trace$ defines expression evaluation. It means that under some
$\pc$, starting with memory $\sigma$ and budget store $\iota$, the
expression $\expr$ evaluates to a value $\TT{n}$ labeled
$(k, \dep)$ resulting in a budget store $\iota'$. Additionally, the
evaluation generates a trace $\trace$, which is a list of values of the
form $\TT{n}_1^{k_1} :: \TT{n}_2^{k_2} :: \ldots ::
\TT{n}_j^{k_j}$. Every declassified value is recorded on the trace along
with the level to which it is declassified. The two immutable maps ---
initial label map ($\ilabel$) and budget-label map ($\blabel$) are
assumed to be available and omitted from the rules for clarity.

\begin{figure}[!htbp]
\begin{framed}
\begin{mathparpagebreakable}
%%%%%%%%%%%%%%%%%%CONST 
\inferrule*[left=\mbox{\labelthis{lir:exp:c}{const}}]
{ }
{\langle\sigma,  \iota,  \TT{n} \rangle \lirsem
  \TT{n}^{(\bot, \{\})}, \iota,  \emptyTrace}
\and 
%%%%%%%%%%%%%%%%%%VAR
\inferrule*[left=\mbox{\labelthis{lir:exp:v}{var}}]
{\sigma(\TT{x}) = \TT{n}^{(k_o, \dep_o)} \\ 
  \dep' = \Reduce(\dep_o, \iota, k_o, \blabel) \\
  \dep = \dep_o \setminus \dep' \\
  % \dep = \dep' \\ 
  k = k_o \bigsqcup\limits_{\TT{x} \in \dep'} \ilabel(\TT{x}) 
}
{\langle\sigma, \iota,  \TT{x} \rangle \lirsem \TT{n}^{(k, \dep)},
  \iota, \emptyTrace}
\and 
%%%%%%%%%%%%%%%%%%AOP
\inferrule*[left=\mbox{\labelthis{lir:exp:aop}{aop}}]
{\langle \sigma, \iota,  \expr_1 \rangle \lirsem
  \TT{n}_1^{(k_1, \dep_1)}, \iota_1, \trace_1 \\ \langle \sigma,
  \iota_1,   \expr_2 \rangle \lirsem \TT{n}_2^{(k_2, \dep_2)},
  \iota', \trace_2 \\ \TT{n} = \TT{n}_1 \aop \TT{n}_2 \\ (k, \dep) =
  (k_1, \dep_1) \sqcup (k_2, \dep_2)  
}
{ \langle\sigma,  \iota,   (\expr_1~\aop~ \expr_2) \rangle
  \lirsem \TT{n}^{(k, \dep)}, \iota', (\trace_1 ::  \trace_2)
}
\and
%%%%%%%%%%%%%%%%%%OPER-COP-W-D
\inferrule*[left=\mbox{\labelthis{lir:exp:cop}{cop}}]
{\langle \sigma, \iota,  \expr_1 \rangle \lirsem
  \TT{n}_1^{(k_1, \dep_1)}, \iota_1, \trace_1 \\ \langle \sigma,
  \iota_1,   \expr_2 \rangle \lirsem \TT{n}_2^{(k_2, \dep_2)},
  \iota', \trace_2 \\ \TT{n} = \TT{n}_1 \cop \TT{n}_2 \\ (k, \dep) =
  (k_1, \dep_1) \sqcup (k_2, \dep_2)
  % \\  \forall x \in \dep.(l = l \sqcup \ilabel(x))
}
{ \langle\sigma,  \iota,   (\expr_1~\cop~ \expr_2) \rangle
  \lirsem \TT{n}^{(k, \dep)}, \iota', (\trace_1 :: \trace_2)}
\and
%%%%%%%%%%%%%%%%%%OPER-COP
\inferrule*[left=\mbox{\labelthis{lir:exp:dn}{dcopn}}]
{
\langle\sigma,  \iota,   (\expr_1~\cop~ \expr_2) \rangle
  \lirsem \TT{n}^{(k, \dep)}, \iota', \trace \\
\pc \not\sqsubseteq k 
% \langle \sigma,  \iota,  \expr_1 \rangle \lirsem
%   \TT{n}_1^{(k_1, \dep_1)} , \iota_1, \trace_1 \\\\
%  \langle \sigma, \iota_1, \expr_2  \rangle \lirsem
%  \TT{n}_2^{(k_2, \dep_2)},  \iota_2, \trace_2  \\\\
% \TT{n} = \TT{n}_1 \op \TT{n}_2 \\ \dep =\dep_1 \cup \dep_2 \\
% k = k_1 \sqcup k_2 \\\\ 
% \pc \not\sqsubseteq (k, \dep) \bigwedge
% \Lambda(\pc) \not\sqsubseteq k \\ 
% \forall x \in \dep. (l = l \sqcup \ilabel(x)) \qquad
% \trace = \trace_1 :: \trace_2
}
{\langle\sigma,  \iota,  \dec(\expr_1 \op \expr_2)
  \rangle \lirsem \TT{n}^{(k, \dep)},
  \iota', \trace}
\and
%%%%%%%%%%%%%%%%%%OPER-COP
\inferrule*[left=\mbox{\labelthis{lir:exp:dr}{dcopr}}]
{
\langle\sigma,  \iota,   (\expr_1~\cop~ \expr_2) \rangle
  \lirsem \TT{n}^{(k_o, \dep_o)}, \iota_1, \trace_1 \\
  \pc \sqsubseteq k_o \\
 \dep' = \Delta(\dep_o, k_o, \ilabel) \\
\dep'' = \Reduce(\dep',\iota_1, k_o, \blabel) \\
  \dep = \dep' \setminus \dep'' \\ 
k = k_o \bigsqcup\limits_{\TT{x} \in \dep''}
  \ilabel(\TT{x}) \bigsqcup\limits_{\TT{y} \in \dep} \blabel(\TT{y}) \\ 
\iota' = \mathbb{I}(\iota_2, \dep)\\
% \trace = \trace_1 :: (\TT{n},{k},\dep) \\
\trace = 
\left\{\begin{array}{ll}
        \trace_1 :: \trace_2 :: \TT{n}^{k}, & \textit{if } (\dep \neq \emptyset) \arcr
        \trace_1 :: \trace_2, & \textit{otherwise } \arcr
        \end{array}\right\} 
%\left\{\begin{array}{ll}
%        \trace_1 :: (\TT{n},{k},\dep), & \textit{if } (\dep \neq \emptyset) \arcr
%        \trace_1, & \textit{otherwise } \arcr
%        \end{array}\right\} 
}
{\langle\sigma,  \iota,  \dec(\expr_1 \op \expr_2)
  \rangle \lirsem \TT{n}^{(k, \{\})}, \iota', \trace}
%
% \inferrule*[left=\mbox{\labelthis{lir:exp:dr}{dcopr}}]
% {\langle \sigma,  \iota,  \expr_1 \rangle \lirsem
%   \TT{n}_1^{(k_1, \dep_1)} , \iota_1, \trace_1 \\
%  \langle \sigma, \iota_1, \expr_2  \rangle \lirsem
%  \TT{n}_2^{(k_2, \dep_2)},  \iota_2, \trace_2  \\\\
% \TT{n} = \TT{n}_1 \op \TT{n}_2 \\ \dep_o = \dep_1 \cup \dep_2 \\
% k_o = k_1 \sqcup k_2 \\\\
% \pc \sqsubseteq k_o \\
% % \pc \sqsubseteq (k_o, \dep_o)~ \bigvee~ \Gamma(\pc) \sqsubseteq k_o \\
%  \dep' = \Delta(\dep_o, k_o, \ilabel) \\
% \dep'' = \Reduce(\dep',\iota_2, k_o, \blabel) \\\\
%   \dep = \dep' \setminus \dep'' \\ 
% k = k_o \bigsqcup\limits_{\TT{x} \in \dep''}
%   \ilabel(\TT{x}) \bigsqcup\limits_{\TT{y} \in \dep} \blabel(\TT{y}) \\ 
% \iota' = \mathbb{I}(\iota_2, \dep)\\
% \trace = 
% \left\{\begin{array}{ll}
%         \trace_1 :: \trace_2 :: \TT{n}^{k}, & \textit{if } (\dep \neq \emptyset) \arcr
%         \trace_1 :: \trace_2, & \textit{otherwise } \arcr
%         \end{array}\right\} 
% }
% {\langle\sigma,  \iota,  \dec(\expr_1 \op \expr_2)
%   \rangle \lirsem \TT{n}^{(k, \{\})}, \iota', \trace}
% \and
% %%%%%%%%%%%%%%%%%%OPER-COP
% \inferrule*[left=\mbox{\labelthis{lir:exp:dn}{dcopn}}]
% {\langle \sigma,  \iota,  \expr_1 \rangle \lirsem
%   \TT{n}_1^{(k_1, \dep_1)} , \iota_1, \trace_1 \\
%  \langle \sigma, \iota_1, \expr_2  \rangle \lirsem
%  \TT{n}_2^{(k_2, \dep_2)},  \iota_2, \trace_2  \\\\
% \TT{n} = \TT{n}_1 \op \TT{n}_2 \\ \dep =\dep_1 \cup \dep_2 \\
% k = k_1 \sqcup k_2 \\ 
% \pc \not\sqsubseteq k \\
% % \pc \not\sqsubseteq (k, \dep) \bigwedge
% % \Gamma(\pc) \not\sqsubseteq k \\ 
% % \forall x \in \dep. (l = l \sqcup \ilabel(x)) \qquad
% \trace = \trace_1 :: \trace_2
% }
% {\langle\sigma,  \iota,  \dec(\expr_1 \op \expr_2)
%   \rangle \lirsem \TT{n}^{(k, \dep)},
%   \iota_2, \trace}
\end{mathparpagebreakable}
%%%%%%%%%%%%%%%%%%AF used above
\begin{flushleft}
\textbf{Auxiliary functions:}
\begin{align*}
\Delta(\dep, k, \ilabel) =&~ \big\{\TT{x}~\big|~(\TT{x} \in \dep) \wedge
                            (\ilabel(\TT{x}) \not\sqsubseteq k) \big\} \\
\Reduce(\dep, \iota, k, \blabel) =&~ \left\{\TT{x}~\middle|~(\TT{x} \in \dep) \wedge \big(\iota(\TT{x})
= 0 \, \vee \, k \not\sqsubseteq \blabel(\TT{x})\big)\right\} \\
\mathbb{I}(\iota, \dep) =&~\lambda {\TT{x}}. 
\begin{cases}
\iota({\TT{x}}) - 1, &\textit{if} ~ ({\TT{x}} \in \dep) \\
\iota({\TT{x}}), &\textit{otherwise}
\end{cases}
% \mathbb{I}(\iota, \dep) =&~ \left\{(\TT{x},
%                            \TT{n})~\middle|~\big((\TT{x}, \TT{j}) \in
%                            \iota\big) \bigwedge \left(\TT{n}
% =
% \left\{\begin{array}{ll}
%         \TT{j} - 1, & \textit{if} ~ (\TT{x} \in \dep) \\
%         \TT{j}, & \textit{otherwise}
%         \end{array}\right\} 
% \begin{cases}
% \iota(x) - 1 & \mathit{if} ~ x \in \dep \\
% \iota(x) & \mathit{else}
%  % (~?~ \iota(x) - 1 ~:~ \iota(x))\big)\big\} 
% \end{cases}
%\right)\right\}
\end{align*}
\end{flushleft}
\end{framed}
\caption{LIR - Semantics of expressions}
\label{fig:lir:sem-e}
\end{figure}

\begin{figure}[!htbp]
\begin{framed}
\begin{mathparpagebreakable}
%%%%%%%%%%%%%%%%%%SKIP
\inferrule*[left=\mbox{\labelthis{lir:cmd:sk}{skip}}]
{ }
{\langle   \sigma, \iota,   \sk \rangle \lirsem
  \langle \sigma, \iota,  \emptyTrace \rangle}
\and
%%%%%%%%%%%%%%%%%%IF
\inferrule*[left=\mbox{\labelthis{lir:cmd:ie}{if-else}}]
{\langle   \sigma, \iota,   \expr \rangle \lirsem
  \TT{n}^{(k_o, \dep)}, \iota', \trace_1 \\ 
i = \left\{\begin{array}{ll}
        1, & \textit{if } (\TT{n} = \texttt{true}) \arcr
        2, & \textit{otherwise } \arcr
        \end{array}\right\} \\\\
k = k_o \bigsqcup\limits_{\TT{x} \in \dep} \ilabel(\TT{x})
\\ \langle   \sigma,  \iota', \comm_i
  \rangle \lirsemup \langle \sigma', \iota'',
  \trace_2 \rangle \\
}
{\langle   \sigma,  \iota,
  (\texttt{if}~\expr~\texttt{then}~\comm_1~\texttt{else}~\comm_2)
  \rangle \lirsem  \langle \sigma', \iota'',
  \trace_1 :: \trace_2 \rangle}
\and
%%%%%%%%%%%%%%%%%%SEQ
\inferrule*[left=\mbox{\labelthis{lir:cmd:s}{seq}}]
{\langle   \sigma,  \iota,  \comm_1 \rangle \lirsem
  \langle  \sigma', \iota', \trace_1 \rangle  \\\\ 
  \langle   \sigma',
  \iota', \comm_2 \rangle \lirsem \langle \sigma'',
  \iota'', \trace_2 \rangle
}
{\langle   \sigma, \iota,  \comm_1;\comm_2 \rangle
  \lirsem \langle \sigma'', \iota'', (\trace_1
  :: \trace_2) \rangle}
\and
%%%%%%%%%%%%%%%%%%ASSN
\inferrule*[left=\mbox{\labelthis{lir:cmd:an}{assn}}]
{ 
  \langle  \sigma, \iota,  \TT{x} \rangle \lirsem \TT{\_}^{(l, \_)} ,
  \iota, \emptyTrace \\ 
  \pc \sqsubseteq l \\\\
  \langle  \sigma, \iota,  \expr \rangle \lirsem \TT{n}^{(m, \dep')} ,
  \iota', \trace \\\\ 
  k = \pc \sqcup m \\ \dep = \Delta(\dep', k, \ilabel)
}
{\langle   \sigma, \iota,  \TT{x} := \expr \rangle
  \lirsem \langle \sigma[\TT{x} \mapsto
  \TT{n}^{(k, \dep)}], \iota', \trace \rangle}
\and
%%%%%%%%%%%%%%%%%%W_FALSE
\inferrule*[left=\mbox{\labelthis{lir:cmd:wf}{while-f}}]
{\langle   \sigma, \iota,  \expr \rangle \lirsem
  \texttt{false}^{(k, \dep)}, \iota', \trace 
}
{\langle   \sigma,\iota,
  \texttt{while}~\expr~\texttt{do}~\comm
  \rangle \lirsem  \langle \sigma, \iota', \trace \rangle}
\and
%%%%%%%%%%%%%%%%%%W_TRUE
\inferrule*[left=\mbox{\labelthis{lir:cmd:wt}{while-t}}]
{\langle   \sigma, \iota,  \expr \rangle \lirsem
  \texttt{true}^{(k_o, \dep)}, \iota_1, \trace_1 \\ 
k = k_o \bigsqcup\limits_{\TT{x} \in \dep} \ilabel(\TT{x})
\\\\ \langle   \sigma,  \iota_1, \comm;
\texttt{while}~\expr~\texttt{do}~\comm
  \rangle \lirsemup \langle \sigma', \iota',
  \trace_2  \rangle \\
}
{\langle   \sigma,  \iota,
  \texttt{while}~\expr~\texttt{do}~\comm
  \rangle \lirsem  \langle \sigma', \iota',
  \trace_1 :: \trace_2 \rangle}
\end{mathparpagebreakable}
\end{framed}
\caption{LIR - Semantics of commands}
\label{fig:lir:sem-c}
\end{figure}

The evaluation rules for expressions are shown in Figure~\ref{fig:lir:sem-e}.
The rule \refrule{lir:exp:c} evaluates constant values to themselves with
  the label $(\bot, \{\})$ as they do not have any dependencies.  
Rule \refrule{lir:exp:v} evaluates a variable and returns its value $\TT{n}$ along
  with the label $(k_o, \dep_o)$. The function $\Reduce$ returns those
  variables in $\dep_o$ whose budget has expired or have a
  budget-label lower than the secrecy-level $k_o$. The current
  secrecy-level $k_o$ is then joined with the value-labels of the
  variables returned by $\Reduce$ while removing them from the
  dependency-set $\dep_o$.  As no new information is released in both
  the cases, there is no change to the budget store $\iota$ and no
  trace is generated (indicated as an empty trace $\emptyTrace$).  
  
In the rules for evaluating arithmetic (\refrule{lir:exp:aop})
and normal comparison (\refrule{lir:exp:cop}) operations, 
the join of the label of the values that the two sub-expressions 
evaluate to is set as the label of the expression. The individual 
sub-expressions may release some information as captured in 
$\trace_1$ and $\trace_2$, thus updating $\iota$ to $\iota'$.

The \refrule{lir:exp:dr} rule corresponds to the comparison operation
with \dec, and allows information release in certain cases. The
separation of \TT{declassify} from the normal comparison operations is
to offer more control over what information is required to be
released. The rule applies only when the current context $\pc$ is
lower than or equal to the label of the value obtained by evaluation
($\pc \sqsubseteq k_o$). This avoids the leaks specified above
in Section~\ref{aspect:blc} and Section~\ref{aspect:br}. The function
$\Delta(\dep, l, \ilabel)$ returns the variables in the dependency-set
$\dep$ that have an value-label greater than $l$. Additionally, variables
whose budget has expired during the evaluation of the sub-expressions
are removed from the dependency-set (using the function $\Reduce$). 
To prevent the leak described earlier in Section~\ref{aspect:blc}, if the budget-label of any of
the remaining variables in the dependency-set is not at least as high
as the original secrecy-level of the computed value
($k_o \not\sqsubseteq \blabel(\TT{x})$), then the variable is removed
from the dependency-set. For all the variables removed from the
dependency-set, their secrecy-level is joined with that of the actual
label. The budget of all remaining
variables in the dependency-set $\dep$ is deducted by $1$,
corresponding to the $1$-bit Boolean value released on the trace. The
function $\mathbb{I}(\iota, \dep)$ reduces the budget and returns a
new budget store. Their budget-label is also joined with the actual
secrecy-level, which gives the final label of the declassified value.
  Depending on whether the provenance set $\dep$ is empty or not,
  either:  
  \begin{itemize}
  \item the budget of all remaining variables in the provenance set
    $\dep$ is deducted by $1$, corresponding to the $1$-bit Boolean
    value released on the trace. The function $\mathbb{I}(\iota,
    \dep)$ reduces the budget and returns a new budget store. Their
    budget label is also joined  with the actual security level, which
    gives the final label of the declassified value \emph{or}
  \item if the remaining provenance set is empty, no declassification
    occurs
  \end{itemize}
  If declassification occurs, the declassified value is appended to
  the trace.
  
If the current context ($\pc$) is not lower than or equal to the 
secrecy-level of the value obtained by evaluation ($\pc \not\sqsubseteq k$), 
no information is released as shown in the rule \refrule{lir:exp:dn}. 
The final label is a join of the secrecy-levels of all the dependencies. 
The individual sub-expressions may release some information $\trace_1$ and 
$\trace_2$, thus updating $\iota$ to $\iota_2$. 

The judgment for a command execution is given by $\langle \sigma,
\iota, \comm \rangle \lirsem \langle \sigma', \iota', \trace \rangle$ 
--- under a program context $\pc$, the execution of a
command $\comm$ starting with memory $\sigma$ and budget store $\iota$
results in a final memory $\sigma'$ and budget store $\iota'$ while
generating a trace $\trace$ of released values. The semantics is shown in
Figure~\ref{fig:lir:sem-c} whose rules are standard except for some changes
in the assignment and branching rules. The assignment rule
\refrule{lir:cmd:an} does the standard NSU check that disallows
assignment to a public variable in a secret context --- $\pc
\sqsubseteq l$, where $l$ is the secrecy level of the variable
\TT{x} to which the assignment is being made. 
The branching rules \refrule{lir:cmd:ie} and
\refrule{lir:cmd:wt} compute the context label by joining the
value-labels ($\ilabel$) of all the dependencies in $\dep$. 
