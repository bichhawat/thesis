\chapter{Improved Permissive-Upgrade}
\label{ch:ipu}

\begin{lstlisting}[float,label=lst1.1,caption=Impermissiveness of the NSU strategy][escapechar=@]
x = false
if (not(z))
  x = true@\label{lineref1}@
if (y) f() else g()
x = false@\label{lineref2}@
\end{lstlisting}
%
The no-sensitive-upgrade (NSU) check described earlier provides the basic
foundations for sound dynamic IFC. However, terminating a program
preemptively because of the NSU check is quite restrictive in practice. For
example, consider the program of Listing~\ref{lst1.1}, where \TT{z} is
labeled $H$ and \TT{y} is labeled $L$. This program potentially upgrades
variable \TT{x} at line~\ref{lineref1} under a high $pc$, and then
executes function \texttt{f} when \TT{y} is \texttt{true} and
executes function \texttt{g} otherwise. Suppose that \texttt{f}
does not read \TT{x}. Then, for $\TT{y} \mapsto \texttt{true}^L$, this program
leaks no information, but the NSU check would terminate this program
prematurely at line~\ref{lineref1}. (Note: \texttt{g} may read \TT{x},
so \TT{x} is not a dead variable at line~\ref{lineref1}.)

To improve permissiveness, Austin and Flanagan~\cite{plas10} proposed
the permissive-upgrade strategy as a replacement for NSU. However,
that development lacks permissiveness in certain cases. 
This chapter presents the soundness results of the permissive-upgrade
strategy with the improvement for further permissiveness in place.
 % are presented here 
% (using the modified notation for the imperative language)
% and the next
% chapter builds a generalization of the permissive-upgrade strategy to
% arbitrary lattices.


\chapter{Background and Overview}
\label{ch:background}

\section{Information Flow Control}
Information flow control (IFC) refers to controlling the flow of
(confidential) information through a program based on a given security
policy. Typically, pieces of information are classified into security
\emph{labels} and the policy is a lattice over labels. Information is only
allowed to flow up the lattice. For illustration purposes often the
smallest non-trivial lattice $L \sqsubset H$ is used, which specifies that
public (low, $L$) data must not be influenced by confidential (high,
$H$) data. Information flow control can be used to provide
confidentiality (or integrity) of secret (trusted) information; the
work in this thesis is, however, limited to confidentiality guarantees. 
Roughly the idea behind information flow control is that an adversary
can view all the public outputs of a program. By preventing private or
sensitive data to flow to public outputs, the adversary does not get
any information about the private or sensitive data.

The seminal work by Denning~\cite{denning76,denning77,denning82}
defined most of the theoretical ideas pertaining to information flow
control. In general, information can flow along many
channels. However, this thesis considers two of the most important
flows --- \textit{explicit} and \textit{implicit} --- in deterministic
programs. Covert channels like timing or resource usage are beyond the
scope of this thesis.  

An \emph{explicit flow} occurs as a result of direct assignment, e.g.,
the statement \TT{public = secret + 1} causes an explicit flow from
\TT{secret} to \TT{public}. An \emph{implicit flow} occurs due to the
control structure of the program. For instance, consider the program
in Listing~\ref{lst1}.  
% \texttt{public = false; if (secret) public =  true},
 The final value of \TT{y} is equal to the value of
\TT{z} even though there is no direct assignment from \TT{z}
to \TT{y}. Leaking a bit like this can be magnified into leaking
a bigger secret bit-by-bit~\cite{askarov}.

\begin{lstlisting}[float,caption=Implicit flow from \TT{z} to
  \TT{y},label=lst1][escapechar=@]
x = false, y = false
if (not(z))@\label{linerefcond1}@
  x = true@\label{lineref}@
if (not(x))@\label{linerefcond}@
  y = true@\label{lineref5}@
\end{lstlisting}

The correctness of the approaches enforcing information flow control
is often stated in terms of a well-defined property known 
as \emph{non-interference}~\cite{goguen}, which basically stipulates
that high input of a program must not influence its low output. While
non-interference is too strong a property in practice, different
variants of the definition are proven. One such variant of
non-interference usually established for information flow control
techniques is \emph{termination-insensitive
  non-interference}~\cite{volpano}. Roughly, a program is
termination-insensitive non-interferent if any two terminating runs 
of the program starting from low-equivalent heaps (i.e., heaps that
look equivalent to the adversary assuming that an adversary can
observe some part of the heap) end in low-equivalent
heaps. Termination-insensitive means that one-bit of leak is tolerable
when an adversary checks whether or not the program terminated. In
particular, this discounted one-bit leak accounts for termination due
to the failure of a runtime security  check. Askarov et
al.~\cite{askarov} show that for programs with intermediate observable
outputs, termination-insensitivity may leak more than one bit but also
show that this attack is limited to a brute-force attack. 

Information flow control approaches either statically analyze the
program at compile-time and reject or accept a program based on the
security policy, or dynamically analyze the flow of data in a program
by either using a modified runtime or with the help of a reference
monitor, or employ a combination of both the approaches to shrug off 
some of the individual limitations of the two approaches. Static approaches
normally support restricted policies, and are impermissive in
nature. Besides, it becomes extremely complicated to use such methods
with dynamically-typed languages and languages that are loaded with
dynamic features. Dynamic approaches, on the other hand, add runtime
overheads and are less precise as compared to the static
approaches. Dynamic information flow control, which is the central
theme of this thesis, is described in detail later in the
chapter. 

\subsection{Flow-sensitivity}
Static analysis techniques are usually flow-insensitive in
nature, i.e., they do not account for the ordering of the instructions
in the program or the general flow of execution of the program. In
other words, all operations should be individually secure to secure
the program as a whole. 
% For instance, consider the program in Listing~\ref{lst1}. 
% Assume that \TT{z} is a secret variable labeled $H$
% and \TT{x} and \TT{y} are public variables labeled $L$. This program gets
% rejected by a flow-insensitive analysis as the operation on
% line~\ref{lineref} is considered insecure.  
%
% Flow-sensitive analysis improves the permissiveness of static analysis
% techniques and is mostly used in dynamic analysis. Under a
% flow-sensitive static analysis, the program in Listing~\ref{lst1}
% upgrades the labels of \TT{x} and subsequently \TT{y} to $H$ indicating the
% influence of $H$-labeled \TT{z} on \TT{x} and \TT{y}. If the program was to
% output the value of \TT{y} on a $L$-observable channel, the program gets
% rejected by the analysis. Flow-sensitive analysis also accepts
% programs with dead-code like the one shown in Listing~\ref{lst1.1.1},
% thus improving the permissiveness of the analysis. Quite a few
% flow-sensitive static analyses have been proposed to enforce
% information flow control~\cite{hunt2006:types,hammer09ijis} while
% almost all dynamic approaches are flow-sensitive in nature. 
%
For instance, consider the program snippet:
$$\TT{l} = \TT{h} $$
Assume that \TT{h} is a secret variable labeled $H$
and \TT{l} is a public variable labeled $L$. This program gets
rejected by a flow-insensitive analysis as the assignment of a secret
value to a public variable is considered insecure.  

Flow-sensitive analysis improves the permissiveness of static analysis
techniques and is mostly used in dynamic analyses. Under a
flow-sensitive static analysis, the above program 
upgrades the labels of \TT{l} to $H$ indicating the
influence of $H$-labeled \TT{h} on \TT{l}. If the program was to
output the value of \TT{l} on a $L$-observable channel later, the
program gets rejected by the analysis. Flow-sensitive analysis also
accepts programs with dead-code like the one shown in
Listing~\ref{lst1.1.1}, 
thus improving the permissiveness of the analysis. Some 
flow-sensitive static analyses have been proposed to enforce
information flow control~\cite{hunt2006:types,hammer09ijis} while
almost all dynamic approaches are flow-sensitive in nature. 

\begin{lstlisting}[float,caption=Permissiveness of flow-sensitive
  analysis,label=lst1.1.1][escapechar=@] 
var x
if (not(z))
  x = true
x = false
\end{lstlisting}

\section{Information Release}
Non-interference prevents all unauthorized flows without
regarding the severity of the leaks. However, many practical
applications require some fragment of the sensitive data to be
released to programs for providing proper functionality or to improve
the user-experience. For instance, according to non-interference even
a standard login application would be considered insecure because it
leaks information about the user’s password, which is considered 
confidential, by either allowing or denying access. 
As a remedy, researchers have proposed various ways to intentionally
release or \emph{declassify} sensitive data through relaxations of
non-interference~\cite{dimDecl}, which is generally enforced using
annotations like \texttt{declassify} in the program. For instance,
\texttt{declassify(pwd == input)} would treat the result 
of the password check as public, thus allowing access-information to
be released to the user. Most of these approaches require policy
specifications by the developer that define what information can be
released by the program as in many settings it is difficult to trust 
third-parties to include correct and appropriate (declassify)
annotations for information release in the code. Declassification
approaches generally ensure that the adversary is unable to declassify
or force declassification of sensitive information as per his/her
needs. Sabelfeld and Sands~\cite{dimDecl} present a survey
of many such techniques that allow release of information in a secure
manner. 

Some of these declassification techniques are described below.
Cohen~\cite{cohen77,cohen78} in his work on selective
dependency used assertions to eliminate certain information paths
thereby allowing programs that satisfy the constraint to be
accepted. The idea was that information flows to the adversary should
not allow her to deduce more information about secrets than is allowed
by the assertions. Volpano and Smith~\cite{relSec} present relative
secrecy, a property that instead of providing absolute
secrecy allows information to be released through specific
\emph{match} queries relative to the size of the secret, i.e., if the
adversary cannot guess the secret in polynomial time. Giacobazzi and
Mastroeni~\cite{ani} generalize non-interference by modelling what 
property the adversary can observe about the secrets. Li and
Zdancewic's relaxed non-interference characterizes the information
released through downgrading policies~\cite{rel-ni}. Sabelfeld and
Sands~\cite{permodel} proposed the PER model using partial equivalence
relations for specifying information flow policies.
Sabelfeld and Myers~\cite{delRelease} introduced the notion of
delimited release, which enables the specification of declassification
policy using \emph{escape hatches}. The policy specifies upfront what
information can be released through these escape hatches. Localized
delimited release~\cite{ldr} requires that an expression is only released 
after a declassification operation for this expression has appeared
in the code. Lux and Mantel~\cite{Lux} propose explicit reference
points to allow flexible specification of what secrets may be
declassified at the specific reference points. Robust
declassification~\cite{robust} ensures that only trusted code is
allowed to declassify information. Chong and Myers~\cite{chong2004}
propose a framework for specifying application-specific information
release policies. Chong~\cite{rir} further proposes the concept of
required information release, which considers applications that are
obligated to release information under certain conditions. Askarov and 
Sabelfeld introduce the concept of gradual release~\cite{gradRelease}
for allowing release of information at certain release or declassification
points. The policy does not allow the adversary knowledge to refine
its knowledge of secrets at
points other than the release points. Banerjee et al.~\cite{cgrSP}
present a way to specify declassification policies 
that satisfy conditioned gradual release, which is an extension of the
gradual release property. Similarly, another security property,
\emph{policy controlled release}, that extends gradual release was
presented by Rocha et al.~\cite{rochaSP} for declassification in
untrusted and legacy programs.

\section{Dynamic Information Flow Control}
\label{ch:bg-difc}
Dynamic IFC usually works by tracking taints or labels on individual
program values in the language runtime. A label represents a mandatory
access policy on the value. 
% For illustration purposes in this thesis, the
% following notations are used ---  the label $L$ (low
% confidentiality) conventionally means that data may be read by an
% (unspecified but fixed) adversary and $H$ (high confidentiality) means
% the data is sensitive or confidential and is not visible to the
% adversary. More generally, labels may be drawn from any lattice of 
% policies, with higher labels representing more restrictive policies.
A value $v$ labeled $\lab$ is written $v^\lab$. 

Flow-sensitive dynamic IFC analysis \emph{propagates} labels as data
flows during program execution. \emph{Explicit} flows are generally
handled by carrying over the label of the computed value to the
variable being assigned. For example, in the statement \TT{x = y + z},
the result of computing \TT{y + z} will have the label that is a join of
the individual labels on \TT{y} and \TT{z}, which is the final label
of \TT{x}, i.e., if either of \TT{y} or \TT{z} is labeled confidential
or $H$, then the final label of \TT{x} is also labeled
$H$\footnote{``\TT{z} is labeled $H$'' actually means ``the 
  value in \TT{z} is labeled $H$''. This convention is used
  consistently.}. 

\emph{Implicit} flows in a flow-sensitive IFC analysis are tracked by
maintaining an additional taint, usually called the program counter
taint or program context taint or $\pc$, which is an upper bound on
the label of all the control dependencies that lead to the current
instruction being executed. For example, in the program of
Listing~\ref{lst1}, the value in variable \TT{x} at the end of
line~\ref{lineref} depends on the value in \TT{z}. If \TT{z} is labeled $H$,
then at line~\ref{lineref}, $\pc = H$ because of the branch in 
line~\ref{linerefcond1} that depends on \TT{z}. Thus, by tracking $\pc$,
dynamic IFC can enforce that \TT{x} has label $H$ at the end of
line~\ref{lineref}, thus taking into account the control dependency.

However, simply tracking control flow dependencies via $\pc$ is not
enough to guarantee absence of information flows when labels are
flow-sensitive, i.e., when the same variable may hold values with
different labels depending on what program paths are executed. The
program in Listing~\ref{lst1} is a classic counterexample, taken
from~\cite{plas09}. Assume that \TT{z} is labeled $H$ and \TT{x} and \TT{y} are
labeled $L$ initially. The final value in \TT{y} is computed as a function
of the value in \TT{z}. If \TT{z} contains $\texttt{true}^H$, then \TT{y} ends
with $\texttt{true}^L$: The branch on line~\ref{linerefcond1} is not
taken, so \TT{x} remains $\texttt{false}^L$ at
line~\ref{linerefcond}. Hence, the branch on line~\ref{linerefcond} is
taken, but $pc = L$ at line~\ref{lineref5} and \TT{y} ends with
$\texttt{true}^L$. If \TT{z} contains $\texttt{false}^H$, then similar
reasoning shows that \TT{y} ends with $\texttt{false}^L$. Consequently,
in both cases \TT{y} ends with label $L$ and its value is exactly equal
to the value in \TT{z}. Hence, an adversary can deduce the value of \TT{z}
by observing \TT{y} at the end (which is allowed because \TT{y} ends with
label $L$). So, this program leaks information about \TT{z} despite
correct use of $pc$.

\subsection{No-sensitive-upgrade Check}
\label{sec:bg-nsu}
Preventing leaks due to implicit flow in dynamic IFC requires coarse
approximation because a dynamic monitor only sees program branches
that are executed and does not know what assignments may happen in
alternate branches in other executions. One such coarse approximation
is the \emph{no-sensitive-upgrade} (NSU) check proposed by
Zdancewic~\cite{zdancewic02PhD}. In the program in Listing~\ref{lst1},
\TT{x}'s label is upgraded from $L$ to $H$ at line~\ref{lineref} in one of
the two executions above, but not the other. Subsequently, information
leaks in the other execution (where \TT{x}'s label remains $L$) via the
branch on line~\ref{linerefcond}. The NSU check stops the leak by
preventing the assignment on line~\ref{lineref}. More generally, it
stops a program whenever a public variable's label is upgraded due to
a high $\pc$. This check suffices to provide termination-insensitive
non-interference as shown by Austin and Flanagan~\cite{plas09}. 

\subsection{Permissive-Upgrade and Faceted Execution}
To tackle the issue of permissiveness with the no-sensitive-upgrade
strategy, Austin and Flanagan proposed 
the permissive-upgrade strategy~\cite{plas10} and faceted
execution~\cite{austin12POPL}; faceted execution being the most
permissive of the three. Faceted execution simulates multiple executions
simultaneously within a single runtime. They introduce the concept of
faceted values that are pairs of values for both low and high
observers. With multiple levels, each value in the pair is represented
as a pair. When branching on a faceted value, multiple executions are
simulated for the different values in the facet. However, the runtime 
overheads of faceted execution are quite prohibitive for multiple security 
levels. This thesis, thus, considers only the permissive-upgrade
strategy, which is more permissive than the no-sensitive-upgrade
strategy and much less performance-intensive compared to faceted
execution. Permissive-upgrade is described in detail in
Section~\ref{sec:existing}. 

\subsection{Other Approaches for Permissiveness}
% \paragraph{Secure Multi-Execution}
Secure multi-execution~\cite{SME} is another approach for enforcing
non-interference at runtime. Instead of tracking information flow
through the program, the approach executes multiple copies of the
program with different values of sensitive data. Conceptually, one
executes the same code once for each security level (like low and
high) with a few constraints. The private data in the low execution
are replaced by default values, i.e., the public copy of the
program does not see the actual value of the private data but a
pre-determined default value, and outputs on an $\ell$-labeled channel 
are permitted only in the $\ell$-level execution of the program, i.e.,
the high execution of the program outputs on the high channel, if any,
and the low execution of the program outputs on the low channel,
typically the network. 
%
The modification of the semantics forces that private data and the
outputs resulting from it are visible to only high-level
observers. The public-level observers or the adversary observe
inaccurate results for the outputs that depend on the private value as
the value is replaced by the default value in that execution. 
This modification forces even unsafe programs to
adhere to non-interference. Additionally, secure multi-execution
guarantees precision, i.e., the semantics of a secure program is
not altered. Thus, for a secure program the outputs are all
accurate. Secure multi-execution normally guarantees
termination-insensitive non-interference as the high execution may not
terminate in some cases. Flowfox~\cite{flowfox} demonstrates secure 
multi-execution in the context of web browsers.
%
However, executing a program multiple times can be prohibitive for a
security lattice with multiple levels~\cite{austin12POPL}. The runtime
overhead incurred can be reduced if the executions are run in parallel
(which requires more hardware resources), though the program has 
to be run for all levels irrespective of whether it uses private data
or not. In addition to this, secure multi-execution makes
declassification complicated as it requires synchronization between
different executions~\cite{fineSME}. 

Birgisson \emph{et al.}~\cite{esorics12} describe a testing-based
approach that adds variable upgrade annotations to avoid halting on
the NSU check in an implementation of dynamic IFC for
JavaScript~\cite{csf12}. Hritcu \emph{et al.}  improve permissiveness
by making IFC errors recoverable in the language
Breeze~\cite{Hritcu:ifc}. This is accomplished by a combination of two
methods: making all labels public (by upgrading them when necessary in
a public $\pc$) and by delaying exceptions. A different way of handling the 
problem of implicit flows through flow-sensitive labels is to assign a
(fixed) label to each label; this approach has been examined in recent
work by Buiras \emph{et al.}\ in the context of a language with a
dedicated monad for tracking information flows~\cite{buiras14CSF}. The
precise connection between that approach and permissive-upgrade
remains unclear, although Buiras \emph{et al.}\ sketch a technique
related to permissive-upgrade in their system, while also noting that
generalizing permissive-upgrade to arbitrary lattices is
non-obvious. This thesis confirms the latter and shows how it can be
done.

% \Blindtext
\section{Improved Permissive-Upgrade Strategy}
\label{sec:ipus}

The original permissive-upgrade strategy, however, lacks 
permissiveness; it rejects secure programs like the one shown in
Listing~\ref{lst2}. Consider that \TT{x} is labeled $H$
and \TT{w}, \TT{y} are labeled $L$. With the original permissive-upgrade
strategy, the label of \TT{z} on
line~\ref{lineref2} would remain $P$ and the execution would be
terminated when branching on \TT{z} on line~\ref{lineref3}. With the 
improvement $$H \sqcup P = H$$ the analysis can accept such programs
while remaining sound. With the improvement, \TT{z} would be labeled
$H$ on line~\ref{lineref2}, which would allow the execution to branch
on line~\ref{lineref3}, thus, 
taking the execution to completion. 
The idea behind the improvement is that an $H$-labeled value is
never observable at $L$-level. Similarly, the result of any operation
involving an $H$-labeled value is also never observable at
$L$-level. Thus, any operation involving a partially-leaked value and
a $H$-labeled value does not reveal any information to an adversary at
level $L$ about the partially leaked value. 

\begin{lstlisting}[float,label=lst2,caption=Example showing the
  impermissiveness of the original permissive-upgrade strategy][escapechar=@]
y = false
if (not(x))
  y = true@\label{lineref1}@
z = y || x@\label{lineref2}@
if (not(z))@\label{lineref3}@
  w = true
\end{lstlisting}

The final label $k$ in the assignment rule \refrule{bs:cmd:ap} under
the improved permissive-upgrade strategy becomes: 
$$ k = % \left\lbrace
\begin{cases}
m & \mbox{ if } \pc = L \\
H  & \mbox{ if } \pc = H \mbox{ and } l = H\\
P  & \mbox{ otherwise}
\end{cases} % \right.
$$

The soundness results of the original permissive-upgrade strategy can
be extended to show the soundness of the improved permissive-upgrade
strategy. However, a significant difficulty in proving the theorem
using the modified notation for the imperative language is that the
definition of $\sim$ is not transitive. The same problem arises for
the soundness proofs in~\cite{plas10}. There, the authors resolve the
issue by defining a special relation called evolution. The need for
evolution is averted here using the auxiliary lemmas listed
below. Lemma~\ref{lem:evol} proves the required result substituting
evolution.  

\begin{myLemma}[Expression Evaluation]
\label{lem:expeval}
If $\langle \sigma_1, e \rangle \bsexp \emph{\TT{n}}_1^{k_1}$ and $\langle 
\sigma_2, e \rangle \bsexp \emph{\TT{n}}_2^{k_2}$ and $\sigma_1 \sim_L
\sigma_2$, then $\emph{\TT{n}}_1^{k_1} \sim_L \emph{\TT{n}}_2^{k_2}$.
\end{myLemma}
\begin{proof} By induction on $e$.
\end{proof}

\begin{myLemma}[Evolution]
\label{lem:evol}
  If $~\pc = H$ and $\langle \sigma, c \rangle \bscmd \sigma'$, then
  \\ $\forall \emph{\TT{x}}.\Gamma(\sigma(\emph{\TT{x}})) = P
  \implies \Gamma(\sigma'(\emph{\TT{x}})) = P$. 
\end{myLemma}
\begin{proof} By induction on the derivation rules and case
  analysis on the last rule.
\end{proof}

\begin{myLemma}[Confinement for improved permissive-upgrade with a
  two-point lattice]
\label{lem:conf}
  If $~\pc = H$ and $\langle \sigma, c \rangle
  \bscmd \sigma' $, then $\sigma \sim_L \sigma'$.
\end{myLemma}
\begin{proof} By induction on the derivation rules.
\end{proof}

\begin{myThm}[TINI for improved permissive-upgrade with a two-point lattice]
  With the assignment rule \refrule{bs:cmd:ap} and the modified syntax of
  Figure~\ref{pus:syntax}, if 
  $~\sigma_1 \sim_L \sigma_2$ and $\langle \sigma_1, c \rangle
  \bscmd \sigma_1' $ and $\langle \sigma_2, c
  \rangle\bscmd \sigma_2' $, then $\sigma_1' \sim_L
  \sigma_2'$.
\end{myThm}
\begin{proof} By induction on $c$ and case
  analysis on the last step.
\end{proof}

The detailed proofs are provided in Appendix~\ref{app:ipu}. 
Note that the definitions and proofs presented in this chapter are
specific to the two-point lattice and with respect to an adversary at
level $L$. 


% \section{Generalized Permissive-Upgrade on Arbitrary Lattices}
\label{sec:gen:pus}

\begin{figure}
\begin{equation*}
\begin{aligned}[c]
\lab          =~& L~\arrowvert~M~\arrowvert~\ldots~\arrowvert~H\\
\pc          =~& \lab \\
k,l,m =~& \lab~\arrowvert~\lab\pl 
\end{aligned}
\qquad \qquad 
\begin{aligned}[c]
\lab_1 \sqcup \lab_2\pl  =~&(\lab_1 \sqcup \lab_2)\pl \\
\lab_1\pl  \sqcup \lab_2\pl  =~&(\lab_1 \sqcup \lab_2)\pl
\end{aligned}
\end{equation*}
\caption{Labels and label operations}\label{fig:labels}
\end{figure}

This section shows by construction the generalization of the
permissive-upgrade strategy to arbitrary security lattices. For every
element $\lab$ of the lattice, a new label $\lab\pl$ is introduced
which means ``partially-leaked $\lab$'', with the following intuition: 
\begin{framed}
\noindent
A variable labeled $\lab\pl$ may contain partially-leaked data, where
$\lab$ is a \emph{lower-bound} on the $\star$-free labels the variable
may have in alternate executions.
\end{framed}

The syntax of labels is listed in Figure~\ref{fig:labels}. Labels
$k,l,m$ may be lattice elements $\lab$ or $\star$-ed lattice elements
$\lab\pl$. In examples, suggestive lattice element names $L, M, H$
(low, medium, high) are used. Labels of the form $\lab$ are called 
$\star$-free or \emph{pure}. Figure~\ref{fig:labels} also defines the
join operation $\sqcup$ on labels. This definition is based on the intuition
above. When the two operands of $\odot$ are labeled $\lab_1$ and
$\lab_2\pl$, $\lab_1 \sqcup \lab_2$ is a lower bound on the pure label
of the resulting value in any execution (because $\lab_2$ is a lower
bound on the pure label of $\lab_2\pl$ in any run). Hence, $\lab_1
\sqcup \lab_2\pl = (\lab_1 \sqcup \lab_2)\pl$. The reason for the
definition $\lab_1\pl \sqcup \lab_2\pl = (\lab_1 \sqcup \lab_2)\pl$ is
similar.

\begin{figure}
\begin{mathparpagebreakable}
\inferrule*[left=\mbox{\labelthis{bs:cmd:agn}{assn-n}}]
{  \langle \sigma, \expr \rangle \bsexp \TT{n}^m \\ l =
  \Gamma(\sigma(\TT{x})) \\ l = \lab_\TT{x} \vee l = \lab_\TT{x}\pl \\ \pc 
  \sqsubseteq \lab_\TT{x} \\ k = \pc \sqcup m}
{\langle \sigma , \TT{x} := \expr \rangle \bscmd \sigma[\TT{x}
  \mapsto \TT{n}^{k}]}
%%%    
\and
\inferrule*[left=\mbox{\labelthis{bs:cmd:ags}{assn-s}}]
{\langle \sigma, \expr \rangle \bsexp \TT{n}^m \\ l =
  \Gamma(\sigma(\TT{x})) \\ l = \lab_\TT{x} \vee  l = \lab_\TT{x}\pl \\ \pc
  \not\sqsubseteq \lab_\TT{x} \\ k = ((\pc \sqcup m)\sqcap \lab_\TT{x} )\pl}
{\langle \sigma , \TT{x} := \expr \rangle \bscmd \sigma[\TT{x} \mapsto
  \TT{n}^{k}]}
\end{mathparpagebreakable}
\caption[Caption]{Assignment rules for the generalized
  permissive-upgrade}\label{fig:assn-our}
\end{figure}

The rules for assignment are shown in Figure~\ref{fig:assn-our}. They
strictly generalize the rule \refrule{bs:cmd:ap} for the two-point lattice,
treating $P = L\pl$. Rule~\refrule{bs:cmd:agn} applies when the existing label of
the variable being assigned to is $\lab_{\TT{x}}$ or $\lab_{\TT{x}}\pl$ and $\pc
\sqsubseteq \lab_{\TT{x}}$. The key intuition behind the rule is the
following: If $\pc \sqsubseteq \lab_{\TT{x}}$, then it is safe to overwrite
the variable, because $\lab_{\TT{x}}$ is necessarily a lower bound on the
(pure) label of $\TT{x}$ in this and any alternate execution (see the
\framebox{framebox} above). Hence, overwriting the variable cannot
cause an implicit flow. As expected, the label of the overwritten
variable is $\pc \sqcup m$, where $m$ is the label of the value
assigned to $\TT{x}$.

Rule \refrule{bs:cmd:ags} applies in the remaining case --- when $\pc
\not\sqsubseteq \lab_{\TT{x}}$. In this case, there may be an implicit flow,
so the final label on $\TT{x}$ must have the form $\lab\pl$ for some
$\lab$. The question is which $\lab$. Intuitively, it may seem that
one could choose $\lab = \lab_{\TT{x}}$, the pure part of the original label
of $\TT{x}$. The final label on $\TT{x}$ would be $\lab_{\TT{x}}\pl$ and this would
satisfy the intuitive meaning of $\star$ written in the
\framebox{framebox} above. Indeed, this intuition suffices for the
two-point lattice of Section~\ref{sec:existing}
and~\ref{sec:ipus}. However, for a more 
general lattice, this intuition is unsound, as illustrated with an
example below. The correct label is $((\pc \sqcup m) \sqcap
\lab_{\TT{x}})\pl$. 
% (Note that this correct label is independent of the label $m$ of the value
% assigned to $\TT{x}$. This is sound because $\TT{x}$ is $\star$-ed and
% cannot be case-analyzed later, so the label on the value in it is irrelevant.)

\begin{figure}[ht]
\begin{minipage}{0.65\linewidth}
\centering
\begin{lstlisting}[caption=Example explaining rule \refrule{bs:cmd:ags},label=list2]
if ($\TT{x}'$)
  $\TT{z} = \TT{y}_1$@\label{hm1}@
else
  $\TT{z} = \TT{y}_2$@\label{hm2}@
if ($\TT{x}_1$)@\label{bl1}@
  $\TT{z} = \TT{x}_1$@\label{hl1}@
if ($\texttt{not}(\TT{x}_2)$)@\label{if2}@
  $\TT{z} = \TT{x}_2$@\label{hl2}@
if ($\TT{z}$)@\label{if3}@
  $\TT{w} = \TT{z}$@\label{hl3}@
\end{lstlisting}
\end{minipage}
\hspace{-1.5cm}
\begin{minipage}{0.45\linewidth}
\centering
{\includegraphics{chapters/gpu/lattice.tikz}}
\caption{Lattice explaining rule \refrule{bs:cmd:ags}}
\label{fig:lattice}
\end{minipage}
\end{figure}

\paragraph{Example} 
The need for the label $k := ((\pc \sqcup m) \sqcap \lab_{\TT{x}})\pl$
instead of $k := \lab_{\TT{x}}\pl$ in rule \refrule{bs:cmd:ags} is illustrated
below. Consider the lattice of Figure~\ref{fig:lattice} and the
program of Listing~\ref{list2}. Assume 
that, initially, the variables $\TT{z}$, $\TT{w}$, $\TT{x}_1$,
$\TT{x}'$, $\TT{x}_2$, $\TT{y}_1$ and $\TT{y}_2$ have labels $H$,
$L_1$, $L_1$, $L'$, $L_2$, $M_1$ and $M_2$, 
respectively. Fix the attacker at level $L_1$. Fix the value of $\TT{x}_1$
at $\texttt{true}^{L_1}$, so that the branch on line~\ref{bl1} is
always taken and line~\ref{hl1} is always executed. Set $\TT{y}_1 \mapsto
\texttt{false}^{M_1}, \TT{y}_2 \mapsto \texttt{true}^{M_2}, \TT{w} \mapsto
\TT{false}^{L_1}$ initially. The initial value of $\TT{z}$ is
irrelevant. Consider two executions of the program starting from two
stores $\sigma_1$ with $\TT{x}' \mapsto \texttt{true}^{L'}, \TT{x}_2 \mapsto
\texttt{true}^{L_2}$ and $\sigma_2$ with $\TT{x}' \mapsto
\texttt{false}^{L'}, \TT{x}_2 \mapsto \texttt{false}^{L_2}$. Note that
as $L'$ and $L_2$ are incomparable to $L_1$ in the lattice,
$\sigma_1$ and $\sigma_2$ are equivalent for $L_1$. 

\begin{table*}
\centering
\begin{tabular} {|l||@{\,}c@{\,}||@{\,}c@{\,}|@{\,}c@{\,}|}
\hline
&
\multicolumn{3}{c|}{$\TT{w} = \texttt{false}^{L_1},\ \TT{x}_1 =
  \texttt{true}^{L_1},\ \TT{y}_1 = \texttt{false}^{M_1},\ \TT{y}_2 =
  \texttt{true}^{M_2}$}
\\
\cline{2-4}
&
$\TT{x}' = \texttt{true}^{L'}$
&
\multicolumn{2}{c|}{$\TT{x}' = \texttt{false}^{L'}$}
\\
&
$\TT{x}_2 = \texttt{true}^{L_2}$
&
\multicolumn{2}{c|}{$\TT{x}_2 = \texttt{false}^{L_2}$}
\\
\cline{3-4}
&
&
$k:= \lab_{\TT{x}}\pl$
&
$k:= ((\pc \sqcup m) \sqcap \lab_{\TT{x}})\pl$
\\
\hline
\texttt{if} ($\TT{x}'$)&$\pc = L'$&&\\
\quad$\TT{z} = \TT{y}_1$&$\TT{z} = \texttt{false}^{M_1}$&&\\
\texttt{else}&&$\pc = L'$&$\pc = L'$\\
\quad$\TT{z} = \TT{y}_2$&&$\TT{z} = \texttt{true}^{M_2}$&$\TT{z} = \texttt{true}^{M_2}$\\
\texttt{if} ($\TT{x}_1$)&$\pc = L_1$&$\pc = L_1$&$\pc = L_1$\\
\quad$\TT{z} = \TT{x}_1$&$\TT{z} = \texttt{true}^{L_1}$&$\TT{z} = \texttt{true}^{M_2\pl}$&$\TT{z} = \texttt{true}^{L\pl}$\\
\texttt{if} ($\texttt{not}(\TT{x}_2)$)&branch not taken&$\pc = L_2$&$\pc = L_2$\\
\quad$\TT{z} = \TT{x}_2$& %$z = \texttt{true}^{L_1}$
&$\TT{z} = \texttt{false}^{L_2}$&$\TT{z} = \texttt{false}^{L\pl}$\\
\texttt{if} ($\TT{z}$)&$\pc = L_1$&branch not taken&execution halted\\
\quad$\TT{w} = \TT{z}$&$\TT{w} = \texttt{true}^{L_1}$&&\\
\hline
Result & $\TT{w} = \texttt{true}^{L_1}$ & $\TT{w} =
\texttt{false}^{L_1}$ (leak) & no leak\\
\hline
\end{tabular}
\caption{Execution steps in two runs of the program from Listing~\ref{list2}, with two variants of the rule \refrule{bs:cmd:ags}}
\label{tblassn}
\end{table*}

Requiring $k := \lab_{\TT{x}}\pl$ in rule \refrule{bs:cmd:ags} causes an
implicit flow that is observable for $L_1$. The intermediate values
and labels of the variables for executions starting from $\sigma_1$
and $\sigma_2$ are shown in the second and third columns of
Table~\ref{tblassn}. Starting with $\sigma_1$, line~\ref{hm1} is
executed, but line~\ref{hm2} is not, so $\TT{z}$ ends with
$\texttt{false}^{M_1}$ at line~\ref{bl1} (rule \refrule{bs:cmd:agn} applies at
line~\ref{hm1}). At line~\ref{hl1}, $\TT{z}$ contains $\texttt{true}^{L_1}$
(again by rule \refrule{bs:cmd:agn}) and line~\ref{hl2} is not executed. Thus, the
branch on line~\ref{if3} is taken and $\TT{w}$ ends with
$\texttt{true}^{L_1}$ at line~\ref{hl3}. Starting with $\sigma_2$,
line~\ref{hm1} is not executed, but line~\ref{hm2} is, so $\TT{z}$ becomes
$\texttt{true}^{M_2}$ at line~\ref{bl1} (rule \refrule{bs:cmd:agn} applies at
line~\ref{hm2}). At line~\ref{hl1}, rule \refrule{bs:cmd:ags} applies, but because
$k := \lab_{\TT{x}}\pl$ is assumed in that rule, $\TT{z}$ now contains the
value $\texttt{true}^{M_2\pl}$. As the branch on line~\ref{if2} is
taken, at line~\ref{hl2}, $\TT{z}$ becomes $\texttt{false}^{L_2}$ by rule
\refrule{bs:cmd:agn} because $L_2 \sqsubseteq M_2$. Thus, the branch on
line~\ref{if3} is not taken and $\TT{w}$ ends with $\texttt{false}^{L_1}$
in this execution. Hence, $\TT{w}$ ends with $\texttt{true}^{L_1}$ and
$\texttt{false}^{L_1}$ in the two executions, respectively. The
attacker at level $L_1$ can distinguish these two results; hence, the
program leaks the value of $\TT{x}'$ and $\TT{x}_2$ to $L_1$.

With the correct \refrule{bs:cmd:ags} rule in place, this leak is avoided (last
column of Table~\ref{tblassn}). In that case, after the assignment on
line~\ref{hl1} in the second execution, $\TT{z}$ has label $((L_1 \sqcup L_1) \sqcap
M_2)\pl = L\pl$. Subsequently, after line~\ref{hl2}, $\TT{z}$ gets the
label $L\pl$. As case analysis on a $\star$-ed value is not allowed,
the execution is halted on line~\ref{if3}. This guarantees
termination-insensitive non-interference with respect to the attacker
at level $L_1$.

\subsection{Termination-Insensitive Non-interference}

To prove non-interference for the generalized permissive-upgrade,
equivalence of labeled values relative to an adversary at arbitrary
lattice level $\lab$ needs to be defined. The definition is shown
below (Definition~\ref{def:gpua:veq}). Note that clauses (3)--(5) here
refine clause (3) of Definition~\ref{def:eq-existing} for the two-point
lattice. The obvious generalization of clause (3) of
Definition~\ref{def:eq-existing} --- $\TT{n}_1^k \sim_\lab \TT{n}_2^m$ whenever 
either $k$ or $m$ is $\star$-ed --- is too coarse to prove
non-interference inductively. For the degenerate case of the two-point
lattice, this definition also degenerates to
Definition~\ref{def:eq-existing} (there, $\lab$ is fixed at $L$, $P = 
L\pl$ and only $L$ may be $\star$-ed).

\begin{mydef}
\label{def:gpua:veq}
Two values $\emph{\TT{n}}_1^k$ and $\emph{\TT{n}}_2^m$ are
$\lab$-equivalent, written $\emph{\TT{n}}_1^k \sim_\lab
\emph{\TT{n}}_2^m$, iff either 
\begin{enumerate}
\item $k = m = \lab' \sqsubseteq \lab$ and $\emph{\TT{n}}_1 =
  \emph{\TT{n}}_2$, or 
\item $ k = \lab'
  \not\sqsubseteq \lab$ and $m = \lab'' \not\sqsubseteq \lab$, or 
\item $k = \lab_1\pl$ and $m = \lab_2\pl$, or
\item $k = \lab_1\pl$ and $m = \lab_2$ and ($\lab_2 \not\sqsubseteq
  \lab$ or $\lab_1 \sqsubseteq \lab_2 $), or
\item $k = \lab_1$ and $m = \lab_2\pl$ and ($\lab_1 \not\sqsubseteq
  \lab$ or $\lab_2 \sqsubseteq \lab_1$)
\end{enumerate}
\end{mydef}

\begin{mydef}
\label{def:gpua:seq}
  Two stores $\sigma_1$ and $\sigma_2$ are $\lab$-equivalent,
  written $\sigma_1 \sim_\lab \sigma_2$, iff for every variable \emph{\TT{x}},
  $\sigma_1(\emph{\TT{x}}) \sim_\lab \sigma_2(\emph{\TT{x}})$.
\end{mydef}

This definition is obtained by constructing (through examples) an
extensive transition graph of pairs of labels that may be assigned to
a single variable at corresponding program points in two executions of
the same program. The starting point is label-pairs of the form
$(\lab, \lab)$. This characterization of equivalence is both
sufficient and necessary. It is sufficient in the sense that it allows
us to prove TINI inductively. It is necessary in the sense that
example programs can be constructed that end in states exercising
every possible clause of this definition. Appendix~\ref{app:egequi}
lists these examples. 

% \section{Termination-Insensitive Non-Interference}

Using the above definition of equivalence of labeled values, TINI can
be proven for the generalized permissive-upgrade strategy presented
above. A significant difficulty in proving the theorem is that the
definition of $\sim_\lab$ is not transitive unlike the previous definition
of $\sim$. 
% The same problem arises
% for the two-point lattice as shown in the previous section and
% in~\cite{plas10}. There, the authors resolve the 
% issue by defining a special relation called evolution. 
% Here, a more conventional approach based on the standard confinement 
% lemma is taken. 
% The need for evolution is averted using several auxiliary
% lemmas that we list below.
Detailed proofs of all the lemmas and the theorems
are presented in Appendix~\ref{app:gpu}.

\begin{myLemma}[Expression evaluation]
\label{lem:gpua:exp}
If $\langle \sigma_1, e \rangle \bsexp \emph{\TT{n}}_1^{k_1}$ and $\langle 
\sigma_2, e \rangle \bsexp \emph{\TT{n}}_2^{k_2}$ and $\sigma_1 \sim_\lab \sigma_2$,
then $\emph{\TT{n}}_1^{k_1} \sim_\lab \emph{\TT{n}}_2^{k_2}$.
\end{myLemma}
\begin{proof}
By induction on $e$.
\end{proof}

%% Lemma~\ref{sup1} shows that a value labeled $\lab\pl$ remains
%% $\star$-ed if an evaluation is done on it under a higher or unrelated
%% context. Lemma~\ref{pcl} on the other hand shows that if a value gets
%% a pure label after an evaluation, then either the value remained
%% unchanged after the evaluation or the evaluation done was done in a
%% lower context. We state two important corollaries,
%% Corollary~\ref{cor1} and~\ref{cor2} derived from Lemma~\ref{sup1}
%% and~\ref{pcl}, respectively, which are used in the proofs for
%% confinement and non-interference.

\begin{myLemma}[$\star$-preservation]
\label{lem:gpua:sup1}
If $\langle \sigma, c \rangle \bscmd \sigma'$ and
$\Gamma(\sigma(x)) = \lab\pl $ and $\pc \not\sqsubseteq \lab$, then
$\Gamma(\sigma'(x)) = \lab'\pl$ and $\lab' \sqsubseteq \lab$.
\end{myLemma}
\begin{proof}
By induction on the derivation rule.
\end{proof}

\begin{mycor}
\label{cor:gpua:cor1}
If $\langle \sigma, c \rangle \bscmd \sigma'$ and
$\Gamma(\sigma(\emph{\TT{x}})) = \lab\pl $ and
$\Gamma(\sigma'(\emph{\TT{x}})) = \lab'$, then 
$\pc \sqsubseteq \lab$.
\end{mycor}
\begin{proof}
Immediate from Lemma~\ref{lem:gpua:sup1}.
\end{proof}

\begin{myLemma}[$\pc$-lemma]
\label{lem:gpua:pcl}
If $\langle \sigma, c \rangle \bscmd \sigma'$ and
$\Gamma(\sigma'(\emph{\TT{x}})) = \lab$, then $\sigma(\emph{\TT{x}}) =
\sigma'(\emph{\TT{x}})$ or 
$\pc \sqsubseteq \lab$.
\end{myLemma}
\begin{proof}
By induction on the derivation rule.
\end{proof}

\begin{mycor}
\label{cor:gpua:cor2}
If $\langle \sigma, c \rangle \bscmd \sigma'$ and
$\Gamma(\sigma(\emph{\TT{x}})) = \lab\pl $ and
$\Gamma(\sigma'(\emph{\TT{x}})) = \lab'$, then 
$\pc \sqsubseteq \lab'$.
\end{mycor}
\begin{proof}
Immediate from Lemma~\ref{lem:gpua:pcl}.
\end{proof}

Using these lemmas, the standard confinement lemma and non-interference
can be proven. 

\begin{myLemma}[Confinement Lemma]
\label{lem:gpua:conf}
If $\pc \not\sqsubseteq \lab$ and $\langle \sigma, c \rangle
\bscmd \sigma'$, then $\sigma \sim_\lab \sigma'$.
\end{myLemma}
\begin{proof}
By induction on the derivation rule.
\end{proof}

\begin{myThm}[TINI for generalized permissive-upgrade for arbitrary lattices]
  If $~\sigma_1 \sim_\lab \sigma_2$ and $\langle \sigma_1, c \rangle
  \bscmd \sigma_1' $ and $\langle \sigma_2, c
  \rangle\bscmd \sigma_2' $, then $\sigma_1' \sim_\lab
  \sigma_2'$.
\end{myThm}
\begin{proof} By induction on $c$.
\end{proof}


%%%%%%%%%%%%%%%%%%%%%%%%%%%%%%%%%%%%%%%%%%%%%%%%%%%%%%%%

%%%%%%%%%%%%%%%%%%%%%%%%%%%%%%%%%%%%%%%%%%%%%%%%%%%%%%%%

\section{Comparison of the Generalization of Section~\ref{sec:gen:pus}
  with the Generalization of Section~\ref{sec:gen:ipus}}

%\subsection{Permissiveness of the Approach in Section~\ref{sec:gen:pus}}

Two distinct and sound generalizations of the permissive-upgrade
strategy for the two-point lattice have now been described:
The generalization of the improved permissive-upgrade to pointwise
products of two-point lattices or, equivalently, to powerset lattices
as described in Section~\ref{sec:gen:ipus}, and the generalization to
arbitrary lattices described in Section~\ref{sec:gen:pus}. For brevity,
these generalizations are called puP (Section~\ref{sec:gen:ipus}) and
puA (Section~\ref{sec:gen:pus}), 
respectively (P and A stand for \underline{p}owerset and
\underline{a}rbitrary, respectively). Since both puP and puA apply to
powerset lattices, an obvious question is whether one is more
permissive than the other on such lattices. The generalization of puP
presented in this chapter can be more permissive than puA on powerset 
lattices in certain cases as shown by the example
below. The reason for this permissiveness is that puP tracks finer
taints, i.e., it tracks partial leaks for each principal separately. 

% We show here that the
% permissiveness of puP and puA on powerset lattices is
% \emph{incomparable} --- there are examples on which each
% generalization is more permissive than the other. We explain one
% example in each direction below. Roughly, incomparability exists
% because puP tracks finer taints (it tracks partial leaks for each
% principal separately), but puA's rules for overwriting
% partially-leaked variables are more permissive.

\begin{figure}[ht]
\begin{minipage}{0.65\linewidth}
\centering
\begin{lstlisting}[caption=Example where puP is more permissive than puA,label=list3]
if (y) @\label{li32}@
  x = z @\label{li33}@
if (z) @\label{li34}@
  x = z @\label{li35}@
if (x)  @\label{li36}@
  z = x
\end{lstlisting}
\end{minipage}
\hspace{-1.5cm}
\begin{minipage}{0.45\linewidth}
\centering
{\includegraphics{chapters/gpu/powerset.tikz}}
\caption{A powerset/product lattice}\label{fig:lattice1}
\end{minipage}
\end{figure}

\paragraph{Example}
The powerset lattice of Figure~\ref{fig:lattice1} is used for
illustration purpose. This lattice is the pointwise lifting of the order $L
\sqsubset H$ to the set $S = \{L, H\} \times \{L, H\}$. For brevity,
this lattice's elements are written as $LL$, $LH$, etc. When puP is
applied to this lattice, labels are drawn from the set $\{L, H, P\}
\times \{L, H, P\}$. These labels are concisely written as $LP$, $HL$,
etc. For puA, labels are drawn from the set $S \cup S\pl$. These
labels are written $LH$, $LH\pl$, etc. Note that $LH\pl$ parses as
$(LH)\pl$, not $L(H\pl)$ (the latter is not a valid label in puA
applied to this lattice).
%
Consider the program in Listing~\ref{list3}. Assume that $\TT{x}$,
$\TT{y}$ and $\TT{z}$ have 
initial labels $LL$, $HL$ and $LH$, respectively and that the initial
store contains $\TT{y} \mapsto \texttt{true}^{HL}, \TT{z} \mapsto
\texttt{true}^{LH}$, so the branches on lines~\ref{li32}
and~\ref{li34} are both taken. The initial value in $\TT{x}$ is irrelevant
but its label is important.
Under puA, $\TT{x}$ obtains label $(((HL) \sqcup (LH)) \sqcap (LL))\pl = LL\pl$ at
line~\ref{li33} by rule \refrule{bs:cmd:ags}. At line~\ref{li35}, the same rule
applies but the label of $\TT{x}$ remains $LL\pl$. When the program tries
to branch on $\TT{x}$ at line~\ref{li36}, it is stopped.
In contrast, under puP, this program executes to completion. At
line~\ref{li33}, the label of $\TT{x}$ changes to $PH$ by rule
\refrule{bs:cmd:agp}. At line~\ref{li35}, the label changes to $LH$ because $pc$
and the label of $\TT{z}$ are both $LH$. Since this new label has no $P$,
line~\ref{li36} executes without halting.
Hence, for this example, puP is more permissive than puA.

% \section{Remarks}
% On powerset lattices, the resulting IFC monitor is different
% in the two cases and the generalization of improved permissive-upgrade
% is more permissive than puA in certain cases. 
% The approach presented in this section is, thus, more general
% and applies to a broader set of lattices. 
An example for which  puA is more permissive than puP in the case of
poweset lattices was not found. But the generalization puA presented in
Section~\ref{sec:gen:pus} is more general than the product
construction puP (Section~\ref{sec:gen:ipus}) when applied to arbitrary
lattices (and hence, applicable to a broader set of lattices) as it is
unclear whether or how  the improved permissive 
upgrade strategy generalises to arbitrary lattices. By developing this
generalization, this work makes permissive-upgrade applicable to
arbitrary security lattices like other IFC techniques.
% and, hence,
% constitutes a useful contribution to IFC literature.



% \Blindtext