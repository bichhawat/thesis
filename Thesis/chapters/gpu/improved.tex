\section{Improved Permissive-Upgrade Strategy}
\label{sec:ipus}

The original permissive-upgrade strategy, however, lacks 
permissiveness; it rejects secure programs like the one shown in
Listing~\ref{lst2}. Consider that \TT{x} is labeled $H$
and \TT{w}, \TT{y} are labeled $L$. With the original permissive-upgrade
strategy, the label of \TT{z} on
line~\ref{lineref2} would remain $P$ and the execution would be
terminated when branching on \TT{z} on line~\ref{lineref3}. With the 
improvement $$H \sqcup P = H$$ the analysis can accept such programs
while remaining sound. With the improvement, \TT{z} would be labeled
$H$ on line~\ref{lineref2}, which would allow the execution to branch
on line~\ref{lineref3}, thus, 
taking the execution to completion. 
The idea behind the improvement is that an $H$-labeled value is
never observable at $L$-level. Similarly, the result of any operation
involving an $H$-labeled value is also never observable at
$L$-level. Thus, any operation involving a partially-leaked value and
a $H$-labeled value does not reveal any information to an adversary at
level $L$ about the partially leaked value. 

\begin{lstlisting}[float,label=lst2,caption=Example showing the
  impermissiveness of the original permissive-upgrade strategy][escapechar=@]
y = false
if (not(x))
  y = true@\label{lineref1}@
z = y || x@\label{lineref2}@
if (not(z))@\label{lineref3}@
  w = true
\end{lstlisting}

The final label $k$ in the assignment rule \refrule{bs:cmd:ap} under
the improved permissive-upgrade strategy becomes: 
$$ k = % \left\lbrace
\begin{cases}
m & \mbox{ if } \pc = L \\
H  & \mbox{ if } \pc = H \mbox{ and } l = H\\
P  & \mbox{ otherwise}
\end{cases} % \right.
$$

The soundness results of the original permissive-upgrade strategy can
be extended to show the soundness of the improved permissive-upgrade
strategy. However, a significant difficulty in proving the theorem
using the modified notation for the imperative language is that the
definition of $\sim$ is not transitive. The same problem arises for
the soundness proofs in~\cite{plas10}. There, the authors resolve the
issue by defining a special relation called evolution. The need for
evolution is averted here using the auxiliary lemmas listed
below. Lemma~\ref{lem:evol} proves the required result substituting
evolution.  

\begin{myLemma}[Expression Evaluation]
\label{lem:expeval}
If $\langle \sigma_1, e \rangle \bsexp \emph{\TT{n}}_1^{k_1}$ and $\langle 
\sigma_2, e \rangle \bsexp \emph{\TT{n}}_2^{k_2}$ and $\sigma_1 \sim_L
\sigma_2$, then $\emph{\TT{n}}_1^{k_1} \sim_L \emph{\TT{n}}_2^{k_2}$.
\end{myLemma}
\begin{proof} By induction on $e$.
\end{proof}

\begin{myLemma}[Evolution]
\label{lem:evol}
  If $~\pc = H$ and $\langle \sigma, c \rangle \bscmd \sigma'$, then
  \\ $\forall \emph{\TT{x}}.\Gamma(\sigma(\emph{\TT{x}})) = P
  \implies \Gamma(\sigma'(\emph{\TT{x}})) = P$. 
\end{myLemma}
\begin{proof} By induction on the derivation rules and case
  analysis on the last rule.
\end{proof}

\begin{myLemma}[Confinement for improved permissive-upgrade with a
  two-point lattice]
\label{lem:conf}
  If $~\pc = H$ and $\langle \sigma, c \rangle
  \bscmd \sigma' $, then $\sigma \sim_L \sigma'$.
\end{myLemma}
\begin{proof} By induction on the derivation rules.
\end{proof}

\begin{myThm}[TINI for improved permissive-upgrade with a two-point lattice]
  With the assignment rule \refrule{bs:cmd:ap} and the modified syntax of
  Figure~\ref{pus:syntax}, if 
  $~\sigma_1 \sim_L \sigma_2$ and $\langle \sigma_1, c \rangle
  \bscmd \sigma_1' $ and $\langle \sigma_2, c
  \rangle\bscmd \sigma_2' $, then $\sigma_1' \sim_L
  \sigma_2'$.
\end{myThm}
\begin{proof} By induction on $c$ and case
  analysis on the last step.
\end{proof}

The detailed proofs are provided in Appendix~\ref{app:ipu}. 
Note that the definitions and proofs presented in this chapter are
specific to the two-point lattice and with respect to an adversary at
level $L$. 

