\section{Overview}
This section presents a \emph{formal} description of the
no-sensitive-upgrade check. The technical development in this thesis
is mostly based on the simple imperative language shown in
Figure~\ref{basic:syntax}. However, the key ideas are orthogonal to
the choice of language and generalize to 
other languages easily. The use of a simpler language is to simplify
non-essential technical details. The parts in this thesis that require
a more complex language define the additional features in place. 
The language's expressions include constants or values (\TT{n}, \TT{b}),
variables ($\TT{x}$) and unspecified binary operators ($\odot$) to 
combine them. The set of variables is fixed upfront. Labels ($\lab$)
are drawn from a fixed security lattice. The lattice contains
different labels $\{L, M, H, \ldots \}$ with a partial ordering
between the elements. Join ($\sqcup$) and meet ($\sqcap$) operations
are defined as usual on the lattice. The program counter label $\pc$
is an element of the lattice.

\subsection{Basic IFC Semantics}
%%%%%%%%%%%%%%%%%%%%%%%%%%%%%%%%%%%%
\begin{figure}
\begin{align*}
\expr	=~& \TT{n}~\arrowvert~\TT{x}~\arrowvert~\expr_1 \odot \expr_2 \\
\comm	=~& \sk~\arrowvert~\TT{x} := \expr~\arrowvert 
 ~\comm_1;\comm_2~\arrowvert~\texttt{if}~\expr~\texttt{then}~\comm_1~\texttt{else}~\comm_2~\arrowvert~\texttt{while}~\expr~\texttt{do}~\comm\\
\lab          =~& L~\arrowvert~M~\arrowvert~H~\arrowvert~\ldots\\
k,l,m, \pc =~& \lab
\end{align*}
\caption{Syntax of the Language}\label{basic:syntax}
\end{figure}

%%%%%%%%%%%%%%%%%%%%%%%%%%%%%%%%%%%%

%%%%%%%%%%%%%%%%%%%%%%%%%%%%%%%%%%%%
\begin{figure*}
\begin{framed}
\begin{mathparpagebreakable}
%\onecolumn
%%%    
\textsc{Expressions:}\\
\inferrule*[left=\mbox{\labelthis{bs:exp:c}{const}}]
{ }
{\langle \sigma, \TT{n} \rangle \bsexp \TT{n}^{\perp}}
\and
\inferrule*[left=\mbox{\labelthis{bs:exp:v}{var}}]
{\TT{n}^k := \sigma(\TT{x})}
{\langle \sigma, \TT{x} \rangle \bsexp \TT{n}^k}
\and
\inferrule*[left=\mbox{\labelthis{bs:exp:o}{oper}}]
{\langle \sigma, \expr' \rangle \bsexp
  \TT{n}'^{k'} \\ \langle \sigma, \expr'' \rangle \bsexp
  \TT{n}''^{k''} \\\\ \TT{n} := \TT{n}' \odot \TT{n}'' \\ 
  k := k' \sqcup k''} 
{\langle \sigma, \expr' \odot \expr'' \rangle \bsexp \TT{n}^k}
%%%    
\\\\
\textsc{Statements:}\\
\inferrule*[left=\mbox{\labelthis{bs:cmd:sk}{skip}}]
{ }
{\langle \sigma, \sk \rangle \bscmd \sigma}
\and
%%%
\inferrule*[left=\mbox{\labelthis{bs:cmd:s}{seq}}]
{\langle \sigma, \comm_1 \rangle \bscmd \sigma'' \\ \langle \sigma'', \comm_2 \rangle \bscmd \sigma' }
{\langle \sigma, \comm_1;\comm_2 \rangle \bscmd \sigma'}
\and
%%%
\inferrule*[left=\mbox{\labelthis{bs:cmd:wf}{while-f}}]
{\langle \sigma, \expr \rangle \bsexp \texttt{false}^{\lab}}
{\langle \sigma, \texttt{while}~\expr~\texttt{do}~\comm \rangle
  \bscmd \sigma }
\and
%%%    
\inferrule*[left=\mbox{\labelthis{bs:cmd:ie}{if-else}}]
{\langle \sigma, \expr \rangle \bsexp \TT{b}^{\lab} \\ 
  i = \left\{\begin{array}{ll}
        1, & \textit{if }~  \TT{b} = \TT{true} \arcr
        2, & \textit{otherwise } \arcr
        \end{array}\right\} \\\\ \langle
  \sigma, \comm_i \rangle \bscmdup \sigma'}
{\langle \sigma, 
  \texttt{if}~\expr~\texttt{then}~\comm_1~\texttt{else}~\comm_2
  \rangle \bscmd \sigma'}
\and
% \inferrule*[left=\mbox{\labelthis{bs:cmd:iet}{if-else-t}}]
% {\langle \sigma, \expr \rangle \bsexp \texttt{true}^{\lab} \\ \langle
%   \sigma, \comm_1 \rangle \bscmdup \sigma'}
% {\langle \sigma, 
%   \texttt{if}~\expr~\texttt{then}~\comm_1~\texttt{else}~\comm_2
%   \rangle \bscmd \sigma'}
% \and
% %%%    
% \inferrule*[left=\mbox{\labelthis{bs:cmd:ief}{if-else-f}}]
% {\langle \sigma, \expr \rangle \bsexp \texttt{false}^{\lab} \\ \langle
%   \sigma, \comm_2 \rangle \bscmdup \sigma'}
% {\langle \sigma, 
%   \texttt{if}~\expr~\texttt{then}~\comm_1~\texttt{else}~\comm_2
%   \rangle \bscmd \sigma'}
% \and
%%%    
\inferrule*[left=\mbox{\labelthis{bs:cmd:wt}{while-t}}]
{\langle \sigma, \expr \rangle \bsexp \texttt{true}^{\lab} \\ \langle
  \sigma, \comm \rangle \bscmdup \sigma'' \\\\ \langle
  \sigma'', \texttt{while}~\expr~\texttt{do}~\comm \rangle
  \bscmdup \sigma' }
{\langle \sigma, \texttt{while}~\expr~\texttt{do}~\comm \rangle
  \bscmd \sigma'}
%%%    
\end{mathparpagebreakable}
\end{framed}
\caption{Semantics}\label{fig:basic-semantics}
\end{figure*}

The rules in Figure~\ref{fig:basic-semantics} define the big-step
semantics of the language, including standard taint propagation for
IFC: the evaluation relation $\langle \sigma, \expr \rangle \bsexp
\TT{n}^k$ for expressions, and the evaluation relation $\langle \sigma,
\comm \rangle \bscmd \sigma'$ for commands. Here, $\sigma$ 
denotes a store, a map from variables to labeled values of the form
$\TT{n}^k$. \TT{b} represents a Boolean constant. For now, labels $k ::=
\lab$; this is generalized later when the ``partially-leaked'' taints
are introduced in Section~\ref{sec:existing}. 

The evaluation relation for expressions evaluates an expression
$\expr$ and returns its value $n$ and label $k$. The label $k$ is the
join of labels of all variables occurring in $\expr$ (according to
$\sigma$). The relation for commands executes a command $\comm$ in the
context of a store $\sigma$, and the current program counter label
$\pc$, and yields a new store $\sigma'$.  The function
$\Gamma(\sigma(\TT{x}))$ returns the label associated with the value in \TT{x}
in store $\sigma$: If $\sigma(\TT{x}) = \TT{n}^k$, then $\Gamma(\sigma(\TT{x})) =
k$. $\bot$ denotes the least element of the lattice. % Here, $\bot = L$.

% The rules for evaluating commands are explained below. 
The sequencing rule (\refrule{bs:cmd:s}) evaluates the
command $c_1$ under store $\sigma$ and the current $\pc$ label; this
yields a new store $\sigma''$. It then evaluates the command $c_2$
under store $\sigma''$ and the same $\pc$ label, which yields the
final store $\sigma'$. 
%
The \refrule{bs:cmd:ie} rule 
evaluates the branch condition $e$ to a Boolean value \TT{b} with
label $\lab$. Based on the value of \TT{b}, one of the branches
$\comm_1$ and $\comm_2$ is executed under a $\pc$ obtained by 
joining the current $\pc$ and the label $\lab$ of \TT{b}. Similarly, the
rules for \texttt{while} (\refrule{bs:cmd:wt} and \refrule{bs:cmd:wf})
evaluate the loop condition $e$ and execute the loop command $c_1$
while $e$ evaluates to \texttt{true}. The $\pc$ for the loop is
obtained by joining the current $\pc$ and the label 
$\lab$ of the result of evaluating $e$.

%%%%%%%%%%%%%%%%%%%%%%%%%%%%%%%%%%%%
\begin{figure*}
\begin{mathparpagebreakable}
\inferrule*[left=\mbox{\labelthis{bs:cmd:an}{assn-nsu}}]
 {l = \Gamma(\sigma(\TT{x})) \\ \pc
    \sqsubseteq l \\ \langle \sigma, \expr \rangle \bsexp \TT{n}^m
    } {\langle \sigma, \TT{x} := \expr \rangle
    \bscmd \sigma[\TT{x} \mapsto \TT{n}^{(\pc \sqcup m)}]}
\end{mathparpagebreakable}
\caption{Assignment rule for NSU}
\label{fig:assn-nsu}
\end{figure*}
%%%%%%%%%%%%%%%%%%%%%%%%%%%%%%%%%%%%

The rule for assignment statements are conspicuously missing from
Figure~\ref{fig:basic-semantics} because they depend on the strategy
used to control implicit flows. 
% In the remainder of this chapter, a number of such rules are
% considered. To start, t
The rule for assignment (\refrule{bs:cmd:an}) corresponding to the NSU
check is shown in Figure~\ref{fig:assn-nsu}. The rule checks that the
label $l$ of the assigned variable \TT{x} in the initial store $\sigma$
is at least as high as $\pc$ (premise $\pc \sqsubseteq l$). If this
condition is not true, the program gets stuck. This is exactly the NSU
check described in Section~\ref{sec:bg-nsu}.

\subsection{Formalization of the No-sensitive-upgrade Check}

% Figure~\ref{fig:basic-syntax} and \ref{fig:basic-semantics}
% describes the syntax and semantics  of a standard while language
% with information flow  control. Notice the assignment rule often
% imposes some condition  on the $pc$ to prohibit low assignments
% under high context. For  instance, for the no-sensitive upgrade
% check as described earlier  the assignment rule.

% \subsection{Termination-Insensitive Non-interference}

For establishing and proving the security
property of termination-insensitive non-interference (TINI), the
observational power of the adversary needs to be defined. 
An adversary at level $\attacker$ in the lattice is allowed to view
all values that have a label less than or equal to $\attacker$. To prove
the security property of non-interference, it is enough to show that
when executing a program beginning with two different memory stores
that are \emph{observationally equivalent} to an adversary, the final
memory stores are also \emph{observationally equivalent} to the
adversary. For this, the observational equivalence of two memory
stores with respect to an adversary needs to be defined. 
Store equivalence is formalized as a relation $\eq$,
indexed by lattice elements $\attacker$, representing the adversary.

\begin{mydef}[Value equivalence]
  Two labeled values $\emph{\TT{n}}_1^k$ and $\emph{\TT{n}}_2^m$ are $\lab$-equivalent,
  written $\emph{\TT{n}}_1^k \sim_\lab \emph{\TT{n}}_2^m$, iff either:
  \begin{enumerate}
  \item $(k = m) \sqsubseteq \lab$ and $\emph{\TT{n}}_1 = \emph{\TT{n}}_2$ or
  \item $k \not\sqsubseteq \lab$ and $m \not\sqsubseteq \lab$
  \end{enumerate}
\end{mydef}

This definition states that for an adversary at security level $\lab$,
two labeled values $\TT{n}_1^k$ and $\TT{n}_2^m$ are equivalent iff either
$\lab$ can access both values and $\TT{n}_1$ and $\TT{n}_2$ are equal, or it
cannot access either value ($k \not\sqsubseteq \lab$ and $m
\not\sqsubseteq \lab$). The additional constraint $k = m$ in clause
(1) is needed to prove non-interference by induction. In the lattice
$L \sqsubset H$, two values labeled $L$ and $H$ are distinguishable
for the $L$-adversary. 

\begin{mydef}[Store equivalence]
  Two stores $\sigma_1$ and $\sigma_2$ are $\lab$-equivalent,
  written $\sigma_1 \sim_\lab \sigma_2$, iff for every variable \emph{\TT{x}},
  $\sigma_1(\emph{\TT{x}}) \sim_\lab \sigma_2(\emph{\TT{x}})$.
\end{mydef}

The following theorem states TINI for the NSU check. The theorem has
been proved for various languages in the past.

\begin{myThm}[TINI for NSU]
  With the assignment rule \refrule{bs:cmd:an} from
  Figure~\ref{fig:assn-nsu}, if 
  $~\sigma_1 \sim_\lab \sigma_2$ and $\langle \sigma_1, c \rangle
  \bscmd \sigma_1' $ and $\langle \sigma_2, c
  \rangle \bscmd \sigma_2' $, then $\sigma_1' \sim_\lab
  \sigma_2'$.
\end{myThm}
\begin{proof} Standard, see e.g.,~\cite{plas09}
\end{proof}

% Although the security lattice here is restricted to two elements $L$
% and $H$, the rules of Figures~\ref{fig:basic-semantics}
% and~\ref{fig:assn-nsu}, the definition of equivalence above and the
% theorem above (for NSU) are all general and work for arbitrary
% lattices.
  
%%%%%%%%%%%%%%%%%%%%%%%%%%%%%%%%%%%%%%%%%%%%%%%%%%%%%%%%%%%%%

%%%%%%%%%%%%%%%%%%%%%%%%%%%%%%%%%%%%%%%%%%%%%%%%%%%%%%%%%%%%%

\section{Austin and Flanagan's Permissive-Upgrade Strategy}
\label{sec:existing}

To allow a dynamic IFC analysis to accept safe executions of programs
with variable upgrades due to high $\pc$, Austin and Flanagan proposed
a less restrictive strategy called the \emph{permissive-upgrade
  strategy}~\cite{plas10}. They study this strategy for a two-point
lattice $L \sqsubset H$ and their strategy does not immediately
generalize to arbitrary security lattices. Whereas NSU stops a program  
when a variable's label is upgraded due to assignment in a high $pc$,
permissive-upgrade allows the assignment, but labels the variable as 
\emph{partially-leaked} or $P$. The exact intuition behind the
partially-leaked label $P$ is the following: 

\begin{framed}
\noindent
  A variable with a value labeled $P$ may have been implicitly
  influenced by $H$-labeled values in this execution, but in other
  executions (obtainable by changing $H$-labeled values in the
  initial store), this implicit influence may not exist and, hence,
  the variable may be labeled $L$.
\end{framed}

%  The taint $P$ roughly means that the
% variable's content in this execution is $H$, but it may be $L$ in
% other executions.
The program must be stopped later if it tries to use
or case-analyze the variable (in particular, branching on a
partially-leaked Boolean variable is stopped). Permissive-upgrade also
ensures termination-insensitive non-interference, but is strictly more
permissive than NSU. For example, permissive-upgrade stops the leaky
program of Listing~\ref{lst1} at line~\ref{linerefcond} when \TT{z}
contains $\texttt{false}^H$, but it allows the program of
Listing~\ref{lst1.1} to execute to completion when \TT{y} contains
$\texttt{true}^L$. 
% Note that in this section, the lattice has
% only two levels: $L$ (public) and $H$ (confidential). 

In the revised syntax of labels, summarized in
Figure~\ref{pus:syntax}, the labels $k,l,m$ on values can be either
elements of the lattice  ($L, H$) or $P$. The $\pc$ can only be one of
$L, H$ because branching on partially-leaked values is
prohibited. The join operation $\sqcup$ is lifted to labels including
$P$. Joining any label with $P$ results in $P$. For brevity in
definitions, they extend the order $\sqsubset$ to $L \sqsubset H
\sqsubset P$. However, $P$ is not a new ``top'' member of the lattice
because it receives special treatment in the semantic rules. 

\begin{figure}
\begin{equation*}
\begin{aligned}[c]
\lab          =~& L~\arrowvert~H\\
\pc          =~& \lab\\
k,l,m =~& \lab~\arrowvert~P
\end{aligned}
\qquad \qquad \qquad
\begin{aligned}[c] 
k \sqcup k ~= ~& k\\
L \sqcup H ~=~ &H\\
L \sqcup P  ~=~&P \\
H\sqcup P  ~=~&P
\end{aligned}
% \end{align*}
\end{equation*}
\caption{Syntax of labels including the partially-leaked label
  $P$}\label{pus:syntax}
\end{figure}


The rule for assignment with permissive-upgrade is
\begin{mathpar}
\inferrule*[left=\mbox{\labelthis{bs:cmd:ap}{assn-pus}}]
{l := \Gamma(\sigma(\TT{x})) \qquad \langle
 \sigma, \expr \rangle \bsexp \TT{n}^m} {\langle \sigma, \TT{x} := \expr
    \rangle \bscmd \sigma[\TT{x} \mapsto \TT{n}^{k}]}
\end{mathpar}
where $k$ is defined as follows:
$$ k = % \left\lbrace
\begin{cases}
m & \mbox{ if } \pc = L \\
m \sqcup H  & \mbox{ if } \pc = H \mbox{ and } l = H\\
P  & \mbox{ otherwise}
\end{cases} % \right.
$$


% The rule for assignment with permissive-upgrade is
% \begin{align*}
%   \inference[assn-PUS: ] {l := \Gamma(\sigma(x)) \qquad \langle
%     \expr, \sigma \rangle \Downarrow n^m} {\langle x := \expr, \sigma
%     \rangle \Downarrow_\pc \sigma[x \mapsto n^{k}]}
% \end{align*}
% where $k$ is defined as follows:
% \[ k = \left\lbrace
% \begin{array}{l}
% m \mbox{ if } \pc = L \\
% m \sqcup H  \mbox{ if } \pc = H \mbox{ and } l = H\\
% P \mbox{ otherwise}
% \end{array} \right.
% \]
The first two conditions in the definition of $k$ correspond to the
NSU rule (Figure~\ref{fig:assn-nsu}). The third condition applies, in
particular, when a variable whose initial label is $L$ is assigned 
with $\pc = H$. The NSU check would stop this assignment. With
permissive-upgrade, however, the updated variable is labeled $P$,
consistent with the intuitive meaning of $P$. This allows 
more permissiveness by allowing the assignment to proceed in all
cases. To compensate, any program (in particular, an adversarial
program) is disallowed from case analyzing any value labeled 
$P$. Consequently, in the rules for \texttt{if-then} and
\texttt{while} (Figure~\ref{fig:basic-semantics}), the label of the
branch condition is of the form $\lab$, which does not include $P$. 
Thus, assignments under high $\pc$ succeed under the
permissive-upgrade check but branching or case-analyzing a
partially-leaked value is not permitted as that can also leak
information. 

The noninterference result obtained for NSU earlier can be extended to
permissive-upgrade by changing the definition of store 
equivalence. Because no program can case-analyze a $P$-labeled value,
such a value is equivalent to any other labeled value.

\begin{mydef}
\label{def:pus}
  Two labeled values $\emph{\TT{n}}_1^k$ and $\emph{\TT{n}}_2^m$ are
  equivalent to an adversary at level $L$,
  written $\emph{\TT{n}}_1^k \sim_L \emph{\TT{n}}_2^m$, iff either:
  \begin{enumerate}
  \item $(k = m) = L$ and $\emph{\TT{n}}_1 = \emph{\TT{n}}_2$ or
  \item $k = H$ and $m = H$ or
  \item $k = P$ or $m = P$
  \end{enumerate}
\end{mydef}

\begin{mydef}
  Two stores $\sigma_1$ and $\sigma_2$ are $L$-equivalent,
  written $\sigma_1 \sim_L \sigma_2$, iff 
  $\forall \emph{\TT{x}}. \sigma_1(\emph{\TT{x}}) \sim_L \sigma_2(\emph{\TT{x}})$.
\end{mydef}

\begin{myThm}[TINI for permissive-upgrade with a two-point lattice]
  With the assignment rule \emph{assn-PUS}, if
  $~\sigma_1 \sim_L \sigma_2$ and $\langle c, \sigma_1 \rangle
  \Downarrow_\pc \sigma_1' $ and $\langle c, \sigma_2
  \rangle\Downarrow_\pc \sigma_2' $, then $\sigma_1' \sim_L
  \sigma_2'$.
\end{myThm}
\begin{proof} See~\cite{plas10}.
\end{proof}

% Note that the above definition and proof are specific to the two-point
% lattice.

