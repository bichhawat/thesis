\chapter{Future Directions}
\label{ch:future}

\section{Evaluating Information Flow Policies on Real-World
  Websites}
The dissertation focused on bridging the gap between the theory
of dynamic information flow control and its practicality in the
real-world. As part of understanding the practicality of such
approaches in real-world websites, it would be interesting to
investigate the types of security policies developers would be
interested in. The idea would be to come up with a set of generic
policies that could help evaluate different real-world websites for
privacy and confidentiality violations. Another interesting direction
for evaluation of information flow policies on real-world websites is
to explore integrity violations by third-party scripts. Broadly, these
policies could be related to cookies, web  storage, and other
user-data used by different websites. For example, a policy related to
cookies could be that third-party scripts should not be able to
extract information from first-party cookies, which is currently
allowed if the cookie is not marked \emph{HttpOnly}. Similarly, a
third-party script (loaded from the same domain) can modify the
\texttt{localStorage} object of the host without any restrictions,
which can be restricted or prevented by ensuring that such information
does not leak via the  third-party script. 

\section{Exploring Alternative Granularities for Enforcing
  Information Flow Control}
The current enforcement of dynamic information flow control tracks the
flow of information between all data structures in the system at a
fine-granularity. Although this increases the precision of the
analysis significantly, it affects the performance
adversely. Alternatively, one could explore approaches that are a 
little coarser than the current approach without giving up much on the 
precision. One such approach would be to only track information flow
globally and at the level of a function-call, rather than tracking it
through all local data-structures. The challenge here would be to
retain the precision of fine-grained techniques, which can probably be  
achieved by code-rewriting. The high-level idea would be to execute
the part of the code not dependent on sensitive data before executing
that part of the code that operates on sensitive data. 

\section{Handling Timing Leaks on the Web}
Primarily, the work done in the area of dynamic information flow
control has focused on handling two types of leaks ---
leaks due to explicit and implicit flows. However, there are various 
other covert channels like timing, resource-usage etc. that can leak
sensitive information. Among these, web-based timing attacks have been  
known to be a non-trivial source of information leaks with no feasible
client-side solutions till date~\cite{timing}. For instance, the time 
taken for performing sensitive computation can be measured by sending
out public requests before and after the computation. Although the two
requests do not carry any sensitive information, the time between the
two requests can leak information about the data on which the
computation was performed~\cite{timing}. As part of future work, it
would be interesting to explore a language-based solution to handling
timing leaks, which would mostly be based on balancing the time taken
for performing sensitive operations. The challenging part would be to 
handle loops and the different browser features that result in such
leaks. 

