\section*{Kurzzusammenfassung}

Seit Jahren werden Computersysteme und -Anwendungen immer komplexer
und verarbeiten eine Unmenge private und sensible Daten. Die
Komplexit\"at der Anwendungen tr\"agt neben der Existenz von (un)gewollt
eingef\"ugten Software Fehlern zur Weitergabe dieser sensiblen
Informationen bei. Information Flow Control (IFC, zu Deutsch
Informations-Fluss-Analyse) Mechanismen sind Gegenstand intensiver
Forschung um diesem Problem entgegen zu wirken. Grunds\"atzlich basieren
diese Ans\"atze auf der Anwendung von vordefinierten Sicherheitsregeln,
die die Unbeeinflussbarkeit (engl. non-interference) garantieren. Der
\"uberwiegende Teil dieser Techniken nutzt statische Analyse zur
Erzeugung der Regeln. Dem gegen\"uber steht die Tatsache, dass
Anwendungen, insbesondere im Bereich Web-Anwendungen, in dynamischen
Sprachen wie JavaScript entwickelt werden, wodurch rein statische
Analysen unzureichend sind. Dynamische Methoden auf der anderen Seite
approximieren das Verhalten einer Anwendung und k\"onnen daher die
grundlegende non-interference nicht garantieren. Sie tendieren dazu
besonders restriktive Regeln zu erzeugen, wodurch auch der rechtm{\"a}{\ss}ige
Zugriff auf Information verweigert wird. Beide Ans\"atze sind daher
nicht zur Anwendung auf Systeme in der realen Welt geeignet.

Das Ziel dieser Arbeit besteht darin die Benutzbarkeit von dynamischen
IFC Mechanismen zu verbessern indem Techniken entwickelt werden, die
die Genauigkeit und Toleranz steigern. Die Arbeit pr\"asentiert eine
korrekte (engl. 'sound') Erweiterung der permissive-upgrade Strategie
(eine Standardstrategie f\"ur dynamische IFC), die die Toleranz der
Strategie verbessert und sie weithin anwendbar macht. Dar\"uber hinaus
pr\"asentiere ich eine neue dynamische IFC Analyse, die auch komplexe
Funktionen, wie unstruktierte Kontrollfl\"usse und Exceptions in
Hochsprachen, abbildet. Obwohl Unbeeinflussbarkeit eine w\"unschenswerte
Eigenschaft ist, gibt es Anwendungen, die rechtm{\"a}{\ss}igen Zugang zu
sensiblen Daten ben\"otigen um ihre Funktion zu erf\"ullen. Um dies zu
erm\"oglichen pr\"asentiert diese Arbeit einen Ansatz, der die ungewollte
Weitergabe von Information quantifiziert und anhand eines
vordefinierten Grenzwertes freigibt. Diese Techniken wurden in einen
Web-Browser integriert, welcher es erlaubt die Definition von
flexiblen und n\"utzlichen Informationsflussregeln f\"ur Web Anwendungen
umzusetzen. 