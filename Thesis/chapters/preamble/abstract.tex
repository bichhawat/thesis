\section*{Abstract}

% \todo[inline]
{
Over the years, computer systems and applications have grown
significantly complex while handling a plethora of private and
sensitive user information. The complexity of these applications is
often assisted by a set of (un)intentional bugs with both malicious
and non-malicious intent leading to information leaks. Information
flow control has been studied extensively as an approach to mitigate
such information leaks. The technique works by enforcing the security
property of non-interference using a specified set of security
policies. A vast majority of existing work in this area
is based on static analyses. However, some of these applications,
especially on the Web, are developed using dynamic languages like
JavaScript that makes the static analyses techniques stale and
ineffective. As a result, there has been a growing interest in recent
years to develop dynamic information flow analysis techniques. In
spite of that, dynamic information flow analysis has not been at the
helm of information flow security in settings like the Web; the prime
reason being that the analysis techniques and the security property
related to them (non-interference) either over-approximate or are too
restrictive in most cases. Concretely, the analysis techniques
generate a lot of false positives, do not allow legitimate release of
sensitive information, support only static and rigid security
policies or are not general enough to be applied to real-world
applications. 

This thesis focuses on improving the usability of dynamic information
flow techniques by presenting mechanisms that can enhance the
precision and permissiveness of the analyses. It begins by presenting
a sound improvement and enhancement of the permissive-upgrade
strategy (a strategy widely used to enforce dynamic information flow
control), which improves the strategy's permissiveness and makes it
generic in applicability. The thesis, then, presents a sound and
precise dynamic information flow analysis for handling complex
features like unstructured control flow and exceptions in higher-order 
languages. Although non-interference is a desired property for
enforcing information flow control, there are program instances that
require legitimate release of some parts of the secret data to provide
the required functionality. Towards this end, this thesis develops a sound 
approach to quantify information leaks dynamically while allowing 
information release in accordance to a pre-specified budget. The thesis
concludes by applying these techniques to an information flow
control-enabled Web browser and explores a policy specification 
mechanism that allows flexible and useful information flow policies to 
be specified for Web applications. 

}
