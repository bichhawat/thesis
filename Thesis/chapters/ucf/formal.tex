\section{Formal Model} 
\label{sec:formalexc}

This section formally models the semantics of the language with
dynamic IFC instrumentation including unstructured control flow and
exceptions. The language from Figure~\ref{basic:syntax} is extended to
include unstructured control flow constructs like \texttt{break},
\TT{continue}, \TT{return} and exceptions. Programs are considered as a 
collection of functions (without parameter-passing). The control flow
analysis is performed on a function before it is executed and is
abstractly represented as a CFG in the formal model.
% \footnote{The
% terms \emph{function} and \emph{CFG} are used interchangeably in the
% rest of this section.}
Thus, the program itself is modeled as a huge
control flow graph ($\cfg$). IPDs are computed using the algorithm by
Lengauer and Tarjan~\cite{Lengauer} when the CFG is created. The CFG
is statically constructed (only once) as new functions are called or
discovered at runtime. For a non-branching node $\inst \in \cfg$,
$\Succ(\inst)$ denotes $\inst$’s unique successor. For a conditional 
branching node $\inst$,  $\Left(\inst)$ and $\Right(\inst)$ denote
successors when the condition is \TT{true} and \TT{false},
respectively.  

The syntax of the language modeling the nodes in a CFG is shown in
Figure~\ref{fig:exc:syntax}. The command 
\texttt{if}~$\expr$~\texttt{then}~$\comm_1$~\texttt{else}~$\comm_2 $
is represented as the node \texttt{branch}~$\expr$ with $\Left(\cfg,
\inst) = \comm_1$ and $\Right(\cfg, \inst) = \comm_2$. Similarly,
\texttt{while}~$\expr$~\texttt{do}~$\comm$ is represented as
\texttt{branch}~$\expr$ with $\Left(\inst) = \comm$ and
$\Right(\inst)$ being the command following \texttt{while} in the
program. A \texttt{jmp} node in the CFG corresponds to \texttt{break}
and \texttt{continue} with $\Succ(\inst)$ pointing to the next
node in the CFG according to the operation. It is also assumed that a
function always ends with a \texttt{return} statement and thus a CFG
normally ends with the \texttt{return} node. Multiple return
statements in a function are represented using a single
\texttt{return} node. $\Succ(\inst)$ points the
node to return to in the previous CFG, while the return value is saved
in a global variable that can be accessed by the program later on. 
When a new CFG is added for a function, the \texttt{return} node of
that CFG points to the successor node of the function call in the
previous CFG. However, the IPDs are computed when the CFGs are
intra-procedural. Every node in the program's CFG is uniquely
identifiable. 

In general, every function has an associated exception table that maps
each potentially exception-throwing instruction in the function to the 
appropriate exception handler within the function. This is represented
by adding a $\Right$ edge in the CFG from the instruction's node to
the handler's node; \texttt{throw} has only one outgoing edge. 
It is conservatively assumed that any unknown code may throw an
exception, so function call is exception-throwing for this purpose. If a
function contains unhandled exceptions, the corresponding edges in the
CFG point to the $\SEN$ of the CFG. The $\SEN$ is only created if one
of the previous functions in the call-stack has an appropriate exception
handler for the unhandled exceptions in the current function. When an
$\SEN$ node is created, an edge is added from the $\SEN$ of the CFG to
a node in the previous CFG, which is either the \texttt{catch} node or
the $\SEN$ of that CFG. $\Succ$ denotes one of these edges. If there
are no appropriate handlers in the call-stack, the exception-throwing
nodes have an edge to the \emph{end} node of the program CFG. For 
simplicity of exposition, it is assumed here that all exceptions
belong to a single class --- for different types of exceptions, the
exception class would also be matched for determining the appropriate 
exception handler.  

%%%%%%%%%%%%%%%%%%%%%%%%%%%%%%%%%%%%
\begin{figure}
\begin{align*}
\expr	:=~& \TT{n}~\arrowvert~\TT{x} ~\arrowvert~\expr_1 \odot \expr_2 \\
% \comm	:=~& \sk~\arrowvert~\TT{x} := \expr~\arrowvert 
%  ~\comm_1;\comm_2~\arrowvert~
%              \texttt{if}~\expr~\texttt{then}~\comm_1~\texttt{else}~\comm_2~\arrowvert~
%              \texttt{while}~\expr~\texttt{do}~\comm~\arrowvert~\texttt{break}~\arrowvert~\\ 
%  & \texttt{continue}~\arrowvert~ \texttt{return}~\expr~\arrowvert~
%    \texttt{try}~\comm~\texttt{catch}~(\TT{x})~\comm~\arrowvert~
%    \texttt{throw}~\expr~\arrowvert~ f(\expr_1, \dots, \expr_n)~\\
\inst :=~& \TT{end}~\arrowvert~\TT{x} := \expr~\arrowvert~
           \texttt{branch}~\expr~\arrowvert~ \texttt{jmp}~\arrowvert~ \texttt{return}~\arrowvert~
           \texttt{throw}~\expr~\arrowvert~\texttt{catch}~\TT{x}~\arrowvert~\SEN~
\end{align*}
\caption{Language Syntax}\label{fig:exc:syntax}
\end{figure}  

Program configurations for commands (nodes) are
represented as $\langle \sigma, \inst, \rho \rangle$, where
$\sigma$ represents the memory store as before, $\inst$ represents the
currently executing node, 
% $\cstack$ represents the call-stack, which is a stack of
% functions called during the execution of the program,
and $\rho$ is the $\pc$-stack. The configuration for expressions is
the same as before: $\langle \sigma, \expr \rangle$. 
% Expressions
% include $t$, which stands for temporary scoped variables. For
% instance in \texttt{catch} ($t$), the scope of $t$ is the
% \texttt{catch} block and $t$ is assigned a value before being
% used. Temporary variables also evaluate to the value contained in
% them, but are not accessed from the memory store $\sigma$ and thus,
% inaccessible to the adversary.

% The call-stack $\cstack$ contains one call-frame for each incomplete 
% function call. A call-frame contains the return address, which is a
% node in the CFG of the previous frame. $|\cstack|$ denotes the size of
% the call-stack and $!\cstack$ its top frame. 

Each entry of the $\pc$-stack $\rho$ is a pair $(\ell, \inst)$,
where $\ell$ is a security label, and $\inst$ is a node in the
CFG. When a new control context is entered, the new $\pc$-label, which
is a join of the current context label and the existing $\pc$-label
(the label on the top of the stack), is pushed together with the IPD
$\inst$ of the entry point of the control  context. $(\inst)$ uniquely
identifies where the control of the context ends. In the semantics,
the meta-function $\isIPD$ pops the stack. It takes the current 
instruction and the current $\pc$-stack, and returns a new
$\pc$-stack. $!\rho$ returns the top frame of the
$\pc$-stack. $\Gamma(!\rho)$ returns the current context 
label, also represented as $\pc$ in the semantics.
\begin{equation*}
 \isIPD(\inst, \rho) := 
 \begin{cases}
   \rho.\mathit{pop()} & \mbox{if}~ !\rho = (\_, \inst)
\\ \rho & \mbox{otherwise}
 \end{cases}
\end{equation*} 
As explained in Section~\ref{sec:excsen}, a new node $(\ell,
\inst)$ is pushed onto $\rho$ only when $\inst$ (the IPD)
differs from the corresponding entry on the top of the stack or it is
not $\SEN$ (Theorem~\ref{thm2:exc}). Otherwise, $\ell$ is joined 
with the label on the top of the stack. This is formally represented 
using the function $\rho.\push(\ell, \inst)$. 

%%%%%%%%%%%%%%%%%%%%%%%%%%%%%%%%%%%%
\begin{figure*}
\begin{framed}
\begin{mathparpagebreakable}
\inferrule*[left=\mbox{\labelthis{es:cmd:a}{assn}}]
{\inst = (\texttt{x}:=\expr) \\ \langle \sigma, \expr \rangle \bsexp
  \TT{n}^m \\ l = \Gamma(\sigma(\TT{x})) \\ l = \lab_\TT{x} \vee l =
  \lab_\TT{x}\pl \\ \pc = \Gamma(!\rho) \\
  k =  \left\{\begin{array}{ll}
        \pc \sqcup m, & \pc  \sqsubseteq \lab_\TT{x} \arcr
        ((\pc \sqcup m)\sqcap \lab_\TT{x} )\pl, & \pc  \not\sqsubseteq \lab_\TT{x} \arcr
        \end{array}\right\} \\ 
 \sigma' = \sigma[\TT{x} \mapsto \TT{n}^k] \\
  \inst' = \Succ(\inst) \\ \rho' =
  \isIPD(\inst', \rho)}
{\langle \sigma, \inst,  \rho \rangle \rightarrow \langle
  \sigma', \inst',  \rho' \rangle}
\and
%%%    
\inferrule*[left=\mbox{\labelthis{es:cmd:b}{branch}}]
{\inst = \texttt{branch}~\expr \\ 
  \langle \sigma, \expr \rangle \bsexp \TT{b}^{\lab} \\ 
  \inst' = \left\{\begin{array}{ll}
        \Left(\inst), & \textit{if }~  \TT{b} = \TT{true} \arcr
        \Right(\inst), & \textit{otherwise } \arcr
        \end{array}\right\} \\ 
  \rho'' = \rho.\push(\lab, \IPD{\inst}) \\
  \rho' = \isIPD(\inst', \rho'')
}
{\langle \sigma, \inst,  \rho \rangle \rightarrow \langle
  \sigma, \inst',  \rho' \rangle}
\and
\inferrule*[left=\mbox{\labelthis{es:cmd:j}{jmp, ret, sen}}]
{\inst = \texttt{jmp} ~\textit{or}~ \texttt{return} ~\textit{or}~ \SEN \\ 
  \inst' = \Succ(\inst) \\ 
  \rho' = \isIPD(\inst', \rho)
}
{\langle \sigma, \inst,  \rho \rangle \rightarrow \langle
  \sigma, \inst',  \rho' \rangle}
%%%
% \and
% \inferrule*[left=\mbox{\labelthis{es:cmd:r}{return}}]
% {\inst = \texttt{return} \\ 
%     % \langle \sigma, \expr \rangle \bsexp \TT{n}^{\lab} \\ 
%     % \mathit{retValue} = \TT{n}^{\lab} \\
%   \inst' = \Succ(\inst) \\ 
%   % \cstack' = \cstack.\mathit{pop}() \\
%   \rho' = \isIPD(\inst', \rho)
% }
% {\langle \sigma, \inst,  \rho \rangle \rightarrow \langle
%   \sigma, \inst',  \rho' \rangle}
%%%
\and
\inferrule*[left=\mbox{\labelthis{es:cmd:t}{throw}}]
{\inst = \texttt{throw}~\expr \\ 
    \langle \sigma, \expr \rangle \bsexp \TT{n}^{k} \\ 
    \pc = \Gamma(!\rho) \\
    \eV = \TT{n}^{(k\,\sqcup\,\pc)} \\
  \inst' = \Succ(\inst) \\ 
  \rho' = \isIPD(\inst', \rho)
}
{\langle \sigma, \inst,  \rho \rangle \rightarrow \langle
  \sigma, \inst',  \rho' \rangle}
%%%
\and
\inferrule*[left=\mbox{\labelthis{es:cmd:c}{catch}}]
{\inst = \texttt{catch}~\TT{x} \\ 
  \eV = \TT{n}^m \\ l = \Gamma(\sigma(\TT{x})) \\ l = \lab_\TT{x} \vee l =
  \lab_\TT{x}\pl \\ \pc = \Gamma(!\rho) \\
  k =  \left\{\begin{array}{ll}
        \pc \sqcup m, & \pc  \sqsubseteq \lab_\TT{x} \arcr
        ((\pc \sqcup m)\sqcap \lab_\TT{x} )\pl, & \pc  \not\sqsubseteq \lab_\TT{x} \arcr
        \end{array}\right\} \\ 
 \sigma' = \sigma[\TT{x} \mapsto \TT{n}^k] \\
  \inst' = \Succ(\inst) \\ 
  \rho' = \isIPD(\inst', \rho)
}
{\langle \sigma, \inst,  \rho \rangle \rightarrow \langle
  \sigma', \inst',  \rho' \rangle}
%%%
% \and
% \inferrule*[left=\mbox{\labelthis{es:cmd:f}{func}}]
% {\inst = f() \\ 
%     % \langle \sigma, \expr_i \rangle \bsexp \TT{n}_i^{\lab_i} \\ 
%     % \lab = \bigsqcup_{i = 1 .. n} {\lab_i} \\
%   \inst' = \cfg(f).n_s \\
%   \rho'' = \rho.\push(\lab, \IPD{\inst}) \\
%   \cstack' = \cstack.\push(\cfg(f), \Left(\inst)) \\ 
%   \rho' = \isIPD(\inst', \rho'', \cstack')
% }
% {\langle \sigma, \inst,  \rho \rangle \rightarrow \langle
%   \sigma, \inst',  \rho' \rangle}
%%%    
% \and
% \inferrule*[left=\mbox{\labelthis{es:cmd:s}{sen}}]
% {\inst = \SEN \\ 
%   \inst' = \Succ(\inst) \\
%   \rho' = \isIPD(\inst', \rho)
% }
% {\langle \sigma, \inst,  \rho \rangle \rightarrow \langle
%   \sigma, \inst',  \rho' \rangle}
\and
%%%    
\inferrule*[left=\mbox{\labelthis{es:cmd:e}{end}}]
{\inst = \texttt{end}}
{\langle \sigma, \inst,  \rho \rangle \rightarrow \_}
%%%
\end{mathparpagebreakable}
\end{framed}
\caption{Semantics}\label{fig:sem-exc}
\end{figure*}

% \subsection{Semantics and IFC with CFGs}
% \label{sec:sem-exc}

The derivation rules in Figure~\ref{fig:sem-exc} define small-step
semantics that define the judgment: $\langle \sigma, \inst, 
\rho\rangle \rightarrow \langle\sigma', \inst',  \rho'\rangle$  
(as big-step semantics cannot model unstructured control flow). The
rules for expressions are the same as in
Figure~\ref{fig:basic-semantics}. The rules are informally explained
above. The soundness of the analysis including unstructured control
flow and exceptions is proven below.

To state the theorem formally, the equivalence of different data
structures with respect to the adversary needs to be defined. Value
and memory equivalence is defined as before in
Definitions~\ref{def:gpua:veq} and~\ref{def:gpua:seq}. To prove
termination-insensitive non-interference, some additional definitions
and auxiliary lemmas are defined below. The detailed proofs can be
found in Appendix~\ref{app:exc}.

\begin{mydef}[$\pc$-stack equivalence]
\label{def:exc:pc}
For two $\pc$-stacks $\rho_1, \rho_2$, $\rho_1 \sim_\lab \rho_2$ iff
the corresponding nodes of $\rho_1$ and $\rho_2$ 
having label less than or equal to $\lab$ are equal. 
\end{mydef}

\begin{mydef}[State equivalence]
\label{def:exc:ceq}
Two states $s_1 = \langle \sigma_1, \iota_1, \rho_1\rangle$ and $s_2 = \langle
\sigma_2, \iota_2, \rho_2\rangle$ are equivalent, written as
$s_1 \sim_\lab s_2$, iff $\sigma_1 \eq \sigma_2$, $\iota_1 = \iota_2$,
and $\rho_1  \sim_\lab \rho_2$.
\end{mydef}

\begin{myLemma}[Confinement Lemma]
\label{lem:exc:conf}
If $~\langle \sigma, \iota, \rho \rangle~\rightarrow~\langle \sigma',
\iota', \rho' \rangle$ and $\Gamma(!\rho) \not\sqsubseteq \lab$, then
$\sigma \sim_\lab \sigma'$, and $\rho \sim_\lab \rho'$.
\end{myLemma}

\begin{myThm}
\label{thm:exc:ni}
Suppose:
\begin{enumerate}
\item $\langle \sigma_1, \iota_1, \rho_1 \rangle \sim_\lab
\langle \sigma_2, \iota_2, \rho_2\rangle$
\item $\langle \sigma_1, \iota_1, \rho_1 \rangle
  \rightarrow^{*} \langle \sigma_1', \texttt{end}, [] \rangle$
\item $\langle \sigma_2, \iota_2, \rho_2 \rangle \rightarrow^{*}
  \langle \sigma_2', \texttt{end}, []\rangle$
\end{enumerate}
Then, $\sigma_1' \sim_\lab \sigma_2'$.
\end{myThm}
