\chapter{Dynamic IFC with Unstructured Control Flow and Exceptions}
\label{ch:ucfe}
\blfootnote{The content of this
  chapter is based partly on the work published as part of the paper,
  ``Information Flow Control in WebKit's JavaScript
  Bytecode''~\cite{post14}} 
This chapter presents a mechanism to prevent leaks due to implicit
flows in the presence of unstructured control constructs like break,
return-in-the-middle and exceptions.  

Implicit flow corresponds to \emph{control dependence} in program
analysis, where a predicate governs which program path is executed and
leaks information through the control flow of the program. To avoid
overtainting $\pc$ labels, an important goal in implicit flow 
tracking is to determine when the influence of a control construct has
ended. For block-structured control flow limited to \texttt{if} and
\texttt{while} commands, this is straightforward: The effect of a
control construct ends with its lexical scope, e.g., in 
$$\texttt{if (h) \{l = 1;\} l = 2}$$
\texttt{h} influences the control flow at
\texttt{l = 1} but not at \texttt{l = 2}. This leads to a
straightforward implementation of a $\pc$ upgrading and downgrading
strategy: One maintains a \emph{stack} of $\pc$
labels~\cite{zdancewic02PhD}; the effective $\pc$ is the top one. When
entering a control flow construct like \texttt{if} or \texttt{while},
a new $\pc$ label, equal to the join of labels of all values on which
the construct's guard depends with the previous effective $\pc$, is
pushed. When exiting the construct, the label is popped.

Unfortunately, it is unclear how to extend this simple strategy to
non-block-struct\-ured control flow constructs such as 
\texttt{break}, \texttt{continue} and \texttt{return}-in-the-middle
for functions, all of which occur in high-level languages. For
example, consider the program 
$$\texttt{l = 1; while(1) \{... if(h) \{break;\}; l = 0; break;\}}$$ 
with \texttt{h} labeled $H$. This program leaks the
value of \texttt{h} into \texttt{l}, but no assignment to \texttt{l}
appears in a block-scope guarded by \texttt{h}. Indeed, the $\pc$
upgrading and downgrading strategy just described is ineffective for
this program. 

Implicit flow in the form of error handling is also a  
source of information leak as it helps the adversary to learn about
the system~\cite{leak}. For instance, an exception handler might print 
the stack trace of the error, from which an attacker can determine what
sort of attacks the system is vulnerable to. 

Tracking information flow in the presence of unstructured control
flows is non-trivial as the control breaks out of block
structures. Exceptions are much more difficult to handle as they 
allow for non-local control transfer. Much work on error handling has
focussed in the context of static analysis~\cite{myersJFlow,aslan_plas09}
and the work on IFC for dynamic languages has mostly ignored
exceptions and other unstructured control flow 
constructs~\cite{plas09,plas10,stagedIFC,guarnieri11ISSTA,acsac09}.
Just et al.~\cite{just11PLASTIC} present dynamic IFC for JavaScript
bytecode with static analysis to determine implicit flows precisely
but ignore implicit flows due to exceptions. Hedin and Sabelfeld
propose a dynamic IFC approach for a language 
modeling the core features of JavaScript~\cite{csf12} but ignore
unstructured control flow constructs like \emph{break},
\emph{continue} and \emph{return}-in-the-middle for functions. For
handling exceptions, they introduce annotations and an additional
class of labels. An extension introduces similar annotations to deal
with unstructured control flows~\cite{sac14}. These labels are more
restrictive than needed, e.g., the code indicated by dots in the
example above is executed irrespective of the condition \texttt{h} in
the first iteration, and thus there is no need to raise the $\pc$
before checking that condition. 

To solve this issue, this chapter presents a precise dynamic
analysis approach using \emph{post-dominator}
analysis~\cite{denning82,just11PLASTIC}.  

\section{Control Flow Graphs and Post-dominator Analysis}
The approach presented in this chapter performs on-the-fly
post-dominator analysis at runtime to handle implicit
flows. A control flow graph (CFG), which is a directed graph, is
constructed for every new function before it is executed with every
instruction being represented as a node and whose edges represent the
possible control fows. For every branch node, its immediate
post-dominator (IPD) is
computed~\cite{denning82,just11PLASTIC,post14}. A stack of $\pc$  
labels is maintained. When executing a branch node, a new $pc$ label
is pushed on the stack \emph{along with} the node's IPD. When the IPD
is actually reached, the $pc$ label along with the IPD is popped from
the stack. In~\cite{Xin,Masri}, the authors prove that the IPD marks the
end of the scope of an operation and hence the security context of the
operation, so our strategy is sound. The IPD-based solution works for all
forms of unstructured control flow like \emph{break}, \emph{continue},
\emph{return}-in-the-middle, and exceptions. Multiple \emph{return}
statements in a function are represented by a single \emph{return} 
node. Theorem~\ref{thm1:exc} shows that the IPD of a node is the most
precise node where the context of an operation can be removed. For
proving Theorem~\ref{thm1:exc}, a few definitions are defined below.

\begin{mydef}\label{defcfg}
\emph{(Control flow graph)}\\
A \emph{control flow graph} is a directed graph $\cfg =
(\mathcal{N}, \mathcal{E}, n_s, n_e, \mathcal{L})$. $\mathcal{N}$ is the set of
nodes. $\mathcal{E}$ is the set of control flow edges $(n_1, n_2)
\in \mathcal{E}$, where $n_i \in \mathcal{N}$. $(n_1, n_2)$ represents $n_2$ may
immediately execute after $n_1$. The nodes $n_s, n_e \in \mathcal{N}$
are special nodes representing the start and end point of $\cfg$.  The
function $\mathcal{L}$ maps the edges in $\mathcal{E}$ to labels.
\end{mydef}

\begin{mydef}\label{defpath}
\emph{(Path)}\\
A \emph{path} in a CFG $\cfg$ is a sequence of nodes $(n_1, n_2, ...,
n_m)$ such that $(n_i, n_{i+1}) \in \mathcal{E}$, written as
$\pathG{n_1}{n_m}$. 
A node $n$ that lies on the path $\pathG{n_1}{n_m}$ is written as
$n \in \pathG{n_1}{n_m}$.
The notation $n_1 < n_2$ with respect to two nodes $n_1$ and $n_2$ in
a CFG $\cfg$ indicates that $n_2$ lies on a path $\pathG{n_1}{n_e}$ . 
\end{mydef}

\begin{mydef}\label{defbp}
\emph{(Branch-point)}\\
A \emph{branch-point} $b$ is a node in a CFG $\cfg$ that has more than
one successor, i.e., \\ outdegree$(b) > 1$.
\end{mydef}

\begin{mydef}\label{defpd}
\emph{(Post-dominator)}\\
In a CFG $\cfg$, a node $n_d$ is said to be the \emph{post-dominator} of a node
$n$ if all paths from $n$ to the end-node pass through $n_d$, i.e.,
$\forall p. \pathG{n}{n_e} \implies n_d \in p$. The notation
\emph{$\dom{n_d}{n}$} indicates that $n_d$ is a post-dominator of $n$.
\end{mydef}

\begin{mydef}\label{defipd}
\emph{(Immediate post-dominator)}\\
A node $i$ is the immediate post-dominator of the node $n$, denoted as
$\IPD{n}$, if{f}:
\begin{enumerate}
\item \emph{$\dom{i}{n}$} and
\item \emph{$\not\exists n_o \in \mathcal{N}.((\dom{n_o}{n}) \wedge (n_o <
  i))$} or \\
  \emph{$\forall n_o \in \mathcal{N}.((n_o \neq n) \implies (\dom{n_o}{n}
  \implies \dom{n_o}{i}))$}.
\end{enumerate}
\end{mydef}

\begin{myThm}[Precision]\label{thm1:exc}
Choosing any node other than the IPD to lower the pc-label will either
give unsound results or be less precise.
\end{myThm}
\begin{proof}
The proof of the theorem is described in Appendix~\ref{app:exc-pre}.
\end{proof}

\section{Exceptions and Synthetic Exit Nodes}
\label{sec:excsen}
Maintaining a precise CFG for post-dominator analysis at runtime in the 
presence of exceptions is expensive. The CFG of a function is 
constructed statically on-the-fly when compiling the function at 
runtime. An exception-throwing node in a
function that does not catch that exception should have an outgoing
control flow edge to the next related exception handler in the runtime
call-stack. This means that the CFG is, in general, inter-procedural,
and edges going out of a function depend on its calling context, so
IPDs of nodes in the function must be computed \emph{every time the
  function is called} (the IPDs change based on the earlier functions
in the call-stack that called the particular function where the
exception occurs). Moreover, in the case of recursive functions, the
nodes must be replicated for every call. This is rather
expensive. Ideally, the function's CFG should be built only once when
\emph{the function is compiled} and work intra-procedurally. 

In the design presented in this chapter, every function that may throw
an unhandled exception and has an exception handler present earlier in
the call-stack (which is assumed to be known at runtime through 
additional data structures in the system) has a special
\emph{synthetic exit node} ($\SEN$), which is placed after the regular
return node of the function in the CFG. 
Every exception-throwing node, whose exception will not be caught
within the function, has an outgoing edge to the $\SEN$. In essence, the
$\SEN$ is treated as the IPD for nodes whose actual IPDs lie outside of
the function. By doing this, all cross-function edges are eliminated
and the CFGs become intra-procedural. This allows the computation of
the CFGs just once as compared to the inter-procedural case. 
% As there are instructions other than \emph{throw} can also throw exceptions, we
% treat them as exception-throwing instructions and calculate the IPDs
% accordingly, i.e., 
For every exc\-ep\-tion-throw\-ing instruction that has an associated
handler, its context is maintained during dynamic information flow
analysis until the handler is reached. 
% This is a bit restrictive; hence, this option can be turned
% off in the instrumented interpreter, if required. 
Thus, function calls and all potential exception-throwing
instructions are represented as nodes with multiple edges (bra\-nches) 
and push a node on the $\pc$-stack. However, a new node is \emph{not} 
pushed on the $\pc$-stack if the IPD of the current node is the same
as the IPD on the top of the $\pc$-stack or if the IPD of the current
node is the $\SEN$, as in this case the \emph{real} IPD, which is outside
of this method, is already on top of the $\pc$-stack. In fact, the
actual IPD of a node having $\SEN$ as its IPD is the node that is
currently on the top of the stack. This result is shown in
Theorem~\ref{thm2:exc}. The proof is shown in Appendix~\ref{app:exc-pre}.

\begin{myThm}\label{thm2:exc}
The actual IPD of a node having $\SEN$ as its IPD is the node on
the top of the pc-stack, which lies in a previously called function.
\end{myThm}

In summary, these semantics emulate the effect of having cross-function edges. 
For illustration, consider the following two functions \texttt{f} and
\texttt{g}. The $\diamond$ at the end of \texttt{g} denotes its
$\SEN$. Note that there is an edge from \texttt{throw 9} to $\diamond$
because \texttt{throw 9} is not handled within~\texttt{g}. $\Box$
denotes the IPD of the function call~\texttt{g()} and handler~\texttt{catch(e)}.

%\noindent 
%\begin{tabular}{@{}p{0.5\textwidth}p{0.4\textwidth}@{}}
%\begin{lstlisting}[mathescape=true]
%function f() = {
%  l = 0;
%  try { g(); } catch(e) { l = 1; }
%  $\Box$ return l;
%}
%\end{lstlisting} & 
%\begin{lstlisting}[mathescape=true]
%function g() = {
%  if (h) {throw 9;}
%  return 7;
%} $\diamond$
%\end{lstlisting}
%\end{tabular}

\noindent 
\begin{tabular}{@{}p{0.55\textwidth}p{0.35\textwidth}@{}}
\begin{tabular}[t]{@{}l@{}}
\texttt{function f() = \{} \\
~~\texttt{l = 0;} \\
~~\texttt{try \{ g(); \} catch(e) \{ l = 1; \}} \\
~~$\Box$ \texttt{return l;}\\
\texttt{\}}
\end{tabular} & 
\begin{tabular}[t]{@{}l@{}}
\texttt{function g() = \{}\\
~~\texttt{if (h) \{throw 9;\}}\\
~~\texttt{return 7;}\\
\texttt{\}} $\diamond$
\end{tabular}
\end{tabular} 

In the absence of instrumentation, when
\texttt{f} is invoked with $pc = L$, the two functions together leak
the value of \texttt{h}, which is assumed to have a label $H$, into the
return value of \texttt{f}. When calling \texttt{g}, the current $\pc$
and IPD $(L, \Box)$ are pushed on the $\pc$-stack.  When executing the
condition \texttt{if (h)} a new node is not pushed again, but the top
element is merely updated to $(H, \Box)$ as its IPD is the $\SEN$
$\diamond$. If \texttt{h} is \texttt{false}, control reaches the
\texttt{return} statement but with $pc = H$.
% because the IPD of \texttt{if (h)} is $\diamond$ and the real IPD is $\Box$. 
At $\Box$, $\pc$ is lowered to $L$, so
\texttt{f} ends with the return value \texttt{0} and public label
$L$. If \texttt{h} is \texttt{true}, control reaches the handler, which
is in \texttt{f} and invokes it with the same $\pc$ as at the point of
exception, i.e., $H$. Consequently, permissive-upgrade marks the 
assignment in the catch block as partially-leaked and prevents the
implicit information leak in this case. 

\section{Formal Model} 
\label{sec:formalexc}

This section formally models the semantics of the language with
dynamic IFC instrumentation including unstructured control flow and
exceptions. The language from Figure~\ref{basic:syntax} is extended to
include unstructured control flow constructs like \texttt{break},
\TT{continue}, \TT{return} and exceptions. Programs are considered as a 
collection of functions (without parameter-passing). The control flow
analysis is performed on a function before it is executed and is
abstractly represented as a CFG in the formal model.
% \footnote{The
% terms \emph{function} and \emph{CFG} are used interchangeably in the
% rest of this section.}
Thus, the program itself is modeled as a huge
control flow graph ($\cfg$). IPDs are computed using the algorithm by
Lengauer and Tarjan~\cite{Lengauer} when the CFG is created. The CFG
is statically constructed (only once) as new functions are called or
discovered at runtime. For a non-branching node $\inst \in \cfg$,
$\Succ(\inst)$ denotes $\inst$’s unique successor. For a conditional 
branching node $\inst$,  $\Left(\inst)$ and $\Right(\inst)$ denote
successors when the condition is \TT{true} and \TT{false},
respectively.  

The syntax of the language modeling the nodes in a CFG is shown in
Figure~\ref{fig:exc:syntax}. The command 
\texttt{if}~$\expr$~\texttt{then}~$\comm_1$~\texttt{else}~$\comm_2 $
is represented as the node \texttt{branch}~$\expr$ with $\Left(\cfg,
\inst) = \comm_1$ and $\Right(\cfg, \inst) = \comm_2$. Similarly,
\texttt{while}~$\expr$~\texttt{do}~$\comm$ is represented as
\texttt{branch}~$\expr$ with $\Left(\inst) = \comm$ and
$\Right(\inst)$ being the command following \texttt{while} in the
program. A \texttt{jmp} node in the CFG corresponds to \texttt{break}
and \texttt{continue} with $\Succ(\inst)$ pointing to the next
node in the CFG according to the operation. It is also assumed that a
function always ends with a \texttt{return} statement and thus a CFG
normally ends with the \texttt{return} node. Multiple return
statements in a function are represented using a single
\texttt{return} node. $\Succ(\inst)$ points the
node to return to in the previous CFG, while the return value is saved
in a global variable that can be accessed by the program later on. 
When a new CFG is added for a function, the \texttt{return} node of
that CFG points to the successor node of the function call in the
previous CFG. However, the IPDs are computed when the CFGs are
intraprocedural. Every node in the program's CFG is uniquely
identifiable. 

In general, every function has an associated exception table that maps
each potentially exception-throwing instruction in the function to the 
appropriate exception handler within the function. This is represented
by adding a $\Right$ edge in the CFG from the instruction's node to
the handler's node; \texttt{throw} has only one outgoing edge. 
It is conservatively assumed that any unknown code may throw an
exception, so function call is exception-throwing for this purpose. If a
function contains unhandled exceptions, the corresponding edges in the
CFG point to the $\SEN$ of the CFG. The $\SEN$ is only created if one
of the previous functions in the call-stack has an appropriate exception
handler for the unhandled exceptions in the current function. When an
$\SEN$ node is created, an edge is added from the $\SEN$ of the CFG to
a node in the previous CFG, which is either the \texttt{catch} node or
the $\SEN$ of that CFG. $\Succ$ denotes one of these edges. If there
are no appropriate handlers in the call-stack, the exception-throwing
nodes have an edge to the \emph{end} node of the program CFG. For 
simplicity of exposition, it is assumed here that all exceptions
belong to a single class --- for different types of exceptions, the
exception class would also be matched for determining the appropriate 
exception handler.  

%%%%%%%%%%%%%%%%%%%%%%%%%%%%%%%%%%%%
\begin{figure}
\begin{align*}
\expr	:=~& \TT{n}~\arrowvert~\TT{x} ~\arrowvert~\expr_1 \odot \expr_2 \\
% \comm	:=~& \sk~\arrowvert~\TT{x} := \expr~\arrowvert 
%  ~\comm_1;\comm_2~\arrowvert~
%              \texttt{if}~\expr~\texttt{then}~\comm_1~\texttt{else}~\comm_2~\arrowvert~
%              \texttt{while}~\expr~\texttt{do}~\comm~\arrowvert~\texttt{break}~\arrowvert~\\ 
%  & \texttt{continue}~\arrowvert~ \texttt{return}~\expr~\arrowvert~
%    \texttt{try}~\comm~\texttt{catch}~(\TT{x})~\comm~\arrowvert~
%    \texttt{throw}~\expr~\arrowvert~ f(\expr_1, \dots, \expr_n)~\\
\inst :=~& \TT{end}~\arrowvert~\TT{x} := \expr~\arrowvert~
           \texttt{branch}~\expr~\arrowvert~ \texttt{jmp}~\arrowvert~ \texttt{return}~\arrowvert~
           \texttt{throw}~\expr~\arrowvert~\texttt{catch}~\TT{x}~\arrowvert~\SEN~
\end{align*}
\caption{Language Syntax}\label{fig:exc:syntax}
\end{figure}  

Program configurations for commands (nodes) are
represented as $\langle \sigma, \inst, \rho \rangle$, where
$\sigma$ represents the memory store as before, $\inst$ represents the
currently executing node, 
% $\cstack$ represents the call-stack, which is a stack of
% functions called during the execution of the program,
and $\rho$ is the $\pc$-stack. The configuration for expressions is
the same as before: $\langle \sigma, \expr \rangle$. 
% Expressions
% include $t$, which stands for temporary scoped variables. For
% instance in \texttt{catch} ($t$), the scope of $t$ is the
% \texttt{catch} block and $t$ is assigned a value before being
% used. Temporary variables also evaluate to the value contained in
% them, but are not accessed from the memory store $\sigma$ and thus,
% inaccessible to the adversary.

% The call-stack $\cstack$ contains one call-frame for each incomplete 
% function call. A call-frame contains the return address, which is a
% node in the CFG of the previous frame. $|\cstack|$ denotes the size of
% the call-stack and $!\cstack$ its top frame. 

Each entry of the $\pc$-stack $\rho$ is a pair $(\ell, \inst)$,
where $\ell$ is a security label, and $\inst$ is a node in the
CFG. When a new control context is entered, the new $\pc$-label, which
is a join of the current context label and the existing $\pc$-label
(the label on the top of the stack), is pushed together with the IPD
$\inst$ of the entry point of the control  context. $(\inst)$ uniquely
identifies where the control of the context ends. In the semantics,
the meta-function $\isIPD$ pops the stack. It takes the current 
instruction and the current $\pc$-stack, and returns a new
$\pc$-stack. $!\rho$ returns the top frame of the
$\pc$-stack. $\Gamma(!\rho)$ returns the current context 
label, also represented as $\pc$ in the semantics.
\begin{equation*}
 \isIPD(\inst, \rho) := 
 \begin{cases}
   \rho.\mathit{pop()} & \mbox{if}~ !\rho = (\_, \inst)
\\ \rho & \mbox{otherwise}
 \end{cases}
\end{equation*} 
As explained in Section~\ref{sec:excsen}, a new node $(\ell,
\inst)$ is pushed onto $\rho$ only when $\inst$ (the IPD)
differs from the corresponding entry on the top of the stack or it is
not $\SEN$ (Theorem~\ref{thm2:exc}). Otherwise, $\ell$ is joined 
with the label on the top of the stack. This is formally represented 
using the function $\rho.\push(\ell, \inst)$. 

%%%%%%%%%%%%%%%%%%%%%%%%%%%%%%%%%%%%
\begin{figure*}
\begin{framed}
\begin{mathparpagebreakable}
\inferrule*[left=\mbox{\labelthis{es:cmd:a}{assn}}]
{\inst = (\texttt{x}:=\expr) \\ \langle \sigma, \expr \rangle \bsexp
  \TT{n}^m \\ l = \Gamma(\sigma(\TT{x})) \\ l = \lab_\TT{x} \vee l =
  \lab_\TT{x}\pl \\ \pc = \Gamma(!\rho) \\
  k =  \left\{\begin{array}{ll}
        \pc \sqcup m, & \pc  \sqsubseteq \lab_\TT{x} \arcr
        ((\pc \sqcup m)\sqcap \lab_\TT{x} )\pl, & \pc  \not\sqsubseteq \lab_\TT{x} \arcr
        \end{array}\right\} \\ 
 \sigma' = \sigma[\TT{x} \mapsto \TT{n}^k] \\
  \inst' = \Succ(\inst) \\ \rho' =
  \isIPD(\inst', \rho)}
{\langle \sigma, \inst,  \rho \rangle \rightarrow \langle
  \sigma', \inst',  \rho' \rangle}
\and
%%%    
\inferrule*[left=\mbox{\labelthis{es:cmd:b}{branch}}]
{\inst = \texttt{branch}~\expr \\ 
  \langle \sigma, \expr \rangle \bsexp \TT{b}^{\lab} \\ 
  \inst' = \left\{\begin{array}{ll}
        \Left(\inst), & \textit{if }~  \TT{b} = \TT{true} \arcr
        \Right(\inst), & \textit{otherwise } \arcr
        \end{array}\right\} \\ 
  \rho'' = \rho.\push(\lab, \IPD{\inst}) \\
  \rho' = \isIPD(\inst', \rho'')
}
{\langle \sigma, \inst,  \rho \rangle \rightarrow \langle
  \sigma, \inst',  \rho' \rangle}
\and
\inferrule*[left=\mbox{\labelthis{es:cmd:j}{jmp, ret, sen}}]
{\inst = \texttt{jmp} ~\textit{or}~ \texttt{return} ~\textit{or}~ \SEN \\ 
  \inst' = \Succ(\inst) \\ 
  \rho' = \isIPD(\inst', \rho)
}
{\langle \sigma, \inst,  \rho \rangle \rightarrow \langle
  \sigma, \inst',  \rho' \rangle}
%%%
% \and
% \inferrule*[left=\mbox{\labelthis{es:cmd:r}{return}}]
% {\inst = \texttt{return} \\ 
%     % \langle \sigma, \expr \rangle \bsexp \TT{n}^{\lab} \\ 
%     % \mathit{retValue} = \TT{n}^{\lab} \\
%   \inst' = \Succ(\inst) \\ 
%   % \cstack' = \cstack.\mathit{pop}() \\
%   \rho' = \isIPD(\inst', \rho)
% }
% {\langle \sigma, \inst,  \rho \rangle \rightarrow \langle
%   \sigma, \inst',  \rho' \rangle}
%%%
\and
\inferrule*[left=\mbox{\labelthis{es:cmd:t}{throw}}]
{\inst = \texttt{throw}~\expr \\ 
    \langle \sigma, \expr \rangle \bsexp \TT{n}^{k} \\ 
    \pc = \Gamma(!\rho) \\
    \eV = \TT{n}^{(k\,\sqcup\,\pc)} \\
  \inst' = \Succ(\inst) \\ 
  \rho' = \isIPD(\inst', \rho)
}
{\langle \sigma, \inst,  \rho \rangle \rightarrow \langle
  \sigma, \inst',  \rho' \rangle}
%%%
\and
\inferrule*[left=\mbox{\labelthis{es:cmd:c}{catch}}]
{\inst = \texttt{catch}~\TT{x} \\ 
  \eV = \TT{n}^m \\ l = \Gamma(\sigma(\TT{x})) \\ l = \lab_\TT{x} \vee l =
  \lab_\TT{x}\pl \\ \pc = \Gamma(!\rho) \\
  k =  \left\{\begin{array}{ll}
        \pc \sqcup m, & \pc  \sqsubseteq \lab_\TT{x} \arcr
        ((\pc \sqcup m)\sqcap \lab_\TT{x} )\pl, & \pc  \not\sqsubseteq \lab_\TT{x} \arcr
        \end{array}\right\} \\ 
 \sigma' = \sigma[\TT{x} \mapsto \TT{n}^k] \\
  \inst' = \Succ(\inst) \\ 
  \rho' = \isIPD(\inst', \rho)
}
{\langle \sigma, \inst,  \rho \rangle \rightarrow \langle
  \sigma', \inst',  \rho' \rangle}
%%%
% \and
% \inferrule*[left=\mbox{\labelthis{es:cmd:f}{func}}]
% {\inst = f() \\ 
%     % \langle \sigma, \expr_i \rangle \bsexp \TT{n}_i^{\lab_i} \\ 
%     % \lab = \bigsqcup_{i = 1 .. n} {\lab_i} \\
%   \inst' = \cfg(f).n_s \\
%   \rho'' = \rho.\push(\lab, \IPD{\inst}) \\
%   \cstack' = \cstack.\push(\cfg(f), \Left(\inst)) \\ 
%   \rho' = \isIPD(\inst', \rho'', \cstack')
% }
% {\langle \sigma, \inst,  \rho \rangle \rightarrow \langle
%   \sigma, \inst',  \rho' \rangle}
%%%    
% \and
% \inferrule*[left=\mbox{\labelthis{es:cmd:s}{sen}}]
% {\inst = \SEN \\ 
%   \inst' = \Succ(\inst) \\
%   \rho' = \isIPD(\inst', \rho)
% }
% {\langle \sigma, \inst,  \rho \rangle \rightarrow \langle
%   \sigma, \inst',  \rho' \rangle}
\and
%%%    
\inferrule*[left=\mbox{\labelthis{es:cmd:e}{end}}]
{\inst = \texttt{end}}
{\langle \sigma, \inst,  \rho \rangle \rightarrow \_}
%%%
\end{mathparpagebreakable}
\end{framed}
\caption{Semantics}\label{fig:sem-exc}
\end{figure*}

% \subsection{Semantics and IFC with CFGs}
% \label{sec:sem-exc}

The derivation rules in Figure~\ref{fig:sem-exc} define small-step
semantics that define the judgment: $\langle \sigma, \inst, 
\rho\rangle \rightarrow \langle\sigma', \inst',  \rho'\rangle$  
(as big-step semantics cannot model unstructured control flow). The
rules for expressions are the same as in
Figure~\ref{fig:basic-semantics}. The rules are informally explained
above. The soundness of the analysis including unstructured control
flow and exceptions is proven below.

To state the theorem formally, the equivalence of different data
structures with respect to the adversary needs to be defined. Value
and memory equivalence is defined as before in
Definitions~\ref{def:gpua:veq} and~\ref{def:gpua:seq}. To prove
termination-insensitive non-interference, some additional definitions
and auxiliary lemmas are defined below. The detailed proofs can be
found in Appendix~\ref{app:exc}.

\begin{mydef}[$pc$-stack equivalence]
\label{def:exc:pc}
For two pc-stacks $\rho_1, \rho_2$, $\rho_1 \sim_\lab \rho_2$ iff
the corresponding nodes of $\rho_1$ and $\rho_2$ 
having label less than or equal to $\lab$ are equal. 
\end{mydef}

\begin{mydef}[State equivalence]
\label{def:exc:ceq}
Two states $s_1 = \langle \sigma_1, \iota_1, \rho_1\rangle$ and $s_2 = \langle
\sigma_2, \iota_2, \rho_2\rangle$ are equivalent, written as
$s_1 \sim_\lab s_2$, iff $\sigma_1 \eq \sigma_2$, $\iota_1 = \iota_2$,
and $\rho_1  \sim_\lab \rho_2$.
\end{mydef}

\begin{myLemma}[Confinement Lemma]
\label{lem:exc:conf}
If $~\langle \sigma, \iota, \rho \rangle~\rightarrow~\langle \sigma',
\iota', \rho' \rangle$ and $\Gamma(!\rho) \not\sqsubseteq \lab$, then
$\sigma \sim_\lab \sigma'$, and $\rho \sim_\lab \rho'$.
\end{myLemma}

\begin{myThm}
\label{thm:exc:ni}
Suppose:
\begin{enumerate}
\item $\langle \sigma_1, \iota_1, \rho_1 \rangle \sim_\lab
\langle \sigma_2, \iota_2, \rho_2\rangle$
\item $\langle \sigma_1, \iota_1, \rho_1 \rangle
  \rightarrow^{*} \langle \sigma_1', \texttt{end}, [] \rangle$
\item $\langle \sigma_2, \iota_2, \rho_2 \rangle \rightarrow^{*}
  \langle \sigma_2', \texttt{end}, []\rangle$
\end{enumerate}
Then, $\sigma_1' \sim_\lab \sigma_2'$.
\end{myThm}
