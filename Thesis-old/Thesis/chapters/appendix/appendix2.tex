\clearpage
\section{Proofs of Precision for Dynamic IFC with Unstructured Control
  Flow and Exceptions}
\label{app:exc-pre}
\begin{myThm}[Precision]\label{app:thm1:exc}
Choosing any node other than the IPD to lower the pc-label will either
give unsound results or be less precise.
\end{myThm}
\begin{proof}
Consider a branch-point $b \in \mathcal{N}$ with IPD $\IPD{b} = i \in
\mathcal{N}$. \\ 
Assume that $(n \in \mathcal{N} )\neq i$ is the node where the context 
of the predicate expression in $b$ is removed. Thus, either:
\begin{itemize}
\item $b < n < i$: Then, $\exists p. n \not\in \pathG{b}{n_e}$. Thus,
  if $n$ performs an action that should not 
  have been performed in the context of the predicate expression in
  $b$, it might leak information about the predicate expression in
  $b$. 
\item $b < i < n$: Then, for any $n' \in \mathcal{N}$ such that  $i <
  n' < n$ performing an operation that should not be performed in the context of $b$ would
  be reported illicit as $n'$ would be executed in the context of
  $b$. 
\begin{itemize}
\item   If $\dom{n'}{b}$, then $\forall p. n' \in \pathG{b}{n_e}$. Hence, the
      statement $n'$ executes   irrespective of whether 
      the branch at $b$ is taken or not and hence,   does not depend on
      the predicate expression in $b$, i.e., there is   no implicit flow
      from the predicate expression in $b$ to $n'$, but   still the
      program might be rejected. 
\item   If $n'$ is a statement executing under the context of another
      branch-point $b'$, such that $\dom{b'}{b}$, then as $b'$ does
      not have any implicit flow from the predicate expression in $b$, any
      statement executing under the context of the prediate expression in
      $b'$ should not be influenced by the context of the predicate
      expression in $b$. Hence, the program might be rejected even though
      there is no information leak.
\end{itemize} 
\item $i <> n$: $\forall p.n \not\in \pathG{i}{n_e}~$ or $~\forall p.n
  \not\in \pathG{b}{n_e}$, $n$ will never be reached. Thus, the context
  of $b$ shall not be removed until $n_e$ such that $b < i <
  n_e$. Similar reasoning as  in the second case with $n = n_e$.
\end{itemize} 
 Hence, the most precise node where one can safely remove the context of
$b$ is $n =\mathit{IPD}(b) = i$.
\end{proof}

\begin{myThm}\label{app:thm2:exc}
The actual IPD of a node having $\SEN$ as its IPD is the node on
the top of the pc-stack, which lies in a previously called function.
\end{myThm}
\begin{proof}
Assume two functions $F$ and $G$ given by the CFGs $\cfg =
(\mathcal{N}, \mathcal{E}, n_s, n_e, \mathcal{L})$ and $\cfg' =
(\mathcal{N}', \mathcal{E}', n'_s, n'_e, \mathcal{L}')$,
respectively. The program's start and exit node are given by $N_s$ and
$N_e$, respectively. Consider a branch-point $b' \in \mathcal{N}'$ having
$\SEN$ as its IPD. Assume a branch-point $b \in \mathcal{N}$ such
that $b < b' < (i = \mathit{IPD}(b))$, $(i \in \mathcal{N}) \neq
\SEN$, $b$ is the last executed branch-point and 
 top of the pc-stack contains $i$. \\
$\forall p.i \in \pathG{b}{n_e}$. Thus, $\dom{i}{b'}$
such that $b < b' < i$. \\
T.S. $\not\exists n \in \pathG{b'}{i}~|~(\dom{n}{b'})
\wedge (b' < n < i)$. \\
Proof by contradiction:
Assume $\exists n.\dom{n}{b'}~|~b' < n < i$. Then the node $n$ either lies in the
function $F$ or $G$ or in another function $H$ given by the CFG $\cfg'' =
(\mathcal{N}'', \mathcal{E}'', n''_s, n''_e, \mathcal{L}'')$, such
that $F$ calls $H$ and $H$ calls $G$.
\begin{itemize}
\item $n \in \mathcal{N}'$: As $\mathit{IPD}(b') = \SEN$ and $\SEN$ is
  the last node in a function($\cfg'$), $\exists p.n \not\in \pathG{b}{n_e}$. 
\item $n \in \mathcal{N}''$: As $\forall p. n \in \pathG{b'}{N_e}$ and
  $G() < b'$, thus, $\forall p. n \in \pathG{G()}{N_e}$,
  which means $\mathit{IPD}(G()) \neq \SEN$. Hence, the top of the
  pc-stack then would have $\mathit{IPD}(G()) = (n'' \in
  \mathcal{N}'') \leq n$ and   not $i$.
\item $n \in \mathcal{N}$: When the call to $G$ or any other
  function $H$ is made, it would push $i$, IPD of the branch-point on
  the top of the  pc-stack.
\end{itemize}
Thus, $\not\exists n \in \pathG{b'}{i}~|~(\dom{n}{b'})
\wedge (b' < n < i)$. Hence, the top of the pc-stack, $i$ is the
actual IPD of any node $b'$ having $\SEN$ as its intra-procedural IPD. 
\end{proof}

\clearpage
\section{Proofs for IFC with Unstructured Control Flow}
\label{app:exc}

%%%   CONFINEMENT LEMMA
\begin{myLemma}[Confinement Lemma]
\label{app:exc:lem:conf}
If $~\langle \sigma, \iota, \rho \rangle~\rightarrow~\langle \sigma',
\iota', \rho' \rangle$ and $\Gamma(!\rho) \not\sqsubseteq \lab$, then
$\sigma \sim_\lab \sigma'$, and $\rho \sim_\lab \rho'$.
\end{myLemma}
\begin{proof}
$\Gamma(!\rho) = \pc$ in the proof that follows.\\
As $\pc \not\sqsubseteq \lab$, the nodes in the pc-stack 
that have label less than or equal to $\lab$ will remain
unchanged. Branching instructions pushing a new node would have label
of at least $\pc$ due to monotonicity of pc-stack. Even if
$\iota'$ is the IPD corresponding to the $!\rho.\ipd$, it would only
pop the top node. Thus, all the nodes that have label less than or
equal to $\lab$ will remain unchanged. Hence, $\rho \sim_\lab \rho'$.\\
To show: $\sigma \sim_\lab \sigma'$. \\
By induction on the derivation rules and case analysis on the last rule: 
\begin{itemize}
\item \refrule{es:cmd:a}, \refrule{es:cmd:c}: Similar to cases~\refrule{bs:cmd:agn}
  and~\refrule{bs:cmd:ags} of Lemma~\ref{lem:app:gpua:conf}.
\item \refrule{es:cmd:b}, \refrule{es:cmd:j}, \refrule{es:cmd:t}: $\sigma = \sigma'$
\end{itemize}
\end{proof}
%%% End of confinement

\begin{myLemma}
\label{app:exc:lem:pcstack}
If $~\langle \sigma_0, \iota_0, \rho_0 \rangle~\rightarrow^{n}~\langle
\sigma_n, \iota_n, \rho_n \rangle$ and $\forall (0 \leq i \leq
n).\Gamma(!\rho_i) \not\sqsubseteq \lab$, then  $\rho_0 \sim_\lab \rho_n$
\end{myLemma}
\begin{proof}
Proof by induction on $n$. \\
Basis: $\rho_0 \sim_\lab \rho_0$\\
IH : $\rho_0 \sim_\lab \rho_{n-1}$\\
From Definition~\ref{def:exc:pc}, all nodes labeled less than or equal
to $\lab$ of $\rho_0$ and $\rho_{n-1}$ are equal. From
Lemma~\ref{app:exc:lem:conf}, $\rho_{n-1} \sim \rho_n$ so, all nodes
labeled less than or equal to $\lab$ of $\rho_{n-1}$ and $\rho_n$ are
equal. Thus, all nodes labeled less than or equal to $\lab$ of
$\rho_0$ and $\rho_n$ are equal and by Definition~\ref{def:exc:pc},
$\rho_0 \sim \rho_n$.
\end{proof}

\begin{myLemma}{\emph{$\star$-preservation Lemma}}\\
\label{app:exc:lem:evol}
If $~\langle \sigma, \iota, \rho \rangle~ \rightarrow
~\langle \sigma', \iota', \rho' \rangle$, \\
then $\forall
 \emph{\TT{x}}. \Gamma(\sigma(\emph{\TT{x}})) = \lab\pl \wedge
 (\Gamma(!\rho) 
 \not\sqsubseteq \lab) \implies \Gamma(\sigma'(\emph{\TT{x}})) = 
  \lab' \pl   \wedge \lab' \sqsubseteq \lab$
\end{myLemma}
\begin{proof}
Proof by induction on the derivation and case analysis on the last
rule.
\begin{itemize}
\item \refrule{es:cmd:a}, \refrule{es:cmd:c}: from the premise
\item \refrule{es:cmd:b}, \refrule{es:cmd:j}, \refrule{es:cmd:t}: $\sigma = \sigma'$
\end{itemize}
\end{proof}

\begin{mycor}
\label{app:exc:cor:cor1}
If $~\langle \sigma, \iota, \rho \rangle~ \rightarrow
~\langle \sigma', \iota', \rho' \rangle$, and
$\Gamma(\sigma(\emph{\TT{x}})) = \lab\pl $ and
$\Gamma(\sigma'(\emph{\TT{x}})) = \lab'$, then 
$\Gamma(!\rho) \sqsubseteq \lab$.
\end{mycor}
\begin{proof}
Immediate from Lemma~\ref{app:exc:lem:evol}.
\end{proof}

\begin{myLemma}
\label{app:exc:lem:sigeq}
If $~\langle \sigma_0, \iota_0, \rho_0 \rangle~ \rightarrow^{\star}
~\langle \sigma_n, \iota_n, \rho_n \rangle$ and 
$\forall (0 \leq i \leq n).\Gamma(!\rho_i) \not\sqsubseteq \lab$, 
then $\sigma_0 \sim_\lab \sigma_n$
\end{myLemma}
\begin{proof}
By induction on $n$. \\
Basis: $\sigma_0 \sim_\lab \sigma_0$ by Definition~\ref{def:gpua:seq}.\\
IH: $\sigma_0 \sim_\lab \sigma_{n-1}$.\\
From IH and Definition~\ref{def:gpua:seq}, $\forall
\TT{x}.(\sigma_0(\TT{x}) \sim_\lab \sigma_{n-1}(\TT{x})$. 
From Lemma~\ref{app:exc:lem:conf}, $\sigma_{n-1} \sim_\lab \sigma_n$. Thus,
$\forall \TT{x}.(\sigma_{n-1}(\TT{x}) \sim_\lab \sigma_n(\TT{x})$\\
Assume that $\sigma_0(\TT{x}) = \TT{v}_0^k$, 
$\sigma_{n-1}(\TT{x}) = \TT{v}_{n-1}^{k'}$, and
$\sigma_n(\TT{x}) = \TT{v}_n^{k''}$ either:
\begin{itemize}
\item $(k = k') = \lab' \sqsubseteq \lab \wedge v_0 = v_{n-1}$ : 
  \begin{enumerate}
  \item $(k' = k'') = \lab' \sqsubseteq \lab \wedge v_{n-1} = v_n$: 
    $(k = k'') = \lab' \sqsubseteq \lab \wedge v_0 = v_n$. Thus, $\sigma_0(\TT{x})
    \sim_\lab \sigma_n(\TT{x})$.
  \item $k' = \lab'$ and $k''  =  \lab''\pl   \wedge
    \lab'' \sqsubseteq \lab' \sqsubseteq \lab$: By
    definition~\ref{def:gpua:veq}.5. Thus, $\sigma_0(\TT{x})
    \sim_\lab \sigma_n(\TT{x})$.
  \end{enumerate}

\item $k = \lab_1 \not\sqsubseteq \lab \wedge k' =
  \lab_2 \not\sqsubseteq \lab $: 
  \begin{enumerate}
  \item $k' = \lab_2 \not\sqsubseteq \lab \wedge k'' =
    \lab_3 \not\sqsubseteq \lab $. By
    definition~\ref{def:gpua:veq}.2, $\sigma_0(\TT{x}) \sim_\lab \sigma_n(\TT{x})$.
  \item $k' = \lab_2 \not\sqsubseteq \lab \wedge k'' =
    \lab_3\pl  $: $\lab_1 \not\sqsubseteq \lab$. Thus, by
    definition~\ref{def:gpua:veq}.5, $\sigma_0(\TT{x}) \sim_\lab \sigma_n(\TT{x})$. 
  \end{enumerate}

\item $k =   \lab_1  \pl   \wedge k' = \lab_2  \pl  $:  
  \begin{enumerate}
  \item $k' =   \lab_2  \pl   \wedge k'' = \lab_3  \pl  $: By
    definition~\ref{def:gpua:veq}.3, $\sigma_0(\TT{x}) \sim_\lab
    \sigma_n(\TT{x})$.  
  \item $k' =   \lab_2  \pl   \wedge k'' = \lab_3 \wedge (\lab_3
    \not\sqsubseteq \lab)$: By definition~\ref{def:gpua:veq}.4,
    $\sigma_0(\TT{x}) \sim_\lab \sigma_n(\TT{x})$.
  \item $k' =   \lab_2  \pl   \wedge k'' = \lab_3 \wedge (\lab_2
    \sqsubseteq \lab_3)$:   
      By corollary~\ref{app:exc:cor:cor1}, $\Gamma(!\rho_{n-1})
      \sqsubseteq \lab_2$. As $\Gamma(!\rho_{n-1}) \not\sqsubseteq 
      \lab$ and $\lab_2 \sqsubseteq \lab_3$, so $\lab_3
      \not\sqsubseteq \lab$. By definition~\ref{def:gpua:veq}.4, $\sigma_0(\TT{x}) \sim_\lab
      \sigma_n(\TT{x})$.
  \end{enumerate}

\item $k =   \lab_1  \pl   \wedge k' =
  \lab_2~s.t.~(\lab_2 \not\sqsubseteq \lab)$: Either
  \begin{itemize}
  \item $k' = \lab_2 \not\sqsubseteq \lab \wedge k'' =
    \lab_3 \not\sqsubseteq \lab$: By definition~\ref{def:gpua:veq}.4, $\sigma_0(\TT{x}) \sim_\lab
    \sigma_n(\TT{x})$
  \item $k' = \lab_2 \not\sqsubseteq \lab \wedge k'' =     \lab_3  \pl
    $: By definition~\ref{def:gpua:veq}.3, $\sigma_0(\TT{x}) \sim_\lab 
    \sigma_n(\TT{x})$
  \end{itemize}

\item $k =   \lab_1  \pl   \wedge k' = \lab_2~s.t.~ (\lab_1 \sqsubseteq \lab_2)$: 
  \begin{itemize}
  \item $k' = k'' = \lab_2$: By definition~\ref{def:gpua:veq}.4, $\sigma_0(\TT{x}) \sim_\lab
    \sigma_n(\TT{x})$
  \item $k' = \lab_2 \not\sqsubseteq \lab \wedge k'' =
    \lab_3 \not\sqsubseteq \lab$: By definition~\ref{def:gpua:veq}.4, $\sigma_0(\TT{x}) \sim_\lab
    \sigma_n(\TT{x})$
  \item $k' = \lab_2 \not\sqsubseteq \lab \wedge k'' =
    \lab_3  \pl $: By definition~\ref{def:gpua:veq}.3, $\sigma_0(\TT{x}) \sim_\lab
    \sigma_n(\TT{x})$
  \end{itemize}

\item $k = \lab_1  \wedge k' = \lab_2 \pl  ~s.t.~ (\lab_1
  \not\sqsubseteq \lab)$: 
  \begin{enumerate}
  \item $k' =   \lab_2  \pl   \wedge k'' = \lab_3  \pl  $: By
    definition~\ref{def:gpua:veq}.5, $\sigma_0(\TT{x}) \sim_\lab \sigma_n(\TT{x})$ 
  \item  $k' =   \lab_2  \pl   \wedge k'' =
    \lab_3~s.t.~(\lab_3 \not\sqsubseteq \lab)$: By
    definition~\ref{def:gpua:veq}.2, $\sigma_0(\TT{x}) \sim_\lab \sigma_n(\TT{x})$ 
  \item  $k' =   \lab_2  \pl   \wedge k'' = \lab_3~s.t.~(\lab_2
    \sqsubseteq \lab_3)$ : By corollary~\ref{app:exc:cor:cor1}, 
    $\Gamma(!\rho_{n}) \sqsubseteq \lab_2$. As $\Gamma(!\rho_{n})
    \not\sqsubseteq  \lab$ and $\lab_2 \sqsubseteq \lab_3$, so $\lab_3 
    \not\sqsubseteq \lab$. By definition~\ref{def:gpua:veq}.2,
    $\sigma_0(\TT{x}) \sim_\lab \sigma_n(\TT{x})$.
  \end{enumerate}

\item $k = \lab_1  \wedge k' =  
  \lab_2 \pl  ~s.t.~ (\lab_2 \sqsubseteq \lab_1)$: Also,
  $(\lab_2 \sqsubseteq \lab_1 \sqsubseteq \lab)$.
  \begin{enumerate}
  \item $k' =   \lab_2  \pl   \wedge k'' = \lab_3  \pl  $: As $\lab_2
    \sqsubseteq \lab$ and $\Gamma(!\rho_n) \not\sqsubseteq \lab$,
    $\Gamma(!\rho_n) \not\sqsubseteq \lab_2$.  By
    lemma~\ref{app:exc:lem:evol}, $\lab_3 \sqsubseteq \lab_2 $. Thus, 
    $\lab_3\sqsubseteq \lab_2 \sqsubseteq \lab_1$. By
    definition~\ref{def:gpua:veq}.5, $\sigma_0(\TT{x}) \sim_\lab
    \sigma_n(\TT{x})$
  \item  $k' =   \lab_2  \pl   \wedge k'' = \lab_3$: As $\lab_2
    \sqsubseteq \lab$ and $\Gamma(!\rho_n) \not\sqsubseteq \lab$,
    $\Gamma(!\rho_n) \not\sqsubseteq \lab_2$. But, by
    Lemma~\ref{app:exc:lem:pcstack}, $\Gamma(!\rho_n) \sqsubseteq
    \lab_2$. By contradiction, this case does not hold.
  \end{enumerate}
\end{itemize}
\end{proof}


%%% SUPPORTING LEMMA 1
\begin{myLemma}
\label{app:exc:lem:sl1}
Suppose \\
$\langle \sigma_1, \iota, \rho_1 \rangle~\rightarrow~\langle 
\sigma_1', \iota_1',\rho_1' \rangle$, \\
$\langle \sigma_2, \iota, \rho_2 \rangle~\rightarrow~\langle  
\sigma_2', \iota_2',\rho_2' \rangle$, \\
$\sigma_1 \sim_\lab \sigma_2$, 
$\rho_1 \sim_\lab \rho_2$, 
$\Gamma(!\rho_1) = \Gamma(!\rho_2) \sqsubseteq \lab$, and
either $\Gamma(!\rho_1') = \Gamma(!\rho_2') \sqsubseteq \lab)$ or
$\Gamma(!\rho_1') \not\sqsubseteq \lab \wedge \Gamma(!\rho_2')
\not\sqsubseteq \lab)$ \\ 
then $\sigma_1' \sim_\lab \sigma_2'$ and $\rho_1' \sim_\lab \rho_2'$.
\end{myLemma}
\begin{proof}
Every instruction executes \emph{isIPD} at the end of the operation. 
If $\iota_i'$ is the IPD corresponding to the $!\rho_i.\ipd$, then it pops
the first node on the pc-stack. As $\rho_1 \sim \rho_2$ and
$\Gamma(!\rho_1)  = \Gamma(!\rho_2)$, $\iota_i'$ would either pop in
both the runs or in none. Thus, $\rho_1' sim \rho_2'$
(\refrule{es:cmd:b} rule is explained below). \\
Assume $\sigma_1(x) = v_1^{k_1}$, $\sigma_2(x) = v_2^{k_2}$,
$\sigma_1'(x) = v_{1'}^{k_1'}$ and $\sigma_2'(x) = v_{2'}^{k_2'}$. \\
Proof by case analysis on the instruction type:
\begin{itemize}
\item \refrule{es:cmd:a}, \refrule{es:cmd:c}:
  $\Gamma(!\rho_1) = \Gamma(!\rho_2) = \pc \sqsubseteq \lab$
  \begin{itemize} 
  \item $\pc \sqsubseteq \lab_1 \wedge \pc \sqsubseteq \lab_2$: As
    $\TT{n}^m$ is equivalent, $\sigma_1'(x) \sim_\lab \sigma_2'(x)$.  

  \item $\pc \not\sqsubseteq \lab_1 \wedge \pc \not\sqsubseteq
    \lab_2$: By Definition~\ref{def:gpua:veq}.3, $\sigma_1'(x)
    \sim_\lab \sigma_2'(x)$.   

  \item $\pc \sqsubseteq \lab_1 \wedge
    \pc  \not\sqsubseteq \lab_2$: $k_2' \sqsubseteq \pc$ and $\pc
    \sqsubseteq k_1'$. By Definition~\ref{def:gpua:veq}.3
    and~\ref{def:gpua:veq}.4, $\sigma_1'(x) \sim_\lab \sigma_2'(x)$.
    Similarly for the analogous case. 
  \end{itemize}

\item \refrule{es:cmd:b}: As $\TT{b}^{\lab_i}$ is equivalent in the
  two runs, either $\lab_1 = \lab_2 \sqsubseteq \lab$ or $\lab_1
  \not\sqsubseteq \lab \wedge \lab_2 \not\sqsubseteq \lab$ ($\lab_i$
  does not have $\pl$). 
  The IPD of $\iota$ would be the same in both the cases. If the IPD
  is $\SEN$, then the label of $!\rho_i$ is joined with the
  label obtained above, which is either less than or equal to $\lab$
  and same in both the runs (or) not less than or equal to $\lab$ in
  both the runs. Thus, either $\Gamma(!\rho_1') = \Gamma(!\rho_2')$ or
  $\Gamma(!\rho_1') \not\sqsubseteq \lab \wedge \Gamma(!\rho_2')
  \not\sqsubseteq \lab$.
  Because $\rho_1 \sim \rho_2$, $\rho_1' \sim_\lab \rho_2'$. \\
  If the IPD is not $\SEN$, then it is some other node, which makes the
  $\ipd$ field the same.  
  Thus, the pushed node is the same in both the cases or has label not
  less than or equal to $\lab$ and hence, $\rho_1' \sim_\lab \rho_2'$
  $\sigma'_1 = \sigma_1 \sim_\lab \sigma_2 = \sigma'_2$.

\item Other rules: $\sigma'_1 = \sigma_1 \sim_\lab \sigma_2 = \sigma'_2$.
\end{itemize}
\end{proof}
%%% End of supporting lemma 1



%%% SUPPORTING LEMMA 2
\begin{myLemma}
\label{app:exc:lem:sl2}
 Suppose
\begin{enumerate}
\item $\langle \sigma_0', \iota_0, \rho_0' \rangle~\rightarrow~\langle
  \sigma_1', \iota_1', \rho_1' \rangle~\rightarrow^{n-1}~ \langle
  \sigma_n', \iota_n', \rho_n' \rangle$,
\item $\langle \sigma_0'', \iota_0, \rho_0'' \rangle~\rightarrow~
  \langle \sigma_1'', \iota_1'', \rho_1''
  \rangle~\rightarrow^{m-1}~\langle \sigma_m'', \iota_m'', \rho_m''
  \rangle $, 
\item $(\rho_0'  \sim_\lab \rho_0'')$, $(\sigma_0' \sim_\lab \sigma_0'')$, 
\item $ (\Gamma(!\rho_0')=\Gamma(!\rho_0'') \sqsubseteq \lab)$,
  $(\Gamma(!\rho_n')=\Gamma(!\rho_m'') \sqsubseteq \lab)$,
\item $\forall(0 < i < n).(\Gamma(!\rho_i') \not\sqsubseteq \lab)~\wedge~
\forall(0 < j < m).(\Gamma(!\rho_j'') \not\sqsubseteq \lab)$,
\end{enumerate}
then $(\iota_n' = \iota_m'')$, $(\rho_n' \sim_\lab \rho_m'')$, and
$(\sigma_n' \sim_\lab \sigma_m'')$. 
\end{myLemma}
\begin{proof}
  Starting with the same instruction and high context in both the runs
  can result in two different instructions, $\iota_1'$ and
  $\iota_1''$. This is only possible if $\iota$ was some branching
  instruction in the first place and this divergence happened in a
  high context. 
  \begin{enumerate}
  \item To prove $\iota_n' = \iota_m''$: \\
    From the property of the IPDs, 
    if $\iota_0$ pushes a node with label
    $\not\sqsubseteq \lab$ on top of $\pc$-stack
    which was originally $\sqsubseteq \lab$, $\IPD{\iota_0}$ pops
    that node. Since the runs start from the same instruction $\iota_0$,
    $\iota_n' = \iota_m'' = \IPD{\iota}$, where $\Gamma(!\rho)
    \sqsubseteq \lab$.  
  
  \item To prove $\rho_n' \sim_\lab \rho_m''$: 
    \begin{itemize}
      \item $n > 1$ and $m > 1$: $\Gamma(!\rho_1') \not\sqsubseteq
        \lab \wedge \Gamma(!\rho_1'') \not\sqsubseteq \lab$, because
        $\iota_0$ has the same IPD and $\iota_1',\iota_1''$ are not
        the IPDs. As $\rho_0' \sim \rho_0''$ and $\Gamma(!\rho_1')
        \not\sqsubseteq \lab \wedge \Gamma(!\rho_1'') \not\sqsubseteq \lab$, 
        from Lemma~\ref{app:exc:lem:sl1}, 
        $\rho_1' \sim \rho_1''$ and $!\rho_1'.\ipd = !\rho_1''.\ipd =
        \IPD{\iota_0}$, if $\iota_1' \neq \IPD{\iota_0}$ and $\iota_1''
        \neq \IPD{\iota_0}$. As $\iota_n' = \iota_m'' = \IPD{\iota_0}$, it pops
        the $!\rho_1'$ and $!\rho_1''$, which correspond to $\rho_n'$ and $\rho_m''$ in
        the $n$th and $m$th step. 
        Because $\rho_1' \sim \rho_1''$ and from
        Lemma~\ref{app:exc:lem:pcstack}, $\rho_n' \sim \rho_m''$.
      \item $n = 1$ and $m > 1$: If $\iota_1' =
        \IPD{\iota_0}$, and $\Gamma(!\rho_1') \sqsubseteq \lab$. 
        It pops the node pushed by $\iota_0$, i.e., $\Gamma(!\rho_n')
        \sqsubseteq \lab$. In the other run as
        $\Gamma(!\rho_1'') \not\sqsubseteq \lab$ and $\Gamma(!\rho_m'')
        \sqsubseteq \lab$, by the property of IPD $\iota_m'' =
        \IPD{\iota_0}$, which would pop from the $\pc$-stack $!\rho_m''$,
        the first frame labelled $\not\sqsubseteq \lab$ on the
        $\pc$-stack. Thus, $\rho_n' \sim \rho_m''$. 
       \item $n > 1$ and $m = 1$: Similar to the above case. 
       \end{itemize}
     \item To prove $\sigma_n' \sim_\lab \sigma_m''$: 
       \begin{enumerate}
       \item $n > 1$ and $m > 1$: 
         From Lemma~\ref{app:exc:lem:sl1}, $\sigma_1' \sim_\lab
         \sigma_1''$. From Lemma~\ref{app:exc:lem:sigeq}, $\sigma_1'
         \sim_\lab \sigma_{n-1}'$ and $\sigma_1''
         \sim_\lab \sigma_{m-1}''$. And from Lemma~\ref{app:exc:lem:conf}
         $\sigma_{n-1}' \sim_\lab \sigma_n'$ and $\sigma_{m-1}''
         \sim_\lab \sigma_m''$. Similar case analysis as above for
         different cases of equivalence. 
       \item $n = 1$ and $m > 1$: In case of \refrule{es:cmd:b}, $\sigma_0' =
         \sigma_1'$ and $\sigma_0'' = \sigma_1''$. Thus, $\sigma_1' \sim_\lab
         \sigma_1''$. From the above case, if  $\sigma_1' \sim_\lab
        \sigma_1''$, then  $\sigma_n' \sim_\lab \sigma_m''$.
      \item $n > 1$ and $m = 1$: Symmetric case of the above. 
      \end{enumerate}
   
    \end{enumerate}
  
\end{proof}

\begin{mydef}[Trace]
\label{def:trace}
A trace is defined as a sequence of configurations or states resulting from a
program evaluation, i.e., for a program evaluation $\mathcal{P} = s_1 \rightarrow s_2
\rightarrow \ldots \rightarrow s_n$ where $s_i = \langle \sigma_i,
\iota_i, \rho_i\rangle$, the corresponding trace is given 
as $\mathcal{T}(\mathcal{P}) := s_1 :: s_2 :: \ldots :: s_n$.
\end{mydef}

\begin{mydef}[Epoch-trace]
\label{def:etrace}
An epoch-trace for an adversary at level $\lab$, ($\mE_\lab$) over a
trace $\mathcal{T} = s_1 :: s_2 :: \ldots :: s_n$ where $s_i = \langle
\sigma_i, \iota_i, \rho_i\rangle$  is defined inductively as:
\begin{align*}
  \mE_\lab(nil) &:=~nil \\
  \mE_\lab(s_i :: \mathcal{T}) &:=
  \begin{cases}
   s_i :: \mE(\mathcal{T})~ & \mathit{if}~~~\Gamma(!\rho_i)
   \sqsubseteq \lab,\\
   \mE(\mathcal{T}) & \mathit{else~if} ~~\Gamma(!\rho_i)
   \not\sqsubseteq \lab.
  \end{cases}
\end{align*}
\end{mydef}

\begin{myThm}[Termination-Insensitive Non-interference]
\label{app:exc:thm:ni} 
Suppose $\mathcal{P}$ and  $\mathcal{P}'$ are two program
evaluations. \\
Then for their respective epoch-traces with respect to an adversary at
level $\lab$ given by:\\
\hspace{5 mm} $\mE_\lab(\mathcal{T}(\mathcal{P})) = s_1 :: s_2 :: \ldots ::
s_n$,\\
\hspace{5 mm} $\mE_\lab(\mathcal{T}(\mathcal{P}')) = s_1' :: s_2' :: \ldots ::
s_m'$,\\
if $s_1 \sim_\lab s_1'$ and $n \leq m$,\\
then\\
\hspace{5 mm} $s_n \sim_\lab s_n'$
\end{myThm}
\begin{proof}
Proof by induction on $n$.\\
Basis: $s_1 \sim_\lab s_1'$, by assumption. \\
IH: $s_k \sim_\lab s_k'$ \\
To prove: $s_{k+1}
\sim_\lab s_{k+1}'$. \\
Let $s_k \rightarrow_i s_{k+1}$ and $s_k \rightarrow_{i'} s'_{k+1}$, then:
\begin{itemize}
\item $i = i' = 1$: From Lemma~\ref{app:exc:lem:sl1}, $s_{k+1}
  \sim_\lab s_{k+1}'$.
\item $i > 1$ or $i' > 1$: From Lemma~\ref{app:exc:lem:sl2}, $s_{k+1}
  \sim_\lab s_{k+1}'$.
\end{itemize}
\end{proof}

\begin{mycor}
\label{app:exc:cor:ni}
Suppose:
\begin{enumerate}
\item $\langle \sigma_1, \iota_1, \rho_1 \rangle \sim_\lab
\langle \sigma_2, \iota_2, \rho_2\rangle$
\item $\langle \sigma_1, \iota_1, \rho_1 \rangle
  \rightarrow^{*} \langle \sigma_1', \texttt{end}, [] \rangle$
\item $\langle \sigma_2, \iota_2, \rho_2 \rangle \rightarrow^{*}
  \langle \sigma_2', \texttt{end}, []\rangle$
\end{enumerate}
Then, $\sigma_1' \sim_\lab \sigma_2'$.
\end{mycor}
\begin{proof}
$\sigma_1, \sigma_2$ and $\rho_1, \rho_2$ are empty at the end of $*$
steps. From the semantics, in context $\sqsubseteq \lab$
both runs would push and pop the same number of nodes. Thus, both take
same number of steps in the epoch-trace. Assume it to be $k$ . Then in
Theorem~\ref{app:exc:thm:ni}, $n = m = k$. Thus, $s_k \sim_\lab s_k'$, where
$s_k = \langle \sigma_1', \texttt{end}, []\rangle$ and $s_k' = \langle
\sigma_2', \texttt{end}, [] \rangle$. By Definition~\ref{def:exc:ceq},
$\sigma_1' \sim_\lab  \sigma_2'$.
\end{proof}
