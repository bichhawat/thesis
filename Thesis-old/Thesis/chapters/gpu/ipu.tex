\chapter{Improved Permissive-Upgrade}
\label{ch:ipu}

\begin{lstlisting}[float,label=lst1.1,caption=Impermissiveness of the NSU strategy][escapechar=@]
x = false
if (not(z))
  x = true@\label{lineref1}@
if (y) f() else g()
x = false@\label{lineref2}@
\end{lstlisting}
%
The no-sensitive-upgrade (NSU) check described earlier provides the basic
foundations for sound dynamic IFC. However, terminating a program
preemptively because of the NSU check is quite restrictive in practice. For
example, consider the program of Listing~\ref{lst1.1}, where \TT{z} is
labeled $H$ and \TT{y} is labeled $L$. This program potentially upgrades
variable \TT{x} at line~\ref{lineref1} under a high $pc$, and then
executes function \texttt{f} when \TT{y} is \texttt{true} and
executes function \texttt{g} otherwise. Suppose that \texttt{f}
does not read \TT{x}. Then, for $\TT{y} \mapsto \texttt{true}^L$, this program
leaks no information, but the NSU check would terminate this program
prematurely at line~\ref{lineref1}. (Note: \texttt{g} may read \TT{x},
so \TT{x} is not a dead variable at line~\ref{lineref1}.)

To improve permissiveness, Austin and Flanagan~\cite{plas10} proposed
the permissive-upgrade strategy as a replacement for NSU. However,
that development lacks permissiveness in certain cases. 
This chapter presents the soundness results of the permissive-upgrade
strategy with the improvement for further permissiveness in place.
 % are presented here 
% (using the modified notation for the imperative language)
% and the next
% chapter builds a generalization of the permissive-upgrade strategy to
% arbitrary lattices.


\section{Overview}
This section presents a \emph{formal} description of the
no-sensitive-upgrade check. The technical development in this thesis
is mostly based on the simple imperative language shown in
Figure~\ref{basic:syntax}. However, the key ideas are orthogonal to
the choice of language and generalize to 
other languages easily. The use of a simpler language is to simplify
non-essential technical details. The parts in this thesis that require
a more complex language define the additional features in place. 
The language's expressions include constants or values (\TT{n}),
variables ($\TT{x}$) and unspecified binary operators ($\odot$) to 
combine them. The set of variables is fixed upfront. Labels ($\lab$)
are drawn from a fixed security lattice. The lattice contains
different labels $\{L, M, H, \ldots \}$ with a partial ordering
between the elements. Join ($\sqcup$) and meet ($\sqcap$) operations
are defined as usual on the lattice. The program counter label $\pc$
is an element of the lattice.

\subsection{Basic IFC Semantics}
%%%%%%%%%%%%%%%%%%%%%%%%%%%%%%%%%%%%
\begin{figure}
\begin{align*}
\expr	=~& \TT{n}~\arrowvert~\TT{x}~\arrowvert~\expr_1 \odot \expr_2 \\
\comm	=~& \sk~\arrowvert~\TT{x} := \expr~\arrowvert 
 ~\comm_1;\comm_2~\arrowvert~\texttt{if}~\expr~\texttt{then}~\comm_1~\texttt{else}~\comm_2~\arrowvert~\texttt{while}~\expr~\texttt{do}~\comm\\
\lab          =~& L~\arrowvert~M~\arrowvert~H~\arrowvert~\ldots\\
k,l,m, \pc =~& \lab
\end{align*}
\caption{Syntax of the Language}\label{basic:syntax}
\end{figure}

%%%%%%%%%%%%%%%%%%%%%%%%%%%%%%%%%%%%

%%%%%%%%%%%%%%%%%%%%%%%%%%%%%%%%%%%%
\begin{figure*}
\begin{framed}
\begin{mathparpagebreakable}
%\onecolumn
%%%    
\textsc{Expressions:}\\
\inferrule*[left=\mbox{\labelthis{bs:exp:c}{const}}]
{ }
{\langle \sigma, \TT{n} \rangle \bsexp \TT{n}^{\perp}}
\and
\inferrule*[left=\mbox{\labelthis{bs:exp:v}{var}}]
{\TT{n}^k := \sigma(\TT{x})}
{\langle \sigma, \TT{x} \rangle \bsexp \TT{n}^k}
\and
\inferrule*[left=\mbox{\labelthis{bs:exp:o}{oper}}]
{\langle \sigma, \expr' \rangle \bsexp
  \TT{n}'^{k'} \\ \langle \sigma, \expr'' \rangle \bsexp
  \TT{n}''^{k''} \\\\ \TT{n} := \TT{n}' \odot \TT{n}'' \\ 
  k := k' \sqcup k''} 
{\langle \sigma, \expr' \odot \expr'' \rangle \bsexp \TT{n}^k}
%%%    
\\\\
\textsc{Statements:}\\
\inferrule*[left=\mbox{\labelthis{bs:cmd:sk}{skip}}]
{ }
{\langle \sigma, \sk \rangle \bscmd \sigma}
\and
%%%
\inferrule*[left=\mbox{\labelthis{bs:cmd:s}{seq}}]
{\langle \sigma, \comm_1 \rangle \bscmd \sigma'' \\ \langle \sigma'', \comm_2 \rangle \bscmd \sigma' }
{\langle \sigma, \comm_1;\comm_2 \rangle \bscmd \sigma'}
\and
%%%
\inferrule*[left=\mbox{\labelthis{bs:cmd:wf}{while-f}}]
{\langle \sigma, \expr \rangle \bsexp \texttt{false}^{\lab}}
{\langle \sigma, \texttt{while}~\expr~\texttt{do}~\comm \rangle
  \bscmd \sigma }
\and
%%%    
\inferrule*[left=\mbox{\labelthis{bs:cmd:ie}{if-else}}]
{\langle \sigma, \expr \rangle \bsexp \TT{b}^{\lab} \\ 
  i = \left\{\begin{array}{ll}
        1, & \textit{if }~  \TT{b} = \TT{true} \arcr
        2, & \textit{otherwise } \arcr
        \end{array}\right\} \\\\ \langle
  \sigma, \comm_i \rangle \bscmdup \sigma'}
{\langle \sigma, 
  \texttt{if}~\expr~\texttt{then}~\comm_1~\texttt{else}~\comm_2
  \rangle \bscmd \sigma'}
\and
% \inferrule*[left=\mbox{\labelthis{bs:cmd:iet}{if-else-t}}]
% {\langle \sigma, \expr \rangle \bsexp \texttt{true}^{\lab} \\ \langle
%   \sigma, \comm_1 \rangle \bscmdup \sigma'}
% {\langle \sigma, 
%   \texttt{if}~\expr~\texttt{then}~\comm_1~\texttt{else}~\comm_2
%   \rangle \bscmd \sigma'}
% \and
% %%%    
% \inferrule*[left=\mbox{\labelthis{bs:cmd:ief}{if-else-f}}]
% {\langle \sigma, \expr \rangle \bsexp \texttt{false}^{\lab} \\ \langle
%   \sigma, \comm_2 \rangle \bscmdup \sigma'}
% {\langle \sigma, 
%   \texttt{if}~\expr~\texttt{then}~\comm_1~\texttt{else}~\comm_2
%   \rangle \bscmd \sigma'}
% \and
%%%    
\inferrule*[left=\mbox{\labelthis{bs:cmd:wt}{while-t}}]
{\langle \sigma, \expr \rangle \bsexp \texttt{true}^{\lab} \\ \langle
  \sigma, \comm \rangle \bscmdup \sigma'' \\\\ \langle
  \sigma'', \texttt{while}~\expr~\texttt{do}~\comm \rangle
  \bscmdup \sigma' }
{\langle \sigma, \texttt{while}~\expr~\texttt{do}~\comm \rangle
  \bscmd \sigma'}
%%%    
\end{mathparpagebreakable}
\end{framed}
\caption{Semantics}\label{fig:basic-semantics}
\end{figure*}

The rules in Figure~\ref{fig:basic-semantics} define the big-step
semantics of the language, including standard taint propagation for
IFC: the evaluation relation $\langle \sigma, \expr \rangle \bsexp
\TT{n}^k$ for expressions, and the evaluation relation $\langle \sigma,
\comm \rangle \bscmd \sigma'$ for commands. Here, $\sigma$ 
denotes a store, a map from variables to labeled values of the form
$\TT{n}^k$. \TT{b} represents a Boolean constant. For now, labels $k ::=
\lab$; this is generalized later when the ``partially-leaked'' taints
are introduced in Section~\ref{sec:existing}. 

The evaluation relation for expressions evaluates an expression
$\expr$ and returns its value $n$ and label $k$. The label $k$ is the
join of labels of all variables occurring in $\expr$ (according to
$\sigma$). The relation for commands executes a command $\comm$ in the
context of a store $\sigma$, and the current program counter label
$\pc$, and yields a new store $\sigma'$.  The function
$\Gamma(\sigma(\TT{x}))$ returns the label associated with the value in \TT{x}
in store $\sigma$: If $\sigma(\TT{x}) = \TT{n}^k$, then $\Gamma(\sigma(\TT{x})) =
k$. $\bot$ denotes the least element of the lattice. % Here, $\bot = L$.

% The rules for evaluating commands are explained below. 
The rule for
sequencing $\comm_1; \comm_2$ (\refrule{bs:cmd:s}) evaluates the
command $c_1$ under store $\sigma$ and the current $\pc$ label; this
yields a new store $\sigma''$. It then evaluates the command $c_2$
under store $\sigma''$ and the same $\pc$ label, which yields the
final store $\sigma'$. 
%
The rule for \texttt{if-else} (\refrule{bs:cmd:ie}) 
% and \refrule{bs:cmd:ief}) 
evaluates the branch condition $e$ to a Boolean value \TT{b} with
label $\lab$. Based on the value of \TT{b}, one of the branches
$\comm_1$ and $\comm_2$ is executed under a $\pc$ obtained by 
joining the current $\pc$ and the label $\lab$ of \TT{b}. Similarly, the
rules for \texttt{while} (\refrule{bs:cmd:wt} and \refrule{bs:cmd:wf})
evaluate the loop condition $e$ and execute the loop command $c_1$
while $e$ evaluates to \texttt{true}. The $\pc$ for the loop is
obtained by joining the current $\pc$ and the label 
$\lab$ of the result of evaluating $e$.

%%%%%%%%%%%%%%%%%%%%%%%%%%%%%%%%%%%%
\begin{figure*}
\begin{mathparpagebreakable}
\inferrule*[left=\mbox{\labelthis{bs:cmd:an}{assn-nsu}}]
 {l = \Gamma(\sigma(\TT{x})) \\ \pc
    \sqsubseteq l \\ \langle \sigma, \expr \rangle \bsexp \TT{n}^m
    } {\langle \sigma, \TT{x} := \expr \rangle
    \bscmd \sigma[\TT{x} \mapsto \TT{n}^{(\pc \sqcup m)}]}
\end{mathparpagebreakable}
\caption{Assignment rule for NSU}
\label{fig:assn-nsu}
\end{figure*}
%%%%%%%%%%%%%%%%%%%%%%%%%%%%%%%%%%%%

The rule for assignment statements are conspicuously missing from
Figure~\ref{fig:basic-semantics} because they depend on the strategy
used to control implicit flows. 
% In the remainder of this chapter, a number of such rules are
% considered. To start, t
The rule for assignment (\refrule{bs:cmd:an}) corresponding to the NSU
check is shown in Figure~\ref{fig:assn-nsu}. The rule checks that the
label $l$ of the assigned variable \TT{x} in the initial store $\sigma$
is at least as high as $\pc$ (premise $\pc \sqsubseteq l$). If this
condition is not true, the program gets stuck. This is exactly the NSU
check described in Section~\ref{sec:bg-nsu}.

\subsection{Formalization of the No-sensitive-upgrade Check}

% Figure~\ref{fig:basic-syntax} and \ref{fig:basic-semantics}
% describes the syntax and semantics  of a standard while language
% with information flow  control. Notice the assignment rule often
% imposes some condition  on the $pc$ to prohibit low assignments
% under high context. For  instance, for the no-sensitive upgrade
% check as described earlier  the assignment rule.

% \subsection{Termination-Insensitive Non-interference}

For establishing and proving the security
property of termination-insensitive non-interference (TINI), the
observational power of the adversary needs to be defined. 
An adversary at level $\attacker$ in the lattice is allowed to view
all values that have a label less than or equal to $\attacker$. To prove
the security property of non-interference, it is enough to show that
when executing a program beginning with two different memory stores
that are \emph{observationally equivalent} to an adversary, the final
memory stores are also \emph{observationally equivalent} to the
adversary. For this, the observational equivalence of two memory
stores with respect to an adversary needs to be defined. 
Store equivalence is formalized as a relation $\eq$,
indexed by lattice elements $\attacker$, representing the adversary.

\begin{mydef}[Value equivalence]
  Two labeled values $\emph{\TT{n}}_1^k$ and $\emph{\TT{n}}_2^m$ are $\lab$-equivalent,
  written $\emph{\TT{n}}_1^k \sim_\lab \emph{\TT{n}}_2^m$, iff either:
  \begin{enumerate}
  \item $(k = m) \sqsubseteq \lab$ and $\emph{\TT{n}}_1 = \emph{\TT{n}}_2$ or
  \item $k \not\sqsubseteq \lab$ and $m \not\sqsubseteq \lab$
  \end{enumerate}
\end{mydef}

This definition states that for an adversary at security level $\lab$,
two labeled values $\TT{n}_1^k$ and $\TT{n}_2^m$ are equivalent iff either
$\lab$ can access both values and $\TT{n}_1$ and $\TT{n}_2$ are equal, or it
cannot access either value ($k \not\sqsubseteq \lab$ and $m
\not\sqsubseteq \lab$). The additional constraint $k = m$ in clause
(1) is needed to prove non-interference by induction. In the lattice
$L \sqsubset H$, two values labeled $L$ and $H$ are distinguishable
for the $L$-adversary. 

\begin{mydef}[Store equivalence]
  Two stores $\sigma_1$ and $\sigma_2$ are $\lab$-equivalent,
  written $\sigma_1 \sim_\lab \sigma_2$, iff for every variable \emph{\TT{x}},
  $\sigma_1(\emph{\TT{x}}) \sim_\lab \sigma_2(\emph{\TT{x}})$.
\end{mydef}

The following theorem states TINI for the NSU check. The theorem has
been proved for various languages in the past.

\begin{myThm}[TINI for NSU]
  With the assignment rule \refrule{bs:cmd:an} from
  Figure~\ref{fig:assn-nsu}, if 
  $~\sigma_1 \sim_\lab \sigma_2$ and $\langle \sigma_1, c \rangle
  \bscmd \sigma_1' $ and $\langle \sigma_2, c
  \rangle \bscmd \sigma_2' $, then $\sigma_1' \sim_\lab
  \sigma_2'$.
\end{myThm}
\begin{proof} Standard, see e.g.,~\cite{plas09}
\end{proof}

% Although the security lattice here is restricted to two elements $L$
% and $H$, the rules of Figures~\ref{fig:basic-semantics}
% and~\ref{fig:assn-nsu}, the definition of equivalence above and the
% theorem above (for NSU) are all general and work for arbitrary
% lattices.
  
%%%%%%%%%%%%%%%%%%%%%%%%%%%%%%%%%%%%%%%%%%%%%%%%%%%%%%%%%%%%%

%%%%%%%%%%%%%%%%%%%%%%%%%%%%%%%%%%%%%%%%%%%%%%%%%%%%%%%%%%%%%

\section{Austin and Flanagan's Permissive-Upgrade Strategy}
\label{sec:existing}

To allow a dynamic IFC analysis to accept safe executions of programs
with variable upgrades due to high $\pc$, Austin and Flanagan proposed
a less restrictive strategy called the \emph{permissive-upgrade
  strategy}~\cite{plas10}. They study this strategy for a two-point
lattice $L \sqsubset H$ and their strategy does not immediately
generalize to arbitrary security lattices. Whereas NSU stops a program  
when a variable's label is upgraded due to assignment in a high $pc$,
permissive-upgrade allows the assignment, but labels the variable as 
\emph{partially-leaked} or $P$. The exact intuition behind the
partially-leaked label $P$ is the following: 

\begin{framed}
\noindent
  A variable with a value labeled $P$ may have been implicitly
  influenced by $H$-labeled values in this execution, but in other
  executions (obtainable by changing $H$-labeled values in the
  initial store), this implicit influence may not exist and, hence,
  the variable may be labeled $L$.
\end{framed}

%  The taint $P$ roughly means that the
% variable's content in this execution is $H$, but it may be $L$ in
% other executions.
The program must be stopped later if it tries to use
or case-analyze the variable (in particular, branching on a
partially-leaked Boolean variable is stopped). Permissive-upgrade also
ensures termination-insensitive non-interference, but is strictly more
permissive than NSU. For example, permissive-upgrade stops the leaky
program of Listing~\ref{lst1} at line~\ref{linerefcond} when \TT{z}
contains $\texttt{false}^H$, but it allows the program of
Listing~\ref{lst1.1} to execute to completion when \TT{y} contains
$\texttt{true}^L$. 
% Note that in this section, the lattice has
% only two levels: $L$ (public) and $H$ (confidential). 

In the revised syntax of labels, summarized in
Figure~\ref{pus:syntax}, the labels $k,l,m$ on values can be either
elements of the lattice  ($L, H$) or $P$. The $\pc$ can only be one of
$L, H$ because branching on partially-leaked values is
prohibited. The join operation $\sqcup$ is lifted to labels including
$P$. Joining any label with $P$ results in $P$. For brevity in
definitions, they extend the order $\sqsubset$ to $L \sqsubset H
\sqsubset P$. However, $P$ is not a new ``top'' member of the lattice
because it receives special treatment in the semantic rules. 

\begin{figure}
\begin{equation*}
\begin{aligned}[c]
\lab          =~& L~\arrowvert~H\\
\pc          =~& \lab\\
k,l,m =~& \lab~\arrowvert~P
\end{aligned}
\qquad \qquad \qquad
\begin{aligned}[c] 
k \sqcup k ~= ~& k\\
L \sqcup H ~=~ &H\\
L \sqcup P  ~=~&P \\
H\sqcup P  ~=~&P
\end{aligned}
% \end{align*}
\end{equation*}
\caption{Syntax of labels including the partially-leaked label
  $P$}\label{pus:syntax}
\end{figure}


The rule for assignment with permissive-upgrade is
\begin{mathpar}
\inferrule*[left=\mbox{\labelthis{bs:cmd:ap}{assn-pus}}]
{l := \Gamma(\sigma(\TT{x})) \qquad \langle
 \sigma, \expr \rangle \bsexp \TT{n}^m} {\langle \sigma, \TT{x} := \expr
    \rangle \bscmd \sigma[\TT{x} \mapsto \TT{n}^{k}]}
\end{mathpar}
where $k$ is defined as follows:
$$ k = % \left\lbrace
\begin{cases}
m & \mbox{ if } \pc = L \\
m \sqcup H  & \mbox{ if } \pc = H \mbox{ and } l = H\\
P  & \mbox{ otherwise}
\end{cases} % \right.
$$


% The rule for assignment with permissive-upgrade is
% \begin{align*}
%   \inference[assn-PUS: ] {l := \Gamma(\sigma(x)) \qquad \langle
%     \expr, \sigma \rangle \Downarrow n^m} {\langle x := \expr, \sigma
%     \rangle \Downarrow_\pc \sigma[x \mapsto n^{k}]}
% \end{align*}
% where $k$ is defined as follows:
% \[ k = \left\lbrace
% \begin{array}{l}
% m \mbox{ if } \pc = L \\
% m \sqcup H  \mbox{ if } \pc = H \mbox{ and } l = H\\
% P \mbox{ otherwise}
% \end{array} \right.
% \]
The first two conditions in the definition of $k$ correspond to the
NSU rule (Figure~\ref{fig:assn-nsu}). The third condition applies, in
particular, when a variable whose initial label is $L$ is assigned 
with $\pc = H$. The NSU check would stop this assignment. With
permissive-upgrade, however, the updated variable is labeled $P$,
consistent with the intuitive meaning of $P$. This allows 
more permissiveness by allowing the assignment to proceed in all
cases. To compensate, any program (in particular, an adversarial
program) is disallowed from case analyzing any value labeled 
$P$. Consequently, in the rules for \texttt{if-then} and
\texttt{while} (Figure~\ref{fig:basic-semantics}), the label of the
branch condition is of the form $\lab$, which does not include $P$. 
Thus, assignments under high $\pc$ succeed under the
permissive-upgrade check but branching or case-analyzing a
partially-leaked value is not permitted as that can also leak
information. 
% To illustrate the permissivness of the permissive-upgrade
% check, consider the different executions of example in
% Listing~\ref{lst1} taken from~\cite{plas10} as shown in
% Table~\ref{tbl:nsu}. The assignment  on line~\ref{lineref} is not
% permitted by the NSU check. However, the permissive-upgrade check
% updates \TT{y} to \texttt{false} with a partially-leaked
% label. Further case analysis on \TT{y} is prohibited (if 
% not, the branch on line~\ref{linerefcond} would not be taken as \TT{y} is
% \texttt{false} and \TT{z} would return $\texttt{true}^{L}$, thus leaking
% the value of \TT{x}).

% \begin{table*}
% \centering
% \begin{tabular} {|l||@{\,}c@{\,}||@{\,}c@{\,}|@{\,}c@{\,}|}
% \hline
% &
% $\TT{x} = \texttt{false}^{H}$
% &
% \multicolumn{2}{c|}{$\TT{x} = \texttt{true}^{H}$}
% \\
% \cline{2-4}
% &
% \textit{All strategies}
% &
% \textit{NSU}
% &
% \textit{Permissive-upgrade}
% \\
% \hline
% \TT{y} = \texttt{true} &$\TT{y} = \texttt{true}^{L}$&$\TT{y} =
%                                               \texttt{true}^{L}$&$\TT{y} =
%                                                                   \texttt{true}^{L}$\\
% \TT{z} = \texttt{true} &$\TT{z} = \texttt{true}^{L}$&$\TT{z} =
%                                               \texttt{true}^{L}$&$\TT{z} =
%                                                                   \texttt{true}^{L}$\\
% \texttt{if} (\TT{x})&branch not taken&$\pc = H$&$\pc = H$\\
% \quad\TT{y} = \texttt{false}&$\TT{y} = \texttt{true}^{L}$&execution halted&$\TT{y} = \texttt{false}^{P}$\\
% \texttt{if} (\TT{y})&$\pc = L$&&execution halted\\
% \quad$\TT{z} = \texttt{false}$&$\TT{z} = \texttt{false}^{L}$&&\\
% \texttt{return} \TT{z}&$\texttt{false}^{L}$&&\\
% \hline
% Result & $\TT{z} = \texttt{false}^{L}$ & no leak & no leak\\
% \hline
% \end{tabular}
% \caption{Execution steps in two runs of the program from
%   Listing~\ref{lst1}, with NSU and the permissive-upgrade check}
% \label{tbl:nsu}
% \end{table*}


The noninterference result obtained for NSU earlier can be extended to
permissive-upgrade by changing the definition of store 
equivalence. Because no program can case-analyze a $P$-labeled value,
such a value is equivalent to any other labeled value.

\begin{mydef}
\label{def:pus}
  Two labeled values $\emph{\TT{n}}_1^k$ and $\emph{\TT{n}}_2^m$ are
  equivalent to an adversary at level $L$,
  written $\emph{\TT{n}}_1^k \sim_L \emph{\TT{n}}_2^m$, iff either:
  \begin{enumerate}
  \item $(k = m) = L$ and $\emph{\TT{n}}_1 = \emph{\TT{n}}_2$ or
  \item $k = H$ and $m = H$ or
  \item $k = P$ or $m = P$
  \end{enumerate}
\end{mydef}

\begin{mydef}
  Two stores $\sigma_1$ and $\sigma_2$ are $L$-equivalent,
  written $\sigma_1 \sim_L \sigma_2$, iff 
  $\forall \emph{\TT{x}}. \sigma_1(\emph{\TT{x}}) \sim_L \sigma_2(\emph{\TT{x}})$.
\end{mydef}

\begin{myThm}[TINI for permissive-upgrade with a two-point lattice]
  With the assignment rule \emph{assn-PUS}, if
  $~\sigma_1 \sim_L \sigma_2$ and $\langle c, \sigma_1 \rangle
  \Downarrow_\pc \sigma_1' $ and $\langle c, \sigma_2
  \rangle\Downarrow_\pc \sigma_2' $, then $\sigma_1' \sim_L
  \sigma_2'$.
\end{myThm}
\begin{proof} See~\cite{plas10}.
\end{proof}

% Note that the above definition and proof are specific to the two-point
% lattice.


\section{Improved Permissive-Upgrade Strategy}
\label{sec:ipus}

% As described in the previous chapter, the NSU check is restrictive
% and halts many programs that do not leak information. 
% To improve
% permissiveness, the permissive-upgrade strategy was proposed as a
% replacement for NSU by Austin and Flanagan~\cite{plas10}. 
% To allow a dynamic IFC to accept safe executions of programs with
% variable upgrades due to high $\pc$, Austin and Flanagan proposed a
% less restrictive strategy called
% \emph{permissive-upgrade}~\cite{plas10}. Whereas NSU stops a program 
% when a variable's label is upgraded due to assignment in a high $pc$,
% permissive-upgrade allows the assignment, but labels the variable
% \emph{partially-leaked} or $P$. The taint $P$ roughly means that the
% variable's content in this execution is $H$, but it may be $L$ in
% other executions. The program must be stopped later if it tries to use
% or case-analyze the variable (in particular, branching on a
% partially-leaked Boolean variable is stopped). Permissive-upgrade also
% ensures termination-insensitive non-interference, but is strictly more
% permissive than NSU. For example, permissive-upgrade stops the leaky
% program of Listing~\ref{lst1} at line~\ref{linerefcond} when \TT{z}
% contains $\texttt{false}^H$, but it allows the program of
% Listing~\ref{lst1.1} to execute to completion when \TT{y} contains
% $\texttt{true}^L$. Please note that in this section, the lattice has
% only two levels: $L$ (public) and $H$ (confidential). 

% \subsection{Improved Permissive-Upgrade on a Two-Point Lattice}
% This section presents an enhancement of the original
% permissive-upgrade strategy with improved permissiveness. 

% Austin and Flanagan~\cite{plas10} introduce a new label $P$ for
% ``partially-leaked''. The intuition behind the partially-leaked label
% $P$ is the following: 

% \begin{framed}
% \noindent
%   A variable with a value labeled $P$ may have been implicitly
%   influenced by $H$-labeled values in this execution, but in other
%   executions (obtainable by changing $H$-labeled values in the
%   initial store), this implicit influence may not exist and, hence,
%   the variable may be labeled $L$.
% \end{framed}

% Thus, assignments under high $\pc$ succeed under the
% permissive-upgrade check but branching or case-analyzing a
% partially-leaked value is not permitted as that can also leak
% information. For illustration, consider the different executions of
% example in Listing~\ref{lst1} taken from~\cite{plas10} as shown in
% Table~\ref{tbl:nsu}. The assignment 
% on line~\ref{lineref} is not permitted by the NSU check. However, the
% permissive-upgrade check updates \TT{y} to \texttt{false} with a
% partially-leaked label. Further case analysis on \TT{y} is prohibited (if
% not, the branch on line~\ref{linerefcond} would not be taken as \TT{y} is
% \texttt{false} and \TT{z} would return $\texttt{true}^{L}$, thus leaking
% the value of \TT{x}).

% \begin{table*}
% \centering
% \begin{tabular} {|l||@{\,}c@{\,}||@{\,}c@{\,}|@{\,}c@{\,}|}
% \hline
% &
% $\TT{x} = \texttt{false}^{H}$
% &
% \multicolumn{2}{c|}{$\TT{x} = \texttt{true}^{H}$}
% \\
% \cline{2-4}
% &
% \textit{All strategies}
% &
% \textit{NSU}
% &
% \textit{Permissive-upgrade}
% \\
% \hline
% \TT{y} = \texttt{true} &$\TT{y} = \texttt{true}^{L}$&$\TT{y} =
%                                               \texttt{true}^{L}$&$\TT{y} =
%                                                                   \texttt{true}^{L}$\\
% \TT{z} = \texttt{true} &$\TT{z} = \texttt{true}^{L}$&$\TT{z} =
%                                               \texttt{true}^{L}$&$\TT{z} =
%                                                                   \texttt{true}^{L}$\\
% \texttt{if} (\TT{x})&branch not taken&$\pc = H$&$\pc = H$\\
% \quad\TT{y} = \texttt{false}&$\TT{y} = \texttt{true}^{L}$&execution halted&$\TT{y} = \texttt{false}^{P}$\\
% \texttt{if} (\TT{y})&$\pc = L$&&execution halted\\
% \quad$\TT{z} = \texttt{false}$&$\TT{z} = \texttt{false}^{L}$&&\\
% \texttt{return} \TT{z}&$\texttt{false}^{L}$&&\\
% \hline
% Result & $\TT{z} = \texttt{false}^{L}$ & no leak & no leak\\
% \hline
% \end{tabular}
% \caption{Execution steps in two runs of the program from
%   Listing~\ref{lst1}, with NSU and the permissive-upgrade check}
% \label{tbl:nsu}
% \end{table*}

The original permissive-upgrade strategy, however, lacks 
permissiveness; it rejects secure programs like the one shown in
Listing~\ref{lst2}. Consider that \TT{x} is labeled $H$
and \TT{w}, \TT{y} are labeled $L$. With the original permissive-upgrade
strategy, the label of \TT{z} on
line~\ref{lineref2} would remain $P$ and the execution would be
terminated when branching on \TT{z} on line~\ref{lineref3}. The
improvement $$H \sqcup P = H$$ allows the analysis to accept such programs
while remaining sound. With the improvement, \TT{z} would be labeled
$H$ on line~\ref{lineref2}, which would allow the execution to branch
on line~\ref{lineref3}, thus, 
taking the execution to completion. 
The idea behind the improvement is that an $H$-labeled value is
never observable at $L$-level. Similarly, the result of any operation
involving an $H$-labeled value is also never observable at
$L$-level. Thus, any operation involving a partially-leaked value and
a $H$-labeled value does not reveal any information to an adversary at
level $L$ about the partially leaked value. 

\begin{lstlisting}[float,label=lst2,caption=Example showing the
  impermissiveness of the original permissive-upgrade strategy][escapechar=@]
y = false
if (not(x))
  y = true@\label{lineref1}@
z = y || x@\label{lineref2}@
if (not(z))@\label{lineref3}@
  w = true
\end{lstlisting}

% \begin{figure}[htbp]
% \begin{align*}
% \lab          :=~& L~\arrowvert~H\\
% \pc          :=~& \lab\\
% k,l,m :=~& \lab~\arrowvert~P\\ \\
% k \sqcup k ~= ~& k\\
% L \sqcup H ~=~ &H\\
% L \sqcup P  ~=~&P \\
% \color{blue}{H\sqcup P  ~=~}&\color{blue}{H} 
% % \\
% % \mathit{if} ~\pc = L ~\mathit{then}~ H\sqcup P  ~=~&H
% \end{align*}
% \caption{Syntax of labels including the partially-leaked label
%   $P$}\label{pus:syntax}
% \end{figure}

% Labels $k,l,m$ on values are either elements of the lattice ($L, H$)
% or $P$. The $\pc$ can only be one of $L, H$ because branching on
% partially-leaked values is prohibited. This is summarized by the
% revised syntax of labels in Figure~\ref{pus:syntax}. The figure also
% lifts the join operation $\sqcup$ to labels including
% $P$. In the original permissive-upgrade~\cite{plas10}, joining any
% label with $P$ resulted in $P$. 

The final label $k$ in the assignment rule \refrule{bs:cmd:ap} under
the improved permissive-upgrade strategy becomes: 
% \begin{mathpar}
% \inferrule*[left=\mbox{\labelthis{bs:cmd:ap}{assn-pus}}]
% {l := \Gamma(\sigma(\TT{x})) \qquad \langle
%  \sigma, \expr \rangle \bsexp \TT{n}^m} {\langle \sigma, \TT{x} := \expr
%     \rangle \bscmd \sigma[\TT{x} \mapsto \TT{n}^{k}]}
% \end{mathpar}
% where $k$ is defined as follows:
$$ k = % \left\lbrace
\begin{cases}
m & \mbox{ if } \pc = L \\
H  & \mbox{ if } \pc = H \mbox{ and } l = H\\
P  & \mbox{ otherwise}
\end{cases} % \right.
$$

% The first two conditions in the definition of $k$ correspond to the
% NSU rule (Figure~\ref{fig:assn-nsu}). The third condition applies, in
% particular, when a variable whose initial label is $L$ is assigned to
% under $\pc = H$. The NSU check would stop this assignment. With
% permissive-upgrade, however, the updated variable is labeled $P$,
% consistent with the intuitive meaning of $P$. This allows more
% permissiveness by allowing the assignment to proceed in all 
% cases. To compensate, any program (in particular, an adversarial
% program) is disallowed from case analyzing any value labeled 
% $P$. Consequently, in the rules for \texttt{if-then} and
% \texttt{while} (Figure~\ref{fig:basic-semantics}), it is required that
% the label of the branch condition be of form $\lab$, which does not
% include $P$.

The soundness results of the original permissive-upgrade strategy can
be extended to show the soundness of the improved permissive-upgrade
strategy. However, a significant difficulty in proving the theorem
using the modified notation for the imperative language is that the
definition of $\sim$ is not transitive. The same problem arises for
the soundness proofs in~\cite{plas10}. There, the authors resolve the
issue by defining a special relation called evolution. The need for
evolution is averted here using the auxiliary lemmas listed
below. Lemma~\ref{lem:evol} proves the required result substituting
evolution.  

\begin{myLemma}[Expression Evaluation]
\label{lem:expeval}
If $\langle \sigma_1, e \rangle \bsexp \emph{\TT{n}}_1^{k_1}$ and $\langle 
\sigma_2, e \rangle \bsexp \emph{\TT{n}}_2^{k_2}$ and $\sigma_1 \sim_L
\sigma_2$, then $\emph{\TT{n}}_1^{k_1} \sim_L \emph{\TT{n}}_2^{k_2}$.
\end{myLemma}
\begin{proof} By induction on $e$.
\end{proof}

\begin{myLemma}[Evolution]
\label{lem:evol}
  If $~\pc = H$ and $\langle \sigma, c \rangle \bscmd \sigma'$, then
  \\ $\forall \emph{\TT{x}}.\Gamma(\sigma(\emph{\TT{x}})) = P
  \implies \Gamma(\sigma'(\emph{\TT{x}})) = P$. 
\end{myLemma}
\begin{proof} By induction on the derivation rules and case
  analysis on the last rule.
\end{proof}

\begin{myLemma}[Confinement for improved permissive-upgrade with a
  two-point lattice]
\label{lem:conf}
  If $~\pc = H$ and $\langle \sigma, c \rangle
  \bscmd \sigma' $, then $\sigma \sim_L \sigma'$.
\end{myLemma}
\begin{proof} By induction on the derivation rules.
\end{proof}

\begin{myThm}[TINI for improved permissive-upgrade with a two-point lattice]
  With the assignment rule \refrule{bs:cmd:ap} and the modified syntax of
  Figure~\ref{pus:syntax}, if 
  $~\sigma_1 \sim_L \sigma_2$ and $\langle \sigma_1, c \rangle
  \bscmd \sigma_1' $ and $\langle \sigma_2, c
  \rangle\bscmd \sigma_2' $, then $\sigma_1' \sim_L
  \sigma_2'$.
\end{myThm}
\begin{proof} By induction on $c$ and case
  analysis on the last step.
\end{proof}

The detailed proofs are provided in Appendix~\ref{app:ipu}. 
Note that the definitions and proofs presented in this chapter are
specific to the two-point lattice and with respect to an adversary at
level $L$. 


% \section{Generalized Permissive-Upgrade on Arbitrary Lattices}
\label{sec:gen:pus}

\begin{figure}
\begin{equation*}
\begin{aligned}[c]
\lab          =~& L~\arrowvert~M~\arrowvert~\ldots~\arrowvert~H\\
\pc          =~& \lab \\
k,l,m =~& \lab~\arrowvert~\lab\pl 
\end{aligned}
\qquad \qquad 
\begin{aligned}[c]
\lab_1 \sqcup \lab_2\pl  =~&(\lab_1 \sqcup \lab_2)\pl \\
\lab_1\pl  \sqcup \lab_2\pl  =~&(\lab_1 \sqcup \lab_2)\pl
\end{aligned}
\end{equation*}
\caption{Labels and label operations}\label{fig:labels}
\end{figure}

This section shows by construction the generalization of the
permissive-upgrade strategy to arbitrary security lattices. For every
element $\lab$ of the lattice, a new label $\lab\pl$ is introduced
which means ``partially-leaked $\lab$'', with the following intuition: 
\begin{framed}
\noindent
A variable labeled $\lab\pl$ may contain partially-leaked data, where
$\lab$ is a \emph{lower-bound} on the $\star$-free labels the variable
may have in alternate executions.
\end{framed}

The syntax of labels is listed in Figure~\ref{fig:labels}. Labels
$k,l,m$ may be lattice elements $\lab$ or $\star$-ed lattice elements
$\lab\pl$. In examples, suggestive lattice element names $L, M, H$
(low, medium, high) are used. Labels of the form $\lab$ are called 
$\star$-free or \emph{pure}. Figure~\ref{fig:labels} also defines the
join operation $\sqcup$ on labels. This definition is based on the intuition
above. When the two operands of $\odot$ are labeled $\lab_1$ and
$\lab_2\pl$, $\lab_1 \sqcup \lab_2$ is a lower bound on the pure label
of the resulting value in any execution (because $\lab_2$ is a lower
bound on the pure label of $\lab_2\pl$ in any run). Hence, $\lab_1
\sqcup \lab_2\pl = (\lab_1 \sqcup \lab_2)\pl$. The reason for the
definition $\lab_1\pl \sqcup \lab_2\pl = (\lab_1 \sqcup \lab_2)\pl$ is
similar.

\begin{figure}
\begin{mathparpagebreakable}
\inferrule*[left=\mbox{\labelthis{bs:cmd:agn}{assn-n}}]
{  \langle \sigma, \expr \rangle \bsexp \TT{n}^m \\ l =
  \Gamma(\sigma(\TT{x})) \\ l = \lab_\TT{x} \vee l = \lab_\TT{x}\pl \\ \pc 
  \sqsubseteq \lab_\TT{x} \\ k = \pc \sqcup m}
{\langle \sigma , \TT{x} := \expr \rangle \bscmd \sigma[\TT{x}
  \mapsto \TT{n}^{k}]}
%%%    
\and
\inferrule*[left=\mbox{\labelthis{bs:cmd:ags}{assn-s}}]
{\langle \sigma, \expr \rangle \bsexp \TT{n}^m \\ l =
  \Gamma(\sigma(\TT{x})) \\ l = \lab_\TT{x} \vee  l = \lab_\TT{x}\pl \\ \pc
  \not\sqsubseteq \lab_\TT{x} \\ k = ((\pc \sqcup m)\sqcap \lab_\TT{x} )\pl}
{\langle \sigma , \TT{x} := \expr \rangle \bscmd \sigma[\TT{x} \mapsto
  \TT{n}^{k}]}
\end{mathparpagebreakable}
\caption[Caption]{Assignment rules for the generalized
  permissive-upgrade}\label{fig:assn-our}
\end{figure}

The rules for assignment are shown in Figure~\ref{fig:assn-our}. They
strictly generalize the rule \refrule{bs:cmd:ap} for the two-point lattice,
treating $P = L\pl$. Rule~\refrule{bs:cmd:agn} applies when the existing label of
the variable being assigned to is $\lab_{\TT{x}}$ or $\lab_{\TT{x}}\pl$ and $\pc
\sqsubseteq \lab_{\TT{x}}$. The key intuition behind the rule is the
following: If $\pc \sqsubseteq \lab_{\TT{x}}$, then it is safe to overwrite
the variable, because $\lab_{\TT{x}}$ is necessarily a lower bound on the
(pure) label of $\TT{x}$ in this and any alternate execution (see the
\framebox{framebox} above). Hence, overwriting the variable cannot
cause an implicit flow. As expected, the label of the overwritten
variable is $\pc \sqcup m$, where $m$ is the label of the value
assigned to $\TT{x}$.

Rule \refrule{bs:cmd:ags} applies in the remaining case --- when $\pc
\not\sqsubseteq \lab_{\TT{x}}$. In this case, there may be an implicit flow,
so the final label on $\TT{x}$ must have the form $\lab\pl$ for some
$\lab$. The question is which $\lab$. Intuitively, it may seem that
one could choose $\lab = \lab_{\TT{x}}$, the pure part of the original label
of $\TT{x}$. The final label on $\TT{x}$ would be $\lab_{\TT{x}}\pl$ and this would
satisfy the intuitive meaning of $\star$ written in the
\framebox{framebox} above. Indeed, this intuition suffices for the
two-point lattice of Section~\ref{sec:existing}
and~\ref{sec:ipus}. However, for a more 
general lattice, this intuition is unsound, as illustrated with an
example below. The correct label is $((\pc \sqcup m) \sqcap
\lab_{\TT{x}})\pl$. 
% (Note that this correct label is independent of the label $m$ of the value
% assigned to $\TT{x}$. This is sound because $\TT{x}$ is $\star$-ed and
% cannot be case-analyzed later, so the label on the value in it is irrelevant.)

\begin{figure}[ht]
\begin{minipage}{0.65\linewidth}
\centering
\begin{lstlisting}[caption=Example explaining rule \refrule{bs:cmd:ags},label=list2]
if ($\TT{x}'$)
  $\TT{z} = \TT{y}_1$@\label{hm1}@
else
  $\TT{z} = \TT{y}_2$@\label{hm2}@
if ($\TT{x}_1$)@\label{bl1}@
  $\TT{z} = \TT{x}_1$@\label{hl1}@
if ($\texttt{not}(\TT{x}_2)$)@\label{if2}@
  $\TT{z} = \TT{x}_2$@\label{hl2}@
if ($\TT{z}$)@\label{if3}@
  $\TT{w} = \TT{z}$@\label{hl3}@
\end{lstlisting}
\end{minipage}
\hspace{-1.5cm}
\begin{minipage}{0.45\linewidth}
\centering
{\includegraphics{chapters/gpu/lattice.tikz}}
\caption{Lattice explaining rule \refrule{bs:cmd:ags}}
\label{fig:lattice}
\end{minipage}
\end{figure}

\paragraph{Example} 
The need for the label $k := ((\pc \sqcup m) \sqcap \lab_{\TT{x}})\pl$
instead of $k := \lab_{\TT{x}}\pl$ in rule \refrule{bs:cmd:ags} is illustrated
below. Consider the lattice of Figure~\ref{fig:lattice} and the
program of Listing~\ref{list2}. Assume 
that, initially, the variables $\TT{z}$, $\TT{w}$, $\TT{x}_1$,
$\TT{x}'$, $\TT{x}_2$, $\TT{y}_1$ and $\TT{y}_2$ have labels $H$,
$L_1$, $L_1$, $L'$, $L_2$, $M_1$ and $M_2$, 
respectively. Fix the attacker at level $L_1$. Fix the value of $\TT{x}_1$
at $\texttt{true}^{L_1}$, so that the branch on line~\ref{bl1} is
always taken and line~\ref{hl1} is always executed. Set $\TT{y}_1 \mapsto
\texttt{false}^{M_1}, \TT{y}_2 \mapsto \texttt{true}^{M_2}, \TT{w} \mapsto
\TT{false}^{L_1}$ initially. The initial value of $\TT{z}$ is
irrelevant. Consider two executions of the program starting from two
stores $\sigma_1$ with $\TT{x}' \mapsto \texttt{true}^{L'}, \TT{x}_2 \mapsto
\texttt{true}^{L_2}$ and $\sigma_2$ with $\TT{x}' \mapsto
\texttt{false}^{L'}, \TT{x}_2 \mapsto \texttt{false}^{L_2}$. Note that
as $L'$ and $L_2$ are incomparable to $L_1$ in the lattice,
$\sigma_1$ and $\sigma_2$ are equivalent for $L_1$. 

\begin{table*}
\centering
\begin{tabular} {|l||@{\,}c@{\,}||@{\,}c@{\,}|@{\,}c@{\,}|}
\hline
&
\multicolumn{3}{c|}{$\TT{w} = \texttt{false}^{L_1},\ \TT{x}_1 =
  \texttt{true}^{L_1},\ \TT{y}_1 = \texttt{false}^{M_1},\ \TT{y}_2 =
  \texttt{true}^{M_2}$}
\\
\cline{2-4}
&
$\TT{x}' = \texttt{true}^{L'}$
&
\multicolumn{2}{c|}{$\TT{x}' = \texttt{false}^{L'}$}
\\
&
$\TT{x}_2 = \texttt{true}^{L_2}$
&
\multicolumn{2}{c|}{$\TT{x}_2 = \texttt{false}^{L_2}$}
\\
\cline{3-4}
&
&
$k:= \lab_{\TT{x}}\pl$
&
$k:= ((\pc \sqcup m) \sqcap \lab_{\TT{x}})\pl$
\\
\hline
\texttt{if} ($\TT{x}'$)&$\pc = L'$&&\\
\quad$\TT{z} = \TT{y}_1$&$\TT{z} = \texttt{false}^{M_1}$&&\\
\texttt{else}&&$\pc = L'$&$\pc = L'$\\
\quad$\TT{z} = \TT{y}_2$&&$\TT{z} = \texttt{true}^{M_2}$&$\TT{z} = \texttt{true}^{M_2}$\\
\texttt{if} ($\TT{x}_1$)&$\pc = L_1$&$\pc = L_1$&$\pc = L_1$\\
\quad$\TT{z} = \TT{x}_1$&$\TT{z} = \texttt{true}^{L_1}$&$\TT{z} = \texttt{true}^{M_2\pl}$&$\TT{z} = \texttt{true}^{L\pl}$\\
\texttt{if} ($\texttt{not}(\TT{x}_2)$)&branch not taken&$\pc = L_2$&$\pc = L_2$\\
\quad$\TT{z} = \TT{x}_2$& %$z = \texttt{true}^{L_1}$
&$\TT{z} = \texttt{false}^{L_2}$&$\TT{z} = \texttt{false}^{L\pl}$\\
\texttt{if} ($\TT{z}$)&$\pc = L_1$&branch not taken&execution halted\\
\quad$\TT{w} = \TT{z}$&$\TT{w} = \texttt{true}^{L_1}$&&\\
\hline
Result & $\TT{w} = \texttt{true}^{L_1}$ & $\TT{w} =
\texttt{false}^{L_1}$ (leak) & no leak\\
\hline
\end{tabular}
\caption{Execution steps in two runs of the program from Listing~\ref{list2}, with two variants of the rule \refrule{bs:cmd:ags}}
\label{tblassn}
\end{table*}

Requiring $k := \lab_{\TT{x}}\pl$ in rule \refrule{bs:cmd:ags} causes an
implicit flow that is observable for $L_1$. The intermediate values
and labels of the variables for executions starting from $\sigma_1$
and $\sigma_2$ are shown in the second and third columns of
Table~\ref{tblassn}. Starting with $\sigma_1$, line~\ref{hm1} is
executed, but line~\ref{hm2} is not, so $\TT{z}$ ends with
$\texttt{false}^{M_1}$ at line~\ref{bl1} (rule \refrule{bs:cmd:agn} applies at
line~\ref{hm1}). At line~\ref{hl1}, $\TT{z}$ contains $\texttt{true}^{L_1}$
(again by rule \refrule{bs:cmd:agn}) and line~\ref{hl2} is not executed. Thus, the
branch on line~\ref{if3} is taken and $\TT{w}$ ends with
$\texttt{true}^{L_1}$ at line~\ref{hl3}. Starting with $\sigma_2$,
line~\ref{hm1} is not executed, but line~\ref{hm2} is, so $\TT{z}$ becomes
$\texttt{true}^{M_2}$ at line~\ref{bl1} (rule \refrule{bs:cmd:agn} applies at
line~\ref{hm2}). At line~\ref{hl1}, rule \refrule{bs:cmd:ags} applies, but because
$k := \lab_{\TT{x}}\pl$ is assumed in that rule, $\TT{z}$ now contains the
value $\texttt{true}^{M_2\pl}$. As the branch on line~\ref{if2} is
taken, at line~\ref{hl2}, $\TT{z}$ becomes $\texttt{false}^{L_2}$ by rule
\refrule{bs:cmd:agn} because $L_2 \sqsubseteq M_2$. Thus, the branch on
line~\ref{if3} is not taken and $\TT{w}$ ends with $\texttt{false}^{L_1}$
in this execution. Hence, $\TT{w}$ ends with $\texttt{true}^{L_1}$ and
$\texttt{false}^{L_1}$ in the two executions, respectively. The
attacker at level $L_1$ can distinguish these two results; hence, the
program leaks the value of $\TT{x}'$ and $\TT{x}_2$ to $L_1$.

With the correct \refrule{bs:cmd:ags} rule in place, this leak is avoided (last
column of Table~\ref{tblassn}). In that case, after the assignment on
line~\ref{hl1} in the second execution, $\TT{z}$ has label $((L_1 \sqcup L_1) \sqcap
M_2)\pl = L\pl$. Subsequently, after line~\ref{hl2}, $\TT{z}$ gets the
label $L\pl$. As case analysis on a $\star$-ed value is not allowed,
the execution is halted on line~\ref{if3}. This guarantees
termination-insensitive non-interference with respect to the attacker
at level $L_1$.

\subsection{Termination-insensitive Non-interference (TINI)}

To prove non-interference for the generalized permissive-upgrade,
equivalence of labeled values relative to an adversary at arbitrary
lattice level $\lab$ needs to be defined. The definition is shown
below (Definition~\ref{def:gpua:veq}). Note that clauses (3)--(5) here
refine clause (3) of Definition~\ref{def:eq-existing} for the two-point
lattice. The obvious generalization of clause (3) of
Definition~\ref{def:eq-existing} --- $\TT{n}_1^k \sim_\lab \TT{n}_2^m$ whenever 
either $k$ or $m$ is $\star$-ed --- is too coarse to prove
non-interference inductively. For the degenerate case of the two-point
lattice, this definition also degenerates to
Definition~\ref{def:eq-existing} (there, $\lab$ is fixed at $L$, $P = 
L\pl$ and only $L$ may be $\star$-ed).

\begin{mydef}
\label{def:gpua:veq}
Two values $\emph{\TT{n}}_1^k$ and $\emph{\TT{n}}_2^m$ are
$\lab$-equivalent, written $\emph{\TT{n}}_1^k \sim_\lab
\emph{\TT{n}}_2^m$, iff either 
\begin{enumerate}
\item $k = m = \lab' \sqsubseteq \lab$ and $\emph{\TT{n}}_1 =
  \emph{\TT{n}}_2$, or 
\item $ k = \lab'
  \not\sqsubseteq \lab$ and $m = \lab'' \not\sqsubseteq \lab$, or 
\item $k = \lab_1\pl$ and $m = \lab_2\pl$, or
\item $k = \lab_1\pl$ and $m = \lab_2$ and ($\lab_2 \not\sqsubseteq
  \lab$ or $\lab_1 \sqsubseteq \lab_2 $), or
\item $k = \lab_1$ and $m = \lab_2\pl$ and ($\lab_1 \not\sqsubseteq
  \lab$ or $\lab_2 \sqsubseteq \lab_1$)
\end{enumerate}
\end{mydef}

\begin{mydef}
\label{def:gpua:seq}
  Two stores $\sigma_1$ and $\sigma_2$ are $\lab$-equivalent,
  written $\sigma_1 \sim_\lab \sigma_2$, iff for every variable \emph{\TT{x}},
  $\sigma_1(\emph{\TT{x}}) \sim_\lab \sigma_2(\emph{\TT{x}})$.
\end{mydef}

This definition is obtained by constructing (through examples) an
extensive transition graph of pairs of labels that may be assigned to
a single variable at corresponding program points in two executions of
the same program. The starting point is label-pairs of the form
$(\lab, \lab)$. This characterization of equivalence is both
sufficient and necessary. It is sufficient in the sense that it allows
us to prove TINI inductively. It is necessary in the sense that
example programs can be constructed that end in states exercising
every possible clause of this definition. Appendix~\ref{app:egequi}
lists these examples. 

% \section{Termination-Insensitive Non-Interference}

Using the above definition of equivalence of labeled values, TINI can
be proven for the generalized permissive-upgrade strategy presented
above. A significant difficulty in proving the theorem is that the
definition of $\sim_\lab$ is not transitive unlike the previous definition
of $\sim$. 
% The same problem arises
% for the two-point lattice as shown in the previous section and
% in~\cite{plas10}. There, the authors resolve the 
% issue by defining a special relation called evolution. 
% Here, a more conventional approach based on the standard confinement 
% lemma is taken. 
% The need for evolution is averted using several auxiliary
% lemmas that we list below.
Detailed proofs of all the lemmas and the theorems
are presented in Appendix~\ref{app:gpu}.

\begin{myLemma}[Expression evaluation]
\label{lem:gpua:exp}
If $\langle \sigma_1, e \rangle \bsexp \emph{\TT{n}}_1^{k_1}$ and $\langle 
\sigma_2, e \rangle \bsexp \emph{\TT{n}}_2^{k_2}$ and $\sigma_1 \sim_\lab \sigma_2$,
then $\emph{\TT{n}}_1^{k_1} \sim_\lab \emph{\TT{n}}_2^{k_2}$.
\end{myLemma}
\begin{proof}
By induction on $e$.
\end{proof}

%% Lemma~\ref{sup1} shows that a value labeled $\lab\pl$ remains
%% $\star$-ed if an evaluation is done on it under a higher or unrelated
%% context. Lemma~\ref{pcl} on the other hand shows that if a value gets
%% a pure label after an evaluation, then either the value remained
%% unchanged after the evaluation or the evaluation done was done in a
%% lower context. We state two important corollaries,
%% Corollary~\ref{cor1} and~\ref{cor2} derived from Lemma~\ref{sup1}
%% and~\ref{pcl}, respectively, which are used in the proofs for
%% confinement and non-interference.

\begin{myLemma}[$\star$-preservation]
\label{lem:gpua:sup1}
If $\langle \sigma, c \rangle \bscmd \sigma'$ and
$\Gamma(\sigma(x)) = \lab\pl $ and $\pc \not\sqsubseteq \lab$, then
$\Gamma(\sigma'(x)) = \lab'\pl$ and $\lab' \sqsubseteq \lab$.
\end{myLemma}
\begin{proof}
By induction on the derivation rule.
\end{proof}

\begin{mycor}
\label{cor:gpua:cor1}
If $\langle \sigma, c \rangle \bscmd \sigma'$ and
$\Gamma(\sigma(\emph{\TT{x}})) = \lab\pl $ and
$\Gamma(\sigma'(\emph{\TT{x}})) = \lab'$, then 
$\pc \sqsubseteq \lab$.
\end{mycor}
\begin{proof}
Immediate from Lemma~\ref{lem:gpua:sup1}.
\end{proof}

\begin{myLemma}[$\pc$-lemma]
\label{lem:gpua:pcl}
If $\langle \sigma, c \rangle \bscmd \sigma'$ and
$\Gamma(\sigma'(\emph{\TT{x}})) = \lab$, then $\sigma(\emph{\TT{x}}) =
\sigma'(\emph{\TT{x}})$ or 
$\pc \sqsubseteq \lab$.
\end{myLemma}
\begin{proof}
By induction on the derivation rule.
\end{proof}

\begin{mycor}
\label{cor:gpua:cor2}
If $\langle \sigma, c \rangle \bscmd \sigma'$ and
$\Gamma(\sigma(\emph{\TT{x}})) = \lab\pl $ and
$\Gamma(\sigma'(\emph{\TT{x}})) = \lab'$, then 
$\pc \sqsubseteq \lab'$.
\end{mycor}
\begin{proof}
Immediate from Lemma~\ref{lem:gpua:pcl}.
\end{proof}

Using these lemmas, the standard confinement lemma and non-interference
can be proven. 

\begin{myLemma}[Confinement Lemma]
\label{lem:gpua:conf}
If $\pc \not\sqsubseteq \lab$ and $\langle \sigma, c \rangle
\bscmd \sigma'$, then $\sigma \sim_\lab \sigma'$.
\end{myLemma}
\begin{proof}
By induction on the derivation rule.
\end{proof}

\begin{myThm}[TINI for generalized permissive-upgrade for arbitrary lattices]
  If $~\sigma_1 \sim_\lab \sigma_2$ and $\langle \sigma_1, c \rangle
  \bscmd \sigma_1' $ and $\langle \sigma_2, c
  \rangle\bscmd \sigma_2' $, then $\sigma_1' \sim_\lab
  \sigma_2'$.
\end{myThm}
\begin{proof} By induction on $c$.
\end{proof}


%%%%%%%%%%%%%%%%%%%%%%%%%%%%%%%%%%%%%%%%%%%%%%%%%%%%%%%%

%%%%%%%%%%%%%%%%%%%%%%%%%%%%%%%%%%%%%%%%%%%%%%%%%%%%%%%%

\section{Comparison of the Generalization of Section~\ref{sec:gen:pus}
  with the Generalization of Section~\ref{sec:gen:ipus}}

%\subsection{Permissiveness of the Approach in Section~\ref{sec:gen:pus}}

Two distinct and sound generalizations of the permissive-upgrade
strategy for the two-point lattice have now been described:
The generalization of the improved permissive-upgrade to pointwise
products of two-point lattices or, equivalently, to powerset lattices
as described in Section~\ref{sec:gen:ipus}, and the generalization to
arbitrary lattices described in Section~\ref{sec:gen:pus}. For brevity,
these generalizations are called puP (Section~\ref{sec:gen:ipus}) and
puA (Section~\ref{sec:gen:pus}), 
respectively (P and A stand for \underline{p}owerset and
\underline{a}rbitrary, respectively). Since both puP and puA apply to
powerset lattices, an obvious question is whether one is more
permissive than the other on such lattices. The generalization of puP
presented in this chapter can be more permissive than puA on powerset 
lattices in certain cases as shown by the example
below. The reason for this permissiveness is that puP tracks finer
taints, i.e., it tracks partial leaks for each principal separately. 

\begin{figure}[ht]
\begin{minipage}{0.65\linewidth}
\centering
\begin{lstlisting}[caption=Example where puP is more permissive than puA,label=list3]
if (y) @\label{li32}@
  x = z @\label{li33}@
if (z) @\label{li34}@
  x = z @\label{li35}@
if (x)  @\label{li36}@
  z = x
\end{lstlisting}
\end{minipage}
\hspace{-1.5cm}
\begin{minipage}{0.45\linewidth}
\centering
{\includegraphics{chapters/gpu/powerset.tikz}}
\caption{A powerset/product lattice}\label{fig:lattice1}
\end{minipage}
\end{figure}

\paragraph{Example}
The powerset lattice of Figure~\ref{fig:lattice1} is used for
illustration purpose. This lattice is the pointwise lifting of the order $L
\sqsubset H$ to the set $S = \{L, H\} \times \{L, H\}$. For brevity,
this lattice's elements are written as $LL$, $LH$, etc. When puP is
applied to this lattice, labels are drawn from the set $\{L, H, P\}
\times \{L, H, P\}$. These labels are concisely written as $LP$, $HL$,
etc. For puA, labels are drawn from the set $S \cup S\pl$. These
labels are written $LH$, $LH\pl$, etc. Note that $LH\pl$ parses as
$(LH)\pl$, not $L(H\pl)$ (the latter is not a valid label in puA
applied to this lattice).
%
Consider the program in Listing~\ref{list3}. Assume that $\TT{x}$,
$\TT{y}$ and $\TT{z}$ have 
initial labels $LL$, $HL$ and $LH$, respectively and that the initial
store contains $\TT{y} \mapsto \texttt{true}^{HL}, \TT{z} \mapsto
\texttt{true}^{LH}$, so the branches on lines~\ref{li32}
and~\ref{li34} are both taken. The initial value in $\TT{x}$ is irrelevant
but its label is important.
Under puA, $\TT{x}$ obtains label $(((HL) \sqcup (LH)) \sqcap (LL))\pl = LL\pl$ at
line~\ref{li33} by rule \refrule{bs:cmd:ags}. At line~\ref{li35}, the same rule
applies but the label of $\TT{x}$ remains $LL\pl$. When the program tries
to branch on $\TT{x}$ at line~\ref{li36}, it is stopped.
In contrast, under puP, this program executes to completion. At
line~\ref{li33}, the label of $\TT{x}$ changes to $PH$ by rule
\refrule{bs:cmd:agp}. At line~\ref{li35}, the label changes to $LH$ because $pc$
and the label of $\TT{z}$ are both $LH$. Since this new label has no $P$,
line~\ref{li36} executes without halting.
Hence, for this example, puP is more permissive than puA.

% \section{Remarks}
% On powerset lattices, the resulting IFC monitor is different
% in the two cases and the generalization of improved permissive-upgrade
% is more permissive than puA in certain cases. 
% The approach presented in this section is, thus, more general
% and applies to a broader set of lattices. 
An example for which  puA is more permissive than puP in the case of
poweset lattices was not found. But the generalization puA presented in
Section~\ref{sec:gen:pus} is more general than the product
construction puP (Section~\ref{sec:gen:ipus}) when applied to arbitrary
lattices (and hence, applicable to a broader set of lattices) as it is
unclear whether or how  the improved permissive 
upgrade strategy generalizes to arbitrary lattices. By developing this
generalization, this work makes permissive-upgrade applicable to
arbitrary security lattices like other IFC techniques.
% and, hence,
% constitutes a useful contribution to IFC literature.



% \Blindtext