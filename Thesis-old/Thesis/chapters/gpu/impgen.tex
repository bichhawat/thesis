\section{Generalization of the Improved Permissive-Upgrade Strategy}
\label{sec:gen:ipus}
Austin and Flanagan point out that permissive-upgrade on a two-point
lattice can be generalized to a pointwise product of such
lattices. This generalization can also be extended to the improved 
permissive-upgrade strategy presented in the previous chapter. 
Specifically, let $X$ be an index set --- these
indices are called principals in~\cite{plas10}. Let a label $l$ be a
map of type $X \rightarrow \{L, H, P\}$ and let the subclass of pure
labels contain maps $\lab, \pc$ of type $X \rightarrow \{L, H\}$. The
order $\sqsubset$ and the join operation $\sqcup$ can be generalized
pointwise to these labels. Finally, the rule \refrule{bs:cmd:ap} can be
generalized pointwise by replacing it with the following rule:
\begin{mathpar}
\inferrule*[left=\mbox{\labelthis{bs:cmd:agp}{assn-gpus}}]
{l := \Gamma(\sigma(\TT{x})) \qquad \langle
    \sigma, \expr \rangle \bsexp \TT{n}^m} {\langle  \sigma,
    \TT{x} := \expr  \rangle \bscmd \sigma[\TT{x} \mapsto
    \TT{n}^{k}]} 
\end{mathpar}
where $k$ is defined as follows:
\[ k(a) = % \left\lbrace
\begin{cases}
m(a) & \mbox{ if } \pc(a) = L \\
H  & \mbox{ if } \pc(a) = H \mbox{ and } l(a) = H\\
P & \mbox{ otherwise}
\end{cases} % \right.
\]
It can be shown that for any semantic
derivation in this generalized system, projecting all labels to a
given principal yields a valid semantic derivation in the system with
a two-point lattice. This immediately implies non-interference for the
generalized system, where observations are \textbf{limited} to individual
principals. 


\begin{mydef}
\label{def:eq-existing}
  Two labeled values $\emph{\TT{n}}_1^k$ and $\emph{\TT{n}}_2^m$ are $a$-equivalent,
  written $\emph{\TT{n}}_1^k \approx^a \emph{\TT{n}}_2^m$, iff either:
  \begin{enumerate}
  \item $k(a) = m(a) = L$ and $\emph{\TT{n}}_1 = \emph{\TT{n}}_2$ or
  \item $k(a) = m(a) = H$ or
  \item $k(a) = P$ or $m(a) = P$
  \end{enumerate}
\end{mydef}

\begin{mydef}[Store equivalence]
  Two stores $\sigma_1$ and $\sigma_2$ are $\lab$-equivalent,
  written $\sigma_1 \approx^a  \sigma_2$, iff for every variable \emph{\TT{x}},
  $\sigma_1(\emph{\TT{x}}) \approx^a  \sigma_2(\emph{\TT{x}})$.
\end{mydef}

\begin{myThm}[TINI for permissive-upgrade with a product lattice]
  With the assignment rule \refrule{bs:cmd:agp}, if $~\sigma_1
  \approx^a \sigma_2$ and $\langle \sigma_1, c \rangle \bscmd
  \sigma_1' $ and $\langle \sigma_2, c \rangle \bscmd \sigma_2'
  $, then $\sigma_1' \approx^a \sigma_2'$.
\end{myThm}
\begin{proof} Outlined above.
\end{proof}

% \paragraph{Remark} This generalization also makes sense if the 
% principals are pre-ordered by a relation, say, $\leq$, with $a \leq b$
% meaning that ``if $a$ has access, then $b$ must have access''. It can
% be proved that the following is an \emph{invariant} on all labels $l$
% that arise during program execution: $((a \leq b) \mathrel{\wedge}
% (l(a) = L)) \implies l(b) = L$. Hence, the intuitive meaning of the
% order $\leq$ is preserved during execution.

This generalization of the two-point lattice to a \emph{product}
of such lattices is interesting because a \emph{powerset} lattice
can be simulated using such a product. However, this still leaves open
the question of constructing a generalization of permissive-upgrade to
an \emph{arbitrary} lattice (for instance, lattices like the one shown
in Figure~\ref{fig:lattice}). Such a generalization is developed in
the next section.

