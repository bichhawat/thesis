\section{Formalization of LIR}
\label{sec:lir-formal}

The following section formalizes the property of LIR and prove it for
the semantics presented above. 
LIR is essentially a property stating
that the total number of bits that can be leaked about a secret (by
a program) is upper bounded by its pre-specified budget 
(in an average sense). The trace
generated in the LIR semantics captures the data about the declassified 
value. An adversary at level $\attacker$ can view those values in a 
memory store $\sigma$ that have a label less than or equal to $\attacker$ 
($\forall x. \Gamma(\sigma(x)) \sqsubseteq \attacker$). The adversary can 
also observe the projection of the trace generated as part of the 
semantics, defined formally in Definition~\ref{def:lir:tp}. Similarly, 
the adversary can view a projection of the budget-map as defined 
in Definition~\ref{def:lir:bmp}.  

\begin{mydef}[Trace projection]
\label{def:lir:tp}
Given a trace, $\trace$, the trace projection w.r.t. an adversary at
level $\attacker$, written $\trace\proj$, is 
% just a standard filter over a list 
defined as: 
  \begin{align*}
  []\proj &=~[] \\
  (\emph{\TT{n}}^m :: \trace)\proj &=
  \begin{cases}
   \emph{\TT{n}}^m :: \trace\proj~ & \mathit{if }~ m \sqsubseteq \attacker,\\
   \trace\proj & \mathit{else}.
  \end{cases}
  \end{align*}
\end{mydef}

\begin{mydef}[Budget-map projection]
\label{def:lir:bmp}
The projection of a budget-map, $\iota$, w.r.t. an adversary at
level $\attacker$, written $\iota\proj$, is defined as:
\begin{align*}
\iota\proj = \lambda \emph{\TT{x}}. 
\begin{cases}
\iota(\emph{\TT{x}}), &\text{if} ~ \big(\blabel(\emph{\TT{x}}) \sqsubseteq \attacker\big) \\
0, &\text{otherwise}
\end{cases}
\end{align*}
\end{mydef}
% \begin{mydef}[Budget map projection]
% \label{def:lir:bmp}
% The projection of a budget map, $\iota$, w.r.t. an adversary at
% level $\attacker$, written $\iota\proj$, is defined as:
% $$\iota\proj = 
% \left \{
% (\emph{\TT{x}}, \emph{\TT{n}})~\middle|~\big((\emph{\TT{x}},
% \emph{\TT{j}}) \in \iota\big) \bigwedge \left(\emph{\TT{n}} =
% \left\{\begin{array}{ll}
%         0, & \text{if} ~ (\blabel(x) \not\sqsubseteq \attacker) \\
%         \emph{\TT{j}}, & \text{otherwise}
%         \end{array}\right\} 
% % \forall x \in \iota. \big(\iota\proj(x) := (\blabel(x)
% % \not\sqsubseteq \attacker ~?~ 0 : \iota(x))
% \right)\right\}
% $$
% \end{mydef}

% \begin{mydef}
% \label{def:lir:dbmp}
% The difference between two budget maps, $\iota$ and $\iota'$, written
% $\iota - \iota'$, is defined as:
% $$\iota - \iota' = \sum\limits_{\emph{\TT{x}} \in \iota}
% \big(\iota(\emph{\TT{x}}) - \iota'(\emph{\TT{x}})\big)$$ 
% \end{mydef}

The observational equivalence of various data structures used in the
semantics with respect to an adversary needs to be defined for
formally defining LIR. Definition~\ref{def:lir:veq}
and~\ref{def:lir:seq} define the  observational equivalence of two
values and two memory stores,  respectively, with respect to an
adversary at level $\attacker$. Definition~\ref{def:lir:beq}
and~\ref{def:lir:teq} define  the observational equivalence of the
budget-maps and the generated traces. 

\begin{mydef}
	\label{def:lir:veq}
	Two labeled values $v_1 = \emph{\TT{n}}_1^{(l_1,\dep_1)}$ and $v_2 =
	\emph{\TT{n}}_2^{(l_2,\dep_2)}$ are observationally equivalent at level
	$\attacker$, written $v_1\eq v_2$ iff either:
	\begin{enumerate}
		\item $\emph{\TT{n}}_1 = \emph{\TT{n}}_2$, % $\Gamma(v_1) = \Gamma(v_2) \sqsubseteq \attacker$,
		$l_1 = l_2 \sqsubseteq \attacker$ and $\dep_1 = \dep_2$ (or)
		\item $\Gamma(v_1) \not\sqsubseteq  \attacker$ and $\Gamma(v_2)
		\not\sqsubseteq \attacker$ and $l_1 = l_2 \sqsubseteq
		\attacker$ and $\dep_1 = \dep_2$
		% $\exists x^m \in \dep_1.(m \not\sqsubseteq
		% \attacker)$ and $\exists y^k \in \dep_2.(k \not\sqsubseteq \attacker)$
		(or)
		\item $\Gamma(v_1) \not\sqsubseteq  \attacker$ and $\Gamma(v_2)
		\not\sqsubseteq \attacker$ and $l_1 \not\sqsubseteq
		\attacker \vee \exists \emph{\TT{x}} \in
		\dep_1. \blabel(\emph{\TT{x}}) \not\sqsubseteq 
		\attacker$ and $l_2 \not\sqsubseteq \attacker \vee \exists
		\emph{\TT{x}} \in   \dep_2. \blabel(\emph{\TT{x}}) \not\sqsubseteq
		\attacker$ 
	\end{enumerate}
\end{mydef}
% \dg{The two ``such that''s in the last two points should also be
%   ``and''s.}

\begin{mydef}
	\label{def:lir:seq}
	Two memory stores $\sigma_1$ and $\sigma_2$ are observationally equivalent
	at level $\attacker$, written $\sigma_1 \eq \sigma_2$ iff
	$\forall \emph{\TT{x}}. \; \sigma_1(\emph{\TT{x}}) \eq \sigma_2(\emph{\TT{x}})$.
\end{mydef}

\begin{mydef}
	\label{def:lir:beq}
	Two budget-maps $\iota$ and $\iota'$ are equivalent at level
	$\attacker$, written $\iota \eq \iota'$, iff
	$\forall \emph{\TT{x}}. \blabel(\emph{\TT{x}}) \sqsubseteq \attacker
	\implies \iota(\emph{\TT{x}}) = \iota'(\emph{\TT{x}})$.
\end{mydef}

\begin{mydef}
	\label{def:lir:teq}
	Two traces $\trace$ and $\trace'$ are equivalent at level
	$\attacker$, written $\trace \eq \trace'$, iff
	$\trace\proj = \trace'\proj$.
\end{mydef}

As per LIR, the length of the projected trace is bounded by 
the total budget deducted. The length of a trace $\trace$ is
represented as $|\trace|$. This intuition of LIR is formalized in 
Definition~\ref{def:lir:lir}. 

\begin{mydef}[Limited information release]
\label{def:lir:lir}
A program $\comm$ is said to satisfy \emph{limited information release}
w.r.t.\ an adversary at level $\attacker$ if for any given memory
store $\sigma$, % containing \emph{a} secret input
$\iota$, and $\pc$,
$\langle \sigma, \iota, \comm \rangle \lirsem \langle
\sigma', \iota', \trace \rangle$ then $|\trace\proj| \leq
\iota\proj - \iota'\proj$ 
% $$\iota\swarrow_\attacker = \mathop {\Large\Sigma}_{x \in \iota} \big(\blabel(x) \sqsubseteq
% \attacker ~? ~\iota(x)~ :~ 0 \big) $$
\end{mydef}

Theorem~\ref{thm:lir} shows that the monitored semantics presented in 
Section~\ref{sec:semantics} satisfy limited information release for every 
secret in the memory. 
Lemma~\ref{lem:lir:conf} proves that in a secret context with respect to
an adversary at level $\attacker$, no declassification visible to the 
adversary occurs. Theorem~\ref{thm:lir:cmdtrli} proves that in
a store containing only one secret value with respect to 
$\attacker$-level adversary (the other secrets can be assumed to be visible 
to the adversary), the length of the trace is bounded by the budget 
reduced, i.e., the number of bits declassified is bounded by the 
deduction in budget of that secret value. Theorem~\ref{thm:lir} 
generalizes this result to all secrets in the system. The proofs are
detailed in Appendix~\ref{sec:app:lir}. 

\begin{myLemma}[Confinement]
  \label{lem:lir:conf}
  If $\langle \sigma, \iota, \comm \rangle
  \lirsem \langle \sigma', \iota', \trace \rangle$,
  and $\Gamma(\pc) \not\sqsubseteq \attacker$, then 
  $\sigma \eq \sigma'$, $\iota \eq \iota'$ and $\trace\proj =
  \emptyTrace$ 
\end{myLemma}

\begin{myThm}[Limited information release for a single secret]
\label{thm:lir:cmdtrli}
If 
$\langle \sigma, \iota, \comm \rangle \lirsem \langle
  \sigma', \iota', \trace \rangle$, and 
$\exists \emph{\TT{x}} \in \sigma. \Big(\Gamma(\sigma(\emph{\TT{x}}))
\not\sqsubseteq \attacker \wedge \big(\forall \emph{\TT{y}} \in
\sigma. \emph{\TT{y}} \neq \emph{\TT{x}} \wedge 
\Gamma(\sigma(\emph{\TT{y}})) \sqsubseteq \attacker\big)\Big)$, 
then  $|\trace\proj| \leq \iota\proj(\emph{\TT{x}}) - \iota'\proj(\emph{\TT{x}})$. 
\end{myThm}

\begin{myThm}[Limited information release]
\label{thm:lir}
For any memory store $\sigma$, budget store $\iota$ and program
$\comm$ if $\langle \sigma, \iota, \comm \rangle \lirsem \langle \sigma',
\iota', \trace \rangle$, then $\comm$ satisfies limited information release for
$\sigma$ and $\iota$
\end{myThm}
