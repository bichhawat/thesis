\chapter{Related Work}
\label{ch:related}

Browser security is a very widely-studied topic. Here, only closely
related work on browser security policies and policy enforcement 
techniques is described. 

\medskip\noindent\textbf{Information flow control and script
  isolation.} With the widespread use of JavaScript,
research in dynamic techniques for IFC has regained momentum.
Nonetheless, static analyses are not completely futile. Guarn\-ieri
\emph{et al.}~\cite{guarnieri11ISSTA} present a static abstract
interpretation for tracking taints in JavaScript. However, the
omnipresent \texttt{eval} construct is not supported and this approach
does not take implicit flows into account.  Chugh \emph{et al.}
propose a staged information flow approach for 
JavaScript~\cite{stagedIFC}. They perform static server-side policy
checks on statically available code and generate residual
policy-checks that are applied to dynamically loaded code.  This
approach is limited to certain JavaScript constructs excluding dynamic 
features like dynamic field access or the \texttt{with} construct.

Austin and Flanagan~\cite{plas09} propose purely dynamic IFC for
dynamically-typed languages like JavaScript. They use the
no-sensitive-upgrade (NSU) check~\cite{zdancewic02PhD} to handle
implicit flows. Their per\-mis\-sive-upgrade strategy~\cite{plas10} is
more permissive than NSU but retains ter\-mi\-na\-tion-insensitive
non-interference. This work builds on the permissive-upgrade
strategy. They also present faceted evaluation~\cite{austin12POPL},
which is the most permissive technique of the three. However, given
the performance considerations, the technique is not suitable for
enforcing information flow control in browsers.  Just
et al.~\cite{just11PLASTIC} present dynamic IFC for JavaScript
bytecode with static analysis to determine implicit flows precisely
even in the presence of semi-unstructured control flow like
\texttt{break} and \texttt{continue}. Again, NSU is leveraged to
prevent implicit flows. This thesis builds on top of their work to
enforce information flow control in browsers.

JSFlow~\cite{csf12,jsflow} is a stand-alone implementation of
a JavaScript interpreter with fine-grained taint tracking. Many
seminal ideas for labeling and tracking flows in JavaScript owe their
lineage to JSFlow, but since JSFlow is written from scratch it has
very high overheads and  introduces annotations to deal with
semi-structured control flow. It detects
security violations due to branches that have not been executed and
injects annotations to prevent these in subsequent runs. The approach
presented in this thesis relies on analyzing CFGs and does not require
annotations. To improve permissiveness,
their subsequent work~\cite{esorics12} uses testing. 

Kerschbaumer et al.~\cite{crowdflow} build an implementation of an
information flow monitor for WebKit but do not handle all implicit
flows.  A black-box approach to enforcing non-interference is based on
secure multi-execution (SME)~\cite{SME}. Bielova \emph{et
  al.}~\cite{rnib} and De Groef \emph{et al.}~\cite{flowfox} implement
SME for web browsers. These systems do not attach labels to specific
fields in the DOM. Instead, labels are attached to individual DOM
APIs.

Chudnov and Naumann~\cite{chudnov-ccs} present another approach to
fine-grained IFC for JavaScript. They rewrite source programs to add
shadow variables that hold labels and additional code that tracks
taints. This approach is inherently more portable than that of JSFlow
or the work presented in this thesis, both of which are tied to
specific, instrumented browsers. However, it is unclear how this
approach could be extended with a policy framework like {\sys} that
assigns state-dependent labels at runtime.

The work most closely related to $\sys$ is that of
Vanhoef \emph{et al.}~\cite{csf14} on stateful declassification
policies in reactive systems, including web browsers. Their policies
are similar to the ones presented here, but there are significant
differences. First, 
their policies are attached to the browser and they are managed by the
browser user rather than website developers. Second, the policies have 
coarse-granularity: They apply uniformly to all events of a certain
type. Hence, it is impossible to specify a policy that makes
keypresses in a password field secret, but makes other keypresses
public. Third, the enforcement is based on secure
multi-execution~\cite{SME}, which is, so far, not compatible with
shared state like the DOM.

COWL~\cite{cowl} enforces mandatory access control at
coarse-granularity. In COWL, third-party scripts are sandboxed. Each
script gets access to either remote servers or the host's DOM, but not
both. Scripts that need both must be re-factored to pass DOM elements
over a message-passing API (\texttt{postMessage}). This can be both
difficult and have high overhead. For scripts that do not need this
factorization, COWL is more efficient than solutions based on FGTT.

Mash-IF~\cite{mashif} uses static analysis to enforce IFC
policies. Mash-IF's model is different from {\sys}'s model. Mash-IF
policies are attached only to DOM nodes and there is no support for
adding policies to new objects or events. Also, in Mash-IF, the
browser user (not the website developer) decides what
declassifications are allowed. Mash-IF is limited to a JavaScript
subset that excludes commonly used features such as \texttt{eval} and
dynamic property access.

JSand~\cite{jsand} uses server-side changes to the host page to
introduce wrappers around sensitive objects, in the style of object
capabilities~\cite{millerphd}. These wrappers mediate every access by
third-party scripts and can enforce rich access policies. Through
secure multi-execution, coarse-grained information flow policies are
also supported. However, as mentioned earlier, it is unclear how
secure multi-execution can be used with scripts that share state with
the host page.

{\sys} policies are enforced using an underlying IFC
component. Although, in principle, any IFC technique such as
fine-grained taint tracking~\cite{jang10CCS,jsflow,post14},
coarse-grained taint tracking~\cite{cowl} or secure
multi-execution~\cite{SME} can be used with {\sys}, to leverage the
full expressiveness of {\sys}'s finely-granular policies, a
fine-grained IFC technique is needed. 

\medskip\noindent\textbf{Access control.}  The traditional browser
security model is based on restricting scripts' access to data, not on
tracking how scripts use data. In the traditional model, it is
impossible to allow scripts access to data they need for legitimate
purposes and, simultaneously, to prevent them from leaking the data on
the side, which is the goal of IFC and {\sys}. More broadly, no
mechanism based only on access control can solve this
problem. Nonetheless, some closely related work on access
control in web browsers is discussed below.

All browsers today implement the same-origin policy (SOP)~\cite{sop},
which prevents a page and scripts included in it from making requests
to domains other than the page's host. However, for pragmatic reasons,
image requests are exempt, which is sufficient to leak
information. Consequently, the SOP is not effective against malicious
or buggy scripts. Cross-origin resource sharing (CORS)~\cite{cors}
relaxes the SOP further to allow some cross-origin requests.
%
%% Besides IFC, there is a significant amount of work on \emph{access
%%   control} in web browsers. Access control either allows or denies a
%% script access to data. It is ineffective in cases where a script
%% legitimately needs access to the data, but the risk is that it may
%% leak data on the side. Consequently, access control targets a simpler
%% problem than IFC and {\sys}---that of protecting data that the script
%% does not need in the first place. Nonetheless, we compare to some
%% closely related work on access control in web browsers.
%
Content security policies (CSPs)~\cite{csp} allow white- and
black-listing scripts. Unlike {\sys}, CSPs offer no protection against
scripts that have been included by the developer without realizing
that they leak information.

Conscript~\cite{conscript} allows the specification of fine-grained
access policies on individual scripts, limiting what actions every
script can perform. Similarly, AdJail~\cite{adjail} limits the
execution of third-party scripts to a shadow page and restricts
communication between the script and the host page.

Zhou and Evans~\cite{zhouESORICS11} take a dual approach, where
fine-grained access control rules are attached to DOM elements. The
rules specify which scripts can and cannot access individual
elements. Along similar lines, Dong \emph{et al.}~\cite{ccs13crypton}
present a technique to isolate sensitive data using authenticated
encryption. Their goal is to reduce the size of the trusted computing
base.

ADsafe~\cite{adsafe} and FBJS~\cite{fbjs} restrict third-party code to
subsets of JavaScript, and use static analysis to check for
illegitimate access. Caja~\cite{caja} uses object capabilities to
mediate all access by third-party scripts. Webjail~\cite{webjail}
supports least privilege integration of third-party scripts by
restricting script access based on high-level policies specified by
the developer.

All these techniques enforce only access policies and cannot control
what a script does with data it has access to.

