\chapter{Implementation in WebKit}
\label{ch:impl}

The ideas presented in the earlier chapters are incorporated into
WebKit, the browser engine used in Safari. The instrumentation is
built on top of the work by Just et al.~\cite{just11PLASTIC} and
instruments the three main components of the engine --- JavaScript 
bytecode interpreter, the document object model (DOM) engine, and the
event handling mechanism. The major benefit of working in the bytecode
interpreter as opposed to source is that the benefits of these years
of engineering efforts in optimizing the production interpreter and
the source to bytecode compiler can be retained.

WebKit’s JavaScript bytecode interpreter (JavaScriptCore) is
instrumented to implement the IFC semantics presented earlier for
enforcing dynamic IFC for JavaScript. In WebKit, bytecode is generated
by a source-code compiler. The goal here is to not modify the
compiler, but slight changes are requried to it to make it compliant
with the instrumentation. The modification is to emit a slightly
different, but functionally equivalent bytecode sequence for
\texttt{finally} blocks; this is needed for accurate 
computation of IPDs. 

In WebKit, bytecode is organized into code blocks. Each code block is
a sequence of bytecodes with line numbers and corresponds to the
instructions for a function or an \texttt{eval} statement. A code block is
generated when a function is created or an \texttt{eval} is executed.  
The instrumentation performs control flow analysis on a code block
when it is created and generates a CFG for it before it starts
executing. For exceptions, the synthetic exit node is added to the CFG
along with the edges as described earlier. The IPDs of its nodes are
calculated by static analysis of its bytecode; they are computed using
an algorithm by Lengauer and Tarjan with CFG as an input to the
algorithm~\cite{Lengauer}. 

The interpreter is a rather standard stack machine, with several
additional data structures for JavaScript-specific features like scope
chains, variable environments, prototype chains and function 
objects. The bytecode interpreter executes in a shared space with
other browser components also referred to as the heap. Local variables
are held in registers on the call stack. The instrumentation adds a
label to all data structures, including registers, object properties
and scope chain pointers, adds code to propagate explicit and implicit
labels and implements the permissive upgrade check. This also handles
the JavaScript-specific IFC concerns proposed
in~\cite{csf12,post14}. Additionally, all native JavaScript methods in
the Array, RegExp, and String objects are instrumented. 

The label is a word size bit-set (currently 64 
bits); each bit in the bit-set represents label from a distinct domain
(like google.com). Join on labels is simply bitwise or. The
formalization of the bytecodes, the semantics of its bytecode
interpreter with the instrumentation of dynamic IFC, and the proof of
correctness of instrumentation of the bytecodes with the IFC 
semantics is shown in~\cite{post14Extended}. The instrumentation
optimizes the common case of computations that mostly use local
variables (registers in the bytecode) by implementing a variant of
\emph{sparse labeling}~\cite{plas09}. Until a function reads a value
from the heap with a label different from the current $\pc$, labels  
are propagated only on heap-writes, but not on in-register
computations. Until that point, all registers are assumed to be
implicitly labeled with the $\pc$. 

Security labels are also attached to every node in the DOM graph and
all its properties, including pointer to other nodes. Appropriate 
IFC checks are added in the native C code implementing all DOM APIs up
to Level 3. The instrumentation includes changes to some of the
additional data structures in the DOM to prevent leaks shown
in~\cite{csf15}. Additional instrumentation carries the labels from the
native C code to the JavaScript interpreter. 

Input events on webpages like mouse clicks, key presses and network
receives trigger JavaScript functions called handlers. The event
handling logic of a browser is complex. Each input event can trigger
handlers registered not just on the node on which the event occurs
(e.g., the button which is clicked), but also on its ancestors in the
HTML parse tree. This is called \emph{event dispatch.} 
This can leak information implicitly through the
presence or absence of links between the node and its ancestors or
through different phases of event handling. To handle such leaks,
changes are made to the event handling loop, labeling every event and
event handler. The instrumentation is based on the formalization of
the event handling loop with the IFC checks as presented in~\cite{csf15}. 

The instrumentation adds approximately 7000 lines of code above the
existing code.  


