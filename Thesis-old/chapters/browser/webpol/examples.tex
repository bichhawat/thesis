\section{Expressiveness of {\sys}}
\label{sec:examples}

The expressiveness of {\sys} policies is illustrated through the
following examples.

\begin{lstlisting}[float, caption=Password strength checking script that leaks the password,label=egscript1,language=C]
var p = document.getElementById("pwd");
p.addEventListener("keypress", function (e){
  var score = checkPwdStrength(p.value);
  document.getElementById("pwdStrength").innerText = score; 
  new Image().src = "http://stealer.com/pwd.jsp?pwd="+p +score;
});
\end{lstlisting}

\subsection{Example 1: Password strength checker}  
Many websites deploy \emph{password strength checkers} on pages where
users set new passwords. A password strength checker is an event
handler from a third-party library that is triggered each time the
user enters a character in the new password field. The handler
provides visual feedback to the user about the strength of the
password entered so far. Strength checkers usually check the length of
the password and the diversity of characters used. Consequently, they
do not require any network communication. However, standard browser
policies cannot enforce this and the password strength checker can
easily leak the password if it wants to. Listing~\ref{egscript1} shows
such a ``leaky'' password checker. The checker installs a listener for 
keypresses in the password field (line 2). In response to every
keypress, the listener delivers its expected functionality by checking
the strength of the password and indicating this to the user (lines 3,
4), but then it leaks out the password to \texttt{stealer.com} by
requesting an image at a url that includes the password (lines 5, 6). 

With {\sys}, the developer of the host webpage can prevent any
exfiltration of the password by including the policy script:

\medskip
\texttt{document.getElementById("pwd").setLabel("HOST");}

\medskip 
\noindent This policy sets the label of the password field to the
host's own domain using the function
\texttt{setLabel()}. Subsequently, the IFC enforcement restricts all
outgoing communication that depends on the password field to the host.

Conceptually, this example is simple because it does not really
leverage the fine-granularity of {\sys} policies and fine-grained
dynamic IFC. Here, the third-party script does not need any network
communication for its intended functionality and, hence, simpler
confinement mechanisms that prohibit a third-party script from
communicating with remote servers would also suffice. The next example 
is a scenario where the third-party script legitimately needs remote
communication, which leverages the fine-granularity of {\sys} policies
and fine-grained dynamic IFC.

\begin{lstlisting}[float, caption=Currency converter script that leaks a private amount,label=egcc,escapechar=\%]
function currencyConverter() {
	var toCur = document.getElementById("to").value;
	var xh = new XMLHttpRequest();
	xh.onreadystatechange = function() { %\label{startcallback}%
		if (xh.readyState == 4) { %\iffalse && xhttp.status == 200 \fi %
			currencyRate = eval(xhttp.responseText);%\label{startsend}%
			var aAmt = document.getElementById("amt").value;
			var convAmt = aAmt * currencyRate;
			document.getElementById("camt").innerHTML = convAmt;%\label{ressend}%
			xh.open("GET","http://currConv.com/amount.jsp?atc=" + aAmt);%\label{endsend}%
			xh.send(); }} %\label{endcallback}%
	xh.open("GET","http://currConv.com/conv.jsp?toCur=" + toCur, true);
	xh.send(); }
\end{lstlisting}


\subsection{Example 2: Currency conversion}
Consider a webpage from an e-commerce website which displays the cost
of an item that the user intends to buy. The amount is listed in the
site's native currency, say US dollars (USD), but for the user's
convenience, the site also allows the user to see the amount converted
to a currency of his/her choice. For this, the user selects a currency
from a drop-down list. A third-party JavaScript library reads both the
USD amount and the second currency, converts the amount to the second
currency and inserts it into the webpage, next to the USD amount.
%
The third-party script fetches the current conversion rate from its
backend service at \texttt{currConv.com}. Consequently, it must send
the \emph{name} of the second currency to its backend service, but
must not send the amount being converted (that is private
information). The web browser's same-origin policy has been relaxed
(using, say, CORS~\cite{cors}) to allow the script to talk to its
backend service at \texttt{currConv.com}. The risk is that the script
can now exfiltrate the private amount. Listing~\ref{egcc} shows a
leaky script that does this. On line~\ref{endcallback}, the script
makes a request to its backend service passing to it the two
currencies. The callback handler (lines
\ref{startcallback}--\ref{endcallback}) reads the amount from the page
element \texttt{amt}, converts it and inserts the result into the page
(lines \ref{startsend}--\ref{ressend}). Later, it leaks out the amount
to the backend service on line~\ref{endsend}, in contravention of the
intended policy.

With {\sys}, this leak can be prevented with the following policy that
sets the label of the amount to the host only:

\medskip
\texttt{document.getElementById("amt").setLabel("HOST")}

\medskip
\noindent This policy will prevent exfiltration of the amount and will not
interfere with the requirement to exfiltrate the second
currency. Importantly, no modifications are required to a script that
does not try to leak data (e.g., the script obtained by dropping the
leaky line~\ref{endsend} of Listing~\ref{egcc}).

\begin{lstlisting}[float,caption=Policy that allows counting clicks but hides details of the clicks,label=eganal1]
var p = document.getElementbyId("sect_name");
p.addEventListener("click",function(event){
  event.setLabel("HOST"); });
\end{lstlisting}

\subsection{Example 3: Web analytics} 
To better understand how
users interact with their websites, web developers often include
third-party analytics scripts that track user clicks and keypresses to
generate useful artifacts like page heat-maps (which part of the page
did the user interact with most?). Although a web developer might be
interested in tracking only certain aspects of their users'
interaction, the inclusion of the third-party scripts comes with the
risk that the scripts will also record and exfiltrate other private
user behavior (possibly for monetizing it later). Using {\sys}, the
web developer can write precise policies on which user events an
analytics script can access and when. Several examples of this are
shown. 

\begin{lstlisting}[float, caption=Analytics script that counts clicks,label=egan]
clickCount = 0;
var p = document.getElementbyId("sect_name");  
p.addEventListener("click",function clkHdlr(e){ clickCount += 1; });
\end{lstlisting}

To allow a script to only count the number of occurrences of a class
of events (e.g., mouse clicks) on a section of the page, but to hide
the details of the individual events (e.g., the coordinates of every
individual click), the web developer can add a policy handler on the
top-most element of the section to set the label of the individual
event objects to \texttt{HOST}. This prevents the analytics script's
listening handler from examining the details of individual events, but
since the handler is still invoked at each event, it can count their
total number. Listings~\ref{eganal1} and~\ref{egan} show the policy
handler and the corresponding analytics script that counts clicks in a
page section named \texttt{sect\_name}.

%% \begin{lstlisting}[float,caption=Policy allowing selected access,label=egpol3]
%% window.addEventListener("click",function(event){
%%     event.setLabel("host.com");
%%     event.setContext("analytics.com");
%% });
%% \end{lstlisting}

%% \begin{lstlisting}[float, caption=Modified analytics script as per policy,label=eganp]
%% window.addEventListener("click",function clkHdlr(e){
%%     var coord = new Image(); 
%%     coord.src ="http://analytics.com/tracker.jsp?click=true";
%% });
%% \end{lstlisting}

\begin{lstlisting}[float,caption=Policy that only tracks whether a click
  happened or not,label=eganal2]
var alreadyClicked = false;
var p = document.getElementById("sect_name");
p.addEventListener("click",function(event){
  if (alreadyClicked = true) 
     event.setContext("HOST");
  else {
     alreadyClicked = true; 
     event.setLabel("HOST");
}});
\end{lstlisting}

Next, consider a restriction of this policy, which allows the
analytics script to learn only whether or not \emph{at least one}
click happened in the page section, completely hiding clicks beyond
the first. This policy can be represented in {\sys} using a local
state variable in the policy to track whether or not a click has
happened and the function \texttt{setContext()}. Listing~\ref{eganal2}
shows the policy. The policy uses a variable \texttt{alreadyClicked}
to track whether or not the user has clicked in the section. Upon the
user's first click, the policy handler sets the event's label to the
host's domain (line~8). This makes the event object private but allows
the analytics handler to trigger and record the occurrence of the
event. On every subsequent click, the policy handler sets the event's
\emph{context} to the host domain using \texttt{setContext()}
(line~5). This prevents the analytics script from exfiltrating any
information about the event, including the fact that it occurred.

Finally, note that a developer can subject different page sections to
different policies by attaching different policy handlers to them. The
most sensitive sections may have a policy that unconditionally sets
the event context to the host's, effectively hiding all user events in
those sections. Less sensitive sections may have policies like those
of Listings~\ref{eganal2} and~\ref{eganal1}. Non-sensitive sections
may have no policies at all, allowing analytics scripts to see all
events in them.

\begin{lstlisting}[float, caption=Example policy to prevent overlay-based
  stealing of keystrokes,label=overlay,language=C]
document.body.addEventListener("keypress", function(event){
    var o = window.getComputedStyle(event.target).getPropertyValue("opacity");
    if (o < 0.5) 
         event.setLabel("HOST");
});
\end{lstlisting}

\subsection{Example 4: Defending against overlay-based attacks}
To bypass {\sys} policies, an adversarial script may ``trick'' a user
using transparent overlays. For example, suppose a script wants to
exfiltrate the contents of a password field that is correctly
protected by a {\sys} policy. The script can create a transparent
overlay on top of the password field. Any password the user enters
will go into the overlay, which \emph{isn't} protected by any policy
and, hence, the script can leak the password.

Such attacks can be prevented easily in {\sys} using a \emph{single}
policy, attached to the top element of the page, that labels data
entered into all significantly transparent overlays as
\texttt{HOST}. Listing~\ref{overlay} shows such a policy. This
particular policy labels all keypress events on elements of opacity
below $0.5$ as \texttt{HOST}, thus preventing their exfiltration. The
threshold value $0.5$ can be changed, and the policy can be easily
extended to other user events like mouse clicks.

\subsection{Summary of {\sys} expressiveness} 
The security community has extensively studied several
aspects of policy labeling, colloquially called the \emph{dimensions
  of declassification} (see~\cite{dimDecl} for a survey). Broadly
speaking, {\sys} policies cover three of these dimensions---the
policies specify what data is declassified (dimension: what), to which
domains (dimension: to whom) and under what state (dimension:
when). ``What data is declassified'' is specified by selectively
attaching policies to elements of the page. ``Which domains get
access'' is determined directly by the labels that the policy
sets. Finally, labels generated by policy handlers can depend on
state, as illustrated in Listing~\ref{eganal2}.

There are two other common dimensions of labeling---who can label the
data (dimension: who) and where in the code can the labels change
(dimension: where). These dimensions are fixed in {\sys} due to the
specifics of the problem: All policies are specified by the host page
in statically defined policies.

