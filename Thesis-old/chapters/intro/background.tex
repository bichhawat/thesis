\chapter{Background and Overview}
\label{ch:background}

\section{Information Flow Control}
Information flow control (IFC) refers to controlling the flow of
(confidential) information through a program based on a given security
policy. Typically, pieces of information are classified into security
\emph{labels} and the policy is a lattice over labels. Information is only
allowed to flow up the lattice. For illustration purposes often the
smallest non-trivial lattice $L \sqsubset H$ is used, which specifies that
public (low, $L$) data must not be influenced by confidential (high,
$H$) data. Information flow control can be used to provide
confidentiality (or integrity) of secret (trusted) information; the
work in this thesis is, however, limited to confidentiality guarantees. 
Roughly the idea behind information flow control is that an adversary
can view all the public outputs of a program. By preventing private or
sensitive data to flow to public outputs, the adversary does not get
any information about the private or sensitive data.

The seminal work by Denning~\cite{denning76,denning77,denning82}
defined most of the theoretical ideas pertaining to information flow
control. In general, information can flow along many
channels. However, this thesis considers two of the most important
flows --- \textit{explicit} and \textit{implicit} --- in deterministic
programs. Covert channels like timing or resource usage are beyond the
scope of this thesis.  

An \emph{explicit flow} occurs as a result of direct assignment, e.g.,
the statement \TT{public = secret + 1} causes an explicit flow from
\TT{secret} to \TT{public}. An \emph{implicit flow} occurs due to the
control structure of the program. For instance, consider the program
in Listing~\ref{lst1}.  
% \texttt{public = false; if (secret) public =  true},
 The final value of \TT{y} is equal to the value of
\TT{z} even though there is no direct assignment from \TT{z}
to \TT{y}. Leaking a bit like this can be magnified into leaking
a bigger secret bit-by-bit~\cite{askarov}.

\begin{lstlisting}[float,caption=Implicit flow from \TT{z} to
  \TT{y},label=lst1][escapechar=@]
x = false, y = false
if (not(z))@\label{linerefcond1}@
  x = true@\label{lineref}@
if (not(x))@\label{linerefcond}@
  y = true@\label{lineref5}@
\end{lstlisting}

The correctness of the approaches enforcing information flow control
is often stated in terms of a well-defined property known 
as \emph{non-interference}~\cite{goguen}, which basically stipulates
that high input of a program must not influence its low output. While
non-interference is too strong a property in practice, different
variants of the definition are proven. One such variant of
non-interference usually established for information flow control
techniques is \emph{termination-insensitive
  non-interference}~\cite{volpano}. Roughly, a program is
termination-insensitive non-interferent if any two terminating runs 
of the program starting from low-equivalent heaps (i.e., heaps that
look equivalent to the adversary assuming that an adversary can
observe some part of the heap) end in low-equivalent
heaps. Termination-insensitive means that one-bit of leak is tolerable
when an adversary checks whether or not the program terminated. In
particular, this discounted one-bit leak accounts for termination due
to the failure of a runtime security  check. Askarov et
al.~\cite{askarov} show that for programs with intermediate observable
outputs, termination-insensitivity may leak more than one bit but also
show that this attack is limited to a brute-force attack. 

Information flow control approaches either statically analyze the
program at compile-time and reject or accept a program based on the
security policy, or dynamically analyze the flow of data in a program
by either using a modified runtime or with the help of a reference
monitor, or employ a combination of both the approaches to shrug off 
some of the individual limitations of the two approaches. Static approaches
normally support restricted policies, and are impermissive in
nature. Besides, it becomes extremely complicated to use such methods
with dynamically-typed languages and languages that are loaded with
dynamic features. Dynamic approaches, on the other hand, add runtime
overheads and are less precise as compared to the static
approaches. Dynamic information flow control, which is the central
theme of this thesis, is described in detail later in the
chapter. 

\subsection{Flow-sensitivity}
Static analysis techniques are usually flow-insensitive in
nature, i.e., they do not account for the ordering of the instructions
in the program or the general flow of execution of the program. In
other words, all operations should be individually secure to secure
the program as a whole. 
% For instance, consider the program in Listing~\ref{lst1}. 
% Assume that \TT{z} is a secret variable labeled $H$
% and \TT{x} and \TT{y} are public variables labeled $L$. This program gets
% rejected by a flow-insensitive analysis as the operation on
% line~\ref{lineref} is considered insecure.  
%
% Flow-sensitive analysis improves the permissiveness of static analysis
% techniques and is mostly used in dynamic analysis. Under a
% flow-sensitive static analysis, the program in Listing~\ref{lst1}
% upgrades the labels of \TT{x} and subsequently \TT{y} to $H$ indicating the
% influence of $H$-labeled \TT{z} on \TT{x} and \TT{y}. If the program was to
% output the value of \TT{y} on a $L$-observable channel, the program gets
% rejected by the analysis. Flow-sensitive analysis also accepts
% programs with dead-code like the one shown in Listing~\ref{lst1.1.1},
% thus improving the permissiveness of the analysis. Quite a few
% flow-sensitive static analyses have been proposed to enforce
% information flow control~\cite{hunt2006:types,hammer09ijis} while
% almost all dynamic approaches are flow-sensitive in nature. 
%
For instance, consider the program snippet:
$$\TT{l} = \TT{h} $$
Assume that \TT{h} is a secret variable labeled $H$
and \TT{l} is a public variable labeled $L$. This program gets
rejected by a flow-insensitive analysis as the assignment of a secret
value to a public variable is considered insecure.  

Flow-sensitive analysis improves the permissiveness of static analysis
techniques and is mostly used in dynamic analyses. Under a
flow-sensitive static analysis, the above program 
upgrades the labels of \TT{l} to $H$ indicating the
influence of $H$-labeled \TT{h} on \TT{l}. If the program was to
output the value of \TT{l} on a $L$-observable channel later, the
program gets rejected by the analysis. Flow-sensitive analysis also
accepts programs with dead-code like the one shown in
Listing~\ref{lst1.1.1}, 
thus improving the permissiveness of the analysis. Some 
flow-sensitive static analyses have been proposed to enforce
information flow control~\cite{hunt2006:types,hammer09ijis} while
almost all dynamic approaches are flow-sensitive in nature. 

\begin{lstlisting}[float,caption=Permissiveness of flow-sensitive
  analysis,label=lst1.1.1][escapechar=@] 
var x
if (not(z))
  x = true
x = false
\end{lstlisting}

\section{Information Release}
Non-interference prevents all unauthorized flows without
regarding the severity of the leaks. However, many practical
applications require some fragment of the sensitive data to be
released to programs for providing proper functionality or to improve
the user-experience. For instance, according to non-interference even
a standard login application would be considered insecure because it
leaks information about the user’s password, which is considered 
confidential, by either allowing or denying access. 
As a remedy, researchers have proposed various ways to intentionally
release or \emph{declassify} sensitive data through relaxations of
non-interference~\cite{dimDecl}, which is generally enforced using
annotations like \texttt{declassify} in the program. For instance,
\texttt{declassify(pwd == input)} would treat the result 
of the password check as public, thus allowing access-information to
be released to the user. Most of these approaches require policy
specifications by the developer that define what information can be
released by the program as in many settings it is difficult to trust 
third-parties to include correct and appropriate (declassify)
annotations for information release in the code. Declassification
approaches generally ensure that the adversary is unable to declassify
or force declassification of sensitive information as per his/her
needs. Sabelfeld and Sands~\cite{dimDecl} present a survey
of many such techniques that allow release of information in a secure
manner. 

Some of these declassification techniques are described below.
Cohen~\cite{cohen77,cohen78} in his work on selective
dependency used assertions to eliminate certain information paths
thereby allowing programs that satisfy the constraint to be
accepted. The idea was that information flows to the adversary should
not allow her to deduce more information about secrets than is allowed
by the assertions. Volpano and Smith~\cite{relSec} present relative
secrecy, a property that instead of providing absolute
secrecy allows information to be released through specific
\emph{match} queries relative to the size of the secret, i.e., if the
adversary cannot guess the secret in polynomial time. Giacobazzi and
Mastroeni~\cite{ani} generalize non-interference by modelling what 
property the adversary can observe about the secrets. Li and
Zdancewic's relaxed non-interference characterizes the information
released through downgrading policies~\cite{rel-ni}. Sabelfeld and
Sands~\cite{permodel} proposed the PER model using partial equivalence
relations for specifying information flow policies.
Sabelfeld and Myers~\cite{delRelease} introduced the notion of
delimited release, which enables the specification of declassification
policy using \emph{escape hatches}. The policy specifies upfront what
information can be released through these escape hatches. Localized
delimited release~\cite{ldr} requires that an expression is only released 
after a declassification operation for this expression has appeared
in the code. Lux and Mantel~\cite{Lux} propose explicit reference
points to allow flexible specification of what secrets may be
declassified at the specific reference points. Robust
declassification~\cite{robust} ensures that only trusted code is
allowed to declassify information. Chong and Myers~\cite{chong2004}
propose a framework for specifying application-specific information
release policies. Chong~\cite{rir} further proposes the concept of
required information release, which considers applications that are
obligated to release information under certain conditions. Askarov and 
Sabelfeld introduce the concept of gradual release~\cite{gradRelease}
for allowing release of information at certain release or declassification
points. The policy does not allow the adversary knowledge to refine
its knowledge of secrets at
points other than the release points. Banerjee et al.~\cite{cgrSP}
present a way to specify declassification policies 
that satisfy conditioned gradual release, which is an extension of the
gradual release property. Similarly, another security property,
\emph{policy controlled release}, that extends gradual release was
presented by Rocha et al.~\cite{rochaSP} for declassification in
untrusted and legacy programs.

\section{Dynamic Information Flow Control}
\label{ch:bg-difc}
Dynamic IFC usually works by tracking taints or labels on individual
program values in the language runtime. A label represents a mandatory
access policy on the value. 
% For illustration purposes in this thesis, the
% following notations are used ---  the label $L$ (low
% confidentiality) conventionally means that data may be read by an
% (unspecified but fixed) adversary and $H$ (high confidentiality) means
% the data is sensitive or confidential and is not visible to the
% adversary. More generally, labels may be drawn from any lattice of 
% policies, with higher labels representing more restrictive policies.
A value $v$ labeled $\lab$ is written $v^\lab$. 

Flow-sensitive dynamic IFC analysis \emph{propagates} labels as data
flows during program execution. \emph{Explicit} flows are generally
handled by carrying over the label of the computed value to the
variable being assigned. For example, in the statement \TT{x = y + z},
the result of computing \TT{y + z} will have the label that is a join of
the individual labels on \TT{y} and \TT{z}, which is the final label
of \TT{x}, i.e., if either of \TT{y} or \TT{z} is labeled confidential
or $H$, then the final label of \TT{x} is also labeled
$H$\footnote{``\TT{z} is labeled $H$'' actually means ``the 
  value in \TT{z} is labeled $H$''. This convention is used
  consistently.}. 

\emph{Implicit} flows in a flow-sensitive IFC analysis are tracked by
maintaining an additional taint, usually called the program counter
taint or program context taint or $\pc$, which is an upper bound on
the label of all the control dependencies that lead to the current
instruction being executed. For example, in the program of
Listing~\ref{lst1}, the value in variable \TT{x} at the end of
line~\ref{lineref} depends on the value in \TT{z}. If \TT{z} is labeled $H$,
then at line~\ref{lineref}, $\pc = H$ because of the branch in 
line~\ref{linerefcond1} that depends on \TT{z}. Thus, by tracking $\pc$,
dynamic IFC can enforce that \TT{x} has label $H$ at the end of
line~\ref{lineref}, thus taking into account the control dependency.

However, simply tracking control flow dependencies via $\pc$ is not
enough to guarantee absence of information flows when labels are
flow-sensitive, i.e., when the same variable may hold values with
different labels depending on what program paths are executed. The
program in Listing~\ref{lst1} is a classic counterexample, taken
from~\cite{plas09}. Assume that \TT{z} is labeled $H$ and \TT{x} and \TT{y} are
labeled $L$ initially. The final value in \TT{y} is computed as a function
of the value in \TT{z}. If \TT{z} contains $\texttt{true}^H$, then \TT{y} ends
with $\texttt{true}^L$: The branch on line~\ref{linerefcond1} is not
taken, so \TT{x} remains $\texttt{false}^L$ at
line~\ref{linerefcond}. Hence, the branch on line~\ref{linerefcond} is
taken, but $pc = L$ at line~\ref{lineref5} and \TT{y} ends with
$\texttt{true}^L$. If \TT{z} contains $\texttt{false}^H$, then similar
reasoning shows that \TT{y} ends with $\texttt{false}^L$. Consequently,
in both cases \TT{y} ends with label $L$ and its value is exactly equal
to the value in \TT{z}. Hence, an adversary can deduce the value of \TT{z}
by observing \TT{y} at the end (which is allowed because \TT{y} ends with
label $L$). So, this program leaks information about \TT{z} despite
correct use of $pc$.

\subsection{No-sensitive-upgrade Check}
\label{sec:bg-nsu}
Preventing leaks due to implicit flow in dynamic IFC requires coarse
approximation because a dynamic monitor only sees program branches
that are executed and does not know what assignments may happen in
alternate branches in other executions. One such coarse approximation
is the \emph{no-sensitive-upgrade} (NSU) check proposed by
Zdancewic~\cite{zdancewic02PhD}. In the program in Listing~\ref{lst1},
\TT{x}'s label is upgraded from $L$ to $H$ at line~\ref{lineref} in one of
the two executions above, but not the other. Subsequently, information
leaks in the other execution (where \TT{x}'s label remains $L$) via the
branch on line~\ref{linerefcond}. The NSU check stops the leak by
preventing the assignment on line~\ref{lineref}. More generally, it
stops a program whenever a public variable's label is upgraded due to
a high $\pc$. This check suffices to provide termination-insensitive
non-interference as shown by Austin and Flanagan~\cite{plas09}. 

\subsection{Permissive-Upgrade and Faceted Execution}
To tackle the issue of permissiveness with the no-sensitive-upgrade
strategy, Austin and Flanagan proposed 
the permissive-upgrade strategy~\cite{plas10} and faceted
execution~\cite{austin12POPL}; faceted execution being the most
permissive of the three. Faceted execution draws inspiration from
secure multi-execution and simulates multiple executions
simultaneously within a single runtime. They introduce the concept of
faceted values that are pairs of values for both low and high
observers. With multiple levels, each value in the pair is represented
as a pair. When branching on a faceted value, multiple executions are
simulated for the different values in the facet. Although the approach
incurs lesser overhead when compared to secure multi-execution, the
overheads are still quite prohibitive for multiple security
levels. This thesis, thus, considers only the permissive-upgrade
strategy, which is more permissive than the no-sensitive-upgrade
strategy and much less performance-intensive compared to faceted
execution. Permissive-upgrade is described in detail in
Section~\ref{sec:existing}. 

\subsection{Other Approaches for Permissiveness}
% \paragraph{Secure Multi-Execution}
Secure multi-execution~\cite{SME} is another approach for enforcing
non-interference at runtime. Instead of tracking information flow
through the program, the approach executes multiple copies of the
program with different values of sensitive data. Conceptually, one
executes the same code once for each security level (like low and
high) with a few constraints. The private data in the low execution
are replaced by default values, i.e., the public copy of the
program does not see the actual value of the private data but a
pre-determined default value, and outputs on an $\ell$-labeled channel 
are permitted only in the $\ell$-level execution of the program, i.e.,
the high execution of the program outputs on the high channel, if any,
and the low execution of the program outputs on the low channel,
typically the network. 
%
The modification of the semantics forces that private data and the
outputs resulting from it are visible to only high-level
observers. The public-level observers or the adversary observe
inaccurate results for the outputs that depend on the private value as
the value is replaced by the default value in that execution. 
This modification forces even unsafe programs to
adhere to non-interference. Additionally, secure multi-execution
guarantees precision, i.e., the semantics of a secure program is
not altered. Thus, for a secure program the outputs are all
accurate. Secure multi-execution normally guarantees
termination-insensitive non-interference as the high execution may not
terminate in some cases. FlowFox~\cite{flowfox} demonstrates secure
multi-execution in the context of web browsers. 
%
However, executing a program multiple times can be prohibitive for a
security lattice with multiple levels~\cite{austin12POPL}. The runtime
overhead incurred can be reduced if the executions are run in parallel
(which requires more hardware resources), though the program has 
to be run for all levels irrespective of whether it uses private data
or not. In addition to this, secure multi-execution makes
declassification complicated as it requires synchronization between
different executions~\cite{fineSME}. 

Birgisson \emph{et al.}~\cite{esorics12} describe a testing-based
approach that adds variable upgrade annotations to avoid halting on
the NSU check in an implementation of dynamic IFC for
JavaScript~\cite{csf12}. Hritcu \emph{et al.}  improve permissiveness
by making IFC errors recoverable in the language
Breeze~\cite{Hritcu:ifc}. This is accomplished by a combination of two
methods: making all labels public (by upgrading them when necessary in
a public $\pc$) and by delaying exceptions. A different way of handling the 
problem of implicit flows through flow-sensitive labels is to assign a
(fixed) label to each label; this approach has been examined in recent
work by Buiras \emph{et al.}\ in the context of a language with a
dedicated monad for tracking information flows~\cite{buiras14CSF}. The
precise connection between that approach and permissive-upgrade
remains unclear, although Buiras \emph{et al.}\ sketch a technique
related to permissive-upgrade in their system, while also noting that
generalizing permissive-upgrade to arbitrary lattices is
non-obvious. This thesis confirms the latter and shows how it can be
done.

% \Blindtext