\chapter{Introduction}

With the growth in the use of computers and the Internet for almost
every application, the amount of sensitive information that 
programs compute has also increased dramatically. For instance,
e-commerce websites have access to the personal details of a user
including her address, credit-card details etc. Similarly, a hospital
database stores sensitive and private medical data of its
patients. While the complexity  of such systems has increased
manifold in the last few decades, the privacy and confidentiality
guarantees still remain questionable~\citep{infoleaks}. Often leaks
occur either due to buggy programs or due to malicious
code. Cryptographic techniques protect data by encrypting it but many 
programs need to operate on confidential data, which requires the data 
to be available in plaintext form. % An adversary can then attempt to
% infer the confidential data by observing such operations in the
% system.
To this end, \emph{access control} techniques are widely used
for preventing leakage of confidential data. However, once the
authorized users have access to data, there is no control on how the
data can be used. The data might be input to a program that writes to
publicly-observable locations or outputs data that is accessible to
all users.  
% For instance, a system
% can restrict access of particular confidential files to a certain set
% of authorized users.

Consider the Web, for instance. Web applications rely extensively on
third-party JavaScript to provide useful libraries, page analytics,
advertisements and many other features~\cite{nick12CCS}. JavaScript
works on a mashup model, wherein the hosting page and included scripts
share the page’s state (called the DOM). Consequently, by design, all
included third-party scripts run with the same access privileges as
the hosting page. While some third-party scripts are developed by
large, well-known, trustworthy vendors, many other scripts are
developed by small, domain-specific vendors whose commercial motives
do not always align with those of the 
webpage providers and users. This leaves sensitive information such as
passwords, credit card numbers, email addresses, click histories,
cookies and location information vulnerable to inadvertent bugs and
deliberate exfiltration by third-party scripts. In many cases,
developers are fully aware that a third-party script accesses
sensitive data to provide useful functionality, but they are unaware
that the script also leaks that data on the side. In fact, this is a
widespread problem~\cite{jang10CCS}. 
%
The traditional browser security
model is based on restricting scripts’ access to data, not on tracking
how scripts use data. Existing web security standards
and web browsers address this problem unsatisfactorily, favoring
functionality over privacy. The same-origin policy implemented
in all major browsers restricts a webpage and third-party scripts
included in it to communicating with web servers from the including
webpage’s domain only. However, broad exceptions are allowed. For
instance, there is no restriction on request parameters in urls that
fetch images and, unsurprisingly, third-party scripts leak information
by encoding it in image urls. The candidate web standard Content
Security Policy~\cite{csp}, also implemented in most browsers,
allows a page to white list scripts that may be included, but places
no restriction on scripts that have been included, thus not helping
with the problem above. 
%
Quite a few \emph{fine-grained} access control
techniques have also been proposed~\cite{conscript, adjail,
  zhouESORICS11, ccs13crypton, adsafe, fbjs, caja, webjail}. 
% For example, Conscript~\cite{conscript} allows the specification of  
% fine-grained access policies on individual scripts, limiting what
% actions every script can perform. Similarly, AdJail~\cite{adjail}
% limits the execution of third-party scripts to a shadow page and
% restricts communication between the script and the host page. Zhou and
% Evans~\cite{zhouESORICS11} take a dual approach, where fine-grained
% access control rules are attached to DOM elements. The rules specify
% which scripts can and cannot access individual elements. Along similar
% lines, Dong et al.~\cite{ccs13crypton} present a technique to isolate
% sensitive data using authenticated encryption. ADsafe~\cite{adsafe}
% and FBJS~\cite{fbjs} restrict third-party code to subsets of
% JavaScript, and use static analysis to check for illegitimate
% access. Caja~\cite{caja} uses object capabilities to mediate all
% access by third-party scripts. Webjail~\cite{webjail} supports least 
% privilege integration of third-party scripts by restricting script
% access based on high-level policies specified by the
% developer.
However, all these techniques enforce only access policies
and cannot control what a script does with data it has been provided
in good faith, i.e., if a third-party script was allowed access to
some data only for local computations, the policy places no
restriction on the script for sending it on the network while
performing the computation. More broadly, no mechanism based only
on access control can solve the problem of information leakage.  

The academic community has proposed solutions based on
\emph{information flow control} (IFC), also known as mandatory access
control.
% \emph{Information flow control} (IFC) is an
% elegant solution for such problems.
It ensures the security of
confidential information even in the presence of untrusted and buggy
code. The idea is to track the flow of information through the program
and prevent any undesired flows based on a security policy. 
Research has considered static methods such as type checking and
program analysis, which verify the security policy at compile
time~\cite{denning76, denning77, myersJFlow, volpano, pottier2003,
  hunt2006:types, LBIFS, hammer09ijis}, 
dynamic methods that track information flow through program execution
at runtime~\cite{fenton, plas09, plas10, Askarov09, Sabelfeld10, SME,
  csf12, austin12POPL, jeeves, plas14, cowl},    
and hybrid approaches that combine both static and dynamic analyses to add
precision to the analysis~\cite{Nentwich07, Gurvan06, Gurvan07,
  russo10CSF, vogt07NDSS, post14, csf15Hedin} 
for handling information leaks described above. 

Dynamic analysis has the drawback of being less precise and
introducing significant performance overheads at runtime, though it
can be more permissive than static analysis methods in some
cases~\cite{russo10CSF}. Although helpful in most 
scenarios, static analyses are mostly ineffective when working with
dynamic languages like JavaScript, which is an indispensable part of
the modern Web. The dynamic nature of
JavaScript~\cite{richards11ECOOP, oopsla13} with features like dynamic
typing, \texttt{eval},  scope-chains, prototype chains etc. makes sound
static analysis difficult. Thus, recent research has focussed on
dynamic analysis for enforcing information flow control especially in
languages like JavaScript. Even though research in dynamic information
flow control has made significant inroads in the last decade, the
applicability of these techniques still remains bleak, 
\emph{permissiveness} being the major challenge to
the practicality of dynamic information flow control. 

In recent years, researchers have explored methods to quantify the
amount of  sensitive information leaked by a program as an alternative
to the qualitative notion of information flow control. The developer
determines an ``acceptable'' upper-bound on the number of bits of
sensitive information a program may leak. Programs that leak less than
this ``acceptable'' amount of information are considered secure. The
information released is generally quantified as the
knowledge gained by an adversary about the sensitive data as a result
of the information flow in the program. Specifically, the adversary
has knowledge about the possible set of initial values that can be
associated with the sensitive data. A flow in the program could reduce
the number of possibilities, thereby increasing the knowledge of the
adversary about the specific value associated with the sensitive data
in that execution of the program. This change in knowledge can be
measured as the number of bits of information of sensitive data
released to the adversary~\cite{CoverQIF}. Statically quantifying the
information leaked by a program has been well studied in the
literature and various static quantitative information flow
measures~\cite{shannon, guessing, smith2009, clarkson2009,
  csf12GLeakage} and techniques~\cite{denning82, clark, clarkson2009,
  smith2009, backes, kopf} have been proposed. However,
quantifying information leaks in a purely dynamic setting is still
largely an open problem (prior work is limited to~\cite{mccamant}) as
dynamic analysis is generally limited to the current execution of the
program in contrast to the static methods, which analyze the program
as a whole.  

% However, in highly dynamic scenarios like the web where the
% third-party code is unknown upfront, it is difficult for the developer
% to specify an appropriate policy with declassification specifications
% that caters to all possible codes. 

% \todo[inline]{Static quantification of information leaks}

\section{Contributions of the Thesis}
This thesis focuses on improving the usability of dynamic information
flow techniques by presenting mechanisms that enhance the precision
and permissiveness of the analyses. The main contributions are as
follows: 
\begin{itemize}
\item \textbf{Generalized Permissive-Upgrade.} To improve the
    permissiveness of dynamic information flow analysis, the thesis 
    presents a sound improvement and enhancement of the
    permissive-upgrade strategy (a strategy widely used to enforce
    dynamic information flow control). The development improves the
    original strategy's permissiveness and applicability by
    generalizing the approach to an arbitrary security lattice, in
    place of the two-point lattice considered in prior work. 
\item \textbf{Precise Dynamic Information Flow Control.} Most of the
  existing work in dynamic information flow control does not handle
  complex features like unstructured control flow and exceptions. The
  proposals that handle these features are too conservative and
  require additional annotations in the program. This thesis presents a
  sound and precise dynamic information flow analysis for handling
  these features without requiring any additional annotations from the
  programmer. 
\item \textbf{Bounding Information Leaks Dynamically.} Although
  non-interference is a desired property for enforcing information
  flow control, there are program instances that  require legitimate
  release of some parts of the secret data to provide the required
  functionality. Towards this end, this thesis develops a sound
  approach to quantify information leaks dynamically while allowing
  information release in accordance with a pre-specified \emph{budget}. 
\item \textbf{Application to a Web Browser.} The thesis concludes by
  applying these techniques to an information flow control-enabled Web
  browser. An information flow control system enforces security
  policies that are a collection of rules for labeling private
  information sources, generally specified by a policy component in
  the system. To complement the work on enforcement components in Web
  browsers, this thesis explores a policy specification mechanism
  to specify flexible and useful information flow policies for Web applications. 
\end{itemize}
 
% % \Blindtext