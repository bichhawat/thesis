\section{Improved Permissive-Upgrade Strategy}
\label{sec:ipus}

% As described in the previous chapter, the NSU check is restrictive
% and halts many programs that do not leak information. 
% To improve
% permissiveness, the permissive-upgrade strategy was proposed as a
% replacement for NSU by Austin and Flanagan~\cite{plas10}. 
% To allow a dynamic IFC to accept safe executions of programs with
% variable upgrades due to high $\pc$, Austin and Flanagan proposed a
% less restrictive strategy called
% \emph{permissive-upgrade}~\cite{plas10}. Whereas NSU stops a program 
% when a variable's label is upgraded due to assignment in a high $pc$,
% permissive-upgrade allows the assignment, but labels the variable
% \emph{partially-leaked} or $P$. The taint $P$ roughly means that the
% variable's content in this execution is $H$, but it may be $L$ in
% other executions. The program must be stopped later if it tries to use
% or case-analyze the variable (in particular, branching on a
% partially-leaked Boolean variable is stopped). Permissive-upgrade also
% ensures termination-insensitive non-interference, but is strictly more
% permissive than NSU. For example, permissive-upgrade stops the leaky
% program of Listing~\ref{lst1} at line~\ref{linerefcond} when \TT{z}
% contains $\texttt{false}^H$, but it allows the program of
% Listing~\ref{lst1.1} to execute to completion when \TT{y} contains
% $\texttt{true}^L$. Please note that in this section, the lattice has
% only two levels: $L$ (public) and $H$ (confidential). 

% \subsection{Improved Permissive-Upgrade on a Two-Point Lattice}
% This section presents an enhancement of the original
% permissive-upgrade strategy with improved permissiveness. 

% Austin and Flanagan~\cite{plas10} introduce a new label $P$ for
% ``partially-leaked''. The intuition behind the partially-leaked label
% $P$ is the following: 

% \begin{framed}
% \noindent
%   A variable with a value labeled $P$ may have been implicitly
%   influenced by $H$-labeled values in this execution, but in other
%   executions (obtainable by changing $H$-labeled values in the
%   initial store), this implicit influence may not exist and, hence,
%   the variable may be labeled $L$.
% \end{framed}

% Thus, assignments under high $\pc$ succeed under the
% permissive-upgrade check but branching or case-analyzing a
% partially-leaked value is not permitted as that can also leak
% information. For illustration, consider the different executions of
% example in Listing~\ref{lst1} taken from~\cite{plas10} as shown in
% Table~\ref{tbl:nsu}. The assignment 
% on line~\ref{lineref} is not permitted by the NSU check. However, the
% permissive-upgrade check updates \TT{y} to \texttt{false} with a
% partially-leaked label. Further case analysis on \TT{y} is prohibited (if
% not, the branch on line~\ref{linerefcond} would not be taken as \TT{y} is
% \texttt{false} and \TT{z} would return $\texttt{true}^{L}$, thus leaking
% the value of \TT{x}).

% \begin{table*}
% \centering
% \begin{tabular} {|l||@{\,}c@{\,}||@{\,}c@{\,}|@{\,}c@{\,}|}
% \hline
% &
% $\TT{x} = \texttt{false}^{H}$
% &
% \multicolumn{2}{c|}{$\TT{x} = \texttt{true}^{H}$}
% \\
% \cline{2-4}
% &
% \textit{All strategies}
% &
% \textit{NSU}
% &
% \textit{Permissive-upgrade}
% \\
% \hline
% \TT{y} = \texttt{true} &$\TT{y} = \texttt{true}^{L}$&$\TT{y} =
%                                               \texttt{true}^{L}$&$\TT{y} =
%                                                                   \texttt{true}^{L}$\\
% \TT{z} = \texttt{true} &$\TT{z} = \texttt{true}^{L}$&$\TT{z} =
%                                               \texttt{true}^{L}$&$\TT{z} =
%                                                                   \texttt{true}^{L}$\\
% \texttt{if} (\TT{x})&branch not taken&$\pc = H$&$\pc = H$\\
% \quad\TT{y} = \texttt{false}&$\TT{y} = \texttt{true}^{L}$&execution halted&$\TT{y} = \texttt{false}^{P}$\\
% \texttt{if} (\TT{y})&$\pc = L$&&execution halted\\
% \quad$\TT{z} = \texttt{false}$&$\TT{z} = \texttt{false}^{L}$&&\\
% \texttt{return} \TT{z}&$\texttt{false}^{L}$&&\\
% \hline
% Result & $\TT{z} = \texttt{false}^{L}$ & no leak & no leak\\
% \hline
% \end{tabular}
% \caption{Execution steps in two runs of the program from
%   Listing~\ref{lst1}, with NSU and the permissive-upgrade check}
% \label{tbl:nsu}
% \end{table*}

The original permissive-upgrade strategy, however, lacks 
permissiveness; it rejects secure programs like the one shown in
Listing~\ref{lst2}. Consider that \TT{x} is labeled $H$
and \TT{w}, \TT{y} are labeled $L$. With the original permissive-upgrade
strategy, the label of \TT{z} on
line~\ref{lineref2} would remain $P$ and the execution would be
terminated when branching on \TT{z} on line~\ref{lineref3}. The
improvement $$H \sqcup P = H$$ allows the analysis to accept such programs
while remaining sound. With the improvement, \TT{z} would be labeled
$H$ on line~\ref{lineref2}, which would allow the execution to branch
on line~\ref{lineref3}, thus, 
taking the execution to completion. 
The idea behind the improvement is that an $H$-labeled value is
never observable at $L$-level. Similarly, the result of any operation
involving an $H$-labeled value is also never observable at
$L$-level. Thus, any operation involving a partially-leaked value and
a $H$-labeled value does not reveal any information to an adversary at
level $L$ about the partially leaked value. 

\begin{lstlisting}[float,label=lst2,caption=Example showing the
  impermissiveness of the original permissive-upgrade strategy][escapechar=@]
y = false
if (not(x))
  y = true@\label{lineref1}@
z = y || x@\label{lineref2}@
if (not(z))@\label{lineref3}@
  w = true
\end{lstlisting}

% \begin{figure}[htbp]
% \begin{align*}
% \lab          :=~& L~\arrowvert~H\\
% \pc          :=~& \lab\\
% k,l,m :=~& \lab~\arrowvert~P\\ \\
% k \sqcup k ~= ~& k\\
% L \sqcup H ~=~ &H\\
% L \sqcup P  ~=~&P \\
% \color{blue}{H\sqcup P  ~=~}&\color{blue}{H} 
% % \\
% % \mathit{if} ~\pc = L ~\mathit{then}~ H\sqcup P  ~=~&H
% \end{align*}
% \caption{Syntax of labels including the partially-leaked label
%   $P$}\label{pus:syntax}
% \end{figure}

% Labels $k,l,m$ on values are either elements of the lattice ($L, H$)
% or $P$. The $\pc$ can only be one of $L, H$ because branching on
% partially-leaked values is prohibited. This is summarized by the
% revised syntax of labels in Figure~\ref{pus:syntax}. The figure also
% lifts the join operation $\sqcup$ to labels including
% $P$. In the original permissive-upgrade~\cite{plas10}, joining any
% label with $P$ resulted in $P$. 

The final label $k$ in the assignment rule \refrule{bs:cmd:ap} under
the improved permissive-upgrade strategy becomes: 
% \begin{mathpar}
% \inferrule*[left=\mbox{\labelthis{bs:cmd:ap}{assn-pus}}]
% {l := \Gamma(\sigma(\TT{x})) \qquad \langle
%  \sigma, \expr \rangle \bsexp \TT{n}^m} {\langle \sigma, \TT{x} := \expr
%     \rangle \bscmd \sigma[\TT{x} \mapsto \TT{n}^{k}]}
% \end{mathpar}
% where $k$ is defined as follows:
$$ k = % \left\lbrace
\begin{cases}
m & \mbox{ if } \pc = L \\
H  & \mbox{ if } \pc = H \mbox{ and } l = H\\
P  & \mbox{ otherwise}
\end{cases} % \right.
$$

% The first two conditions in the definition of $k$ correspond to the
% NSU rule (Figure~\ref{fig:assn-nsu}). The third condition applies, in
% particular, when a variable whose initial label is $L$ is assigned to
% under $\pc = H$. The NSU check would stop this assignment. With
% permissive-upgrade, however, the updated variable is labeled $P$,
% consistent with the intuitive meaning of $P$. This allows more
% permissiveness by allowing the assignment to proceed in all 
% cases. To compensate, any program (in particular, an adversarial
% program) is disallowed from case analyzing any value labeled 
% $P$. Consequently, in the rules for \texttt{if-then} and
% \texttt{while} (Figure~\ref{fig:basic-semantics}), it is required that
% the label of the branch condition be of form $\lab$, which does not
% include $P$.

The soundness results of the original permissive-upgrade strategy can
be extended to show the soundness of the improved permissive-upgrade
strategy. However, a significant difficulty in proving the theorem
using the modified notation for the imperative language is that the
definition of $\sim$ is not transitive. The same problem arises for
the soundness proofs in~\cite{plas10}. There, the authors resolve the
issue by defining a special relation called evolution. The need for
evolution is averted here using the auxiliary lemmas listed
below. Lemma~\ref{lem:evol} proves the required result substituting
evolution.  

\begin{myLemma}[Expression Evaluation]
\label{lem:expeval}
If $\langle \sigma_1, e \rangle \bsexp \emph{\TT{n}}_1^{k_1}$ and $\langle 
\sigma_2, e \rangle \bsexp \emph{\TT{n}}_2^{k_2}$ and $\sigma_1 \sim_L
\sigma_2$, then $\emph{\TT{n}}_1^{k_1} \sim_L \emph{\TT{n}}_2^{k_2}$.
\end{myLemma}
\begin{proof} By induction on $e$.
\end{proof}

\begin{myLemma}[Evolution]
\label{lem:evol}
  If $~\pc = H$ and $\langle \sigma, c \rangle \bscmd \sigma'$, then
  \\ $\forall \emph{\TT{x}}.\Gamma(\sigma(\emph{\TT{x}})) = P
  \implies \Gamma(\sigma'(\emph{\TT{x}})) = P$. 
\end{myLemma}
\begin{proof} By induction on the derivation rules and case
  analysis on the last rule.
\end{proof}

\begin{myLemma}[Confinement for improved permissive-upgrade with a
  two-point lattice]
\label{lem:conf}
  If $~\pc = H$ and $\langle \sigma, c \rangle
  \bscmd \sigma' $, then $\sigma \sim_L \sigma'$.
\end{myLemma}
\begin{proof} By induction on the derivation rules.
\end{proof}

\begin{myThm}[TINI for improved permissive-upgrade with a two-point lattice]
  With the assignment rule \refrule{bs:cmd:ap} and the modified syntax of
  Figure~\ref{pus:syntax}, if 
  $~\sigma_1 \sim_L \sigma_2$ and $\langle \sigma_1, c \rangle
  \bscmd \sigma_1' $ and $\langle \sigma_2, c
  \rangle\bscmd \sigma_2' $, then $\sigma_1' \sim_L
  \sigma_2'$.
\end{myThm}
\begin{proof} By induction on $c$ and case
  analysis on the last step.
\end{proof}

The detailed proofs are provided in Appendix~\ref{app:ipu}. 
Note that the definitions and proofs presented in this chapter are
specific to the two-point lattice and with respect to an adversary at
level $L$. 

