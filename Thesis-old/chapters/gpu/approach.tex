\section{Generalized Permissive-Upgrade on Arbitrary Lattices}
\label{sec:gen:pus}

\begin{figure}
\begin{equation*}
\begin{aligned}[c]
\lab          =~& L~\arrowvert~M~\arrowvert~\ldots~\arrowvert~H\\
\pc          =~& \lab \\
k,l,m =~& \lab~\arrowvert~\lab\pl 
\end{aligned}
\qquad \qquad 
\begin{aligned}[c]
\lab_1 \sqcup \lab_2\pl  =~&(\lab_1 \sqcup \lab_2)\pl \\
\lab_1\pl  \sqcup \lab_2\pl  =~&(\lab_1 \sqcup \lab_2)\pl
\end{aligned}
\end{equation*}
\caption{Labels and label operations}\label{fig:labels}
\end{figure}

This section shows by construction the generalization of the
permissive-upgrade strategy to arbitrary security lattices. For every
element $\lab$ of the lattice, a new label $\lab\pl$ is introduced
which means ``partially-leaked $\lab$'', with the following intuition: 
\begin{framed}
\noindent
A variable labeled $\lab\pl$ may contain partially-leaked data, where
$\lab$ is a \emph{lower-bound} on the $\star$-free labels the variable
may have in alternate executions.
\end{framed}

The syntax of labels is listed in Figure~\ref{fig:labels}. Labels
$k,l,m$ may be lattice elements $\lab$ or $\star$-ed lattice elements
$\lab\pl$. In examples, suggestive lattice element names $L, M, H$
(low, medium, high) are used. Labels of the form $\lab$ are called 
$\star$-free or \emph{pure}. Figure~\ref{fig:labels} also defines the
join operation $\sqcup$ on labels. This definition is based on the intuition
above. When the two operands of $\odot$ are labeled $\lab_1$ and
$\lab_2\pl$, $\lab_1 \sqcup \lab_2$ is a lower bound on the pure label
of the resulting value in any execution (because $\lab_2$ is a lower
bound on the pure label of $\lab_2\pl$ in any run). Hence, $\lab_1
\sqcup \lab_2\pl = (\lab_1 \sqcup \lab_2)\pl$. The reason for the
definition $\lab_1\pl \sqcup \lab_2\pl = (\lab_1 \sqcup \lab_2)\pl$ is
similar.

\begin{figure}
\begin{mathparpagebreakable}
\inferrule*[left=\mbox{\labelthis{bs:cmd:agn}{assn-n}}]
{  \langle \sigma, \expr \rangle \bsexp \TT{n}^m \\ l =
  \Gamma(\sigma(\TT{x})) \\ l = \lab_\TT{x} \vee l = \lab_\TT{x}\pl \\ \pc 
  \sqsubseteq \lab_\TT{x} \\ k = \pc \sqcup m}
{\langle \sigma , \TT{x} := \expr \rangle \bscmd \sigma[\TT{x}
  \mapsto \TT{n}^{k}]}
%%%    
\and
\inferrule*[left=\mbox{\labelthis{bs:cmd:ags}{assn-s}}]
{\langle \sigma, \expr \rangle \bsexp \TT{n}^m \\ l =
  \Gamma(\sigma(\TT{x})) \\ l = \lab_\TT{x} \vee  l = \lab_\TT{x}\pl \\ \pc
  \not\sqsubseteq \lab_\TT{x} \\ k = ((\pc \sqcup m)\sqcap \lab_\TT{x} )\pl}
{\langle \sigma , \TT{x} := \expr \rangle \bscmd \sigma[\TT{x} \mapsto
  \TT{n}^{k}]}
\end{mathparpagebreakable}
\caption[Caption]{Assignment rules for the generalized
  permissive-upgrade}\label{fig:assn-our}
\end{figure}

The rules for assignment are shown in Figure~\ref{fig:assn-our}. They
strictly generalize the rule \refrule{bs:cmd:ap} for the two-point lattice,
treating $P = L\pl$. Rule~\refrule{bs:cmd:agn} applies when the existing label of
the variable being assigned to is $\lab_{\TT{x}}$ or $\lab_{\TT{x}}\pl$ and $\pc
\sqsubseteq \lab_{\TT{x}}$. The key intuition behind the rule is the
following: If $\pc \sqsubseteq \lab_{\TT{x}}$, then it is safe to overwrite
the variable, because $\lab_{\TT{x}}$ is necessarily a lower bound on the
(pure) label of $\TT{x}$ in this and any alternate execution (see the
\framebox{framebox} above). Hence, overwriting the variable cannot
cause an implicit flow. As expected, the label of the overwritten
variable is $\pc \sqcup m$, where $m$ is the label of the value
assigned to $\TT{x}$.

Rule \refrule{bs:cmd:ags} applies in the remaining case --- when $\pc
\not\sqsubseteq \lab_{\TT{x}}$. In this case, there may be an implicit flow,
so the final label on $\TT{x}$ must have the form $\lab\pl$ for some
$\lab$. The question is which $\lab$. Intuitively, it may seem that
one could choose $\lab = \lab_{\TT{x}}$, the pure part of the original label
of $\TT{x}$. The final label on $\TT{x}$ would be $\lab_{\TT{x}}\pl$ and this would
satisfy the intuitive meaning of $\star$ written in the
\framebox{framebox} above. Indeed, this intuition suffices for the
two-point lattice of Section~\ref{sec:existing}
and~\ref{sec:ipus}. However, for a more 
general lattice, this intuition is unsound, as illustrated with an
example below. The correct label is $((\pc \sqcup m) \sqcap
\lab_{\TT{x}})\pl$. 
% (Note that this correct label is independent of the label $m$ of the value
% assigned to $\TT{x}$. This is sound because $\TT{x}$ is $\star$-ed and
% cannot be case-analyzed later, so the label on the value in it is irrelevant.)

\begin{figure}[ht]
\begin{minipage}{0.65\linewidth}
\centering
\begin{lstlisting}[caption=Example explaining rule \refrule{bs:cmd:ags},label=list2]
if ($\TT{x}'$)
  $\TT{z} = \TT{y}_1$@\label{hm1}@
else
  $\TT{z} = \TT{y}_2$@\label{hm2}@
if ($\TT{x}_1$)@\label{bl1}@
  $\TT{z} = \TT{x}_1$@\label{hl1}@
if ($\texttt{not}(\TT{x}_2)$)@\label{if2}@
  $\TT{z} = \TT{x}_2$@\label{hl2}@
if ($\TT{z}$)@\label{if3}@
  $\TT{w} = \TT{z}$@\label{hl3}@
\end{lstlisting}
\end{minipage}
\hspace{-1.5cm}
\begin{minipage}{0.45\linewidth}
\centering
{\includegraphics{chapters/gpu/lattice.tikz}}
\caption{Lattice explaining rule \refrule{bs:cmd:ags}}
\label{fig:lattice}
\end{minipage}
\end{figure}

\paragraph{Example} 
The need for the label $k := ((\pc \sqcup m) \sqcap \lab_{\TT{x}})\pl$
instead of $k := \lab_{\TT{x}}\pl$ in rule \refrule{bs:cmd:ags} is illustrated
below. Consider the lattice of Figure~\ref{fig:lattice} and the
program of Listing~\ref{list2}. Assume 
that, initially, the variables $\TT{z}$, $\TT{w}$, $\TT{x}_1$,
$\TT{x}'$, $\TT{x}_2$, $\TT{y}_1$ and $\TT{y}_2$ have labels $H$,
$L_1$, $L_1$, $L'$, $L_2$, $M_1$ and $M_2$, 
respectively. Fix the attacker at level $L_1$. Fix the value of $\TT{x}_1$
at $\texttt{true}^{L_1}$, so that the branch on line~\ref{bl1} is
always taken and line~\ref{hl1} is always executed. Set $\TT{y}_1 \mapsto
\texttt{false}^{M_1}, \TT{y}_2 \mapsto \texttt{true}^{M_2}, \TT{w} \mapsto
\TT{false}^{L_1}$ initially. The initial value of $\TT{z}$ is
irrelevant. Consider two executions of the program starting from two
stores $\sigma_1$ with $\TT{x}' \mapsto \texttt{true}^{L'}, \TT{x}_2 \mapsto
\texttt{true}^{L_2}$ and $\sigma_2$ with $\TT{x}' \mapsto
\texttt{false}^{L'}, \TT{x}_2 \mapsto \texttt{false}^{L_2}$. Note that
as $L'$ and $L_2$ are incomparable to $L_1$ in the lattice,
$\sigma_1$ and $\sigma_2$ are equivalent for $L_1$. 

\begin{table*}
\centering
\begin{tabular} {|l||@{\,}c@{\,}||@{\,}c@{\,}|@{\,}c@{\,}|}
\hline
&
\multicolumn{3}{c|}{$\TT{w} = \texttt{false}^{L_1},\ \TT{x}_1 =
  \texttt{true}^{L_1},\ \TT{y}_1 = \texttt{false}^{M_1},\ \TT{y}_2 =
  \texttt{true}^{M_2}$}
\\
\cline{2-4}
&
$\TT{x}' = \texttt{true}^{L'}$
&
\multicolumn{2}{c|}{$\TT{x}' = \texttt{false}^{L'}$}
\\
&
$\TT{x}_2 = \texttt{true}^{L_2}$
&
\multicolumn{2}{c|}{$\TT{x}_2 = \texttt{false}^{L_2}$}
\\
\cline{3-4}
&
&
$k:= \lab_{\TT{x}}\pl$
&
$k:= ((\pc \sqcup m) \sqcap \lab_{\TT{x}})\pl$
\\
\hline
\texttt{if} ($\TT{x}'$)&$\pc = L'$&&\\
\quad$\TT{z} = \TT{y}_1$&$\TT{z} = \texttt{false}^{M_1}$&&\\
\texttt{else}&&$\pc = L'$&$\pc = L'$\\
\quad$\TT{z} = \TT{y}_2$&&$\TT{z} = \texttt{true}^{M_2}$&$\TT{z} = \texttt{true}^{M_2}$\\
\texttt{if} ($\TT{x}_1$)&$\pc = L_1$&$\pc = L_1$&$\pc = L_1$\\
\quad$\TT{z} = \TT{x}_1$&$\TT{z} = \texttt{true}^{L_1}$&$\TT{z} = \texttt{true}^{M_2\pl}$&$\TT{z} = \texttt{true}^{L\pl}$\\
\texttt{if} ($\texttt{not}(\TT{x}_2)$)&branch not taken&$\pc = L_2$&$\pc = L_2$\\
\quad$\TT{z} = \TT{x}_2$& %$z = \texttt{true}^{L_1}$
&$\TT{z} = \texttt{false}^{L_2}$&$\TT{z} = \texttt{false}^{L\pl}$\\
\texttt{if} ($\TT{z}$)&$\pc = L_1$&branch not taken&execution halted\\
\quad$\TT{w} = \TT{z}$&$\TT{w} = \texttt{true}^{L_1}$&&\\
\hline
Result & $\TT{w} = \texttt{true}^{L_1}$ & $\TT{w} =
\texttt{false}^{L_1}$ (leak) & no leak\\
\hline
\end{tabular}
\caption{Execution steps in two runs of the program from Listing~\ref{list2}, with two variants of the rule \refrule{bs:cmd:ags}}
\label{tblassn}
\end{table*}

Requiring $k := \lab_{\TT{x}}\pl$ in rule \refrule{bs:cmd:ags} causes an
implicit flow that is observable for $L_1$. The intermediate values
and labels of the variables for executions starting from $\sigma_1$
and $\sigma_2$ are shown in the second and third columns of
Table~\ref{tblassn}. Starting with $\sigma_1$, line~\ref{hm1} is
executed, but line~\ref{hm2} is not, so $\TT{z}$ ends with
$\texttt{false}^{M_1}$ at line~\ref{bl1} (rule \refrule{bs:cmd:agn} applies at
line~\ref{hm1}). At line~\ref{hl1}, $\TT{z}$ contains $\texttt{true}^{L_1}$
(again by rule \refrule{bs:cmd:agn}) and line~\ref{hl2} is not executed. Thus, the
branch on line~\ref{if3} is taken and $\TT{w}$ ends with
$\texttt{true}^{L_1}$ at line~\ref{hl3}. Starting with $\sigma_2$,
line~\ref{hm1} is not executed, but line~\ref{hm2} is, so $\TT{z}$ becomes
$\texttt{true}^{M_2}$ at line~\ref{bl1} (rule \refrule{bs:cmd:agn} applies at
line~\ref{hm2}). At line~\ref{hl1}, rule \refrule{bs:cmd:ags} applies, but because
$k := \lab_{\TT{x}}\pl$ is assumed in that rule, $\TT{z}$ now contains the
value $\texttt{true}^{M_2\pl}$. As the branch on line~\ref{if2} is
taken, at line~\ref{hl2}, $\TT{z}$ becomes $\texttt{false}^{L_2}$ by rule
\refrule{bs:cmd:agn} because $L_2 \sqsubseteq M_2$. Thus, the branch on
line~\ref{if3} is not taken and $\TT{w}$ ends with $\texttt{false}^{L_1}$
in this execution. Hence, $\TT{w}$ ends with $\texttt{true}^{L_1}$ and
$\texttt{false}^{L_1}$ in the two executions, respectively. The
attacker at level $L_1$ can distinguish these two results; hence, the
program leaks the value of $\TT{x}'$ and $\TT{x}_2$ to $L_1$.

With the correct \refrule{bs:cmd:ags} rule in place, this leak is avoided (last
column of Table~\ref{tblassn}). In that case, after the assignment on
line~\ref{hl1} in the second execution, $\TT{z}$ has label $((L_1 \sqcup L_1) \sqcap
M_2)\pl = L\pl$. Subsequently, after line~\ref{hl2}, $\TT{z}$ gets the
label $L\pl$. As case analysis on a $\star$-ed value is not allowed,
the execution is halted on line~\ref{if3}. This guarantees
termination-insensitive non-interference with respect to the attacker
at level $L_1$.

\subsection{Termination-Insensitive Non-interference}

To prove non-interference for the generalized permissive-upgrade,
equivalence of labeled values relative to an adversary at arbitrary
lattice level $\lab$ needs to be defined. The definition is shown
below (Definition~\ref{def:gpua:veq}). Note that clauses (3)--(5) here
refine clause (3) of Definition~\ref{def:eq-existing} for the two-point
lattice. The obvious generalization of clause (3) of
Definition~\ref{def:eq-existing} --- $\TT{n}_1^k \sim_\lab \TT{n}_2^m$ whenever 
either $k$ or $m$ is $\star$-ed --- is too coarse to prove
non-interference inductively. For the degenerate case of the two-point
lattice, this definition also degenerates to
Definition~\ref{def:eq-existing} (there, $\lab$ is fixed at $L$, $P = 
L\pl$ and only $L$ may be $\star$-ed).

\begin{mydef}
\label{def:gpua:veq}
Two values $\emph{\TT{n}}_1^k$ and $\emph{\TT{n}}_2^m$ are
$\lab$-equivalent, written $\emph{\TT{n}}_1^k \sim_\lab
\emph{\TT{n}}_2^m$, iff either 
\begin{enumerate}
\item $k = m = \lab' \sqsubseteq \lab$ and $\emph{\TT{n}}_1 =
  \emph{\TT{n}}_2$, or 
\item $ k = \lab'
  \not\sqsubseteq \lab$ and $m = \lab'' \not\sqsubseteq \lab$, or 
\item $k = \lab_1\pl$ and $m = \lab_2\pl$, or
\item $k = \lab_1\pl$ and $m = \lab_2$ and ($\lab_2 \not\sqsubseteq
  \lab$ or $\lab_1 \sqsubseteq \lab_2 $), or
\item $k = \lab_1$ and $m = \lab_2\pl$ and ($\lab_1 \not\sqsubseteq
  \lab$ or $\lab_2 \sqsubseteq \lab_1$)
\end{enumerate}
\end{mydef}

\begin{mydef}
\label{def:gpua:seq}
  Two stores $\sigma_1$ and $\sigma_2$ are $\lab$-equivalent,
  written $\sigma_1 \sim_\lab \sigma_2$, iff for every variable \emph{\TT{x}},
  $\sigma_1(\emph{\TT{x}}) \sim_\lab \sigma_2(\emph{\TT{x}})$.
\end{mydef}

This definition is obtained by constructing (through examples) an
extensive transition graph of pairs of labels that may be assigned to
a single variable at corresponding program points in two executions of
the same program. The starting point is label-pairs of the form
$(\lab, \lab)$. This characterization of equivalence is both
sufficient and necessary. It is sufficient in the sense that it allows
us to prove TINI inductively. It is necessary in the sense that
example programs can be constructed that end in states exercising
every possible clause of this definition. Appendix~\ref{app:egequi}
lists these examples. 

% \section{Termination-Insensitive Non-Interference}

Using the above definition of equivalence of labeled values, TINI can
be proven for the generalized permissive-upgrade strategy presented
above. A significant difficulty in proving the theorem is that the
definition of $\sim_\lab$ is not transitive unlike the previous definition
of $\sim$. 
% The same problem arises
% for the two-point lattice as shown in the previous section and
% in~\cite{plas10}. There, the authors resolve the 
% issue by defining a special relation called evolution. 
% Here, a more conventional approach based on the standard confinement 
% lemma is taken. 
% The need for evolution is averted using several auxiliary
% lemmas that we list below.
Detailed proofs of all the lemmas and the theorems
are presented in Appendix~\ref{app:gpu}.

\begin{myLemma}[Expression evaluation]
\label{lem:gpua:exp}
If $\langle \sigma_1, e \rangle \bsexp \emph{\TT{n}}_1^{k_1}$ and $\langle 
\sigma_2, e \rangle \bsexp \emph{\TT{n}}_2^{k_2}$ and $\sigma_1 \sim_\lab \sigma_2$,
then $\emph{\TT{n}}_1^{k_1} \sim_\lab \emph{\TT{n}}_2^{k_2}$.
\end{myLemma}
\begin{proof}
By induction on $e$.
\end{proof}

%% Lemma~\ref{sup1} shows that a value labeled $\lab\pl$ remains
%% $\star$-ed if an evaluation is done on it under a higher or unrelated
%% context. Lemma~\ref{pcl} on the other hand shows that if a value gets
%% a pure label after an evaluation, then either the value remained
%% unchanged after the evaluation or the evaluation done was done in a
%% lower context. We state two important corollaries,
%% Corollary~\ref{cor1} and~\ref{cor2} derived from Lemma~\ref{sup1}
%% and~\ref{pcl}, respectively, which are used in the proofs for
%% confinement and non-interference.

\begin{myLemma}[$\star$-preservation]
\label{lem:gpua:sup1}
If $\langle \sigma, c \rangle \bscmd \sigma'$ and
$\Gamma(\sigma(x)) = \lab\pl $ and $\pc \not\sqsubseteq \lab$, then
$\Gamma(\sigma'(x)) = \lab'\pl$ and $\lab' \sqsubseteq \lab$.
\end{myLemma}
\begin{proof}
By induction on the derivation rule.
\end{proof}

\begin{mycor}
\label{cor:gpua:cor1}
If $\langle \sigma, c \rangle \bscmd \sigma'$ and
$\Gamma(\sigma(\emph{\TT{x}})) = \lab\pl $ and
$\Gamma(\sigma'(\emph{\TT{x}})) = \lab'$, then 
$\pc \sqsubseteq \lab$.
\end{mycor}
\begin{proof}
Immediate from Lemma~\ref{lem:gpua:sup1}.
\end{proof}

\begin{myLemma}[$\pc$-lemma]
\label{lem:gpua:pcl}
If $\langle \sigma, c \rangle \bscmd \sigma'$ and
$\Gamma(\sigma'(\emph{\TT{x}})) = \lab$, then $\sigma(\emph{\TT{x}}) =
\sigma'(\emph{\TT{x}})$ or 
$\pc \sqsubseteq \lab$.
\end{myLemma}
\begin{proof}
By induction on the derivation rule.
\end{proof}

\begin{mycor}
\label{cor:gpua:cor2}
If $\langle \sigma, c \rangle \bscmd \sigma'$ and
$\Gamma(\sigma(\emph{\TT{x}})) = \lab\pl $ and
$\Gamma(\sigma'(\emph{\TT{x}})) = \lab'$, then 
$\pc \sqsubseteq \lab'$.
\end{mycor}
\begin{proof}
Immediate from Lemma~\ref{lem:gpua:pcl}.
\end{proof}

Using these lemmas, the standard confinement lemma and non-interference
can be proven. 

\begin{myLemma}[Confinement Lemma]
\label{lem:gpua:conf}
If $\pc \not\sqsubseteq \lab$ and $\langle \sigma, c \rangle
\bscmd \sigma'$, then $\sigma \sim_\lab \sigma'$.
\end{myLemma}
\begin{proof}
By induction on the derivation rule.
\end{proof}

\begin{myThm}[TINI for generalized permissive-upgrade for arbitrary lattices]
  If $~\sigma_1 \sim_\lab \sigma_2$ and $\langle \sigma_1, c \rangle
  \bscmd \sigma_1' $ and $\langle \sigma_2, c
  \rangle\bscmd \sigma_2' $, then $\sigma_1' \sim_\lab
  \sigma_2'$.
\end{myThm}
\begin{proof} By induction on $c$.
\end{proof}


%%%%%%%%%%%%%%%%%%%%%%%%%%%%%%%%%%%%%%%%%%%%%%%%%%%%%%%%

%%%%%%%%%%%%%%%%%%%%%%%%%%%%%%%%%%%%%%%%%%%%%%%%%%%%%%%%

\section{Comparison of the Generalization of Section~\ref{sec:gen:pus}
  with the Generalization of Section~\ref{sec:gen:ipus}}

%\subsection{Permissiveness of the Approach in Section~\ref{sec:gen:pus}}

Two distinct and sound generalizations of the permissive-upgrade
strategy for the two-point lattice have now been described:
The generalization of the improved permissive-upgrade to pointwise
products of two-point lattices or, equivalently, to powerset lattices
as described in Section~\ref{sec:gen:ipus}, and the generalization to
arbitrary lattices described in Section~\ref{sec:gen:pus}. For brevity,
these generalizations are called puP (Section~\ref{sec:gen:ipus}) and
puA (Section~\ref{sec:gen:pus}), 
respectively (P and A stand for \underline{p}owerset and
\underline{a}rbitrary, respectively). Since both puP and puA apply to
powerset lattices, an obvious question is whether one is more
permissive than the other on such lattices. The generalization of puP
presented in this chapter can be more permissive than puA on powerset 
lattices in certain cases as shown by the example
below. The reason for this permissiveness is that puP tracks finer
taints, i.e., it tracks partial leaks for each principal separately. 

% We show here that the
% permissiveness of puP and puA on powerset lattices is
% \emph{incomparable} --- there are examples on which each
% generalization is more permissive than the other. We explain one
% example in each direction below. Roughly, incomparability exists
% because puP tracks finer taints (it tracks partial leaks for each
% principal separately), but puA's rules for overwriting
% partially-leaked variables are more permissive.

\begin{figure}[ht]
\begin{minipage}{0.65\linewidth}
\centering
\begin{lstlisting}[caption=Example where puP is more permissive than puA,label=list3]
if (y) @\label{li32}@
  x = z @\label{li33}@
if (z) @\label{li34}@
  x = z @\label{li35}@
if (x)  @\label{li36}@
  z = x
\end{lstlisting}
\end{minipage}
\hspace{-1.5cm}
\begin{minipage}{0.45\linewidth}
\centering
{\includegraphics{chapters/gpu/powerset.tikz}}
\caption{A powerset/product lattice}\label{fig:lattice1}
\end{minipage}
\end{figure}

\paragraph{Example}
The powerset lattice of Figure~\ref{fig:lattice1} is used for
illustration purpose. This lattice is the pointwise lifting of the order $L
\sqsubset H$ to the set $S = \{L, H\} \times \{L, H\}$. For brevity,
this lattice's elements are written as $LL$, $LH$, etc. When puP is
applied to this lattice, labels are drawn from the set $\{L, H, P\}
\times \{L, H, P\}$. These labels are concisely written as $LP$, $HL$,
etc. For puA, labels are drawn from the set $S \cup S\pl$. These
labels are written $LH$, $LH\pl$, etc. Note that $LH\pl$ parses as
$(LH)\pl$, not $L(H\pl)$ (the latter is not a valid label in puA
applied to this lattice).
%
Consider the program in Listing~\ref{list3}. Assume that $\TT{x}$,
$\TT{y}$ and $\TT{z}$ have 
initial labels $LL$, $HL$ and $LH$, respectively and that the initial
store contains $\TT{y} \mapsto \texttt{true}^{HL}, \TT{z} \mapsto
\texttt{true}^{LH}$, so the branches on lines~\ref{li32}
and~\ref{li34} are both taken. The initial value in $\TT{x}$ is irrelevant
but its label is important.
Under puA, $\TT{x}$ obtains label $(((HL) \sqcup (LH)) \sqcap (LL))\pl = LL\pl$ at
line~\ref{li33} by rule \refrule{bs:cmd:ags}. At line~\ref{li35}, the same rule
applies but the label of $\TT{x}$ remains $LL\pl$. When the program tries
to branch on $\TT{x}$ at line~\ref{li36}, it is stopped.
In contrast, under puP, this program executes to completion. At
line~\ref{li33}, the label of $\TT{x}$ changes to $PH$ by rule
\refrule{bs:cmd:agp}. At line~\ref{li35}, the label changes to $LH$ because $pc$
and the label of $\TT{z}$ are both $LH$. Since this new label has no $P$,
line~\ref{li36} executes without halting.
Hence, for this example, puP is more permissive than puA.

% \section{Remarks}
% On powerset lattices, the resulting IFC monitor is different
% in the two cases and the generalization of improved permissive-upgrade
% is more permissive than puA in certain cases. 
% The approach presented in this section is, thus, more general
% and applies to a broader set of lattices. 
An example for which  puA is more permissive than puP in the case of
poweset lattices was not found. But the generalization puA presented in
Section~\ref{sec:gen:pus} is more general than the product
construction puP (Section~\ref{sec:gen:ipus}) when applied to arbitrary
lattices (and hence, applicable to a broader set of lattices) as it is
unclear whether or how  the improved permissive 
upgrade strategy generalises to arbitrary lattices. By developing this
generalization, this work makes permissive-upgrade applicable to
arbitrary security lattices like other IFC techniques.
% and, hence,
% constitutes a useful contribution to IFC literature.

