\chapter{Conclusion}
\label{ch:conclusion}

Security and privacy have been of paramount importance since computer
systems and applications have started handling private, sensitive and
confidential information. The need of the hour is to carefully handle
such data as it interacts with various untrusted third-party
applications. Although helpful in many scenarios, static information
flow analyses are mostly ineffective when working with dynamic
languages like JavaScript, which is an indispensable part of the
modern web. The dynamic nature of JavaScript makes sound static
analysis difficult. Dynamic information flow control is a promising
step forward, though the permissiveness of dynamic analyses presents a
major challenge to the practicality of these techniques. This thesis
focuses on improving the usability of dynamic information flow
techniques by:  
 
\begin{itemize}
\item \textbf{Developing mechanisms that enhance the precision and 
  permissiveness of the analyses:} To improve the precision of dynamic
information flow analysis, this thesis develops a sound approach for  
handling complex language features like unstructured control flow and
exceptions. The existing approaches to handle these features are too
conservative generating a lot of false-positives and often require
additional annotations by the developer. In contrast, the methodology
presented in this thesis performs a sound and precise dynamic
information flow analysis using post-dominator analysis at runtime to
handle these features without requiring any additional annotations
from the developer. The approach is also shown to be the most precise 
approach for handling such complex features dynamically. 

To further improve the permissiveness of dynamic information flow
analysis, this thesis presents the design of a sound improvement and
enhancement of the permissive-upgrade strategy. The development
improves the original strategy's permissiveness by relaxing the rules
for handling partially-leaked data while retaining soundness. The
original strategy's enforcement was limited to a two-point security
lattice, and lacked generalization to an arbitrary lattice (as is
required for real-world scenarios). To this end, this thesis presents
a non-trivial approach to generalize the applicability of the approach
to an arbitrary security lattice.  

\item \textbf{Proposing a technique to bound the release of sensitive
  information in realistic applications:} Although dynamic
quantification has been studied earlier, bounding information leaks
dynamically remains an open problem. To this end, this thesis 
develops a sound approach to bound information leaks dynamically by 
allowing information release in accordance to a pre-specified budget
that specifies the amount of information that can be released about
the secret. The thesis proposes the property of limited information
release to capture this security condition and proves its enforcement
sound, information-theoretically. 

\item \textbf{Describing a comprehensive policy mechanism for easy
  specification of security policies:} To complement the work on
enforcement components in web browsers, the thesis also  explores a
policy specification mechanism to specify flexible and useful
information flow policies for web applications. 
\end{itemize}

