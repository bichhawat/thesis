\section{LIR Enforcement}
\label{sec:formalization}

\subsection{Language and Syntax}
This section describes a runtime enforcement of LIR for the simple
imperative language shown in Figure~\ref{basic:syntax} extended with
the \dec~operator as shown in Figure~\ref{fig:syntaxLIR}. The
comparison and arithmetic operators are separated as $\op$ and $\aop$,
respectively, for the rules. The language
is sufficient for describing the key idea of LIR --- appropriate
deduction of budgets at declassification of boolean expressions involving
secrets. 

\begin{figure}[!htbp]
\begin{align*}
  \expr	:=~&\TT{n}~\arrowvert~\TT{x}~\arrowvert~\expr_1 \aop \expr_2~\arrowvert~
             \expr_1 \op \expr_2~\arrowvert~\dec(\expr_1 \op \expr_2)\\
  \comm	:=~&\sk~\arrowvert~\TT{x} :=
             \expr~\arrowvert~\comm_1;\comm_2~\arrowvert~\texttt{if}~\expr~\texttt{then}~\comm_1~\texttt{else}~\comm_2~\arrowvert~ 
   \texttt{while}~\expr~\texttt{do}~\comm
\end{align*}
\caption{Syntax of the language}
\label{fig:syntaxLIR}
\end{figure}

Every input (coming in via store variables) to the program has a confidentiality
label (part of a security lattice and represented using 
$k, m$) associated with it referred to as the \emph{initial label}. An
immutable map $\ilabel$ represents the mapping from a variable to
its initial label. Along with the initial confidentiality label, every
input is also associated with a \emph{budget} and an immutable
\emph{budget-label}. The budget on an input represents an upper bound 
on the amount of information that is allowed to leak about that
input. The budget-label serves the purpose of indicating the
confidentiality level to which the value can be
declassified. Additionally, the budget-label helps in increasing the 
permissiveness of the approach as described later. An immutable map
$\blabel$ tracks the budget-label associated with different variables.
Since the budgets change as the program executes, they themselves can
be a channel of information leak. The budgets are represented as
$\TT{n}^l$, where $\TT{n}$ is the budget and $l$ is the budget-label, i.e., if a
variable has a budget of $1$ and the budget-label $L$, it is denoted
as $1^L$.
% To counter that, we
% associate the budgets with a confidentiality label of their own, i.e.,
% the budget-label. 
% As a result, the budget label should never be above
% the the initial label of the input.

Instead of directly tracking the label on the values, the variable
dependencies (also referred to as provenance) are tracked.
As every variable has its initial label, its budget and its budget
label associated with it, tracking dependencies enables access to 
this metadata for every dependency individually. This assists the
enforcement by choosing the right variable to deduct budget from, 
which is required at the point of declassification. These dependencies
are split into two parts and represented as a pair: $(l, \dep)$. 
$\dep$ is a set of variables on whose initial values the current
value depends. Information can be released only about the variables
that are part of the provenance set ($\dep$). The security level $l$
of the label is the join of security levels (upper-bound) of all the
provenances for which no more information is allowed to be
released. Once the budget of a variable expires (becomes $0$), the
security level of that provenance is joined with the security level of
the value and the variable is removed from the provenance set, i.e.,
for initial $\TT{x} = \TT{n}^{(\bot, \{\TT{x}, \TT{y}\})}$ such that
$\ilabel(\TT{x}) = l$ and $\ilabel(\TT{y}) = m$, once the budget of
$\TT{y}$ expires, $\TT{x} = \TT{n}^{(m, \{\TT{x}\})}$.  Thus, the
security level of a value ($l$) represents the minimal level at which
the value can be observed. The current value in the variable can never
be released to a level below $l$. Initially, every variable depends
only on itself. Every initial variable $\TT{x}$ is labeled $(\bot,
\{\TT{x}\})$, where $\bot$ represents the least element in the lattice. 

The function $\Gamma(\TT{n}^{(l, \dep)})$ returns the label of the
value, i.e., $\Gamma(\TT{n}^{(l, \dep)}) = l \bigsqcup\limits_{\TT{x}
  \in \dep} \ilabel(\TT{x}) \bigsqcup\limits_{\TT{x} \in \dep}
\blabel(\TT{x})$.  The function $\Gamma_f(\TT{n}^{(l, \dep)})$ returns
the security level $l$ of the label.  The function $\dep(\TT{x})$
returns all the provenances of $\TT{x}$. 
The join operation on the labels returns the join of the security levels and the
union of the provenance sets as the final label, i.e.,
$$(l, \dep) \sqcup (l', \dep') = (l \sqcup l', \dep \cup \dep')$$
Similarly, the ordering on the labels is defined as:
$$(l, \dep) \sqsubseteq (l', \dep') = (l \sqsubseteq l', \dep \subseteq \dep')$$
$\pc$ denotes the current program-context level, and % $\Gamma_f(pc)$
% and $\Gamma(\pc)$ function as before. 
contains \emph{only} the security level, represented as $(l, \{\})$,
i.e., none of the provenances in the $\pc$ are allowed to release any
information through the variables being assigned within the branch and
are hence joined with the security level $l$ of the $\pc$.  This
decision is justified below with an example. The function
$\Gamma(\pc)$ returns the security level $l$ of the 
$\pc$. When the meaning is clear from the context, $\pc$ is used
instead of $\Gamma(\pc)$.  

\subsection{Key Aspects of the Enforcement}
\label{sec:enf-design}

\subsubsection{Provenance Tracking}
As the monitor is \emph{flow-sensitive}, information can be released
through other variables that are dependent on the initial value of
some of the secret variables. Instead of directly tracking abstract
labels, which gives a bound on the secrecy but loses the actual input
variable dependency, variable dependencies are tracked. These
dependencies determine the  variables to deduct the budget
from. Provenance tracking is required for a correct analysis 
and to prevent an adversary from laundering secret information as
described below. Consider the example in Listing~\ref{egDep} with the 
security lattice $L \sqsubseteq H$. Assume that the initial values in
\texttt{sec} and \texttt{h} are both secret with the initial label $H$
and an initial budget of $1^L$; the initial values in \texttt{pub} and
\texttt{i} are both public with an initial label $L$.

\begin{lstlisting}[float,caption=Leak due to dependent variables, label=egDep]
 pub $=$ 0;
 h $=$ sec $\%$ 2;  @\label{a:t}@
 sec $=$ sec $/$ 2;
 if (declassify(h $==$ 1))  @\label{if:h}@
    pub $=$ pub $|$ 1; @\label{if:a}@
 i $=$ sec $\%$ 2; @\label{a:2}@
 if (declassify(i $==$ 1)) @\label{if:i}@
    pub $=$ pub $|$ 2; @\label{if:a1}@
\end{lstlisting}

Suppose that at the assignment on line~\ref{a:t} only the label of
\texttt{sec} is carried over to the label of \texttt{h} (like any flow-sensitive IFC
monitor would) but the actual variable dependencies are not tracked.  Without 
tracking the dependent variables, the declassification on line~\ref{if:h}
would release the value of \texttt{h} (the last bit of \texttt{sec}) to
\texttt{pub} deducting the budget of \texttt{h} but not \texttt{sec}. As a
result another bit of information about \texttt{sec} can be leaked later
(through declassify on line~\ref{a:2}), which should not have been
allowed as the budget of \texttt{sec} is just $1$.

With dependency tracking, initially \texttt{h} and \texttt{sec} will
be $\dep(\texttt{h}) = \texttt{h}$ and $\dep(\texttt{sec}) = \texttt{sec}$.
The monitor would update the dependence of \texttt{h} on
\texttt{sec} on line~\ref{a:t}. As a result, the declassification on
line~\ref{if:h} would deduct the budget from \texttt{sec} and not
\texttt{h}. On line~\ref{a:2}, \texttt{i} is updated with the next bit
of \texttt{sec} and so is the dependence of \texttt{i} on
\texttt{sec}, i.e., $\dep(\texttt{i}) = \texttt{sec}$. When executing
the branch on line~\ref{if:i} as \texttt{sec} has expired its budget,
the monitor joins the security level of \texttt{sec} in the new
program context making it secret $H$, hence limiting the amount of
information released about the initial value of \texttt{sec}. 
% At every assignment, the LIR monitor embedded in the semantics determines the
% variables on which the assigned value depends and adds these variables to the
% provenance set for the assigned variable.

\begin{lstlisting}[float,caption=Example to illustrate budget reduction,label=egbr]
 x $=$ (a $\%$ 2) @\label{brass}@
 if (a $==$ 0) @\label{br0}@
   x $=$ b @\label{brxb}@
 z $=$ declassify(x $==$ 1) @\label{brz}@
 y $=$ declassify(b $==$ 0) @\label{bry}@
\end{lstlisting}
\subsubsection{Budget-label Constraints}
When declassifying a value labeled $(l, \dep)$, budget is reduced
for only those variables in the provenance set $\dep$ that have a
budget-label at least as high as $l$, i.e., $\forall \TT{x} \in \dep. l
\sqsubseteq \blabel(\TT{x})$ as when the budget-label is below $l$ no 
useful information can be released. 
The reduction of budget and release of information is subject to the 
condition that declassification of a value in a secret context is not
allowed, as this could leak information about the context itself. In
the current setting, this would mean that if the label of the value
being declassified is $(l, \dep)$, then $\pc \sqsubseteq l$. It
turns out that this suffices to prevent leaks via change of budgets,
as for declassification it is required that the budget-label is at
least as secret as $l$, i.e., no value is declassified to a level
below $\pc$. Without these checks in place, the monitor could leak
additional information as shown below.

Consider, for illustration, the example in Listing~\ref{egbr}. Assume
that \texttt{a} and \texttt{b} are two secrets with initial labels
$(M, \{\})$ and $(L, \{\texttt{b}\})$, initial labels $M$ and $H$ such
that $L \sqsubseteq M \sqsubseteq H$, and initial budgets of $0$ and
$1^L$, respectively. On line~\ref{brass}, \texttt{x}'s label becomes
$(M, \{\})$ as the budget of \texttt{a} has already expired. Consider
two executions of the program depending on whether \texttt{a} is $0$
or not. When \texttt{a} is $0$, \texttt{x} is assigned \texttt{b} on
line~\ref{brxb} with a label $(M, \{\texttt{b}\})$. On line~\ref{brz},
\texttt{x} can be declassified to level $M \sqcup L$ (as
$\blabel(\texttt{b}) = L$). Assume that the value that is declassified
is \texttt{true}. Thus, the label of the declassified value in
\texttt{z} is $(M, \{\})$ and the budget of \texttt{b} becomes $0$. On
line~\ref{bry}, as \texttt{b} has expired its budget, \texttt{y} is
labeled $(H, \{\})$. The value released in this run is
\texttt{true}$^M$.  In the other run when \texttt{a} is not $0$,
\texttt{x} has a label of $(M, \{\})$ on line~\ref{brz}. Thus, no
declassification occurs on line~\ref{brz}. As the budget of \texttt{b}
on line~\ref{bry} is still $1^L$, the value on line~\ref{bry} is
declassified to $L$. Thus, the label of \texttt{y} after the
assignment becomes $(L, \{\})$ and the value released for this run is
\texttt{false}$^L$.  Thus, in the first run an $L$-level adversary
cannot observe any value while in the second run, the adversary can
see the value as it is labeled $L$. As the adversary knows the budget
of \texttt{b}, he/she can conclude that in the first case the branch on
line~\ref{br0} was taken and in the second case it was not taken. This
leaks an additional bit of information about \texttt{a} although the
budget of \texttt{a} doesn't allow any information release.

To prevent this leak, in the first case, the value of \texttt{x} on
line~\ref{brz} is not declassified as the budget-label of
\texttt{b} ($\blabel(\texttt{b}) = L$) is lower than the
security level of the value ($M$).

% Intuitively, the program context
% affecting a variable need to be captured in the variable's label to
% avoid leaking information about the context. As any variable's value
% at a given point in the program can only be affected by security
% contexts ($\pc$) that are lower than or equal to the variable's
% current security level (i.e., if the label of a variable is $(l,
% \dep)$, then the value in the variable could have only been modified
% in a $\pc \sqsubseteq (l, \dep)$, else the assignment would have
% failed the NSU check or be labeled partially-leaked), we impose this
% check at declassification points. In general, if a variable in the
% provenance set has expired its budget no further information should be
% released about the other 
% provenances as this could also leak information about the variable
% with expired budget as shown above. However, as the information can
% only be released to a level specified by the budget-label of the
% variable, we have the additional check.

% \medskip
\subsubsection{Handling implicit flows}
Implicit flows are handled as per the permissive-upgrade strategy ---
assignments to public variables in a secret context ($\pc$) are
labeled partially-leaked. However, for simplicity of exposition the
no-sensitive-upgrade check is used here and further in the chapter
instead of the permissive-upgrade strategy. 

Since variable dependencies are used in
addition to the security levels, those dependencies also need to be
checked for during assignment operations. For an assignment to
succeed, it is important that
no new dependencies are handed over to the variable as part of the
assignment operation. Thus, it is required that % either:
% \begin{itemize}
% \item $\pc \sqcup (m, \dep') \sqsubseteq (l, \dep)$ where $(l,
%   \dep)$ is the label of the variable \TT{x} being assigned to and $(m,
%   \dep')$ is the label of the expression being assigned to the variable
%   \TT{x}. The reason being that the assignments that happen in one of
%   the runs would carry the dependencies of the value being assigned
%   too while in an alternate run those dependencies might not be
%   present in the label of the assigned variable. 
% \item 
  $\pc \sqsubseteq k$, where $k$ is the security level of the
  variable being assigned, i.e., the security
  level of the variable being assigned is at least as high as the
  security level of the $\pc$. 
  % In this case, the final security
  % level of \TT{x} would be at least as high as $\Gamma(\pc)$, thus,
  % carrying the security level of the dependency variables rather than
  % carrying the dependencies.
  This prevents leaks as the assignments
  that happen under a $\pc$ would carry the dependencies of the $\pc$
  too while in an alternate run those dependencies might not be
  present in the label of the assigned variable. 
% \end{itemize}
The mismatch of dependencies in the two cases can leak information as
illustrated using an example in Listing~\ref{egimp}. 

%% To handle implicit flows, the assignments are subject to the standard
%% no-sensitive-upgrade (NSU) check described earlier in
%% Section~\ref{sec:bgdifc}, i.e., assignment to any variable $x$ is allowed
%% only if $\pc \sqsubseteq \Gamma(x)$. However, we do not allow $\pc$ to carry
%% provenances, i.e., $\pc = (l, \{\})$. All the provenances in the label of the
%% predicate are joined with the security level $l$ as no information is allowed
%% to be released about those provenances. This is done to prevent the
%% occurrence of unaccounted information leaks.


\begin{lstlisting}[float,caption=Example illustrating handling of
  implicit flows,label=egimp]
 if (x $\leq$ 10) @\label{impbr}@
   x $=$ y @\label{impassn}@
 z $=$ declassify(x $\leq$ 5) @\label{impdec}@
\end{lstlisting}

For the program in Listing~\ref{egimp}, assume that \texttt{x}, \texttt{y} and
\texttt{z} have the initial confidentiality labels of $H$, $H$ and $L$ ($L
\sqsubseteq H$), and budgets of $1^L$, $0$ and $0$ assigned to them,
respectively. Also assume that their current dependencies are of the form $(L,
\{\texttt{x}\})$, $(H, \{\})$ and $(L, \{\})$. Consider two runs of the program
with $\texttt{x} \leq 10$ in one and $\texttt{x} > 10$ in the
other. Suppose that check for provenances are kept in the $\pc$, i.e.,
the assignment succeeds if $\pc = (l, \dep) \sqsubseteq (k_o, \dep_o)$
where $(k_o, \dep_o)$ is the label of the variable \texttt{x}.  When
$\texttt{x} \leq 10$, the branch  on line~\ref{impbr} is taken and the
$\pc$ on line~\ref{impassn} would be $(L, 
\{\texttt{x}\})$, which would allow the assignment (as dependency of
\TT{x} is not less sensitive than the dependencies in $\pc$). As a
result the label of \texttt{x} would be updated to $(H, \{x\})$ (join
of $\pc$ and label of \texttt{y}). On line~\ref{impdec}, when the
value of $\texttt{x} \leq 5$ is declassified, no information is
released as the budget-label of \texttt{x} is lower than its security
level ($H$) (as explained above). In the other run when $\texttt{x}
> 10$, the branch is not taken and the value of $\texttt{x} \leq 5$ on
line~\ref{impdec} is released to the level $L$ as the label of
\texttt{x} on line~\ref{impdec} is $(L, \{\texttt{x}\})$ and 
the budget is $1^L$. This leads to different sensitivity of \texttt{z} in the
two runs which leads to leaking the value of $x \leq 10$ in the first
run without being accounted for in its budget.

With the modification suggested above in place, in the first run ---
the value of $\pc = H$ (because $\ilabel(\texttt{x}) = H$) and
$\Gamma_f(\TT{x}) = L$ and $H \not\sqsubseteq L$.  
Thus, in the first run of the above example, the assignment is not
allowed as per the no-sensitive-upgrade check. In case of the
permissive-upgrade strategy, the label would be computed as
$((\Gamma(\pc \sqcup m)) \sqcap l)\pl$. A partially-leaked value
does not carry any provenances as no declassification is allowed on a
partially-leaked value.

%% Intuitively, this happens because we allow declassification of the
%% provenances in the $\pc$ that carried over to the variables through the
%% assignments within the branch, although these provenances were used as
%% ``secret'' within the branch.

% \medskip

\subsubsection{Permissiveness}
The overall confidentiality of the underlying value with a
label $(l, \dep)$ is the join of $l$ with the initial label of all the
variables in the dependency set. Thus, all those dependencies whose
initial label is below the first part of the label can be removed as
keeping those dependencies does not affect the confidentiality of the
value.  This improves the permissiveness of the technique by not
marking the release of information for variables that have a label
lesser than or equal to the current security level or the $\pc$
because no information can be released below the current security
level of the variable or the current $\pc$. For instance, consider a
value $n^{(k, \{\TT{x}\})}$ such that its label's first part is $k$ and the
dependency set consists of just one variable \TT{x}. If
$\ilabel(\TT{x}) \sqsubseteq k$ then \TT{x} is removed from the
dependency set and represent the value as $n^{(k, \{\})}$. As
explained earlier, budgets are deducted only when $\pc \sqsubseteq k$, 
thus, the budgets of any variable whose initial label
is below the $\pc$ are not deducted from. 
% This improves the permissiveness of the enforcement.

%% When evaluating a comparison expression that releases some information, budgets
%% are deducted for only those provenances that have labels greater than the
%% current context ($l \not\sqsubseteq \Gamma_f(\pc)$), where $\pc$ returns the
%% current program-context label). This improves the permissiveness of the
%% technique by not marking the release of information for variables that have a
%% label lesser than or equal to the current $\pc$-label because no information can
%% be released below the current $\pc$-label in the program.

\begin{lstlisting}[float,caption=Example to illustrate permissiveness,label=egp]
 if (med $\leq$ 0) 
    x $=$ declassify(sec $==$ y); @\label{ifb}@
 if (declassify(y $\neq$ 100))   @\label{if1}@
    pub $=$ 1;  @\label{ifb1}@
\end{lstlisting}

For illustration, consider the program in Listing~\ref{egp}. Assume that,
\texttt{med}, \texttt{x}, \texttt{sec}, \texttt{y}, and \texttt{pub}
are labeled $(M, \{\})$, $(M, \{\})$, $(M, \{\texttt{sec}\}),
(L, \{\texttt{y}\})$, and $(L, \{\texttt{pub}\})$,
respectively, $\ilabel(\texttt{sec}) = H, \ilabel(\texttt{y}) = M,
\ilabel(\texttt{pub}) = L$ such that $L \sqsubset M \sqsubset H$ and the budgets
of \texttt{sec} and \texttt{y} are $1^M$ and $1^L$, respectively. The program
context label ($\pc$) on line~\ref{ifb} is $(M, \{\})$. The expression
$\TT{sec} == \TT{y}$ on line~\ref{ifb} has the label $(M, \{\TT{sec},
\TT{y} \})$. Without the check for the current security level, the
expression evaluation on line~\ref{ifb} releases the value of 
\texttt{sec} and \texttt{y} into \texttt{x}. However as the security
level is $M$, the final value of \texttt{x} would have the security
level of at least $M$. Thus, information about \texttt{y} is released
to $M$-level observers, who already had access to the value of
\texttt{y}. Subsequently, on line~\ref{if1} as the budget of
\texttt{y} has expired, no more information about \texttt{y} is
released and the assignment on line~\ref{ifb1} fails. Thus, no useful
information about \texttt{y} is released at any point in the program,
yet the program is terminated because of the NSU check. With the check
for current security level and the context label, the budget of
\texttt{y} is not deducted on line~\ref{ifb}, which allows the check
on line~\ref{if1} to release information about \texttt{y} 
thereby allowing the assignment on line~\ref{ifb1} to succeed.


\subsection{Semantics}
\label{sec:semantics}
To formally define the information released by a program starting from some
initial memory containing secret values, big-step semantics is defined
for the language shown in Figure~\ref{fig:syntaxLIR} that releases
some information about initial secret values at comparison operations.
% For simplicity of
% exposition, we explain the enforcement for a simple WHILE
% language. The key idea of the policy, however, can be generalized to
% any language.
% We define a big-step semantics for the language in
% Figure~\ref{fig:syntaxLIR}.
% and instrument them with the kind
% of budget and label checks described above.

Program configurations are extended with $\iota$ representing the
budget store, which is a map from the variables to the budget of these
initial  values (inputs to the program). The configurations for
expressions ($\expr$) and commands ($\comm$) are, thus, represented
by $\langle \sigma, \iota, \expr \rangle$ and $\langle \sigma, \iota, 
\comm \rangle$, respectively where $\sigma$ represents the memory
store as before. 
$\langle\sigma, \iota, \expr \rangle \lirsem \TT{n}^{(k, \dep)}, \iota',
\trace$ defines expression evaluation. It means that under some
$\pc$, starting with memory $\sigma$ and budget store $\iota$, the
expression $\expr$ evaluates to a value $\TT{n}$ labeled
$(k, \dep)$ resulting in a budget store $\iota'$. Additionally, the
evaluation generates a trace $\trace$, which is a list of values of the
form $\TT{n}_1^{k_1} :: \TT{n}_2^{k_2} :: \ldots ::
\TT{n}_j^{k_j}$. Every declassified value is recorded on the trace along
with the level to which it is declassified. The two immutable maps ---
initial label map ($\ilabel$) and budget-label map ($\blabel$) are
assumed to be available and omitted from the rules for clarity.

\begin{figure}[!htbp]
\begin{framed}
\begin{mathparpagebreakable}
%%%%%%%%%%%%%%%%%%CONST 
\inferrule*[left=\mbox{\labelthis{lir:exp:c}{const}}]
{ }
{\langle\sigma,  \iota,  \TT{n} \rangle \lirsem
  \TT{n}^{(\bot, \{\})}, \iota,  \emptyTrace}
\and 
%%%%%%%%%%%%%%%%%%VAR
\inferrule*[left=\mbox{\labelthis{lir:exp:v}{var}}]
{\sigma(\TT{x}) = \TT{n}^{(k_o, \dep_o)} \\ 
  \dep' = \Reduce(\dep_o, \iota, k_o, \blabel) \\
  \dep = \dep_o \setminus \dep' \\
  % \dep = \dep' \\ 
  k = k_o \bigsqcup\limits_{\TT{x} \in \dep'} \ilabel(\TT{x}) 
}
{\langle\sigma, \iota,  \TT{x} \rangle \lirsem \TT{n}^{(k, \dep)},
  \iota, \emptyTrace}
\and 
%%%%%%%%%%%%%%%%%%AOP
\inferrule*[left=\mbox{\labelthis{lir:exp:aop}{aop}}]
{\langle \sigma, \iota,  \expr_1 \rangle \lirsem
  \TT{n}_1^{(k_1, \dep_1)}, \iota_1, \trace_1 \\\\ \langle \sigma,
  \iota_1,   \expr_2 \rangle \lirsem \TT{n}_2^{(k_2, \dep_2)},
  \iota', \trace_2 \\\\ \TT{n} = \TT{n}_1 \aop \TT{n}_2 \\ k = k_1
  \sqcup k_2   \\\\ \dep = \dep_1 \cup \dep_2 \\ \trace = \trace_1 :: 
  \trace_2}
{ \langle\sigma,  \iota,   (\expr_1~\aop~ \expr_2) \rangle
  \lirsem \TT{n}^{(k, \dep)}, \iota', \trace}
\and
%%%%%%%%%%%%%%%%%%OPER-COP-W-D
\inferrule*[left=\mbox{\labelthis{lir:exp:cop}{cop}}]
{\langle \sigma, \iota,  \expr_1 \rangle \lirsem
  \TT{n}_1^{(k_1, \dep_1)}, \iota_1, \trace_1 \\\\ \langle \sigma,
  \iota_1,   \expr_2 \rangle \lirsem \TT{n}_2^{(k_2, \dep_2)},
  \iota', \trace_2 \\\\ \TT{n} = \TT{n}_1 \cop \TT{n}_2 \\ k = k_1
  \sqcup k_2 \\\\ \dep = \dep_1 \cup \dep_2 \\ \trace = \trace_1 ::
  \trace_2
  % \\  \forall x \in \dep.(l = l \sqcup \ilabel(x))
}
{ \langle\sigma,  \iota,   (\expr_1~\cop~ \expr_2) \rangle
  \lirsem \TT{n}^{(k, \dep)}, \iota', \trace}
\and
%%%%%%%%%%%%%%%%%%OPER-COP
\inferrule*[left=\mbox{\labelthis{lir:exp:dr}{dcopr}}]
{\langle \sigma,  \iota,  \expr_1 \rangle \lirsem
  \TT{n}_1^{(k_1, \dep_1)} , \iota_1, \trace_1 \\
 \langle \sigma, \iota_1, \expr_2  \rangle \lirsem
 \TT{n}_2^{(k_2, \dep_2)},  \iota_2, \trace_2  \\\\
\TT{n} = \TT{n}_1 \op \TT{n}_2 \\ \dep_o = \dep_1 \cup \dep_2 \\
k_o = k_1 \sqcup k_2 \\\\
\pc \sqsubseteq k_o \\
% \pc \sqsubseteq (k_o, \dep_o)~ \bigvee~ \Gamma(\pc) \sqsubseteq k_o \\
 \dep' = \Delta(\dep_o, k_o, \ilabel) \\
\dep'' = \Reduce(\dep',\iota_2, k_o, \blabel) \\\\
  \dep = \dep' \setminus \dep'' \\ 
k = k_o \bigsqcup\limits_{\TT{x} \in \dep''}
  \ilabel(\TT{x}) \bigsqcup\limits_{\TT{y} \in \dep} \blabel(\TT{y}) \\ 
\iota' = \mathbb{I}(\iota_2, \dep)\\
\trace = 
\left\{\begin{array}{ll}
        \trace_1 :: \trace_2 :: \TT{n}^{k}, & \textit{if } (\dep \neq \emptyset) \arcr
        \trace_1 :: \trace_2, & \textit{otherwise } \arcr
        \end{array}\right\} 
}
{\langle\sigma,  \iota,  \dec(\expr_1 \op \expr_2)
  \rangle \lirsem \TT{n}^{(k, \{\})}, \iota', \trace}
\and
%%%%%%%%%%%%%%%%%%OPER-COP
\inferrule*[left=\mbox{\labelthis{lir:exp:dn}{dcopn}}]
{\langle \sigma,  \iota,  \expr_1 \rangle \lirsem
  \TT{n}_1^{(k_1, \dep_1)} , \iota_1, \trace_1 \\
 \langle \sigma, \iota_1, \expr_2  \rangle \lirsem
 \TT{n}_2^{(k_2, \dep_2)},  \iota_2, \trace_2  \\\\
\TT{n} = \TT{n}_1 \op \TT{n}_2 \\ \dep =\dep_1 \cup \dep_2 \\
k = k_1 \sqcup k_2 \\ 
\pc \not\sqsubseteq k \\
% \pc \not\sqsubseteq (k, \dep) \bigwedge
% \Gamma(\pc) \not\sqsubseteq k \\ 
% \forall x \in \dep. (l = l \sqcup \ilabel(x)) \qquad
\trace = \trace_1 :: \trace_2
}
{\langle\sigma,  \iota,  \dec(\expr_1 \op \expr_2)
  \rangle \lirsem \TT{n}^{(k, \dep)},
  \iota_2, \trace}
\end{mathparpagebreakable}
%%%%%%%%%%%%%%%%%%AF used above
\begin{flushleft}
\textbf{Auxiliary functions:}
\begin{align*}
\Delta(\dep, k, \ilabel) =&~ \big\{\TT{x}~\big|~(\TT{x} \in \dep) \wedge
                            (\ilabel(\TT{x}) \not\sqsubseteq k) \big\} \\
\Reduce(\dep, \iota, k, \blabel) =&~ \left\{\TT{x}~\middle|~(\TT{x} \in \dep) \wedge \big(\iota(\TT{x})
= 0 \, \vee \, k \not\sqsubseteq \blabel(\TT{x})\big)\right\} \\
\mathbb{I}(\iota, \dep) =&~ \left\{(\TT{x},
                           \TT{n})~\middle|~\big((\TT{x}, \TT{j}) \in
                           \iota\big) \bigwedge \left(\TT{n}
=
\left\{\begin{array}{ll}
        \TT{j} - 1, & \textit{if} ~ (\TT{x} \in \dep) \\
        \TT{j}, & \textit{otherwise}
        \end{array}\right\} 
% \begin{cases}
% \iota(x) - 1 & \mathit{if} ~ x \in \dep \\
% \iota(x) & \mathit{else}
%  % (~?~ \iota(x) - 1 ~:~ \iota(x))\big)\big\} 
% \end{cases}
\right)\right\}
\end{align*}
\end{flushleft}
\end{framed}
\caption{LIR - Semantics of expressions}
\label{fig:lir:sem-e}
\end{figure}

The evaluation rules for expressions shown in Figure~\ref{fig:lir:sem-e}
are explained below:
\begin{itemize}
\item \refrule{lir:exp:c}: Constant values evaluate to themselves with
  the label $(\bot, \{\})$ as they do not have any dependencies. There
  is no change to the budget store and it generates an empty trace ($\emptyTrace$).
\item \refrule{lir:exp:v}: 
  The evaluation of a variable returns its value $\TT{n}$ along
  with the label $(k_o, \dep_o)$. The function $\Reduce$ returns those
  variables in $\dep_o$ which have a budget $0$. The current security
  level $k_o$ is joined with the initial labels of those variables
  while removing them from the provenance set $\dep_o$.  As no new
  information is released, $\iota$ remains unchanged and no trace is
  generated. 
\item \refrule{lir:exp:aop} and \refrule{lir:exp:cop}: For arithmetic
  operations and normal comparison operations, the label of the
  expression is the join of the label of the values that the two
  sub-expressions evaluate to. The individual sub-expressions can
  release some information $\trace_1$ and $\trace_2$, thus updating
  $\iota$ to $\iota'$.
\item \refrule{lir:exp:dr}: This rule corresponds to the comparison
  operation with \dec, and allows information release in certain
  cases. The separation of \TT{declassify} from the normal comparison operations
  is to offer more control over what information is required to be
  released. The rule applies only when the current context $\pc$ is
  lower than or equal to the label of the value obtained by evaluation
  ($\pc \sqsubseteq k_o$). This is to ensure that the declassification
  does not happen 
  in a high context. The function $\Delta(\dep, l, \ilabel)$ returns
  the variables in the provenance set $\dep$ that have an initial
  label greater than $l$. Additionally, if the budget of the variables
  in the provenance set has expired during the evaluation of the
  sub-expressions, those variables are removed from the provenance set
  (using the function $\Reduce$). To prevent the leak described earlier,
  if the budget label of any of the remaining variables in the
  provenance set is not at least as high as the original security 
  level of the computed value ($k_o \not\sqsubseteq \blabel(\TT{x})$),
  then the variable is removed from the dependency set. For all the
  variables removed from the provenance set, their security level is
  joined with that of the actual label. 
  Depending on whether the provenance set $\dep$ is empty or not,
  either:  
  \begin{itemize}
  \item the budget of all remaining variables in the provenance set
    $\dep$ is deducted by $1$, corresponding to the $1$-bit Boolean
    value released on the trace. The function $\mathbb{I}(\iota,
    \dep)$ reduces the budget and returns a new budget store. Their
    budget label is also joined  with the actual security level, which
    gives the final label of the declassified value \emph{or}
  \item if the remaining provenance set is empty, no declassification
    occurs
  \end{itemize}
  If declassification occurs, the declassified value is appended to
  the trace.
\item \refrule{lir:exp:dn}: If the current context ($\pc$) is not
  lower than or equal to the security level of the value obtained by evaluation
  ($\pc \not\sqsubseteq k$), no information release is allowed. The
  final label is a join of the security 
  levels of all the provenances. The individual sub-expressions can
  release some information $\trace_1$ and $\trace_2$, thus updating
  $\iota$ to $\iota_2$.
\end{itemize}
 


\begin{figure}[!htbp]
\begin{framed}
\begin{mathparpagebreakable}
%%%%%%%%%%%%%%%%%%SKIP
\inferrule*[left=\mbox{\labelthis{lir:cmd:sk}{skip}}]
{ }
{\langle   \sigma, \iota,   \sk \rangle \lirsem
  \langle \sigma, \iota,  \emptyTrace \rangle}
\and
%%%%%%%%%%%%%%%%%%ASSN
\inferrule*[left=\mbox{\labelthis{lir:cmd:an}{assn}}]
{ % \sigma(\TT{x}) = \_^{(k_o, \dep_o)} \\ \pc = (k_p, \dep_p) \\
  \pc \sqsubseteq \Gamma_f(\TT{x}) \\
  \langle  \sigma, \iota,  \expr \rangle \lirsem \TT{n}^{(m, \dep')} ,
  \iota', \trace \\ 
  % \pc \sqcup (m, \dep') \sqsubseteq (k_o, \dep_o) \bigvee \Gamma(\pc)
  % \sqsubseteq k_o  \\
  k = \pc \sqcup m \\ \dep = \Delta(\dep', k, \ilabel)
}
{\langle   \sigma, \iota,  \TT{x} := \expr \rangle
  \lirsem \langle \sigma[\TT{x} \mapsto
  \TT{n}^{(k, \dep)}], \iota', \trace \rangle}
\and
%%%%%%%%%%%%%%%%%%SEQ
\inferrule*[left=\mbox{\labelthis{lir:cmd:s}{seq}}]
{\langle   \sigma,  \iota,  \comm_1 \rangle \lirsem
  \langle  \sigma', \iota', \trace_1 \rangle  \\ 
  \langle   \sigma',
  \iota', \comm_2 \rangle \lirsem \langle \sigma'',
  \iota'', \trace_2 \rangle
}
{\langle   \sigma, \iota,  \comm_1;\comm_2 \rangle
  \lirsem \langle \sigma'', \iota'', \trace_1
  :: \trace_2 \rangle}
\and
%%%%%%%%%%%%%%%%%%IF
\inferrule*[left=\mbox{\labelthis{lir:cmd:ie}{if-else}}]
{\langle   \sigma, \iota,   \expr \rangle \lirsem
  \TT{n}^{(k_o, \dep)}, \iota', \trace_1 \\ 
i = \left\{\begin{array}{ll}
        1, & \textit{if } (\TT{n} = \texttt{true}) \arcr
        2, & \textit{otherwise } \arcr
        \end{array}\right\} \\
k = k_o \bigsqcup\limits_{\TT{x} \in \dep} \ilabel(\TT{x})
\\ \langle   \sigma,  \iota', \comm_i
  \rangle \lirsemup \langle \sigma', \iota'',
  \trace_2 \rangle \\
}
{\langle   \sigma,  \iota,
  (\texttt{if}~\expr~\texttt{then}~\comm_1~\texttt{else}~\comm_2)
  \rangle \lirsem  \langle \sigma', \iota'',
  \trace_1 :: \trace_2 \rangle}
\and
%%%%%%%%%%%%%%%%%%W_FALSE
\inferrule*[left=\mbox{\labelthis{lir:cmd:wf}{while-f}}]
{\langle   \sigma, \iota,  \expr \rangle \lirsem
  \texttt{false}^{(k, \dep)}, \iota', \trace 
}
{\langle   \sigma,\iota,
  \texttt{while}~\expr~\texttt{do}~\comm
  \rangle \lirsem  \langle \sigma, \iota', \trace \rangle}
\and
%%%%%%%%%%%%%%%%%%W_TRUE
\inferrule*[left=\mbox{\labelthis{lir:cmd:wt}{while-t}}]
{\langle   \sigma, \iota,  \expr \rangle \lirsem
  \texttt{true}^{(k_o, \dep)}, \iota_1, \trace_1 \\ 
k = k_o \bigsqcup\limits_{\TT{x} \in \dep} \ilabel(\TT{x})
\\ \langle   \sigma,  \iota_1, \comm;
\texttt{while}~\expr~\texttt{do}~\comm
  \rangle \lirsemup \langle \sigma', \iota',
  \trace_2  \rangle \\
}
{\langle   \sigma,  \iota,
  \texttt{while}~\expr~\texttt{do}~\comm
  \rangle \lirsem  \langle \sigma', \iota',
  \trace_1 :: \trace_2 \rangle}
\end{mathparpagebreakable}
\end{framed}
\caption{LIR - Semantics of commands}
\label{fig:lir:sem-c}
\end{figure}


The judgment for a command execution is given by $\langle \sigma,
\iota, \comm \rangle \lirsem \langle \sigma', \iota', \trace \rangle$,
which means that under a program context $\pc$, the execution of a
command $\comm$ starting with memory $\sigma$ and budget store $\iota$
results in a final memory $\sigma'$ and budget store $\iota'$ while
generating a trace $\trace$ of released values from expression
declassification. The semantics of commands is shown in
Figure~\ref{fig:lir:sem-c}. The rules are standard except for some changes
in the assignment rule and branching rules. The assignment rule
\refrule{lir:cmd:an} does the standard NSU check that disallows
assignment to a public variable in a secret context --- $\Gamma(\pc)
\sqsubseteq \Gamma_f(\TT{x})$. The branching rules
\refrule{lir:cmd:ie} and \refrule{lir:cmd:wt} compute the context
label by joining the initial labels ($\ilabel$) of all the provenances
in $\dep$.
