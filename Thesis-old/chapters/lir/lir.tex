\chapter{Bounding Information Leaks Dynamically}
\label{ch:lir}

Information flow control allows tracking of sensitive and private
information while preventing leaks to unauthorized public 
channels. However, in some cases it is required that (some part of) 
the sensitive data be accessible over certain public channels. This
violates the security property of non-interference but is needed for  
practical reasons. This is generally achieved by specifying special
declassification policies, which allow for such release of
information. For instance, the declassification policy 
\lstinline{declassify(pwd == input)} would treat the result of the
equality check (\lstinline{pwd == input})  as public, thus allowing
information to be released to the user. The enforcement generally
requires \TT{declassify} annotations to be added to the code to
specify which operations need to release the value. Most of the
approaches require policy specifications by the developer that define
\emph{what} information can be released by the program when the data
to be declassified is part of the untrusted code; else the untrusted
code can arbitrarily add \TT{declassify} annotations and release
information at will. 

% This ensures that
% the untrusted code does not get access to the data not intended for
% it, i.e., in the example above, the code can specify
% \lstinline{declassify(pwd)} and leak the complete password if there is
% no specification of what (part of) data in \TT{pwd} can be
% declassified.  

% An alternative approach to declassification is to quantify the amount
% of information about a sensitive value that can be released by
% a program and to determine an acceptable upper bound, which accepts 
% or rejects the program. The information released is generally
% quantified by computing the difference in the entropy, a measure of
% uncertainty, of information contained in the sensitive value before
% and after the release with respect to an adversary. Most analysis
% methods that have been proposed until now are static in nature and
% quite a few of these determine an average worst-case bound on the
% amount of information that a program can release as a whole. However,
% all of these measures rely on the fact that the probability
% distribution of the output values is known upfront. 

% The motivation of this work is to provide an analysis that bounds
% information leaks dynamically at runtime and to prove that the
% approach is sound. Previously proposed quantitative information flow
% analysis methods in general yield leakage bounds over all possible
% executions of a program. A dynamic analysis, on the other hand,
% considers only the path being taken in the current execution, which in 
% general consensus seems to be unsound for quantifying information
% leaks. However, such an analysis would prove useful in scenarios like
% the Web where the program to be analyzed might only be available at
% runtime and one needs to only bound the amount of information released
% by the program about a sensitive value to untrusted
% sources~\cite{bielovaPLAS}. Additionally, it seems intuitive to reason about the path
% taken by the current execution of the program alone. Such leaks can be
% accounted for and bounded in the dynamic analysis presented in this
% chapter. 

An alternative approach to declassification is to \emph{quantify} the
amount of information a program releases about a sensitive value and
to determine an acceptable upper bound, which accepts or rejects the
program. The information released is generally quantified by computing
the difference in the entropy, a measure of uncertainty, of
information contained in the sensitive value before and after the
release with respect to an adversary~\referp{shannon, guessing,
  smith2009, clarkson2009, csf12GLeakage}. Research in quantitative
information flow has focussed on statically analyzing the program
and determining a worst-case bound on the amount of information that a
program can release as a whole~\referp{denning82, clark, clarkson2009,
  smith2009, backes, kopf}. These approaches rely on the assumption
that the probability distribution of the output values is known 
upfront. 

Quantitative information flow analysis methods, in general, yield
leakage bounds over all possible executions of a program. A dynamic
analysis, on the other hand, considers only the path being taken in
the current execution, which in general consensus seems to be unsound
for quantifying information leaks. However, such an analysis would prove
useful in scenarios like the Web where the program to be analyzed
might only be available at runtime and one needs to only bound the
amount of information released by the program about a sensitive value
to untrusted sources~\referp{bielovaPLAS}. 

% McCamant and Ernst~\cite{mccamant} provide a dynamic
% analysis for quantifying information flows but do not prove the
% soundness of their approach. They, however, provide a simulation-based 
% proof for quantifying the amount of information released by the
% program about a secret for a simple system with a two-level
% lattice. However, it is unclear how their technique scales for
% multiple security levels or multiple secrets in the system. Moreover, 
% the approach is more suitable for quantification than bounding
% information leaks; bounding information leaks introduces additional
% issues as detailed in later sections. Other prior work~\cite{csf13} used
% self-information to quantify the number of bits of information that
% could be leaked about a secret value. However, this technique requires
% the knowledge of the probability distribution or individual probabilities
% of the outputs that are based on secret values. In cases where it is
% difficult to determine the probability of the outputs upfront, the use
% of this measure for analysis does not help much. 

In the context of dynamic analysis, bounding information leaks helps
control the information leakage at runtime. The motivation of this
work is, thus, to provide an analysis that bounds information leaks
dynamically at runtime and to prove that the approach is sound. 
McCamant and Ernst~\refert{mccamant} provide the only dynamic 
quantitative information flow analysis but do not prove the
soundness of their approach. They, however, provide a simulation-based 
proof~\referp{plas07} for quantifying the amount of information
released by the program about a secret for a simple system with a
two-level lattice. They track information flow more finely at the
granularity of bits by restricting the language's variables to single
bits. They maintain a shadow bit for every variable that
indicates the secrecy level (public or secret) of the value in the
variable. However, it is unclear how their technique scales for
multiple security levels or multiple secrets in the system. 

Moreover, the approach by McCamant and Ernst~\refert{mccamant} 
is more suitable for quantification rather than for \emph{bounding} 
information leaks; bounding information leaks introduces additional 
non-trivial complications and leakage channels as detailed in later
sections. Other prior work by Besson \emph{et al.}~\referp{csf13} 
used self-information to
quantify the number of bits of information that could be leaked about
a secret value. However, this technique requires the knowledge of the
probability distribution or individual probabilities of the outputs
that are based on secret values. In cases where it is difficult to
determine the probability of the outputs upfront, the use of this
measure for analysis does not help much.   

% We build our proof strategy on similar lines. 
% The idea is that for every output that the program produces,
% which depends on a secret value, one can assign a unique code, which
% is generally represented as a bit-stream. Given that we have 
% uniquely decodable codes for different outputs, the average length of
% the code (average number of bits per output) is always greater than
% or equal to the Shannon entropy of the program outputs~\cite{shannon},
% which gives us a clear upper bound on the number of bits of
% information released through different runs of the program.

% Our analysis is flow-sensitive, which concentrates our interest to
% information leaks due to implicit flows. In our approach, every secret
% value has a release budget associated with it, which specifies the
% number of bits about the value that can be released and to what level
% the release can happen. As part of the declassification semantics, we
% generate a trace of all the released values when information release
% occurs at the comparison operations. The trace in this case  
% corresponds to the decodable code. To prove that our approach is
% sound, we show that every trace generated by the program on different
% executions corresponds to a unique set of information about high
% values. Thus, every different trace the program generates is a
% uniquely-decodable code, the average of which would be greater than
% the Shannon entropy of the secrets in the program. We show the
% uniqueness of the code by simulating the program execution for every  
% attacker-level with the attacker's view of the memory and the
% generated trace. The secret parts of the memory are unknown to the
% attacker and marked accordingly. At declassification points, we remove
% the first value from the trace and use it instead of the ``unknown''
% secret value. We then prove that the attacker's view of the final memory
% obtained by running the simulation is observationally equivalent to
% the attacker's view of the final memory obtained by executing the
% program in the declassification semantics. The additional challenge
% lies in the fact that we have multiple levels in the lattice and the
% declassification can happen to any of these levels, which needs to be
% accounted for properly in the generated trace as well. We also show
% that as the information released about a secret value at any program
% point is a 1-bit value, we account for it by deducting one from the
% budget of that secret value at that point, if it is allowed. 

% An orthogonal motivation for this approach is that quite a few flows
% do not leak a lot of information. In practice most implicit flows leak
% a single bit of information only, which might be required for
% providing proper functionality. King \emph{et
%   al.}~\cite{king08implicit} investigate the occurrence of implicit
% flows in some standard algorithms and discuss both the pros and cons
% of handling implicit flows. Russo \emph{et al.}~\cite{implicit} also
% observe that implicit flows cannot be exploited in non-malicious code
% to leak secrets efficiently. Unfortunately, allowing information release
% when only an insignificant amount of information is leaked can be
% widened to leak the complete secret~\cite{relSec,delRelease}. 

This chapter explores an alternative approach, where the developer can
specify a \emph{budget} for every sensitive value in the system
(defaulting to zero if not specified), which is basically an upper
bound on the amount of information allowed to leak about that
value. An underlying enforcement mechanism ensures that no more
information about that value is leaked than what is specified by the
budget.  

\begin{lstlisting}[float,belowskip=-0.5em,caption=Age-based Advertisement, label=egLeak]
 age $=$ getCurrentAge(birthday, birthyear);
 if (age $<$ 18)
    preference $=$ "child";
 else
    preference $=$ "adult";
\end{lstlisting}

Additionally, in scenarios
like the Web where the untrusted third-party code changes quite often,
it is difficult for the developer to keep up with the changes and
modify the \emph{what} policy specifications accordingly. For
instance, consider the program snippet in Listing~\ref{egLeak} where
the displayed advertisements for some webpage depend on the viewer's
age. The exact age is considered sensitive data, but in order to
provide its functionality only  an abstraction of the age is required,
namely whether \lstinline{age < 18}. The developer might specify
\lstinline{declassify(age < 18)} as a declassification policy.
However, different jurisdictions might entail adaptations to this rule  
(requiring a different check on \lstinline{age}), resulting in a
cumbersome process where the developer needs to update the policy
frequently. If, however, the developer associates a budget of $3$ with
the initial value in the variable \lstinline{age} (meaning that only
$3$ bits of information about \lstinline{age} are allowed to leak),
this example will be accepted in the envisioned model. If the
advertiser changes the criteria to include another check whether the
user is a teenager, this comparison can be included without requiring 
any change to the policy. The soundness of the underlying enforcement 
would ensure that no more information than the specified budget can
ever leak.  

Bounding information leaks has additional advantages --- it provides 
a formal meaning for allowing certain operations like the one 
described in Listing~\ref{egLeak} while establishing important
security properties related to them. There are a number of 
real-world programs that allow a limited number of such 
operations to be performed, e.g., a password checker allows $3$ tries
before it locks the user's account. 

Bounding information leaks in a purely dynamic setting is still largely 
an open problem. As the property of non-interference does not allow 
any information release, this chapter proposes a new security 
property, \emph{limited information release}, an end-to-end guarantee 
that allows the flow of secret information to public channels in a 
controlled manner by declassifying fragments of the secret. The 
policy enforces that the only information leaked about the secrets 
by a program is bounded by their pre-specified budgets. 
A sound enforcement of the property is descirbed and proven sound 
information-theoretically by coding the outputs generated through
information releases and showing that the output codes are
uniquely-decodable. We utilize an important result by
Shannon~\refert{shannon}, which states  that the average length of
uniquely-decodable codes upper bounds the Shannon entropy of the
outputs. 

% The challenge, however, is to build an enforcement mechanism that
% quantifies such information leaks soundly at runtime. As  a solution,
% this chapter proposes \emph{limited information release}, an
% information release policy that allows the flow of secret information
% to public channels in a controlled manner by declassifying fragments
% of the secret. The 
% policy enforces that the information leaked about the secrets by a
% program is bounded by the specified budgets. The soundness of
% the approach is proven by coding the outputs generated through
% information leaks and by showing that the codes are
% uniquely-decodable. The proof utilizes an important result
% by~\cite{shannon} that the average length of uniquely-decodable codes
% upper bounds the Shannon entropy of the outputs. 

% \medskip

\section{Quantifying Information Leaks}
\label{sec:bgqif}

The information released, or the \emph{mutual information}, is
generally quantified as the difference in the entropy, which is a
measure of the adversary's uncertainty, of the sensitive information 
before and after the information is released, i.e.,  
$$\text{mutual information = initial uncertainty - final
  uncertainty}$$ 
Roughly speaking, this amounts to the gain in knowledge of the
adversary about a sensitive value. The adversary computes a probable
initial set of values for the sensitive value. By observing the
information released, the adversary can refine his/her knowledge set
by removing the improbable values.
Various information-theoretic measures have been proposed~\cite{shannon, guessing, 
  smith2009, clarkson2009, csf12GLeakage} for quantifying information
leaks. Many of these measures determine an average worst-case bound on 
the amount of information that a program can leak. 

Clark et al.~\cite{clark} propose an approach to statically 
quantify the amount of information leaked in an imperative language.
An important result that they prove in their work is that the information 
released by a program in a deterministic system is equal to the Shannon 
entropy of the outputs given the public inputs. Thus, for computing the 
information released by a program about a secret, it is enough to compute
the Shannon entropy of the public outputs given the public inputs. 
Another important result is the Shannon's source coding 
theorem~\cite{shannon} that intuitively states that 
if one can associate variable-sized bit-codes with different outputs 
of a program such that the codes are uniquely-decodable, 
then the average of the code-lengths of these codes has been shown 
to be bounded by the Shannon entropy of the outputs of the program. 
The soundness of our approach builds on top of these two results. 
We associate bit-codes with the outputs of the program being analyzed 
and show that these codes are uniquely-decodable. In other words, 
the information release as computed by our approach averaged over 
all executions is bounded by the actual information released by the program.

% Various measures~\cite{shannon,guessing,smith2009,clarkson2009,
%   csf12GLeakage} 
% % like Shannon
% % entropy~\cite{}, min-entropy~\cite{}, guessing entropy~\cite{},
% % g-vulnerability~\cite{}, belief-tracking~\cite{} etc. 
% have been proposed for quantifying information leaks. 
% Many of these measures determine an average worst-case bound on
% the amount of information that a program can leak. These
% measures are based on associating variable-sized bit-codes to
% different outputs. Other measures consider the vulnerability of a 
% secret, which is directly proportional to its probability distribution.
% The entropy of the program outputs is the expected value of a random
% variable formed by the probability distribution of the different
% outputs or the most probable output. 

% Intuitively, the measures computes the average number of bits
% required by an adversary to uniquely encode the different outputs of
% the program related to the sensitive value. Thus, even a single bit
% can encode a 32-bit or 64-bit value.   
% Imagine that a sensitive value can take the following
% values: ``N'', ``S'', ``E'', and ``W''. To accurately determine the
% value, comparisons done twice would suffice. Assuming that only the
% results of the comparison operations are visible, 2 bits can be used
% to reveal any of the four values: ``N'' as \{\TT{true}, \TT{true}\},
% ``S'' as \{\TT{false}, \TT{true}\}, ``E'' as \{\TT{true}, \TT{false}\}
% and ``W'' as \{\TT{false}, \TT{false}\}. In all the four cases, the
% two bits reveal all the 32 or 64 bits contained in the value. However,
% all the entropy-based measures would compute a maximum of 2-bit
% leakage (depending on the probabilities the computed leakage can be
% less than 2 bits). 

% Thus, it is more intuitive to view the measure of mutual information
% as being bounded by the 
% amount of information leaked \emph{about} the value rather
% than viewing it as the number of bits of the value leaked. This
% generally amounts to the
% number of operations (guesses) allowed to reveal information about the
% secret. Its average over different executions is shown to be bounded
% by the Shannon entropy of the different outputs in this chapter using
% the Shannon source-coding theorem~\cite{shannon}.  

% The approach by~\cite{clarkson2009} reasons
% about an adversary's belief with respect to the secret
% values. The adversary associates a probability distribution with
% the secret value, which differs from the actual probability
% distribution of the secret. Based on the different outcomes of the
% program, the adversary alters her belief about the secret value. The
% information leak is based on the change in the adversary's
% belief. Most of these measures rely on the availability of a
% probability distribution for the output values.     

\section{Limited Information Release}
\label{sec:lir-desc}
Limited information release (LIR) is an information release policy
that declassifies some (limited) parts of sensitive information. 
% \dg{``Fraction of information'' is not a well-defined
%   phrase. Rephrase this sentence.}
The motivation for LIR is that, in
general, certain information flows leak only an insignificant part of
a secret. As an example, the \emph{comparison} of a secret with a
constant value is largely considered acceptable and its rejection by
standard IFC analyses is too restrictive in
practice~\referp{scd,dta}. Such information leaks are usually
acceptable if one can guarantee that an adversary cannot widen the
declassification to launder information. 
%The need for such policies is motivated with a couple of examples below.

In the real-world, such policies find a wide range of applications. For instance, 
in the advertisement example from Listing~\ref{egLeak}, the third-party 
ad service does not need to know the exact age of the viewer for determining 
the preferred advertisement. It is sufficient to know whether the viewer is 
a child or an adult (or a teenager). A similar example is that of a music 
app that hosts advertisements for music shows and concerts in a town. Based
on the age and the preferences of the user, the advertisements might differ 
in each case. 

\begin{lstlisting}[float,caption=Password Checker, label=egpc]
dbPwd $=$ getActualPassword(user);
uPwd $=$ readUserPassword();
login $=$ (dbPwd $==$ uPwd);
\end{lstlisting}

Another common use-case of such policies is user authentication 
based on a secret password or a PIN. The password checker in 
Listing~\ref{egpc} compares the secret password to 
the password entered by the user. The public variable \texttt{login}
reflects whether these match, indicating whether the login was
successful. Strict non-interference would normally prohibit 
assignments to the low variable \texttt{login} as the value is 
derived from the secret password. However, releasing the login 
status to the user cannot be avoided. Normally, a user can login 
in a single try but the probability of an adversary guessing the
correct password in one (or even a few) tries can be
assumed to be negligible when the password is strong and chosen 
randomly~\referp{relSec}. 
%In general, a brute-force attack on the secret would be 
%required, and if the secret is from a large domain then such an 
%attack is not a threat. 
Thus, the assignment to the \texttt{login} variable may be permitted 
for a few tries before the user's account gets locked.

Generally in these applications, there is a trade-off between some 
private information leak and better services.

%\subsection{Motivating Examples}
%\label{sec:examples}
%\begin{lstlisting}[float,caption=Age-based Advertisement, label=egLeak2]
% age $=$ getCurrentAge(birthday, birthyear);
% if (age $<$ 13)
%    preference $=$ "child";
% else if (age $<$ 20)
%    preference $=$ "teenager";
% else 
%    preference $=$ "adult";
%\end{lstlisting}
%
%\subsubsection{Age-based advertisements}
%A third-party advertising library might want to display ads %to its users
%based on the viewer's age. Consider the program snippet in
%Listing~\ref{egLeak2}, a modified version of the program in
%Listing~\ref{egLeak} which determines the preferred advertisements based on the
%secret value \texttt{age}. The third-party ad service does not 
%need to know the exact age of the viewer for that purpose. It is
%sufficient to know whether the viewer is
%a child, a teenager or an adult. While leaking very little information
%this provides focussed functionality and might
%even be required in some jurisdictions to protect minors from inappropriate
%ads. As another example, consider a music app that hosts
%advertisements for music shows and concerts in a town. Based
%on the age (whether the user is an adult, a teenager or a child) and
%the preferences of the user, the advertisement might display different
%categories. Again, there is a tradeoff between some 
%private information and better services. %do not require access to the
% precise age and would leak very little information about the user's age.  
% Limited information release allows
% such small leaks through comparisons on the sensitive value.



% \medskip
%\subsubsection{Password checker}
%User authentication is often based on a secret password or PIN.
 % in the context of the secret
%information. Limited information release allows assignments under such
% checks and, thus, deems the program secure.

% \begin{lstlisting}[float,caption=Shortcut Key Usage,label=egsk]
% window.addEventListener("keypress", function(event) {
%        if (event.altKey) {
%           send("ALT key pressed!");
%  }});
% \end{lstlisting}

% \dg{I don't see how LIR can help the next two examples. Either explain
%   or drop these examples.}
% % \medskip
% \paragraph{Analytics}
% Many web pages include analytics scripts to track user behavior on
% the page in order to improve the user
% experience. Most of these analytics scripts track events
% on the web page. One such analytics code is shown in
% Listing~\ref{egsk}, which tracks whether or not the
% \texttt{alt} modifier is pressed by the user. To prevent precise
% keylogging, the event
% properties (the keys pressed) are secret in this case, hence this
% program snippet does not satisfy non-interference. However, the only
% information that the script gains in this case is whether or not the
% \texttt{alt} modifier was pressed by the user. Such checks could be
% allowed in practice unless the script tries to track %and keep a log of
% all the keys pressed by the user.

% \begin{lstlisting}[float,caption=Sanity Check,label=egsc]
%  if (textInput $\neq$ NULL)
%     useInput ();
%  else
%     error("no input");
% \end{lstlisting}
% % \medskip
% \paragraph{Sanity check}
% Another common case where information release is acceptable is during
% sanity checks, e.g., whether input has been provided or not, as shown in
% Listing~\ref{egsc}. The \texttt{textInput} could be some secret value
% but in order to use the value it might be necessary to check if the value is
% non-null. If no input is
% provided, the error ``\texttt{no input}'' only reveals that the
% secret \texttt{textInput} is equal to \texttt{NULL}.
% % Information release in such cases could be acceptable and allowed.
% % Limited information release allows this kind of release.


%\subsection{Limited Information Release Policy}
%\label{sec:lir-policy}

In practice, certain operations involving Boolean-valued expressions, 
and implicit flows leak very 
little information, which might be required for providing proper
functionality. King \emph{et al.}~\refert{king08implicit} 
investigate the occurrence of implicit flows in some standard 
algorithms and discuss the pros and cons of handling implicit 
flows. Russo \emph{et al.}~\referp{implicit} also observe
that implicit flows cannot be exploited in non-malicious code to 
leak secrets efficiently. The LIR policy is, thus, guided by two key tenets: 

\begin{itemize}
\item \textbf{Declassification of Boolean-valued expressions.}  As has been
  observed previously~\referp{scd, dta} and is exemplified
  above, comparison operations and other Boolean values often provide a
  low bandwidth channel for information leak. For
  instance, $h \neq l$ only reveals whether the two values in $h$ and
  $l$ are equal or not. Similarly, $h < l$ and $h > l$ only reveal
  that the value of $h$ is lesser and greater than the value of $l$,
  respectively. With LIR, such comparison operations and Boolean-valued  
  expressions are treated as potential points of information release. 
  As all operations need not be declassified, 
  information is released only for those 
  operations that are explicitly annotated with \TT{declassify}.

\begin{lstlisting}[float, caption=Laundering attack via implicit flows, label=lst:laundering]
pub $=$ 0; i $=$ 1;
while (i $\leq 2^{31}$) { @\label{imp:while}@
if ((sec $\&$ i) $==$ i) @\label{imp:if}@
pub $=$ pub $|$ i;
i $=$ i $<<$ 1;
}
\end{lstlisting}

\item \textbf{Bounding the information released.} 
  Unfortunately, allowing information release when
  only an insignificant amount of information is leaked can be widened
  to leak the complete secret~\referp{relSec,delRelease}. 
  The example in Listing~\ref{lst:laundering} is a classical example 
  of a laundering attack. It implicitly leaks the secret value
  in \texttt{sec} to the variable \texttt{pub} without any direct assignments.
  Every time the check on line~\ref{imp:if} is performed, it leaks one
  bit of \texttt{sec}. The whole secret gets implicitly laundered into
  the variable \texttt{pub} as this comparison is performed in a loop of 
  length equal to the size of the secret (in bits).
  
  To limit the amount of information that can be leaked through 
  laundering attacks, LIR introduces a notion of \emph{budget} (a
  natural number) associated with a sensitive value, which is an
  upper bound on the amount of information that is allowed to leak about
  the sensitive value. A budget is associated with every secret in the
  program and defaults to $0$, if left unspecified. 
\end{itemize}

% \medskip
Intuitively, a program is said to satisfy limited information release,
if the information released about the initial value of each secret in
the system is limited by its pre-specified budget. If no information 
is released about any secret or if the budgets for all the secrets 
reduce to $0$, LIR reduces to the standard
non-interference policy. It is important to note that even when
staying within the bounds of the budget one can leak the entire secret
via comparison operations. For instance, if an equality comparison
$h==l$ involving a secret value $h$ and a public value $l$ 
returns \TT{true}, a public adversary knows that the value of $h$ is
the same as the value in $l$. 
% The bounds are only meaningful in an
% average sense like in many other existing entropy-based notions of
% information leakage~\referp{denning82, clark, smith2009}. In other
% words, LIR only ensures that the average information leak over
% multiple executions is bounded by the specified budget and it should
% not be understood as providing leakage bounds over a single
% execution. \dg{The previous sentence directly contradicts the implied
%   meaning of sentences in the introduction. Here, you are claiming
%   that dynamic analysis provides only average bounds; in the
%   introduction, you claimed the exact opposite. This is a serious
%   issue. The story has to be straight.}

%\dg{I think I've said this many times. I don't understand why we are
%  restricting declassification to comparisons only. POPL readers will
%  ask the same question, but somehow we don't seem willing to change
%  this. Why?} 
%
%\TODO{The reason we restrict ourselves to comparisons is that for 
%	other operations the declassification operation would count all 
%	32/64 bits and the budget would expire instantly. It doesn't make 
%	sense to add a policy that allows all 32/64 bits to be leaked. 
%	Rather, I would expect that the policy is specified separately 
%	as we do in WebPol.
%}

Although declassification of only Boolean-valued expressions is 
allowed by LIR, this can be extended to other expressions as well. 
In such cases, where the expression being declassified has a 
data type having $n$ possible values, the leak is 
accounted for by assuming that $\log_{2}n$ bits are leaked. 
As in this paper we consider only Boolean and integer data types, 
we restrict the information release to only Boolean expressions.  
Declassifying an integer value would effectively mean that all bits 
of information about the secret value have been released and the 
secret value could have been simply declassified.




\section{LIR Enforcement}
\label{sec:formalization}

\subsection{Language and Syntax}
This section describes a runtime enforcement of LIR for the simple
imperative language shown in Figure~\ref{basic:syntax} extended with
the \dec~operator as shown in Figure~\ref{fig:syntaxLIR}. The
comparison and arithmetic operators are separated as $\op$ and $\aop$,
respectively, for the rules. The language
is sufficient for describing the key idea of LIR --- appropriate
deduction of budgets at declassification of Boolean expressions involving
secrets. 

\begin{figure}[!htbp]
\begin{align*}
  \expr	:=~&\TT{n}~\arrowvert~\TT{x}~\arrowvert~\expr_1 \aop \expr_2~\arrowvert~
             \expr_1 \op \expr_2~\arrowvert~\dec(\expr_1 \op \expr_2)\\
  \comm	:=~&\sk~\arrowvert~\TT{x} :=
             \expr~\arrowvert~\comm_1;\comm_2~\arrowvert~\texttt{if}~\expr~\texttt{then}~\comm_1~\texttt{else}~\comm_2~\arrowvert~ 
   \texttt{while}~\expr~\texttt{do}~\comm
\end{align*}
\caption{Syntax of the language}
\label{fig:syntaxLIR}
\end{figure}

\subsubsection{\textbf{Value-labels, budgets and budget-labels}}
Every input (coming in via store variables) to the program has an
initial immutable confidentiality label associated with it referred to
as the \emph{value-label}, which is an element of a security lattice
and represented using $k, l, m$. An immutable map $\ilabel$
represents the mapping from a variable to its value-label. Along with
the value-label, every input is also associated with a \emph{budget}
and an immutable \emph{budget-label}, which is also an element 
of the security lattice. The budget of an input
represents an upper bsound on the amount of information that is allowed
to leak about that input. As budgets are publicly visible, changes to
budgets in secret contexts can result in additional leaks as we
demonstrate later. Such leaks are handled using the budget-label by  
disallowing budget reduction if the budget-label is not at least as
high as the current program context ($\pc$). Additionally, the
budget-label indicates the confidentiality level to which information
about the secret value can be released. An immutable budget-map $\blabel$
tracks the budget-label associated with each variable. 
% Since the budgets change as the program
% executes, they themselves can be a channel of information leak.
The budgets are represented as $\TT{n}^l$, where $\TT{n}$ is the
budget and $l$ is the budget-label. We require that the budget-label 
is lower than the value-label for a variable, i.e., 
$\forall \TT{x}.\blabel(\TT{x}) \sqsubset \ilabel(\TT{x})$.
%, i.e., a variable having a budget
%of $1$ and the budget-label $L$ is denoted as $1^L$.

\subsubsection{\textbf{Label format and dependencies}}
Instead of directly tracking the label on the values, variable
dependencies are tracked. As every variable has
an associated value-label, budget, and budget-label, tracking
dependencies enables access to this metadata for every dependency
individually. This assists the enforcement by determining the right
variables to deduct budget from, which is required at the point of
declassification. These dependencies are split into two parts and the
label on a value is represented as a pair: $(l, \dep)$. The variable
is itself represented as $\TT{x} = \TT{n}^{(l, \dep)}$, where \TT{n}
is the value in the variable \TT{x} and $(l, \dep)$ is the label on
the value. $\dep$, referred to as dependency-set, is a set of
variables on whose initial values the current value
depends. Information can only be released about the variables that are
included in the dependency-set. The first part of the label $l$,
referred to as the secrecy-level of the value, is an element of the 
security lattice and the join of the 
value-labels (upper-bound) of all the dependencies for which no more
information is allowed to (or can) be released. 

Once the budget of a variable expires (becomes $0$), the value-label of that
variable is joined with the secrecy-level of the current value and
the variable is removed from the dependency-set of the current value. 
For instance, assume that initially $\TT{x} = \TT{n}^{(\bot, \{\TT{x}, \TT{y}\})}$ 
such that $\ilabel(\TT{x}) = k$ and $\ilabel(\TT{y}) = m$, once the budget of
$\TT{y}$ expires, $\TT{x} = \TT{n}^{(m, \{\TT{x}\})}$. The
secrecy-level of a value $l$ represents the minimal level at which the
value can be observed, i.e., the current value in the variable can never 
be released or declassified to a level below $l$. Initially, every
variable \TT{x} depends only on itself and is labeled 
$(\bot, \{\TT{x}\})$, where $\bot$ represents the least
element in the lattice. 

The join operation on the labels returns the join of the
secrecy-levels and the union of the dependency-sets as the final
label, i.e., 
$$(l, \dep) \sqcup (l', \dep') = (l \sqcup l', \dep \cup \dep')$$
Similarly, the ordering on the labels is defined as:
$$(l, \dep) \sqsubseteq (l', \dep') = (l \sqsubseteq l', \dep \subseteq \dep')$$
The function $\Lambda(\TT{x})$ returns the current label of the value
in \TT{x}, i.e., if $\TT{x} = \TT{n}^{(l, \dep)}$, then
$$\Lambda(\TT{x}) = \Lambda(\TT{n}^{(l, \dep)}) = {(l, \dep)}$$ 
The function $\Gamma(\TT{n}^{(l, \dep)})$ returns the current
confidentiality level of the value, i.e., 
$$\Gamma(\TT{n}^{(l, \dep)}) = l \bigsqcup\limits_{\TT{x} \in
	\dep} \ilabel(\TT{x})$$ 
Note that the value in \TT{x} is visible to an $\attacker$-level
adversary only if $\Gamma(\TT{x}) \sqsubseteq \attacker$ while
$\Lambda(\TT{x})$ allows us to track the required dependencies. 
% The function $\depend(\TT{x})$ returns all the dependencies of
% $\TT{x}$.
$\pc$ denotes the current program-context level, and 
contains \emph{only} the secrecy-level, represented as $(l, \{\})$,
i.e., none of the dependencies in the $\pc$ are allowed to release any
information through the variables being assigned within the branch and
are hence joined with the secrecy-level $l$ of the $\pc$ before
entering the branch or loop. This decision is justified below with an
example. The function $\Gamma(\pc)$ returns the secrecy-level $l$ of
the $\pc$. When the meaning is clear from the context, we use $\pc$ 
instead of $\Gamma(\pc)$.  

\subsection{Key Aspects of the Enforcement}
\label{sec:enf-design}

%% We begin by describing the subtleties involved in enforcement that motivated the
%% design of the semantics:
\subsubsection{\textbf{Dependency Tracking}}
\label{aspect:dt} 
A program execution can release information through variables that are
dependent on the initial value of some secret variables. Instead of
directly tracking abstract labels, which gives a bound on the secrecy
but loses the actual input variable dependency, variable dependencies
are tracked. These dependencies determine the  variables to deduct the
budget from. Dependency tracking is required for a correct analysis 
and to prevent an adversary from laundering secret information. To
illustrate why dependency tracking is imperative for a sound
analysis, consider the example in Listing~\ref{egDep} with the 
security lattice $L \sqsubset H$. Assume that the initial values in
\texttt{sec} and \texttt{h} are both secret having the value-label $H$ 
and a budget of $1^L$; the initial values in \texttt{pub} and
\texttt{i} are both public having the value-label $L$.

\begin{lstlisting}[float,caption=Leak due to dependent variables, label=egDep]
 pub $=$ 0;
 h $=$ sec $\%$ 2;  @\label{a:t}@
 sec $=$ sec $/$ 2;
 if (declassify(h $==$ 1))  @\label{if:h}@
    pub $=$ pub $|$ 1; @\label{if:a}@
 i $=$ sec $\%$ 2; @\label{a:2}@
 if (declassify(i $==$ 1)) @\label{if:i}@
    pub $=$ pub $|$ 2; @\label{if:a1}@
\end{lstlisting}

Suppose that at the assignment on line~\ref{a:t} only the
confidentiality level of \texttt{sec} is carried over to the label of
\texttt{h} but the actual variable dependencies are not tracked.
Without tracking the dependent variables, the declassification on
line~\ref{if:h} would release the value of \texttt{h} (the last bit of
\texttt{sec}) to \texttt{pub} deducting the budget of \texttt{h} but
not \texttt{sec}. As a result another bit of information about
\texttt{sec} can be leaked later (through the \TT{declassify} on
line~\ref{a:2}), which should not have been allowed as the budget of
\texttt{sec} is just $1$. A trivial solution to this problem would be
to reduce the budget of all the dependencies at the point of
declassification. 

With dependency tracking, initially \texttt{h} and \texttt{sec} are 
dependent on themselves, i.e., \texttt{h} and \texttt{sec},
respectively. The assignment on line~\ref{a:t} would update the
dependence of \texttt{h} on \texttt{sec}. As a result, the
declassification on line~\ref{if:h} would deduct the budget from
\texttt{sec} and not \texttt{h}. On line~\ref{a:2}, \texttt{i} is
updated with the next bit of \texttt{sec} and the dependency-set of
\texttt{i} is updated to contain only \texttt{sec}. As \texttt{sec}'s
budget is expired on line~\ref{if:i}, the monitor joins the
secrecy-level of \texttt{sec} in the new program context, making it
$H$, thus limiting the amount of information released about 
the initial value of \texttt{sec}.  

% \medskip
\begin{lstlisting}[float,caption=Example to illustrate budget-label
  constraints,label=egbr] 
 if (a $==$ 0) @\label{br0}@
   x $=$ b @\label{brxb}@
 z $=$ declassify(x $==$ 1) @\label{brz}@
 y $=$ declassify(b $==$ 0) @\label{bry}@
\end{lstlisting}
\subsubsection{\textbf{Budget-label Constraints}}
\label{aspect:blc} 
The reduction of budget for all variables in the dependency-set can 
result in additional unaccounted leaks. 
We illustrate the leak using the example in Listing~\ref{egbr} and 
the security lattice $L \sqsubset M \sqsubset H$. Assume that
\TT{a} and \TT{b} are two secrets having value-labels $\ilabel(\TT{a})
= M$ and $\ilabel(\TT{b}) = H$ and the current labels $\Lambda(\TT{a})
= (M, \{\})$ and $\Lambda(\TT{b}) = (L, \{\texttt{b}\})$, respectively.  
The initial budgets of \TT{a} and \TT{b} are $0$ and $1^L$,
respectively. Also assume that the label of \TT{x} is $\Lambda(\TT{x})
= (M, \{\})$. Considering two executions of the program depending on
whether the value of \TT{a} is $0$ or not, we show that not imposing
the condition $\forall \TT{x} \in \dep. l \sqsubseteq \blabel(\TT{x})$
when reducing the budget of a secret variable labeled $(l, \dep)$ can
result in additional leaks.  

When \texttt{a} is $0$, \texttt{x} is assigned \texttt{b} on
line~\ref{brxb} with a label $(M, \{\texttt{b}\})$ (the join of the
label of \TT{a} and \TT{b}). On line~\ref{brz}, \texttt{x} can be
declassified to level $M$ ($M \sqcup L$ as the budget of \texttt{b} is 
$1^L$). Thus on line~\ref{brz}, the label of the value in \texttt{z}
becomes $(M, \{\})$ and the budget of \texttt{b} becomes $0$. On  
line~\ref{bry}, as \texttt{b}'s budget has expired, \texttt{b}'s
current label becomes $(H, \{\})$; \texttt{y} is also labeled $(H,
\{\})$. In the other run when \texttt{a} is not $0$, \texttt{x}
retains its label of $(M, \{\})$ on line~\ref{brz}. As the
dependency-set is empty, no declassification occurs on
line~\ref{brz}. As the budget of \texttt{b} on line~\ref{bry} is still
$1^L$, the value on line~\ref{bry} is declassified to $L$. Thus, the
label of \texttt{y} after the assignment becomes $(L, \{\})$.  Thus,
in the first run an 
$L$-level adversary cannot observe any value while in the second run,
the adversary can see the value as it is labeled $L$. As the adversary
knows the budget of \texttt{b}, it can conclude that in the first case
the branch on line~\ref{br0} was taken and in the second case it was
not taken. This leaks an additional bit of information about
\texttt{a} although the budget of \texttt{a} doesn't allow any
information release. 

This happens because a purely dynamic information flow monitor
cannot account for dependencies in alternate branches. To prevent 
this leak, when declassifying a value labeled $(l, \dep)$, the 
budget is reduced for only those variables in the dependency-set 
$\dep$ that have a budget-label at least as high as $l$, i.e., 
$\forall \TT{x} \in \dep. l \sqsubseteq \blabel(\TT{x})$. 
By imposing this condition, in the first run of the above example, 
the value of \texttt{x} labeled $(M, \{\texttt{b}\})$ on line~\ref{brz} 
is not declassified as the budget-label of \texttt{b} 
($\blabel(\texttt{b}) = L$) is lower than the secrecy-level of the 
value ($M$). The execution in the first run then proceeds 
similar to the second run, thereby preventing the additional leak.  

% \medskip
\begin{lstlisting}[float,caption=Example to illustrate budget reduction,label=egbred]
 if (a $==$ 0) @\label{egbr0}@
   x $=$ declassify(b $==$ 0) @\label{egbrxb}@
 z $=$ declassify(b $==$ 1) @\label{egbrz}@
\end{lstlisting}
\subsubsection{\textbf{Budget Reduction}}
\label{aspect:br}
The budget of a secret variable is publicly visible, which can lead to
additional information leaks due to budget reduction in a secret context. 
We impose two additional conditions to prevent such leaks: 
\begin{enumerate}
\item A value cannot be declassified to a level lower than 
  the current program context $\pc$, as this could leak information
  about the context itself
\item The budget of a variable can only be reduced in a $\pc$ lower
  than or equal to the budget-label of the variable.
\end{enumerate} 
In the current setting, this would mean that if the 
label of the value being declassified is $(l, \dep)$ where $dep$
contains a variable \TT{x}, then information about \TT{x} is only
declassified (the budget of \TT{x} is only reduced) if $\pc
\sqsubseteq l \sqcup \blabel(\TT{x})$. 

Without these checks in place, the monitor could leak additional
information as illustrated by the example in Listing~\ref{egbred}. 
As before, assume that \TT{a} and \TT{b} are two secrets having
value-labels $\ilabel(\TT{a}) = M$ and $\ilabel(\TT{b}) = H$ and the
current labels $\Lambda(\TT{a}) = (M, \{\})$ and $\Lambda(\TT{b}) = (L,
\{\texttt{b}\})$, respectively, such that $L \subseteq M \subseteq H$.   
The initial budgets of \TT{a} and \TT{b} are $0$ and $1^L$,
respectively. Also assume that the label of \TT{x} is $\Lambda(\TT{x})
= (M, \{\})$. Again, consider two executions of the program depending
on whether the value of \TT{a} is $0$ or not. 
Assume that the current context ($\pc$) 
is not checked for before declassifying a value. 

When \texttt{a} is $0$, the value of \texttt{b == 0} is declassified
on line~\ref{egbrxb} making \TT{b}'s budget $0$. \TT{x} is assigned
the label $(M, \{\})$ (the join of the label of \TT{a} and the
budget-label of \TT{b}). On line~\ref{egbrz}, as \texttt{b}'s budget
has expired, \texttt{b}'s current label becomes $(H, \{\})$ and
\texttt{z} is labeled $(H, \{\})$. In the other run when \texttt{a} is
not $0$, \texttt{x} retains its label of $(M, \{\})$ on
line~\ref{egbrz} and the budget of \TT{b} also remains unchanged. As
the budget of \texttt{b} on line~\ref{egbrz} is still $1^L$, the value is
declassified to the level $L$. Thus, in the second run the label of
\texttt{z} after the assignment becomes $(L, \{\})$.  While in the
first run an $L$-level adversary cannot observe any value, in the
second run  the adversary can see the value as it is labeled $L$. As
the adversary knows the budget of \texttt{b}, it can conclude that in
the first case  the branch on line~\ref{egbr0} was taken and in the
second case it was not taken. This leaks an additional bit of
information about \texttt{a} although the budget of \texttt{a} doesn't
allow any information release. 

By allowing the declassification operation only when $\pc \sqsubseteq l \sqcup
\blabel(\TT{x})$, the additional leak can be prevented. With this
condition in place, in the first run of the program, the value of 
\TT{b == 0} on line~\ref{egbrxb} is not declassified as the current
program context $M$ is not lesser than or equal to $L$, which is the
join of the secrecy-level and the budget-label of \TT{b}. Instead, the
declassification occurs on line~\ref{egbrz} thereby preventing any
additional information leaks. 

% \medskip
\subsubsection{\textbf{Handling implicit flows}}
\label{aspect:if}
Dynamic IFC approaches handle implicit flows via control constructs by
employing a program context ($\pc$) label. One commonly used method
for preventing implicit leaks in a dynamic and termination-insensitive 
setting is the no-sensitive-upgrade (NSU) check
(Section~\ref{sec:bgdifc}). The NSU check does not allow assignments
to public variables in a sensitive context ($\pc$). Since variable
dependencies are used in addition to the secrecy-levels, those
dependencies also need to be checked for during assignment
operations. For an assignment to succeed, it is important that
no new dependencies are handed over to the variable as part of the
assignment operation. 

Directly using dependencies in the $\pc$ could lead to
additional information leaks. This happens because the assignments
that happen under a $\pc$ would carry the dependencies of
the $\pc$ too while in an alternate run those dependencies might not
be present in the label of the assigned variable. Such mismatch of
dependencies can leak information as illustrated using an example in
Listing~\ref{egimp}.

\begin{lstlisting}[float,caption=Illustration of handling of
  implicit flows,label=egimp]
 if (x $\leq$ 10) @\label{impbr}@
   x $=$ y @\label{impassn}@
 z $=$ declassify(x $\leq$ 5) @\label{impdec}@
\end{lstlisting}

For the program in Listing~\ref{egimp}, assume that \texttt{x},
\texttt{y} and \texttt{z} have value-labels of $H$, $H$ and $L$ ($L
\sqsubseteq H$), and budgets of $1^L$, $0$ and $0$ assigned to them,
respectively. Also assume that their current labels are of the
form $\Lambda(\TT{x}) = (L, \{\texttt{x}\})$, $\Lambda(\TT{y}) = (H,
\{\})$ and $\Lambda(\TT{z}) = (L, \{\})$. 
Consider two runs of the program with $\texttt{x} \leq 10$ and
$\texttt{x} > 10$. Suppose that the check for dependencies is included
in the $\pc$, i.e., the assignment succeeds only if $\pc = (l, \dep)
\sqsubseteq (k_o, \dep_o)$ where $(k_o, \dep_o)$ is the label of the
variable \texttt{x}.  When $\texttt{x} \leq 10$, the branch on
line~\ref{impbr} is taken and the $\pc$ on line~\ref{impassn} would be
$(L, \{\texttt{x}\})$, which would allow the assignment (as the label 
of \TT{x} is not less sensitive than the $\pc$). As a result the label
of \texttt{x} would be updated to $(H, \{\TT{x}\})$ (join of $\pc$ and
label of \texttt{y}). On line~\ref{impdec}, no information is
released as the budget-label of \texttt{x} is lower than its
secrecy-level ($H$) (as explained above in
Section~\ref{aspect:blc}). In the other run when 
$\texttt{x} > 10$, the branch is not taken and the value of
$\texttt{x} \leq 5$ on line~\ref{impdec} is released to the level $L$
as the label of \texttt{x} on line~\ref{impdec} is $(L,
\{\texttt{x}\})$ and its budget is $1^L$. This leads to different
sensitivity of \texttt{z} in the two runs which leads to leaking an
additional bit about \TT{x} in the first run without being accounted
for in its budget. 

To prevent this leak, we compute the combined label of all the
dependencies of the value in the predicate before using it in the
$\pc$, i.e., we carry only the secrecy-level in the $\pc$. Thus, in
the first run of the above example, the $\pc$ on line~\ref{impassn}
becomes $(H, \{\})$ (because $\ilabel(\texttt{x}) = H$) and the
assignment fails due to the no-sensitive-upgrade check. To ensure
soundness, it is required that $\pc \sqsubseteq k$, where $k$ is the
secrecy-level of the variable being assigned, i.e., the secrecy-level
of the variable being assigned is at least as high as the
secrecy-level of the $\pc$. 

% \medskip

\subsubsection{\textbf{Permissiveness}}
\label{aspect:perm}
The overall confidentiality of the underlying value with a
label $(l, \dep)$ is the join of $l$ with the value-label of all the
variables in the dependency-set. Thus, all those dependencies whose 
value-label is below the secrecy-level of the label can be removed as 
keeping those dependencies does not affect the confidentiality of the
value. This improves the permissiveness of the technique by not
marking the release of information for variables that have a label
lesser than or equal to the current secrecy-level or the $\pc$
because no information can be released below the current secrecy-level 
of the variable or the current $\pc$. For instance, consider a 
value $\TT{n}^{(k, \{\TT{x}\})}$. If $\ilabel(\TT{x}) \sqsubseteq k$, 
then \TT{x} is removed from the dependency-set and the value is
represented as $n^{(k, \{\})}$. As explained earlier, budgets are
deducted only when $\pc \sqsubseteq k$, thus, the budgets of any
variable whose value-label is below the $\pc$ are not deducted from.  
 
\begin{lstlisting}[float,caption=Example to illustrate permissiveness,label=egp]
 if (med $\leq$ 0) 
    x $=$ declassify(sec $==$ y); @\label{ifb}@
 if (declassify(y $\neq$ 100))   @\label{if1}@
    pub $=$ 1;  @\label{ifb1}@
\end{lstlisting}

To illustrate the gain in permissiveness by enforcing the condition
specified above, consider the program in Listing~\ref{egp}. Assume
that, \texttt{med}, \texttt{x}, \texttt{sec}, \texttt{y}, and
\texttt{pub}  are labeled $(M, \{\})$, $(M, \{\})$, $(M,
\{\texttt{sec}\}), (L, \{\texttt{y}\})$, and $(L, \{\texttt{pub}\})$,
respectively, with their respective value-labels being
$\ilabel(\texttt{sec}) = H, \ilabel(\texttt{y}) = M, 
\ilabel(\texttt{pub}) = L$ such that $L \sqsubset M \sqsubset
H$. Also, assume that the budgets of \texttt{sec} and \texttt{y} are
$1^M$ and $1^L$, respectively. The program context label ($\pc$) on
line~\ref{ifb} is $(M, \{\})$. The expression 
$\TT{sec} == \TT{y}$ on line~\ref{ifb} has the label $(M, \{\TT{sec},
\TT{y} \})$.

Without the check for the current secrecy-level, the
expression evaluation on line~\ref{ifb} releases the value of 
\texttt{sec} and \texttt{y} into \texttt{x}. However as the
secrecy-level is $M$, the final value of \texttt{x} would have the
secrecy-level of at least $M$. Thus, information about \texttt{y} is
released to the levels $M$ and above, who already had access to the
value of \texttt{y}. Subsequently, on line~\ref{if1} as the budget of
\texttt{y} has expired, no more information about \texttt{y} is
released and the assignment on line~\ref{ifb1} fails. Thus, no useful
information about \texttt{y} is released at any point in the program,
yet the program is terminated because of the NSU check. With the check 
for current secrecy-level and the context label, the budget of
\texttt{y} is not deducted on line~\ref{ifb}, which allows the check
on line~\ref{if1} to release information about \texttt{y} 
thereby allowing the assignment on line~\ref{ifb1} to succeed.


\subsection{Semantics}
\label{sec:semantics}
To formally define the information released by a program starting from some
initial memory containing secret values, big-step semantics is defined
for the language shown in Figure~\ref{fig:syntaxLIR} that releases
some information about initial secret values at Boolean declassifications.
% For simplicity of
% exposition, we explain the enforcement for a simple WHILE
% language. The key idea of the policy, however, can be generalized to
% any language.
% We define a big-step semantics for the language in
% Figure~\ref{fig:syntaxLIR}.
% and instrument them with the kind
% of budget and label checks described above.

Program configurations are extended with $\iota$ representing the
budget store, which is a map from the variables to the budget of these
initial  values (inputs to the program). The configurations for
expressions ($\expr$) and commands ($\comm$) are, thus, represented
by $\langle \sigma, \iota, \expr \rangle$ and $\langle \sigma, \iota, 
\comm \rangle$, respectively where $\sigma$ represents the memory
store as before. 
$\langle\sigma, \iota, \expr \rangle \lirsem \TT{n}^{(k, \dep)}, \iota',
\trace$ defines expression evaluation. It means that under some
$\pc$, starting with memory $\sigma$ and budget store $\iota$, the
expression $\expr$ evaluates to a value $\TT{n}$ labeled
$(k, \dep)$ resulting in a budget store $\iota'$. Additionally, the
evaluation generates a trace $\trace$, which is a list of values of the
form $\TT{n}_1^{k_1} :: \TT{n}_2^{k_2} :: \ldots ::
\TT{n}_j^{k_j}$. Every declassified value is recorded on the trace along
with the level to which it is declassified. The two immutable maps ---
initial label map ($\ilabel$) and budget-label map ($\blabel$) are
assumed to be available and omitted from the rules for clarity.

\begin{figure}[!htbp]
\begin{framed}
\begin{mathparpagebreakable}
%%%%%%%%%%%%%%%%%%CONST 
\inferrule*[left=\mbox{\labelthis{lir:exp:c}{const}}]
{ }
{\langle\sigma,  \iota,  \TT{n} \rangle \lirsem
  \TT{n}^{(\bot, \{\})}, \iota,  \emptyTrace}
\and 
%%%%%%%%%%%%%%%%%%VAR
\inferrule*[left=\mbox{\labelthis{lir:exp:v}{var}}]
{\sigma(\TT{x}) = \TT{n}^{(k_o, \dep_o)} \\ 
  \dep' = \Reduce(\dep_o, \iota, k_o, \blabel) \\
  \dep = \dep_o \setminus \dep' \\
  % \dep = \dep' \\ 
  k = k_o \bigsqcup\limits_{\TT{x} \in \dep'} \ilabel(\TT{x}) 
}
{\langle\sigma, \iota,  \TT{x} \rangle \lirsem \TT{n}^{(k, \dep)},
  \iota, \emptyTrace}
\and 
%%%%%%%%%%%%%%%%%%AOP
\inferrule*[left=\mbox{\labelthis{lir:exp:aop}{aop}}]
{\langle \sigma, \iota,  \expr_1 \rangle \lirsem
  \TT{n}_1^{(k_1, \dep_1)}, \iota_1, \trace_1 \\ \langle \sigma,
  \iota_1,   \expr_2 \rangle \lirsem \TT{n}_2^{(k_2, \dep_2)},
  \iota', \trace_2 \\ \TT{n} = \TT{n}_1 \aop \TT{n}_2 \\ (k, \dep) =
  (k_1, \dep_1) \sqcup (k_2, \dep_2)  
}
{ \langle\sigma,  \iota,   (\expr_1~\aop~ \expr_2) \rangle
  \lirsem \TT{n}^{(k, \dep)}, \iota', (\trace_1 ::  \trace_2)
}
\and
%%%%%%%%%%%%%%%%%%OPER-COP-W-D
\inferrule*[left=\mbox{\labelthis{lir:exp:cop}{cop}}]
{\langle \sigma, \iota,  \expr_1 \rangle \lirsem
  \TT{n}_1^{(k_1, \dep_1)}, \iota_1, \trace_1 \\ \langle \sigma,
  \iota_1,   \expr_2 \rangle \lirsem \TT{n}_2^{(k_2, \dep_2)},
  \iota', \trace_2 \\ \TT{n} = \TT{n}_1 \cop \TT{n}_2 \\ (k, \dep) =
  (k_1, \dep_1) \sqcup (k_2, \dep_2)
  % \\  \forall x \in \dep.(l = l \sqcup \ilabel(x))
}
{ \langle\sigma,  \iota,   (\expr_1~\cop~ \expr_2) \rangle
  \lirsem \TT{n}^{(k, \dep)}, \iota', (\trace_1 :: \trace_2)}
\and
%%%%%%%%%%%%%%%%%%OPER-COP
\inferrule*[left=\mbox{\labelthis{lir:exp:dn}{dcopn}}]
{
\langle\sigma,  \iota,   (\expr_1~\cop~ \expr_2) \rangle
  \lirsem \TT{n}^{(k, \dep)}, \iota', \trace \\
\pc \not\sqsubseteq k 
% \langle \sigma,  \iota,  \expr_1 \rangle \lirsem
%   \TT{n}_1^{(k_1, \dep_1)} , \iota_1, \trace_1 \\\\
%  \langle \sigma, \iota_1, \expr_2  \rangle \lirsem
%  \TT{n}_2^{(k_2, \dep_2)},  \iota_2, \trace_2  \\\\
% \TT{n} = \TT{n}_1 \op \TT{n}_2 \\ \dep =\dep_1 \cup \dep_2 \\
% k = k_1 \sqcup k_2 \\\\ 
% \pc \not\sqsubseteq (k, \dep) \bigwedge
% \Lambda(\pc) \not\sqsubseteq k \\ 
% \forall x \in \dep. (l = l \sqcup \ilabel(x)) \qquad
% \trace = \trace_1 :: \trace_2
}
{\langle\sigma,  \iota,  \dec(\expr_1 \op \expr_2)
  \rangle \lirsem \TT{n}^{(k, \dep)},
  \iota', \trace}
\and
%%%%%%%%%%%%%%%%%%OPER-COP
\inferrule*[left=\mbox{\labelthis{lir:exp:dr}{dcopr}}]
{
\langle\sigma,  \iota,   (\expr_1~\cop~ \expr_2) \rangle
  \lirsem \TT{n}^{(k_o, \dep_o)}, \iota_1, \trace_1 \\
  \pc \sqsubseteq k_o \\
 \dep' = \Delta(\dep_o, k_o, \ilabel) \\
\dep'' = \Reduce(\dep',\iota_1, k_o, \blabel) \\
  \dep = \dep' \setminus \dep'' \\ 
k = k_o \bigsqcup\limits_{\TT{x} \in \dep''}
  \ilabel(\TT{x}) \bigsqcup\limits_{\TT{y} \in \dep} \blabel(\TT{y}) \\ 
\iota' = \mathbb{I}(\iota_2, \dep)\\
% \trace = \trace_1 :: (\TT{n},{k},\dep) \\
\trace = 
\left\{\begin{array}{ll}
        \trace_1 :: \trace_2 :: \TT{n}^{k}, & \textit{if } (\dep \neq \emptyset) \arcr
        \trace_1 :: \trace_2, & \textit{otherwise } \arcr
        \end{array}\right\} 
%\left\{\begin{array}{ll}
%        \trace_1 :: (\TT{n},{k},\dep), & \textit{if } (\dep \neq \emptyset) \arcr
%        \trace_1, & \textit{otherwise } \arcr
%        \end{array}\right\} 
}
{\langle\sigma,  \iota,  \dec(\expr_1 \op \expr_2)
  \rangle \lirsem \TT{n}^{(k, \{\})}, \iota', \trace}
%
% \inferrule*[left=\mbox{\labelthis{lir:exp:dr}{dcopr}}]
% {\langle \sigma,  \iota,  \expr_1 \rangle \lirsem
%   \TT{n}_1^{(k_1, \dep_1)} , \iota_1, \trace_1 \\
%  \langle \sigma, \iota_1, \expr_2  \rangle \lirsem
%  \TT{n}_2^{(k_2, \dep_2)},  \iota_2, \trace_2  \\\\
% \TT{n} = \TT{n}_1 \op \TT{n}_2 \\ \dep_o = \dep_1 \cup \dep_2 \\
% k_o = k_1 \sqcup k_2 \\\\
% \pc \sqsubseteq k_o \\
% % \pc \sqsubseteq (k_o, \dep_o)~ \bigvee~ \Gamma(\pc) \sqsubseteq k_o \\
%  \dep' = \Delta(\dep_o, k_o, \ilabel) \\
% \dep'' = \Reduce(\dep',\iota_2, k_o, \blabel) \\\\
%   \dep = \dep' \setminus \dep'' \\ 
% k = k_o \bigsqcup\limits_{\TT{x} \in \dep''}
%   \ilabel(\TT{x}) \bigsqcup\limits_{\TT{y} \in \dep} \blabel(\TT{y}) \\ 
% \iota' = \mathbb{I}(\iota_2, \dep)\\
% \trace = 
% \left\{\begin{array}{ll}
%         \trace_1 :: \trace_2 :: \TT{n}^{k}, & \textit{if } (\dep \neq \emptyset) \arcr
%         \trace_1 :: \trace_2, & \textit{otherwise } \arcr
%         \end{array}\right\} 
% }
% {\langle\sigma,  \iota,  \dec(\expr_1 \op \expr_2)
%   \rangle \lirsem \TT{n}^{(k, \{\})}, \iota', \trace}
% \and
% %%%%%%%%%%%%%%%%%%OPER-COP
% \inferrule*[left=\mbox{\labelthis{lir:exp:dn}{dcopn}}]
% {\langle \sigma,  \iota,  \expr_1 \rangle \lirsem
%   \TT{n}_1^{(k_1, \dep_1)} , \iota_1, \trace_1 \\
%  \langle \sigma, \iota_1, \expr_2  \rangle \lirsem
%  \TT{n}_2^{(k_2, \dep_2)},  \iota_2, \trace_2  \\\\
% \TT{n} = \TT{n}_1 \op \TT{n}_2 \\ \dep =\dep_1 \cup \dep_2 \\
% k = k_1 \sqcup k_2 \\ 
% \pc \not\sqsubseteq k \\
% % \pc \not\sqsubseteq (k, \dep) \bigwedge
% % \Gamma(\pc) \not\sqsubseteq k \\ 
% % \forall x \in \dep. (l = l \sqcup \ilabel(x)) \qquad
% \trace = \trace_1 :: \trace_2
% }
% {\langle\sigma,  \iota,  \dec(\expr_1 \op \expr_2)
%   \rangle \lirsem \TT{n}^{(k, \dep)},
%   \iota_2, \trace}
\end{mathparpagebreakable}
%%%%%%%%%%%%%%%%%%AF used above
\begin{flushleft}
\textbf{Auxiliary functions:}
\begin{align*}
\Delta(\dep, k, \ilabel) =&~ \big\{\TT{x}~\big|~(\TT{x} \in \dep) \wedge
                            (\ilabel(\TT{x}) \not\sqsubseteq k) \big\} \\
\Reduce(\dep, \iota, k, \blabel) =&~ \left\{\TT{x}~\middle|~(\TT{x} \in \dep) \wedge \big(\iota(\TT{x})
= 0 \, \vee \, k \not\sqsubseteq \blabel(\TT{x})\big)\right\} \\
\mathbb{I}(\iota, \dep) =&~\lambda {\TT{x}}. 
\begin{cases}
\iota({\TT{x}}) - 1, &\textit{if} ~ ({\TT{x}} \in \dep) \\
\iota({\TT{x}}), &\textit{otherwise}
\end{cases}
% \mathbb{I}(\iota, \dep) =&~ \left\{(\TT{x},
%                            \TT{n})~\middle|~\big((\TT{x}, \TT{j}) \in
%                            \iota\big) \bigwedge \left(\TT{n}
% =
% \left\{\begin{array}{ll}
%         \TT{j} - 1, & \textit{if} ~ (\TT{x} \in \dep) \\
%         \TT{j}, & \textit{otherwise}
%         \end{array}\right\} 
% \begin{cases}
% \iota(x) - 1 & \mathit{if} ~ x \in \dep \\
% \iota(x) & \mathit{else}
%  % (~?~ \iota(x) - 1 ~:~ \iota(x))\big)\big\} 
% \end{cases}
%\right)\right\}
\end{align*}
\end{flushleft}
\end{framed}
\caption{LIR - Semantics of expressions}
\label{fig:lir:sem-e}
\end{figure}

\begin{figure}[!htbp]
\begin{framed}
\begin{mathparpagebreakable}
%%%%%%%%%%%%%%%%%%SKIP
\inferrule*[left=\mbox{\labelthis{lir:cmd:sk}{skip}}]
{ }
{\langle   \sigma, \iota,   \sk \rangle \lirsem
  \langle \sigma, \iota,  \emptyTrace \rangle}
\and
%%%%%%%%%%%%%%%%%%IF
\inferrule*[left=\mbox{\labelthis{lir:cmd:ie}{if-else}}]
{\langle   \sigma, \iota,   \expr \rangle \lirsem
  \TT{n}^{(k_o, \dep)}, \iota', \trace_1 \\ 
i = \left\{\begin{array}{ll}
        1, & \textit{if } (\TT{n} = \texttt{true}) \arcr
        2, & \textit{otherwise } \arcr
        \end{array}\right\} \\\\
k = k_o \bigsqcup\limits_{\TT{x} \in \dep} \ilabel(\TT{x})
\\ \langle   \sigma,  \iota', \comm_i
  \rangle \lirsemup \langle \sigma', \iota'',
  \trace_2 \rangle \\
}
{\langle   \sigma,  \iota,
  (\texttt{if}~\expr~\texttt{then}~\comm_1~\texttt{else}~\comm_2)
  \rangle \lirsem  \langle \sigma', \iota'',
  \trace_1 :: \trace_2 \rangle}
\and
%%%%%%%%%%%%%%%%%%SEQ
\inferrule*[left=\mbox{\labelthis{lir:cmd:s}{seq}}]
{\langle   \sigma,  \iota,  \comm_1 \rangle \lirsem
  \langle  \sigma', \iota', \trace_1 \rangle  \\\\ 
  \langle   \sigma',
  \iota', \comm_2 \rangle \lirsem \langle \sigma'',
  \iota'', \trace_2 \rangle
}
{\langle   \sigma, \iota,  \comm_1;\comm_2 \rangle
  \lirsem \langle \sigma'', \iota'', (\trace_1
  :: \trace_2) \rangle}
\and
%%%%%%%%%%%%%%%%%%ASSN
\inferrule*[left=\mbox{\labelthis{lir:cmd:an}{assn}}]
{ 
  \langle  \sigma, \iota,  \TT{x} \rangle \lirsem \TT{\_}^{(l, \_)} ,
  \iota, \emptyTrace \\ 
  \pc \sqsubseteq l \\\\
  \langle  \sigma, \iota,  \expr \rangle \lirsem \TT{n}^{(m, \dep')} ,
  \iota', \trace \\\\ 
  k = \pc \sqcup m \\ \dep = \Delta(\dep', k, \ilabel)
}
{\langle   \sigma, \iota,  \TT{x} := \expr \rangle
  \lirsem \langle \sigma[\TT{x} \mapsto
  \TT{n}^{(k, \dep)}], \iota', \trace \rangle}
\and
%%%%%%%%%%%%%%%%%%W_FALSE
\inferrule*[left=\mbox{\labelthis{lir:cmd:wf}{while-f}}]
{\langle   \sigma, \iota,  \expr \rangle \lirsem
  \texttt{false}^{(k, \dep)}, \iota', \trace 
}
{\langle   \sigma,\iota,
  \texttt{while}~\expr~\texttt{do}~\comm
  \rangle \lirsem  \langle \sigma, \iota', \trace \rangle}
\and
%%%%%%%%%%%%%%%%%%W_TRUE
\inferrule*[left=\mbox{\labelthis{lir:cmd:wt}{while-t}}]
{\langle   \sigma, \iota,  \expr \rangle \lirsem
  \texttt{true}^{(k_o, \dep)}, \iota_1, \trace_1 \\ 
k = k_o \bigsqcup\limits_{\TT{x} \in \dep} \ilabel(\TT{x})
\\\\ \langle   \sigma,  \iota_1, \comm;
\texttt{while}~\expr~\texttt{do}~\comm
  \rangle \lirsemup \langle \sigma', \iota',
  \trace_2  \rangle \\
}
{\langle   \sigma,  \iota,
  \texttt{while}~\expr~\texttt{do}~\comm
  \rangle \lirsem  \langle \sigma', \iota',
  \trace_1 :: \trace_2 \rangle}
\end{mathparpagebreakable}
\end{framed}
\caption{LIR - Semantics of commands}
\label{fig:lir:sem-c}
\end{figure}

The evaluation rules for expressions are shown in Figure~\ref{fig:lir:sem-e}.
The rule \refrule{lir:exp:c} evaluates constant values to themselves with
  the label $(\bot, \{\})$ as they do not have any dependencies.  
Rule \refrule{lir:exp:v} evaluates a variable and returns its value $\TT{n}$ along
  with the label $(k_o, \dep_o)$. The function $\Reduce$ returns those
  variables in $\dep_o$ whose budget has expired or have a
  budget-label lower than the secrecy-level $k_o$. The current
  secrecy-level $k_o$ is then joined with the value-labels of the
  variables returned by $\Reduce$ while removing them from the
  dependency-set $\dep_o$.  As no new information is released in both
  the cases, there is no change to the budget store $\iota$ and no
  trace is generated (indicated as an empty trace $\emptyTrace$).  
  
In the rules for evaluating arithmetic (\refrule{lir:exp:aop})
and normal comparison (\refrule{lir:exp:cop}) operations, 
the join of the label of the values that the two sub-expressions 
evaluate to is set as the label of the expression. The individual 
sub-expressions may release some information as captured in 
$\trace_1$ and $\trace_2$, thus updating $\iota$ to $\iota'$.

The \refrule{lir:exp:dr} rule corresponds to the comparison operation
with \dec, and allows information release in certain cases. The
separation of \TT{declassify} from the normal comparison operations is
to offer more control over what information is required to be
released. The rule applies only when the current context $\pc$ is
lower than or equal to the label of the value obtained by evaluation
($\pc \sqsubseteq k_o$). This avoids the leaks specified above
in Section~\ref{aspect:blc} and Section~\ref{aspect:br}. The function
$\Delta(\dep, l, \ilabel)$ returns the variables in the dependency-set
$\dep$ that have an value-label greater than $l$. Additionally, variables
whose budget has expired during the evaluation of the sub-expressions
are removed from the dependency-set (using the function $\Reduce$). 
To prevent the leak described earlier in Section~\ref{aspect:blc}, if the budget-label of any of
the remaining variables in the dependency-set is not at least as high
as the original secrecy-level of the computed value
($k_o \not\sqsubseteq \blabel(\TT{x})$), then the variable is removed
from the dependency-set. For all the variables removed from the
dependency-set, their secrecy-level is joined with that of the actual
label. The budget of all remaining
variables in the dependency-set $\dep$ is deducted by $1$,
corresponding to the $1$-bit Boolean value released on the trace. The
function $\mathbb{I}(\iota, \dep)$ reduces the budget and returns a
new budget store. Their budget-label is also joined with the actual
secrecy-level, which gives the final label of the declassified value.
  Depending on whether the provenance set $\dep$ is empty or not,
  either:  
  \begin{itemize}
  \item the budget of all remaining variables in the provenance set
    $\dep$ is deducted by $1$, corresponding to the $1$-bit Boolean
    value released on the trace. The function $\mathbb{I}(\iota,
    \dep)$ reduces the budget and returns a new budget store. Their
    budget label is also joined  with the actual security level, which
    gives the final label of the declassified value \emph{or}
  \item if the remaining provenance set is empty, no declassification
    occurs
  \end{itemize}
  If declassification occurs, the declassified value is appended to
  the trace.
  
If the current context ($\pc$) is not lower than or equal to the 
secrecy-level of the value obtained by evaluation ($\pc \not\sqsubseteq k$), 
no information is released as shown in the rule \refrule{lir:exp:dn}. 
The final label is a join of the secrecy-levels of all the dependencies. 
The individual sub-expressions may release some information $\trace_1$ and 
$\trace_2$, thus updating $\iota$ to $\iota_2$. 

The judgment for a command execution is given by $\langle \sigma,
\iota, \comm \rangle \lirsem \langle \sigma', \iota', \trace \rangle$ 
--- under a program context $\pc$, the execution of a
command $\comm$ starting with memory $\sigma$ and budget store $\iota$
results in a final memory $\sigma'$ and budget store $\iota'$ while
generating a trace $\trace$ of released values. The semantics is shown in
Figure~\ref{fig:lir:sem-c} whose rules are standard except for some changes
in the assignment and branching rules. The assignment rule
\refrule{lir:cmd:an} does the standard NSU check that disallows
assignment to a public variable in a secret context --- $\pc
\sqsubseteq l$, where $l$ is the secrecy level of the variable
\TT{x} to which the assignment is being made. 
The branching rules \refrule{lir:cmd:ie} and
\refrule{lir:cmd:wt} compute the context label by joining the
value-labels ($\ilabel$) of all the dependencies in $\dep$. 

\section{Formalization of LIR}
\label{sec:lir-formal}

The following section formalizes the property of LIR and prove it for
the semantics presented above. 
LIR is essentially a property stating
that the total number of bits that can be leaked about a secret (by
a program) is upper bounded by its pre-specified budget 
(in an average sense). The trace
generated in the LIR semantics captures the data about the declassified 
value. An adversary at level $\attacker$ can view those values in a 
memory store $\sigma$ that have a label less than or equal to $\attacker$ 
($\forall x. \Gamma(\sigma(x)) \sqsubseteq \attacker$). The adversary can 
also observe the projection of the trace generated as part of the 
semantics, defined formally in Definition~\ref{def:lir:tp}. Similarly, 
the adversary can view a projection of the budget-map as defined 
in Definition~\ref{def:lir:bmp}.  

\begin{mydef}[Trace projection]
\label{def:lir:tp}
Given a trace, $\trace$, the trace projection w.r.t. an adversary at
level $\attacker$, written $\trace\proj$, is 
% just a standard filter over a list 
defined as: 
  \begin{align*}
  []\proj &=~[] \\
  (\emph{\TT{n}}^m :: \trace)\proj &=
  \begin{cases}
   \emph{\TT{n}}^m :: \trace\proj~ & \mathit{if }~ m \sqsubseteq \attacker,\\
   \trace\proj & \mathit{else}.
  \end{cases}
  \end{align*}
\end{mydef}

\begin{mydef}[Budget-map projection]
\label{def:lir:bmp}
The projection of a budget-map, $\iota$, w.r.t. an adversary at
level $\attacker$, written $\iota\proj$, is defined as:
\begin{align*}
\iota\proj = \lambda \emph{\TT{x}}. 
\begin{cases}
\iota(\emph{\TT{x}}), &\text{if} ~ \big(\blabel(\emph{\TT{x}}) \sqsubseteq \attacker\big) \\
0, &\text{otherwise}
\end{cases}
\end{align*}
\end{mydef}
% \begin{mydef}[Budget map projection]
% \label{def:lir:bmp}
% The projection of a budget map, $\iota$, w.r.t. an adversary at
% level $\attacker$, written $\iota\proj$, is defined as:
% $$\iota\proj = 
% \left \{
% (\emph{\TT{x}}, \emph{\TT{n}})~\middle|~\big((\emph{\TT{x}},
% \emph{\TT{j}}) \in \iota\big) \bigwedge \left(\emph{\TT{n}} =
% \left\{\begin{array}{ll}
%         0, & \text{if} ~ (\blabel(x) \not\sqsubseteq \attacker) \\
%         \emph{\TT{j}}, & \text{otherwise}
%         \end{array}\right\} 
% % \forall x \in \iota. \big(\iota\proj(x) := (\blabel(x)
% % \not\sqsubseteq \attacker ~?~ 0 : \iota(x))
% \right)\right\}
% $$
% \end{mydef}

% \begin{mydef}
% \label{def:lir:dbmp}
% The difference between two budget maps, $\iota$ and $\iota'$, written
% $\iota - \iota'$, is defined as:
% $$\iota - \iota' = \sum\limits_{\emph{\TT{x}} \in \iota}
% \big(\iota(\emph{\TT{x}}) - \iota'(\emph{\TT{x}})\big)$$ 
% \end{mydef}

The observational equivalence of various data structures used in the
semantics with respect to an adversary needs to be defined for
formally defining LIR. Definition~\ref{def:lir:veq}
and~\ref{def:lir:seq} define the  observational equivalence of two
values and two memory stores,  respectively, with respect to an
adversary at level $\attacker$. Definition~\ref{def:lir:beq}
and~\ref{def:lir:teq} define  the observational equivalence of the
budget-maps and the generated traces. 

\begin{mydef}
	\label{def:lir:veq}
	Two labeled values $v_1 = \emph{\TT{n}}_1^{(l_1,\dep_1)}$ and $v_2 =
	\emph{\TT{n}}_2^{(l_2,\dep_2)}$ are observationally equivalent at level
	$\attacker$, written $v_1\eq v_2$ iff either:
	\begin{enumerate}
		\item $\emph{\TT{n}}_1 = \emph{\TT{n}}_2$, % $\Gamma(v_1) = \Gamma(v_2) \sqsubseteq \attacker$,
		$l_1 = l_2 \sqsubseteq \attacker$ and $\dep_1 = \dep_2$ (or)
		\item $\Gamma(v_1) \not\sqsubseteq  \attacker$ and $\Gamma(v_2)
		\not\sqsubseteq \attacker$ and $l_1 = l_2 \sqsubseteq
		\attacker$ and $\dep_1 = \dep_2$
		% $\exists x^m \in \dep_1.(m \not\sqsubseteq
		% \attacker)$ and $\exists y^k \in \dep_2.(k \not\sqsubseteq \attacker)$
		(or)
		\item $\Gamma(v_1) \not\sqsubseteq  \attacker$ and $\Gamma(v_2)
		\not\sqsubseteq \attacker$ and $l_1 \not\sqsubseteq
		\attacker \vee \exists \emph{\TT{x}} \in
		\dep_1. \blabel(\emph{\TT{x}}) \not\sqsubseteq 
		\attacker$ and $l_2 \not\sqsubseteq \attacker \vee \exists
		\emph{\TT{x}} \in   \dep_2. \blabel(\emph{\TT{x}}) \not\sqsubseteq
		\attacker$ 
	\end{enumerate}
\end{mydef}
% \dg{The two ``such that''s in the last two points should also be
%   ``and''s.}

\begin{mydef}
	\label{def:lir:seq}
	Two memory stores $\sigma_1$ and $\sigma_2$ are observationally equivalent
	at level $\attacker$, written $\sigma_1 \eq \sigma_2$ iff
	$\forall \emph{\TT{x}}. \; \sigma_1(\emph{\TT{x}}) \eq \sigma_2(\emph{\TT{x}})$.
\end{mydef}

\begin{mydef}
	\label{def:lir:beq}
	Two budget-maps $\iota$ and $\iota'$ are equivalent at level
	$\attacker$, written $\iota \eq \iota'$, iff
	$\forall \emph{\TT{x}}. \blabel(\emph{\TT{x}}) \sqsubseteq \attacker
	\implies \iota(\emph{\TT{x}}) = \iota'(\emph{\TT{x}})$.
\end{mydef}

\begin{mydef}
	\label{def:lir:teq}
	Two traces $\trace$ and $\trace'$ are equivalent at level
	$\attacker$, written $\trace \eq \trace'$, iff
	$\trace\proj = \trace'\proj$.
\end{mydef}

As per LIR, the length of the projected trace is bounded by 
the total budget deducted. The length of a trace $\trace$ is
represented as $|\trace|$. This intuition of LIR is formalized in 
Definition~\ref{def:lir:lir}. 

\begin{mydef}[Limited information release]
\label{def:lir:lir}
A program $\comm$ is said to satisfy \emph{limited information release}
w.r.t.\ an adversary at level $\attacker$ if for any given memory
store $\sigma$, % containing \emph{a} secret input
$\iota$, and $\pc$,
$\langle \sigma, \iota, \comm \rangle \lirsem \langle
\sigma', \iota', \trace \rangle$ then $|\trace\proj| \leq
\iota\proj - \iota'\proj$ 
% $$\iota\swarrow_\attacker = \mathop {\Large\Sigma}_{x \in \iota} \big(\blabel(x) \sqsubseteq
% \attacker ~? ~\iota(x)~ :~ 0 \big) $$
\end{mydef}

Theorem~\ref{thm:lir} shows that the monitored semantics presented in 
Section~\ref{sec:semantics} satisfy limited information release for every 
secret in the memory. 
Lemma~\ref{lem:lir:conf} proves that in a secret context with respect to
an adversary at level $\attacker$, no declassification visible to the 
adversary occurs. Theorem~\ref{thm:lir:cmdtrli} proves that in
a store containing only one secret value with respect to 
$\attacker$-level adversary (the other secrets can be assumed to be visible 
to the adversary), the length of the trace is bounded by the budget 
reduced, i.e., the number of bits declassified is bounded by the 
deduction in budget of that secret value. Theorem~\ref{thm:lir} 
generalizes this result to all secrets in the system. The proofs are
detailed in Appendix~\ref{sec:app:lir}. 

\begin{myLemma}[Confinement]
  \label{lem:lir:conf}
  If $\langle \sigma, \iota, \comm \rangle
  \lirsem \langle \sigma', \iota', \trace \rangle$,
  and $\Gamma(\pc) \not\sqsubseteq \attacker$, then 
  $\sigma \eq \sigma'$, $\iota \eq \iota'$ and $\trace\proj =
  \emptyTrace$ 
\end{myLemma}

\begin{myThm}[Limited information release for a single secret]
\label{thm:lir:cmdtrli}
If 
$\langle \sigma, \iota, \comm \rangle \lirsem \langle
  \sigma', \iota', \trace \rangle$, and 
$\exists \emph{\TT{x}} \in \sigma. \Big(\Gamma(\sigma(\emph{\TT{x}}))
\not\sqsubseteq \attacker \wedge \big(\forall \emph{\TT{y}} \in
\sigma. \emph{\TT{y}} \neq \emph{\TT{x}} \wedge 
\Gamma(\sigma(\emph{\TT{y}})) \sqsubseteq \attacker\big)\Big)$, 
then  $|\trace\proj| \leq \iota\proj(\emph{\TT{x}}) - \iota'\proj(\emph{\TT{x}})$. 
\end{myThm}

\begin{myThm}[Limited information release]
\label{thm:lir}
For any memory store $\sigma$, budget store $\iota$ and program
$\comm$ if $\langle \sigma, \iota, \comm \rangle \lirsem \langle \sigma',
\iota', \trace \rangle$, then $\comm$ satisfies limited information release for
$\sigma$ and $\iota$
\end{myThm}

\section{Soundness and Decoding Semantics}
\label{sec:lir:decoding}

The LIR policy defined earlier enforces that the information released 
by a program about every secret is bounded by its pre-determined budget. 
However, to prove that the actual information released by the approach is 
soundly accounted for, it needs to be shown that the the budget
reduction performed as part of LIR is well-founded in an information
theoretic sense. To this end, the result of Shannon's source coding
theorem~\referp{shannon} is leveraged to utilize the trace generated
by the semantics containing the declassified values. 

Recall that Shannon's source coding theorem states that if 
different outputs of a program can be associated with uniquely-decodable
codes, i.e., given a bit string there are no two different  
interpretations of the sequence of codewords, the information 
leaked by the program about the secret inputs through these outputs 
as computed using Shannon entropy is upper-bounded by the average 
length of these codes. The trace projected to an adversary contains all  
the bits the adversary can observe about the secret inputs 
to the program. The projected trace can, thus, be regarded as a coding 
of the information released about the secrets to the adversary. 
Hence, if it can be proven that the trace projected to an adversary
is uniquely-decodable then its average length over all executions
is an upper bound on the information leaked by the program as 
computed by the definition of Shannon entropy. 

In order to prove the soundness of the approach, this section presents a  
decoding semantics that simulates the adversary's approach to decode the 
information it obtains via the trace. The purpose of the decoding 
semantics is to show that the adversary-projected trace 
corresponds to a uniquely-decodable code, i.e., decoding the trace would 
result in a unique final memory store for the adversary, thereby showing 
that our enforcement is sound. The salient features of the semantics are:
\begin{enumerate}
\item Decoding semantics is specialized to a fixed adversary at level
   $\attacker$.
\item Decoding semantics operates on adversary-projected data structures,
  i.e., an $\attacker$-projected memory, where the secret inputs are 
  replaced by $\star$ that represents an unknown 
  value (as shown in Definition~\ref{def:lir:msp}) and
  $\attacker$-projected budget store, where the budgets having a 
  budget-label higher than the adversary's level  
  are replaced by $0$ (as in Definition~\ref{def:lir:bmp}). 
  % $$\iota\proj = \mathop{\Large \forall}_{x \in \sigma} \Big(\iota\proj(x) = -^l
  % \wedge l \sqsubseteq \attacker ~?~ \blabel(x) ~:~ 0 \Big)$$
\item At declassification points, the decoding semantics
  reads a value from the $\attacker$-projected trace. 
\item Decoding semantics completely skips the code that is executed 
  under a $\pc$ influenced by secrets (having the value $\star$).
\end{enumerate}

% In order to prove that the projected trace corresponds to a uniquely-decodable
% code, this section presents a decoding semantics that is almost
% similar to the LIR semantics from Section~\ref{sec:semantics}, except
% for the following differences:
% \begin{enumerate}
% \item Decoding semantics is specialized to a fixed adversary,
%   represented as $\attacker$.
% \item Decoding semantics operates on an $\attacker$-projected memory
%   (where the secret inputs are replaced by $\star$, representing an unknown
%   value as shown in Definition~\ref{def:lir:msp}) and
%   $\attacker$-projected budget store (where secret budget 
%   values are replaced by $0$ as in Definition~\ref{def:lir:bmp}). 
%   % $$\iota\proj = \mathop{\Large \forall}_{x \in \sigma} \Big(\iota\proj(x) = -^l
%   % \wedge l \sqsubseteq \attacker ~?~ \blabel(x) ~:~ 0 \Big)$$
% \item At declassification points, the decoding semantics
%   reads a value from the trace, which is also $\attacker$-projected. 
% \item Decoding semantics completely skips the code that is executed under a
%   $\pc$ influenced by $\star$.
% \end{enumerate}

\begin{mydef}[Memory store projection]
\label{def:lir:msp}
The projection of a memory store, $\sigma$, w.r.t. an adversary at
level $\attacker$, written $\sigma\proj$, is defined as:
\begin{align*}
\sigma\proj = \lambda \emph{\TT{x}}. 
\begin{cases}
\sigma(\emph{\TT{x}}), &\text{if} ~ \big(\Gamma(\sigma(\emph{\TT{x}})) \sqsubseteq \attacker\big) \\
\star^{(\bot,\{\emph{\TT{x}}\})}, &\text{otherwise}
\end{cases}
\end{align*}
\end{mydef}
% \begin{mydef}[Memory store projection]
% \label{def:lir:msp}
% The projection of a memory store, $\sigma$, w.r.t. an adversary at
% level $\attacker$, written $\sigma\proj$, is defined as:
% $$\sigma\proj = \Pi(\sigma, \attacker) = 
% \left \{\emph{\TT{x}}~\middle|~ \left(\emph{\TT{x}} =
% \left\{\begin{array}{ll}
%         \emph{\TT{y}}, & \text{if} ~ (\emph{\TT{y}} \in \sigma) \wedge \big(\Gamma(\sigma(\emph{\TT{y}})) \sqsubseteq \attacker\big) \\
%          \star^{(\bot,\{\emph{\texttt{x}}\})}, & \text{otherwise}
%         \end{array}\right\} 
% \right)\right\}
% $$
% $$\sigma\proj = \mathop{\Large \forall}_{x \in \sigma} \Big(\sigma\proj(x) =
%   \big(\Gamma(\sigma(x)) \sqsubseteq \attacker ~?~ \sigma(x) ~:~ \star^{(\bot,
%     \{\texttt{x}\})} \big)\Big)$$
% \end{mydef}

\begin{figure}[!htbp]
\begin{framed}
\begin{mathparpagebreakable}
%%%%%%%%%%%%%%%%%%CONST
\inferrule*[left=\mbox{\labelthis{lir:exp:sc}{s-const}}]
{ } 
{\langle\sigma, \iota, \TT{n}, \trace \rangle \simsem
  \TT{n}^{(\bot,\{\})}, \iota, \trace}
\and
%%%%%%%%%%%%%%%%%%VAR
\inferrule*[left=\mbox{\labelthis{lir:exp:sv}{s-var}}]
{\sigma(\TT{x})  = \TT{v}^{(k, \dep_o)} \\ 
  \dep' = \Reduce(\dep_o, \iota, k_o, \blabel) \\
  \dep = \dep_o \setminus \dep' \\ 
  k = k_o  \bigsqcup\limits_{\TT{x}  \in \dep'} \ilabel(\TT{x})  }
{\langle\sigma, \iota, \TT{x}, \trace \rangle \simsem \TT{v}^{(k, \dep)},
  \iota, \trace}
\and
%%%%%%%%%%%%%%%%%%AOP
\inferrule*[left=\mbox{\labelthis{lir:exp:saop}{s-aop}}]
{\langle \sigma, \iota, \expr_1, \trace \rangle \simsem
  \TT{v}_1^{(k_1, \dep_1)}, \iota_1, \trace_1 \\ \langle \sigma,
  \iota_1,   \expr_2, \trace_1 \rangle \simsem \TT{v}_2^{(k_2, \dep_2)},
  \iota', \trace' \\ \TT{v}= \TT{v}_1 \aop \TT{v}_2 \\ (k, \dep) =
  (k_1, \dep_1) \sqcup (k_2, \dep_2) 
}
{ \langle\sigma,  \iota,  (\expr_1 \aop \expr_2), \trace \rangle
  \simsem\TT{v}^{(k, \dep)}, \iota', \trace'}
\and
%%%%%%%%%%%%%%%%%%COP
\inferrule*[left=\mbox{\labelthis{lir:exp:scop}{s-cop}}]
{\langle \sigma, \iota, \expr_1, \trace \rangle \simsem
  \TT{v}_1^{(k_1, \dep_1)}, \iota_1, \trace_1 \\ \langle \sigma,
  \iota_1,   \expr_2, \trace_1 \rangle \simsem \TT{v}_2^{(k_2, \dep_2)},
  \iota', \trace' \\ \TT{v}= \TT{v}_1 \cop \TT{v}_2 \\ (k, \dep) =
  (k_1, \dep_1) \sqcup (k_2, \dep_2) 
}
{ \langle\sigma,  \iota,  (\expr_1 \cop \expr_2), \trace \rangle
  \simsem\TT{v}^{(k, \dep)}, \iota', \trace'}
\and
%%%%%%%%%%%%%%%%%%OPER-COP
\inferrule*[left=\mbox{\labelthis{lir:exp:sdn}{s-dcopn}}]
{
\langle\sigma,  \iota,  (\expr_1 \cop \expr_2), \trace \rangle
  \simsem\TT{v}^{(k, \dep)}, \iota', \trace' \\
  \pc \not\sqsubseteq k 
}
{\langle\sigma,  \iota, \dec(\expr_1 \op \expr_2), \trace
  \rangle \simsem\TT{v}^{(k, \dep)}, \iota', \trace'}
\and
%%%%%%%%%%%%%%%%%%OPER-COP
\inferrule*[left=\mbox{\labelthis{lir:exp:sdr}{s-dcopr}}]
{
  \langle\sigma,  \iota,  (\expr_1 \cop \expr_2), \trace \rangle
  \simsem\TT{v}_o^{(k_o, \dep_o)}, \iota_1, \trace_1 \\
  pc \sqsubseteq k_o   \\
  \dep' = \Delta(\dep_o, k_o, \ilabel) \\
  \dep'' = \Reduce(\dep',\iota_1, k_o, \blabel) \\ 
  \dep = \dep' \setminus \dep'' \\ 
  m =  k_o \bigsqcup\limits_{\TT{x} \in \dep''}
  \ilabel(\TT{x}) \bigsqcup\limits_{\TT{y} \in \dep} \blabel(\TT{y}) \\
%\iota' = \mathbb{I}(\iota_1, \dep)\\
 ( \TT{v}, k, \iota', \trace') =  \left\{\begin{array}{ll}
        (\TT{v}_t, k_t, \mathbb{I}(\iota_1, \dep_t), \trace''), & \textit{if} ~(\trace_1 = \TT{v}_t^{k_t} :: \trace'') \wedge (k_t = m) \wedge (\dep \neq \emptyset)\arcr
        (\TT{v}_{o}, m, \iota_1, \trace_1), & \textit{otherwise}
        \end{array}\right\} 
% pc \sqsubseteq k_o   \\
%  \dep' = \Delta(\dep_o, k_o, \ilabel) \\
% \dep'' = \Reduce(\dep',\iota_2, k_o, \blabel) \\ 
%   \dep = \dep' \setminus \dep'' \\ 
% m =  k_o \bigsqcup\limits_{\TT{x} \in \dep''}
%   \ilabel(\TT{x}) \bigsqcup\limits_{\TT{y} \in \dep} \blabel(\TT{y}) \\
% \iota' = \mathbb{I}(\iota_2, \dep)\\
%  (\trace', \TT{v}, k) =  \left\{\begin{array}{ll}
%         (\trace'', \TT{v}_t, k_t), & \textit{if} ~~ (\dep \neq \emptyset) \wedge (\trace_2 = \TT{v}_t^{k_t} :: \trace'') \arcr
%         (\trace_2, \TT{v}_{o}, m), & \textit{otherwise}
%         \end{array}\right\} 
% \trace' = \left\{\begin{array}{ll}
%         \trace'', & \textit{if} ~ (\dep \neq \emptyset) \wedge (\trace_2 =\TT{v}:: \trace'') \arcr
%         \trace_2, & \textit{otherwise}
%         \end{array}\right\} \\
% (\dep \neq \emptyset) ~?~ (\trace_2 =\TT{v}^l :: \trace') : 
% (\trace' = \trace_2 \wedge v=\TT{v}_o \wedge l = l_o \bigsqcup\limits_{x \in \dep}
%   (\ilabel(\TT{x}) \sqcup \blabel(\TT{x})))
}
{\langle\sigma,  \iota, \dec(\expr_1 \op \expr_2), \trace
  \rangle \simsem\TT{v}^{(k, \{\})}, \iota', \trace'}
% \and
% %%%%%%%%%%%%%%%%%%OPER-COP
% \inferrule*[left=\mbox{\labelthis{lir:exp:sdn}{s-dcopn}}]
% {\langle \sigma,  \iota, \expr_1, \trace \rangle \simsem
%   \TT{v}_1^{(k_1, \dep_1)}, \iota_1, \trace_1 \\
%   \langle \sigma, \iota_1, \expr_2,\trace_1  \rangle \simsem
%   \TT{v}_2^{(k_2, \dep_2)}, \iota_2, \trace_2  \\\\
%   \TT{v} = \TT{v}_1 \op \TT{v}_2
%   \\ \dep = \dep_1  \cup \dep_2 \\ k = k_1 \sqcup k_2
%   \\ pc \not\sqsubseteq k \\
% }
% {\langle\sigma,  \iota, \dec(\expr_1 \op \expr_2), \trace
%   \rangle \simsem\TT{v}^{(k, \dep)}, \iota_2, \trace_2}
\end{mathparpagebreakable}
\end{framed}
\caption{Decoding semantics of expressions}
\label{fig:lir:sem-sim-e}
\end{figure}

\begin{figure}[!tb]
\begin{framed}
\begin{mathpar}
%%%%%%%%%%%%%%%%%%SKIP
\inferrule*[left=\mbox{\labelthis{lir:cmd:ssk}{s-skip}}]
{ }
{\langle   \sigma, \iota, \sk, \trace \rangle \simsem
  \langle \sigma, \iota, \trace \rangle}
\and
%%%%%%%%%%%%%%%%%%ASSN
\inferrule*[left=\mbox{\labelthis{lir:cmd:sa}{s-assn}}]
{ \pc \sqsubseteq \Gamma(\sigma(\TT{x})) \\
\langle   \sigma, \iota, \expr, \trace \rangle
\simsem \TT{n}^{(m, {\dep'})}, \iota', \trace'
\\  k = \pc \sqcup m \\ \dep = \Delta(\dep', k, \ilabel)
}
{\langle   \sigma,  \iota, (\TT{x} := \expr), \trace \rangle
 \simsem \langle \sigma[\TT{x} \mapsto
  \TT{n}^{(k,{\dep})}], \iota', \trace' \rangle}
\and
%%%%%%%%%%%%%%%%%%SEQ
\inferrule*[left=\mbox{\labelthis{lir:cmd:ss}{s-seq}}]
  {\langle   \sigma, \iota, \comm_1, \trace \rangle \simsem
  \langle  \sigma', \iota', \trace_1 \rangle  \\ \langle   \sigma',
  \iota', \comm_2, \trace_1 \rangle \simsem \langle
  \sigma'',\iota'', \trace' \rangle
  }
  {\langle   \sigma, \iota, \comm_1;\comm_2, \trace \rangle
  \simsem \langle \sigma'', \iota'', \trace' \rangle}
\and
%%%%%%%%%%%%%%%%%%IFN
\inferrule*[left=\mbox{\labelthis{lir:cmd:sien}{s-if-else-n}}]
  {\langle   \sigma, \iota,  \expr, \trace \rangle \simsem
 \TT{v}^{(k_o,{\dep})}, \iota', \trace_1 \\\TT{v}\neq \star
  \\
i = \left\{\begin{array}{ll}
        1, & \textit{if} ~ (\TT{v} = \texttt{true}) \arcr
        2, & \textit{otherwise}
        \end{array}\right\} \\
k = k_o \bigsqcup\limits_{\TT{x} \in \dep} \ilabel(\TT{x})
\\ \langle   \sigma, \iota', \comm_i, \trace_1
  \rangle \simsemup \langle \sigma', \iota'',
  \trace' \rangle}
  {\langle   \sigma, \iota,
  (\texttt{if}~\expr~\texttt{then}~\comm_1~\texttt{else}~\comm_2),
  \trace \rangle \simsem  \langle \sigma', \iota'',
  \trace' \rangle}
\and
%%%%%%%%%%%%%%%%%%IFS
\inferrule*[left=\mbox{\labelthis{lir:cmd:sies}{s-if-else-s}}]
  {\langle \sigma, \iota,  \expr, \trace \rangle \simsem
  \star^{(\_, \_)}, \iota', \trace'
  }
  {\langle   \sigma, \iota,
  (\texttt{if}~\expr~\texttt{then}~\comm_1~\texttt{else}~\comm_2),
  \trace \rangle \simsem  \langle \sigma, \iota', \trace' \rangle}
\and
%%%%%%%%%%%%%%%%%%W_F_S
\inferrule*[left=\mbox{\labelthis{lir:cmd:swfs}{s-while-fs}}]
  {\langle   \sigma, \iota, \expr, \trace \rangle \simsem
 \TT{v}^{(\_, \_)}, \iota', \trace' \\ (\TT{v} = \star) \vee (\TT{v} = \texttt{false})
  }
  {\langle   \sigma,\iota,
  \texttt{while}~\expr~\texttt{do}~\comm, \trace
  \rangle \simsem  \langle \sigma, \iota', \trace' \rangle}
\and
%%%%%%%%%%%%%%%%%%W_T
\inferrule*[left=\mbox{\labelthis{lir:cmd:swt}{s-while-t}}]
  {\langle   \sigma, \iota, \expr, \trace \rangle \simsem
  \texttt{true}^{(k_o,{\dep})}, \iota_1, \trace_1 \\
  k = k_o \bigsqcup\limits_{x \in \dep} \ilabel(\TT{x})
  % \forall x \in \dep.(l := l \sqcup \ilabel(\TT{x}))
  \\ \langle   \sigma, \iota_1, \comm;
  \texttt{while}~\expr~\texttt{do}~\comm, \trace_1
  \rangle \simsemup \langle \sigma', \iota', \trace_2 \rangle
  }
  {\langle   \sigma,  \iota,
  \texttt{while}~\expr~\texttt{do}~\comm, \trace
  \rangle \simsem  \langle \sigma'',\iota'', \trace'' \rangle}

\end{mathpar}
\end{framed}
\caption{Decoding semantics of commands}
\label{fig:lir:sem-sim-c}
\end{figure}

%% We describe the different notations used in the formalism.
%% $\sigma\proj$ for any adversary at level $\attacker$ is the projection
%% of the initial memory $\sigma$ with all the high values
%% $(\not\sqsubseteq \attacker)$ replaced with $\star$, i.e.,
%% $$\sigma\proj = \mathop{\Large \forall}_{x \in \sigma} \Big(\sigma\proj(\TT{x})
%% = \big(\Gamma(\sigma(\TT{x})) \sqsubseteq \attacker \implies \sigma(\TT{x}) \,\diamond\,
%% \star^{(\bot, \{\texttt{x}\})} \big)\Big)$$
%%  The values in the projected memory are represented as either
%%  $\emph{\TT{n}}^{(l,\dep)}$ or $\star^{(l,\dep)}$, where
%% $l$ like before captures the label of all the provenances that have
%% expired their budget and $\dep$ is the set of provenances that
%% affected the value through the execution. The initial labels not lower
%% than or equal to the adversary's level are represent by $\top$ instead
%% of their original label in $\ilabel$ and $\blabel$, i.e.,
%% $$\forall x \in \ilabel. (\ilabel(\TT{x}) = l) \not\sqsubseteq \attacker
%% \implies \ilabel(\TT{x}) = \top$$
%% and similarly for $\blabel$. These are referred to as projected
%% label-map and projected budget-label map.
%% Thus, if any of the secrets in the
%% provenance set expires their budget, the security level $l$ becomes
%% $\top$ indicating that no more information can be released by this
%% value to $\attacker$. Additionally, if the budget label of an initial
%% value (the level to which the release happens) is ``secret'' ($\top$)
%% with respect to the adversary, its initial budget is set to $0$. This
%% is represented by taking a projection of the allowed budgets with
%% respect to the adversary, given by $\iota\proj$
%% % $$\iota\proj = (\forall x \in \iota. \blabel(\TT{x}) \not\sqsubseteq \attacker
%% % \implies \iota(\TT{x}) = 0)$$
%% Similarly, given a trace of released values
%% $\trace$, $\trace\proj$ is the projected trace as described before.
%% % $\Gamma(\sigma(\TT{x}))$ is defined as the join of the label of the value
%% % and the labels of all the variables that the value is dependent on,
%% % i.e., if $\sigma(\TT{x}) = \emph{\TT{n}}^{(l, \dep)}$, then $\Gamma(\sigma(\TT{x})) = l \sqcup
%% % \big(\bigsqcup\limits_{y \in \dep} \ilabel(y)\big)$. Similarly, for $\TT{v}_s =
%% % \sigma^s(\TT{x}) = n_s^{(l,\dep')}$, $\Gamma(\TT{v}_s) = l \sqcup \Gamma(\dep') =
%% % l \bigsqcup\limits_{x \in \dep'} \ilabel(\TT{x})$. Like before,
%% % $\Gamma_f(v^{(l, \dep)})$ returns the first part of the label, i.e., $l$.

%% % We define the simulation semantics of expressions and commands in
%% % Figures~\ref{fig:sem-sim-e} and~\ref{fig:sem-sim-c}.
%% % \TODO{describe the simulation semantics}

Program configurations for expressions and commands are extended with
the projected trace and projections of the memory and budget store,
given by $\langle \sigma\proj, \iota\proj, \expr, \trace\proj \rangle$
and $\langle \sigma\proj, \iota\proj, \comm, \trace\proj \rangle$,
respectively. $\sigma\proj$, $\iota\proj$, and $\trace\proj$
represent the $\attacker$-projected memory, budget-map and trace,
respectively.  

The semantics of expression and command evaluation is shown in
Figures~\ref{fig:lir:sem-sim-e} and~\ref{fig:lir:sem-sim-c} and is
mostly similar to LIR semantics presented earlier
(Figures~\ref{fig:lir:sem-e} and~\ref{fig:lir:sem-c}) except that the
decoding semantics takes the trace  $\trace$ as input. The rules
generate a new trace $\trace'$, which is a suffix of the original
trace $\trace$. The  main difference in the semantics of expressions
shows up in the \refrule{lir:exp:sdr} rule --- when declassification
occurs, the  declassified value is read from the trace (as opposed to
the one computed by local evaluation in the sub-expressions, which is
generally $\star$). The main difference in the semantics of commands
is in the branching rules. Branching or looping on $\star$ values
skips the execution as shown in rules \refrule{lir:cmd:sies} and
\refrule{lir:cmd:swfs}.  

The proof for the projected-trace being uniquely-decodable is based on the
decoding semantics progressing on the trace generated by the LIR
semantics and resulting in a unique final memory with respect to the 
adversary. This requires defining equivalence between the memory 
stores in the two semantics. The memory equivalence
(Definition~\ref{def:lir:simseq}) is defined based on the 
observational equivalence of the values in the original memory store and 
the projected memory store (Definition~\ref{def:lir:eq}). 
The idea behind the definition of value equivalence is that if a value is 
visible to the adversary in the original memory store, then the value is 
also visible to the adversary in the projected store 
(Definition~\ref{def:lir:eq-1}). If the value is not visible to the 
adversary in the original memory store, then either the value shall never 
be declassified to the adversary in either semantics 
(Definition~\ref{def:lir:eq-2a} and~\ref{def:lir:eq-2b}) or the value can 
be declassified to the adversary under both the semantics 
(Definition~\ref{def:lir:eq-2c}). 

\begin{mydef}
\label{def:lir:eq}
Two values $\emph{\TT{v}}_o$ and $\emph{\TT{v}}_s$, where $\emph{\TT{v}}_o$ is a value in the original
memory store and $\emph{\TT{v}}_s$ is a value in the projected memory store for
simulation, are observationally equivalent at level $\attacker$,
written $\emph{\TT{v}}_o \sigeq \emph{\TT{v}}_s$ iff
\begin{enumerate}[label=\emph{\arabic*}., ref={\ref{def:lir:eq}.\arabic*}]
\item\label{def:lir:eq-1} $\emph{\TT{v}}_o = \emph{\TT{n}}_o^{(k, \dep)}$, $\emph{\TT{v}}_s =
  \emph{\TT{n}}_s^{(k',\dep')}$, $\emph{\TT{n}}_o = \emph{\TT{n}}_s$ and
  $(k = k') \sqsubseteq \attacker$ and $\dep = \dep'$ (or)
\item\label{def:lir:eq-2} $\emph{\TT{v}}_o = \emph{\TT{n}}_o^{(k, \dep)}$, $\Gamma(\emph{\TT{v}}_o) \not\sqsubseteq
  \attacker$,  $\emph{\TT{v}}_s = \star^{(k',\dep')}$ and either:
  \begin{enumerate}[label=\emph{(\alph*)}, ref={\ref{def:lir:eq-2}(\alph*)}]
  \item\label{def:lir:eq-2a} $k \not\sqsubseteq \attacker ~\wedge ~k'
    \not\sqsubseteq \attacker$
  \item \label{def:lir:eq-2b} $k \sqsubseteq \attacker~\wedge~\exists \emph{\TT{x}} \in
    \dep. (\blabel(\emph{\TT{x}}) \not\sqsubseteq \attacker) ~\wedge ~k' \not\sqsubseteq
    \attacker$
  \item\label{def:lir:eq-2c} $(k = k') \sqsubseteq \attacker~ \wedge
    ~\dep = \dep'$
  \end{enumerate}
\end{enumerate}
\end{mydef}

\begin{mydef}
\label{def:lir:simseq}
Given a memory store $\sigma$ and a projected memory store
$\sigma'\proj$, their equivalence $\sigeq$ at level
$\attacker$ is defined as: 
$\forall \emph{\TT{x}}.~ \sigma(\emph{\TT{x}}) \sigeq
\sigma'\proj(\emph{\TT{x}})$ 
\end{mydef}


Under a given memory $\sigma$ and budget store
$\iota$, if a program executes to completion under the LIR semantics
and generates a trace $\trace$, then for any chosen adversary
$\attacker$ the same program executing under $\attacker$-projected 
memory ($\sigma\proj$) and budget store ($\iota\proj$) along with the
projected trace $\trace\proj$ results in an observationally equivalent
final memory stores obtained at the end of both the executions with
respect to the adversary $\attacker$ (Theorem~\ref{thm:lir:csim}).

\begin{myThm}[Simulation Theorem]
\label{thm:lir:csim}
If 
$\langle \sigma, \iota, \comm \rangle \lirsem \langle \sigma', \iota',
\trace \rangle$,  then 
% $\exists \sigma'. \sigma \sigeq \sigma_1\proj \wedge \langle
$\langle \sigma\proj, \iota\proj, \comm, \trace\proj
\rangle  \simsem \langle \sigma'',\iota'',
\trace'' \rangle$ such that $\sigma' \sigeq \sigma''$,
$\iota' \eq \iota''$ and $\trace'' = \emptyTrace$
\end{myThm}

The proof of the theorem and the various auxiliary lemmas 
required to prove the theorem are described in
Appendix~\ref{sec:app:lir}.  As a sanity check, we show that when the budgets of
all the secrets in the store are $0$, then the LIR semantics satisfy
non-interference (Corollary~\ref{cor:lir:ni}). 

\begin{mycor}[Non-interference]
\label{cor:lir:ni}
If $\langle \sigma_1, \iota_1, \comm \rangle
\lirsem \langle \sigma_1', \iota_1', \trace_1 \rangle$, 
 $\langle \sigma_2, \iota_2, \comm \rangle
\lirsem \langle \sigma_2', \iota_2', \trace_2 \rangle$, 
 $\sigma_1 \eq \sigma_2$, and $\iota_1\proj = \iota_2\proj = 0$, 
then $\sigma_1' \eq \sigma_2'$.
\end{mycor}

